Taking the direct sum of $T^{(k, \ell)} (V)$ for every $k, \ell \geq 0$, we can form a vector space which is \textit{graded} by the rank of the tensors $(k, \ell)$. In this section we will define two operations which move tensors from one grade to another grade, the first of which, the tensor product $\otimes : T(V) \times T(V) \to T(V)$, endows the \emph{tensor algebra}
	\[ T(V) := \bigoplus_{k, \ell} T^{(k, \ell)} (V) \]
with the structure of an \textit{associative graded algebra}. 	

\subsection{Products}

Define the \emph{tensor product} $\otimes : T^{(k, \ell)}(V) \times T^{(p, q)}(V) \to T^{(k + p, \ell + q)}(V)$ defined by 
	\[ (S \otimes T) (\omega^1, \dots, \omega^{k + p}, v_1, \dots, v_{\ell + q}) = S(\omega^1, \dots, \omega^k, v_1, \dots, v_\ell) T(\omega^{k + 1}, \dots, \omega^{k + p}, v_{\ell + 1}, \dots, v_{\ell + q}). \]
The operation in coordinates amounts to pairwise multiplication, i.e.
	\[ (S \otimes T)^{i_1 \dots i_k i_{k + 1} \dots i_{k + p}}_{j_1 \dots j_\ell j_{\ell + 1} \dots j_{\ell + q}} = S^{i_1 \dots i_k}_{j_1 \dots j_\ell} T^{i_{k + 1} \dots i_{k + p}}_{j_{\ell + 1} \dots j_{\ell + q}}. \]
Using the tensor product, we can construct a natural basis for $T^{(k, \ell)} (V)$ provided a basis for $V$: 	

\begin{proposition}[Basis for $T^{(k, p)} (V)$]
	Let $\{ e_i \}_i \subseteq V$ and $\{ \epsilon^j\}_j \subseteq V^*$ form dual bases, then 
		\[ \{ e_{i_1} \otimes \dots \otimes e_{i_k} \otimes \epsilon^{j_1} \otimes \dots \otimes \epsilon^{j_\ell} \}_{I, J} \subseteq T^{(k, \ell)} (V)\]
	forms a basis. More precisely, each $T \in T^{(k, \ell)} (V)$ admits the unique coordinate representation
		\begin{align*}
			T 
				&= T^{i_1\dots i_k}_{j_1 \dots j_\ell} e_{i_1} \otimes \dots \otimes e_{i_k} \otimes \epsilon^{j_1} \otimes \dots \otimes \epsilon^{j_\ell} ,
		\end{align*}
	with coordinates given by $T^{i_1\dots i_k}_{j_1 \dots j_\ell} := T(\epsilon_{i_1}, \dots , \epsilon_{i_k} ,e^{j_1} , \dots, e^{j_\ell})$. Moreover $\dim T^{(k, \ell)} (V) = (\dim V)^{k + \ell}$.
\end{proposition}

\begin{proof}
	Let $\omega^{(\alpha)} = \omega^{(\alpha)}_j \epsilon^j$ be a collection of co-vectors and $v_{(\beta)} = v^i_{(\beta)} e_i$ a collection of vectors, we use multi-linearity to compute
		\begin{align*}
			T( \omega^{(1)},  \dots, \omega^{(k)}, v_{(1)} , \dots, v_{(\ell)})
				&= T(\epsilon_{i_1}, \dots , \epsilon_{i_k} ,e^{j_1} , \dots, e^{j_\ell})  \omega_{j_1}^{(1)} \dots \omega_{j_k}^{(k)}  v^{i_1}_{(1)} \dots, v^{i_\ell}_{(\ell)} \\
				&=T^{i_1\dots i_k}_{j_1 \dots j_\ell} (e_{i_1} \otimes \dots \otimes e_{i_k} \otimes \epsilon^{j_1} \otimes \dots \otimes \epsilon^{j_\ell}) ( \omega^{(1)},  \dots, \omega^{(k)}, v_{(1)} , \dots, v_{(\ell)}).
		\end{align*}
	This proves the formula. Choosing appropriate combinations of vanishing and non-vanishing coefficients $\omega^{(\alpha)}_j, v^i_{(\beta)} \in \R$ shows that $T^{i_1\dots i_k}_{j_1 \dots j_\ell} = 0$ if and only if $T \equiv 0$, proving uniqueness. 
\end{proof}
	
Using the tensor product and symmetrisation, we can similarly define the \emph{symmetric tensor product} of covariant tensors $\odot : \Sigma^k (V^*) \times \Sigma^p (V^*) \to \Sigma^{k + p} (V^*)$ by  
	\begin{align*}
		(S \odot T)(v_1, \dots, v_k, v_{k + 1}, \dots, v_{k + p}) 
			&:= \operatorname{Sym} (S \otimes T)(v_1, \dots, v_k, v_{k + 1}, \dots, v_{k + p})  \\
			&= \sum_{\sigma \in S_{k + p}} S(v_{\sigma(1)}, \dots, v_{\sigma(k)}) T(v_{\sigma(k + 1)}, \dots, v_{\sigma(k + p)}).
	\end{align*}	
In coordinates, 
	\begin{align*}
		(S \odot T)^{i_1 \dots i_k i_{k + 1} \dots i_{k + p}}_{j_1 \dots j_\ell j_{\ell + 1} \dots j_{\ell + q}} = \sum_{\sigma \in S_{k + p}} S^{i_{\sigma(1)} \dots i_{\sigma(k)}}_{j_{\sigma(1)} \dots j_{\sigma(\ell)}} T^{i_{{\sigma(k + 1)}} \dots i_{{\sigma(k + p)}}}_{j_{{\sigma(\ell + 1)}} \dots j_{{\sigma(\ell + q)}}}.
	\end{align*}	
Since symmetrisation is a projection, it maps a basis to a collection which spans $\Sigma^k (V^*)$. It is clear that the redundancies are precisely $\operatorname{Sym}(e^{i_1} \otimes \cdots \otimes e^{i_k})  = \operatorname{Sym} (\epsilon^{\sigma(i_1)} \otimes \dots \otimes \epsilon^{\sigma(i_k)})$. The collection of $k$-tuples of indices $(i_1, \dots, i_k)$ up to permutation by $S_k$ can be identified with the collection of $k$-tuples satisfying $i_1 \leq \dots \leq i_k$. Thus we have 


	
\begin{proposition}[Basis for $\Sigma^k (V^*)$]
	Let $\{ e_i \}_i \subseteq V$ and $\{ \epsilon^j\}_j \subseteq V^*$ form dual bases, then 
		\[ \left\{ \sum_{\sigma \in S_k} \epsilon^{\sigma(i_1)} \otimes \dots \otimes \epsilon^{\sigma(i_k)}  \right\}_{1 \leq i_1 \leq \dots \leq i_k \leq n} \subseteq \Sigma^k (V^*) \]
	forms a basis. More precisely, each $T \in \Sigma^k (V^*)$ admits the unique coordinate representation
		\[ T = \sum_{1 \leq i_1 \leq \dots \leq i_k \leq n} \frac{1}{\alpha (i_1, \dots, i_k)} T_{{i_1} \dots{i_k}} \sum_{\sigma \in S_k} \epsilon^{\sigma(i_1)} \otimes \dots \otimes \epsilon^{\sigma(i_k)} \]
	where $T_{{i_1} \dots{i_k}} = T(e_{i_1}, \dots, e_{i_k})$ and $\alpha(i_1,\dots,i_k) = m_1! \cdots m_n!$ and $m_j$ is the multiplicity of $j \in \{1, \dots, n\}$ in the tuple $(i_1, \dots, i_k)$. Moreover, $\dim \Sigma^k (V^*) = \binom{n + k - 1}{k}$. 
\end{proposition}

\begin{proof}
	Since $T$ is symmetric, the coordinates are invariant under the symmetric group $ T_{{i_1} \dots{i_k}}  = T_{{\sigma(i_1)} \dots{\sigma(i_k)}}$. Expanding $T$ into coordinates and the implicit sum into a double sum over indices $i_1 \leq \dots \leq i_k$ and $\sigma \in S_k$ \textit{a la} our previous remark, we obtain
	\begin{align*}
		T 
			&=  \sum_{1 \leq i_1 \leq \cdots \leq i_k \leq n} \frac 1{\alpha(i_1,\dots,i_k)} \sum_{\sigma \in S_k} T_{{\sigma(i_1)} \dots{\sigma(i_k)}}  e^{\sigma(i_1)} \otimes \cdots \otimes e^{\sigma(i_k)}\\
			&=  \sum_{1 \leq i_1 \leq \cdots \leq i_k \leq n} \frac 1{\alpha(i_1,\dots,i_k)}  T_{{i_1} \dots{i_k}} \sum_{\sigma \in S_k}  e^{\sigma(i_1)} \otimes \cdots \otimes e^{\sigma(i_k)}.
	\end{align*}
	Uniqueness follows from testing against various $k$-tuples of vectors. 
\end{proof}

\subsection{Trace}	

There is a natural mapping $\operatorname{tr}: V \times V^* \to \R$ known as the \emph{trace} given by pairing the vector and co-vector, 
	\[  (v, \omega) \mapsto \langle v, \omega \rangle, \] 
or, in coordinates, pairing a contravariant index with a covariant index, 
	\[ v^i \omega_j \mapsto v^i \omega_i. \]
Indeed, from the coordinate form, we see that the trace operator coincides with the usual trace operator for matrices when viewing them as $(1, 1)$-tensors. We can generalise the trace operator to a family $\operatorname{tr}^{(p, q)} : T^{(k, \ell)} (V) \to T^{(k - 1, \ell - 1)} (V)$ which takes the trace of the $p$-th contravariant index with the $q$-th covariant index. Tracing a tensor product is sometimes known as \textit{contraction}; more on this later. 


