
\subsection{Symmetrisation}

Let $S_k$ denote the \emph{symmetric group}, the set of permutations of the set $\{1, \dots, k\}$. There is a natural action on the space of $k$-tuples $V^k$; we say that a covariant $k$-tensor $T \in T^k (V^*)$ is \emph{symmetric} if $T$ is invariant with respect to this action, i.e.
	\[ T(v_1, \dots, v_k) = T(v_{\sigma(1)}, \dots, v_{\sigma(k)}) \]
for all permutations $\sigma \in S_k$. In particular, given a basis $\{v_i\}_i$, we see that $T$ is symmetric if and only if its coordinates are symmetric in its indices, 
	\[ T_{i_1 \dots i_k} = T_{\sigma(i_1) \dots \sigma(i_k)}. \]
Denote the space of symmetric tensors by $\Sigma^k (V^*)$, there is a projection $\operatorname{Sym} : T^k (V^*) \to \Sigma^k (V^*)$ known as the \emph{symmetrisation},
	\[ \operatorname{Sym} T (v_1, \dots, v_k) := \frac{1}{k !} \sum_{\sigma \in S_k} T(v_{\sigma(1)}, \dots, v_{\sigma(k)}). \]
It is easy to check that symmetrisation acts identically on symmetric tensors. 

\begin{example}
	Inner products are the prototypical example of symmetric tensors. 
\end{example}

\subsection{Alternation}

We say that the tensor $T \in T^k (V^*)$ is \emph{alternating} if it is invariant under action by the symmetric group up to the sign of the permutation 
	\[ T(v_1, \dots, v_k) = (-1)^\sigma T(v_{\sigma(1)}, \dots, v_{\sigma(k)}) \]
for all permutations $\sigma \in S_k$. In particular, given a basis $\{v_i\}_i$, we see that $T$ is alternating if and only if its coordinates are anti-symmetric in its indices, 
	\[ T_{i_1 \dots i_k} = (-1)^\sigma T_{\sigma(i_1) \dots \sigma(i_k)}. \]
Denote the space of alternating tensors by $\Lambda^k (V^*)$, there is a projection $\operatorname{Alt} : T^\ell (V^*) \to \Lambda^k (V^*)$ known as the \emph{alternation},
	\[ \operatorname{Alt} T (v_1, \dots, v_k) := \frac{1}{k !} \sum_{\sigma \in S_k} (-1)^\sigma T(v_{\sigma(1)}, \dots, v_{\sigma(k)}). \]
It is easy to check that alternation acts identically on alternating tensors. 

\begin{remark}
	The following are equivalent:
	\begin{enumerate}
		\item $T$ is alternating, 
		\item $T(v_1, \dots, v_k) =0$ whenever $(v_1, \dots, v_k)$ is linearly dependent, 
		\item $T$ vanishes whenever two of its arguments are equal, $T(v_1, \dots, w, \dots, w, \dots, v_k) = 0$. 
	\end{enumerate}
\end{remark}

\begin{example}
	The prototypical example of an alternating tensor is the determinant. 
\end{example}
