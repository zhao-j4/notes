Before getting into any multi-linear algebra, it is important to get a grasp of coordinates in linear algebra and how these coordinates respond to change of bases. Let $V$ be an $n$-dimensional real vector space, and denote $V^*$ its dual space. Given a basis $\{e_i\}_i \subseteq V$, there exists a dual basis $\{\epsilon^j\}_j \subseteq V^*$ satisfying 
	\[ \langle e_i, \epsilon^j \rangle = \delta^j_i. \]
Choose another basis $\{\widetilde e_i\}_i \subseteq V$ and denote its dual basis by $\{\widetilde \epsilon^j\}_j \subseteq V^*$. There exists a change of basis matrix $C^k_i \in \mathsf{GL} (\R^n)$ sending the original basis to the new basis,
	\[ \widetilde e_i = C^k_i e_k. \]	
On the other hand, the inverse change of basis matrix $(C^{-1})^j_k \in \mathsf{GL} (\R^n)$, i.e. $(C^{-1})^j_k C^k_i = \delta^j_i$, transforms the original dual basis to the new dual basis 	
	\[ \widetilde \epsilon^j = (C^{-1})^j_k \epsilon^k. \]
Throughout these notes, we will use these \emph{Einstein summation notation}, where repeated indices are summed over, e.g. $a^i b_i := \sum_i a^i b_i$.


\subsection{{Contravariance}}

We say an object is \emph{contravariant} if the coordinate representation \textit{contra-varies} with respect to change of basis, that is, transforms by the inverse matrix $(C^{-1})^j_i$. Such coordinates are indexed by \textit{upper indices}. The prototypical example of a contravariant object is a \emph{vector} $v \in V$. Every vector admits a unique coordinate representation $\{v^i\}_i \subseteq \R$ with respect to the basis $\{e_i\}_i$, i.e.
	\[ v = v^i e_i . \]
Let $\{ \widetilde v^j \}_j \subseteq \R$ be the unique coordinates with respect to the basis $\{\widetilde e_j\}_j$, then the change of coordinates from $\{v^i\}_i$ to $\{\widetilde v^j\}_j$ is given by the inverse change of basis matrix,
	\[ \widetilde v^j = {(C^{-1})}^j_i v^i. \]
Indeed, 	
	\[ v = \widetilde v^j \widetilde e_j  = \left( {(C^{-1})}^j_i v^i \right) \left( C^k_j e_k \right) = \delta^k_i v^i e_k = v^i e_i . \]
We can interpret a choice of basis $\{e_i\}_i$ as endowing $V$ with a ``measuring tool'', where the coordinates $\{v^i\}_i$ representing the resulting ``measurement''. A change of basis corresponds to changing the choice of ``measuring tool'', e.g. we can view a change of basis $\widetilde e_i = \tfrac{1}{100} e_i$ as changing from ``meters'' $e_i$ to ``centimeters'' $\widetilde e_i$, so the corresponding change of coordinates is 
	\[ \widetilde v^i \text{ meters } = 100 v^i \text{ centimeters}. \]




\subsection{Covariance}

We say that an object is \emph{covariant} if the coordinate representation \textit{co-varies} with respect to change of basis, that is, transforms by the matrix $C^k_i$.  Such coordinates are indexed by \textit{lower indices}. The prototypical example of a covariant object is a \emph{covector} $\omega \in V^*$. Every covector admits a unique coordinate representation $\{\omega_i \}_i \subseteq \R$ with respect to the basis $\{\epsilon^i\}_i$, i.e.
	\[ \omega = \omega_i \epsilon^i = \widetilde \omega_j \widetilde \epsilon^j. \]
Let $\{\widetilde \omega_j \}_j \subseteq \R$ be the unique coordinates with respect to the basis $\{\widetilde \epsilon^j\}_j$, then the change of coordinates from $\{\omega_i\}_i$ to $\{\widetilde \omega_j\}_j$ is given by the change of basis matrix,
	\[ \widetilde \omega_j = C^i_j \omega_i \]
Indeed, 
	\[ \omega = \widetilde \omega_j \widetilde \epsilon^j = \left( C^i_j \omega_i \right) \left( (C^{-1})_k^j \epsilon^k \right) =  \delta^i_k \omega_i \epsilon^k = \omega_i \epsilon^i. \]
Scalars are regarded as ``dimensionless'' quantities, so since a covector acting on a vector produces a scalar, they have inverse dimensions. For example, we can view a change of basis $\widetilde \epsilon^j = 100 \epsilon^j$ as changing from  ``meters$^{-1}$'' $\epsilon^j$ to ``centimeters$^{-1}$'' $\widetilde \epsilon^j$, so the corresponding change of coordinates is 
	\[ \widetilde \omega_j \text{ meters$^{-1}$} = \frac{1}{100} \omega_j  \text{ centimeters$^{-1}$}. \]

