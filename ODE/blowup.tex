In the previous sections, we have studied the \textit{local} well-posedness theory of the initial data problem. Inductively applying the local theory to the end-time of existence for any solution $u$, we can show that there exists a \emph{maximal time of existence} $T_{\text{max}}$, i.e. there does not exist a solution $v : [0, T^+] \to \R^n$ to (\ref{eq:idp}) such that $u \equiv v$ on $[0, T_{\text{max}})$. We refer to the solution defined on $[0, T_{\text{max}})$ as the \emph{maximal solution}. 

\begin{theorem}[Continuation criterion]
	Let $\Omega \subseteq \R^n$ be a domain, and suppose $F \in \dot C^{0, 1}_{\loc} (\Omega \to \R^n)$ is locally Lipschitz. If $u : [0, T] \to \Omega$ is a solution to the initial data problem (\ref{eq:idp}) which does not approach the boundary of $\Omega$ or blow-up, i.e.
		\begin{align*}
			\lim_{t \to T}\operatorname{dist} (u(t), \partial \Omega) 
				> 0, \qquad \text{and} \qquad
			\lim_{t \to T} |u(t)| 
				< \infty,
		\end{align*}
	then $u$ can be extended to a solution on $[0, T^+]$. 
\end{theorem}

\begin{proof}
	Since $u$ does not approach the boundary or blow-up, there exists a closed ball $\overline{B_\epsilon(u(T))} \subseteq \Omega$ on which we can continue the solution via the local theory. 
\end{proof}

\begin{corollary}[Existence of maximal solutions]
	Let $\Omega \subseteq \R^n$ be a domain, and suppose $F \in \dot C^{0, 1}_{\loc} (\Omega \to \R^n)$ is locally Lipschitz. Then there exists a maximal solution defined on a half-open interval $[0, T_{\text{max}})$. 
\end{corollary}

\begin{proof}
	The maximal time interval of existence must be half-open, since if $u$ is defined on $[0, T]$, then by local well-posedness it can be extended to $[0, T^+]$. Define the maximal time interval of existence $[0, T_{\text{max}})$ as the union of all intervals $[0, T]$ for which one has a solution to the initial data problem (\ref{eq:idp}). By the uniqueness theorem, we may glue these solutions together to obtain a maximal solution.
\end{proof}

\begin{corollary}[Blow-up criterion]
	Let $\Omega \subseteq \R^n$ be a domain, and suppose $F \in \dot C^{0, 1}_{\loc} (\Omega \to \R^n)$ is locally Lipschitz. If $u$ is a maximal solution to the initial data problem (\ref{eq:idp}) and $T_{\text{max}} < \infty$, then either
		\begin{align*}
			\lim_{t \to T_{\text{max}}}\operatorname{dist} (u(t), \partial \Omega) 
				= 0, \qquad \text{or} \qquad
			\lim_{t \to T_{\text{max}}} |u(t)| 
				= \infty.
		\end{align*}
\end{corollary}

\begin{proof}
	This is the contrapositive of the continuation criterion. 
\end{proof}

\begin{remark}
	The proposition does not apply to global solutions, e.g. the constant function $u \equiv u_0$ is a global solution to $u' = 0$ and $u(0) = u_0$, and it does not exhibit blow-up. 
\end{remark}

\begin{example}
	Consider the initial data problem
	\begin{align*}
		\partial_t u	
			&= u^2, \\
		u_{|t = 0}	
			&= u_0.
	\end{align*}
The map $u \mapsto u^2$ is locally Lipschitz, however the solution
	\[ u(t) = \frac{1}{1/u_0 - t} \]	
admits blow-up in finite time, namely $t = 1/u_0$. Following the proof of Picard-Lindelof, we see that the length of time in which a solution exists depends inversely with the local Lipschitz constant, and so the blow-up in this case coincides with the blow-up of the local Lipschitz constant of $u \mapsto u^2$. 
\end{example}

This example suggests that if we had uniform control over the Lipschitz constant, then we could obtain a global solution, which are obviously maximal. 

\begin{proposition}[Existence of a global solution]
	Let $\Omega \subseteq \R^n$ be a domain, and suppose $F \in C^{0, 1} (\Omega \to \R^n)$ is Lipschitz continuous, then there exists a unique global solution to the initial data problem  (\ref{eq:idp}).
\end{proposition}

\begin{proof}
	Denote
		\[ L := ||F||_{\dot C^{0, 1} (\Omega)} \]
	and let $0 < T < 1/L$. It follows that the integral operator $\Phi_{u_0}$ as defined in the proof of the Picard-Lindelof existence theorem is a contraction on the space $C^0([0, T] \to \Omega)$, 
		\[ ||\Phi_{u_0}u  - \Phi_{u_0} v||_{C^0[0, T]}\leq  \int_0^T |F(u(s)) - F(v(s))| \, ds \leq TL ||u - v||_{C^0[0, T]}. \]
	The contraction mapping principle furnishes a unique solution $u : [0, T] \to \Omega$ to the initial data problem. We iterate the Picard-Lindelof scheme at each endpoint of the previous construction, supposing we had a solution $u_k : [0, kT] \to \Omega$ to the initial data problem for some $k \in \N$, and solving 
		\begin{align*}
			u_{k + 1}' (t) &= F(u_{k + 1} (t)), \\
			u_{k + 1} (kT)	&= u_k (kT).
		\end{align*}
	By Picard-Lindelof iteration, we obtain a solution $u_{k + 1} : [0, (k + 1) T] \to \Omega$ satisfying the original problem. Proceeding inductively in $k$ furnishes a global solution $u: [0, \infty) \to \Omega$.
\end{proof}
