Let $1 \leq p, q \leq \infty$ and $s \in \R$, we define the \emph{homogeneous Besov space} $\dot B^{s, p}_q (\R^d)$ as the completion of the homogeneous Schwartz space $\dot \cS (\R^d)$ with respect to the norm 
	\[ ||u||_{\dot B^{s, p}_q} := \left|\left| || N^s u_N ||_{L^p_x} \right|\right|_{\ell^q_N (2^\Z)} =  \left( \sum_{N \in 2^\Z} || N^s u_N ||_{L^p}^q \right)^{1/q}, \]
with the usual modification at the endpoints $p, q = \infty$. For the same range of exponents, we analogously define the \emph{inhomogeneous Besov space} $B^{s, p}_q (\R^d)$ as the completion of Schwartz space $\cS (\R^d)$ with respect to the norm 
	\[ ||u||_{B^{s, p}_q} := ||u_{\leq 1}||_{L^p} + \left|\left| || N^s u_N ||_{L^p_x} \right|\right|_{\ell^q_N (2^\N)} =  ||u_{\leq 1}||_{L^p} + \left( \sum_{N \in 2^\N} || N^s u_N ||_{L^p}^q \right)^{1/q}.\]

\subsection{Properties}

We can read off some basic properties of Besov spaces from the definitions and properties of $L^p$-spaces and $\ell^q$-spaces. 

\begin{proposition}[Monotonicity in $s$ and $q$]
	Let $1 \leq p \leq \infty$ and $s \in \R$. 
\begin{enumerate}
	\item For $1 \leq q_1 \leq q_2 \leq \infty$, we have
			\begin{align*}
				||u||_{B^{s, p}_{q_2}} 
					&\lesssim_{q_1, q_2} ||u||_{B^{s, p}_{q_1}},\\
				||u||_{\dot B^{s, p}_{q_2}} 
					&\lesssim_{q_1, q_2} ||u||_{\dot B^{s, p}_{q_1}}.
			\end{align*}
		In particular, we have the continuous embeddings $B^{s, p}_{q_1} (\R^d) \hookrightarrow B^{s, p}_{q_2} (\R^d)$ and $\dot B^{s, p}_{q_1} (\R^d) \hookrightarrow \dot B^{s, p}_{q_2} (\R^d)$. 
	
	\item For $1 \leq q_1, q_2 \leq \infty$ and $\alpha > 0$, we have
			\begin{align*}
				||u||_{B^{s, p}_{q_2}} \lesssim_\alpha ||u||_{B^{s + \alpha, p}_{q_1}}.
			\end{align*}
		In particular, we have the continuous embedding $B^{s + \alpha, p}_{q_1} (\R^d) \hookrightarrow B^{s, p}_{q_2} (\R^d)$.
		
\end{enumerate}\label{prop:basicembedbesov}	
\end{proposition}

\begin{proof}
\leavevmode
\begin{enumerate}
	\item This follows immediately from the embedding $\ell^{q_1} \hookrightarrow \ell^{q_2}$ and the definition of the Besov norms. 
	
	\item By (a), it suffices to prove the result for $q_1 = \infty$ and $q_2 = 1$. We can write
		\begin{align*}
			 \sum_{N \in 2^\N}   ||N^s u_N||_{L^p} 
				&\leq  \Big| \Big| N^{s + \alpha} ||u_N||_{L^p} \Big|\Big|_{\ell^\infty_N (2^\N)}  \sum_{N \in 2^\N}  N^{-\alpha}  \lesssim \Big| \Big| N^{s + \alpha} ||u_N||_{L^p} \Big|\Big|_{\ell^\infty_N (2^\N)},
		\end{align*}	
		as desired. 
\end{enumerate}
\end{proof}

The index $s$ measures \textit{regularity}, as such we should expect the Besov norm to measure the norms of derivatives, for example the \emph{Riesz potential} $|\nabla|^s$ and the \emph{Bessel potential} $\langle \nabla \rangle^s$. When these differential operators are applied to a Littlewood-Paley piece $P_N u$, we see that in frequency space they are comparable to multiplication by $N^s$ and $\langle N\rangle^s$ respectively. This is made rigorous by the following lemma: 

\begin{lemma}[Sobolev-Bernstein inequalities] Let $1 \leq p \leq \infty$ and $s \in \R$. Then 
				\begin{align*}
					|| |\nabla|^s u_N||_{L^p} 
						&\sim N^s ||u_N||_{L^p}, \\
					|| \langle\nabla\rangle^s u_{N}||_{L^p} 
						&\sim \langle N\rangle^s ||u_{N}||_{L^p}.
				\end{align*}			
\end{lemma}

\begin{proposition}[Lifting property]
	Let $1 \leq p , q\leq \infty$ and $s, \sigma \in \R$. Then 
		\begin{align*}
			|||\nabla|^\sigma u||_{\dot B^{s - \sigma, p}_q}
				&\sim ||u||_{\dot B^{s, p}_q}\\
			 || \langle \nabla \rangle^\sigma u ||_{B^{s - \sigma, p}_q} 
			 	&\sim ||u||_{B^{s, p}_q} ,
		\end{align*}	 
	That is, $|\nabla|^\sigma : 	\dot B^{s - \sigma, p}_q (\R^d) \to \dot B^{s, p}_q (\R^d)$ and $\langle \nabla \rangle^\sigma : 	B^{s - \sigma, p}_q (\R^d) \to B^{s, p}_q (\R^d)$ isomorphically. 
\end{proposition}

\begin{proof}
	Immediate from Sobolev-Bernstein. 
\end{proof}

The expected duality relations hold for Besov spaces; the proofs are essentially identical to that for Lebesgue spaces and Sobolev spaces. Denote $1 \leq p', q' \leq \infty$ the dual exponents for $1 \leq p, q \leq \infty$, i.e.
	\[ \frac1p + \frac{1}{p'} = \frac1q + \frac{1}{q'} = 1. \]

\begin{theorem}[Duality]
	Let $1 \leq p, q < \infty$ and $s \in \R$, then 
		\begin{align*}
			\dot B^{s, p}_{q} (\R^d)^* 
				&\cong \dot B^{-s, p'}_{q'} (\R^d), \\
			B^{s, p}_{q} (\R^d)^* 
				&\cong  B^{-s, p'}_{q'} (\R^d).
		\end{align*}
\end{theorem}

\begin{proof}\cite[Theorem 2.11.2]{Triebel1983}
	The inclusion $\dot B^{-s, p'}_{q'} (\R^d) \hookrightarrow \dot B^{s, p}_{q} (\R^d)^* $ is essentially Holder's inequality while the reverse inclusion follows from the identification $\ell^q L^p (2^\Z \times \R^d)^* \cong \ell^{q'} L^{p'} (2^\Z \times \R^d)$ and Hahn-Banach. 
\end{proof}

\subsection{Embeddings}

As with the Sobolev embedding inequalities, the Besov embeddings boil down to a trade of regularity for integrability. Our main tool will be Bernstein's inequality, which states that frequency localisation allows us to move from $L^p$-integrability to $L^q$-integrability at the cost of $(\tfrac{d}{p} - \tfrac{d}{q})$-many derivatives. 

\begin{lemma}[Bernstein's inequalities]
\label{lem:bernstein}
	Let $1 \leq p \leq q \leq \infty$, then 
	\begin{align*}
					||u_N||_{L^q} 
						&\lesssim N^{\frac{d}{p} - \frac{d}{q}} ||u_N||_{L^p}, \\
					||u_{\leq N}||_{L^q} 
						&\lesssim N^{\frac{d}{p} - \frac{d}{q}} ||u_{\leq N}||_{L^p}. 
	\end{align*}
\end{lemma}

\begin{proof}
	Let $1 \leq r \leq \infty$ satisfy $\tfrac1p + \tfrac1r = \tfrac1q + 1$, then by Young's convolution inequality and a change of variables $Nx = y$, we have the inequality
	\begin{align*}
		|| P_N u ||_{L^q} = || u * \widecheck{\psi_N} ||_{L^p} \leq ||u||_{L^p} ||N^d \widecheck \psi(Nx)||_{L^r_x} = N^{d - \frac{d}{r}} ||u||_{L^p} ||\widecheck \psi ||_{L^r_y} \sim N^{\frac{d}{p} - \frac{d}{q}} ||u||_{L^p}.
	\end{align*}
	To obtain $u_N$ instead of $u$ on the right, observe that the same proof holds replacing $P_N$ with the fattened projection $\widetilde{P_N}$. Since $\widetilde{P_N} P_N = P_N$, replacing $u$ with $P_N u$ completes the proof. Arguing similarly furnishes the inequality replacing $u_N$ with $u_{\leq N}$. 
\end{proof}



\begin{theorem}[Homogeneous Besov embedding]
	Let $1 \leq p_1 \leq p_2 \leq \infty$ and $1 \leq q_1 \leq q_2 \leq \infty$ and $s_2 \leq s_1$, then 
		\[ || u ||_{\dot B^{s_2, p_2}_{q_2}} \lesssim || u ||_{\dot B^{s_1, p_1}_{q_1}} \]
	whenever $\frac{s_1}{d} - \frac{1}{p_1} = \frac{s_2}{d} - \frac{1}{p_2}$. In particular, we have the continuous embedding $\dot B^{s_1, p_1}_{q_1} (\R^d) \hookrightarrow \dot B^{s_2, p_2}_{q_2} (\R^d)$.
\end{theorem}

\begin{proof}
	By Proposition \ref{prop:basicembedbesov} (a), it suffices to prove the result for $q_1 = q_2 = q$. Recall Bernstein's inequality, 
		\[ || u_N||_{L^{p_2}} \lesssim N^{\frac{d}{p_1} - \frac{d}{p_2}} ||u_N||_{L^{p_1}}. \]
	Applying the identity $\frac{s_1}{d} - \frac{1}{p_1} = \frac{s_2}{d} - \frac{1}{p_2}$, rearranging, and summing in $\ell^q (2^\Z)$ furnishes the result. 	
\end{proof}

\begin{theorem}[Inhomogeneous embedding]
	Let $1 \leq p_1 \leq p_2 \leq \infty$ and $1 \leq q_1 \leq q_2 \leq \infty$ and $s_2 \leq s_1$, then 
		\[ || u ||_{B^{s_2, p_2}_{q_2}} \lesssim || u ||_{B^{s_1, p_1}_{q_1}} \]
	whenever $\frac{s_1}{d} - \frac{1}{p_1} \geq \frac{s_2}{d} - \frac{1}{p_2}$. In particular, we have the continuous embedding $B^{s_1, p_1}_{q_1} (\R^d) \hookrightarrow B^{s_2, p_2}_{q_2} (\R^d)$.
\end{theorem}

\begin{proof}
	By Proposition \ref{prop:basicembedbesov} (a), it suffices to prove the result for $q_1 = q_2 = q$. Control over the low frequency term $u_{\leq 1}$ follows from Bernstein's inequality. For high frequencies $N \in 2^\N$, we can apply the inequality $\frac{s_1}{d} - \frac{1}{p_1} \geq \frac{s_2}{d} - \frac{1}{p_2}$ with Bernstein's inequality to write 
		\[ || u_N||_{L^{p_2}} \lesssim N^{\frac{d}{p_1} - \frac{d}{p_2}} ||u_N||_{L^{p_1}} \leq N^{s_1 - s_2} ||u_N||_{L^{p_1}}. \]
	Rearranging and summing in $\ell^q (2^\N)$ furnishes the result. 	
\end{proof}

\subsection{Traces}

In the interest of solving boundary value problems, we study the boundedness properties of the \emph{trace} operator $T$ which sends a function $u : \R^d \to \C$ to its restriction on a hyperplane $u_{|x_d = 0} : \R^{d - 1} \to \C$, 
	\[ T u (x') := u(x', 0) = \langle u, \delta_{x_d = 0} \rangle \]
where we use the notation $x = (x_1, \dots, x_d)$ and $x' = (x_1, \dots, x_{d - 1})$. We similarly denote $\cF'$ the Fourier transform and $P_N'$ the Littlewood-Paley projections on the hyperplane $\R^{d - 1}$. As always, the main ingredient for studying the trace operator on Besov spaces is the uncertainty principle. This is encapsulated by the following two lemmas: 

\begin{lemma}
	Given a tempered distribution $u \in \cS' (\R^d)$, the trace of the Littlewood-Paley projection $P_{\leq N} u$ has Fourier support in $|\xi'| \lesssim N$. In particular, 
		\[ P'_M T P_{\leq N} u = 0 \]
	for $M \gtrsim N$. 
\end{lemma}

\begin{proof}
	From the uncertainty principle, restricting to a point in physical space is equivalent to integrating in frequency space, that is, $\langle \delta_0, \phi \rangle = \langle 1, \widehat \phi \rangle$. Moreover, the composition of the Fourier transform on $\R^{d - 1}_{x'}$ with the Fourier transform on $\R_{x_d}$ gives the Fourier transform on $\R^d_x$. Hence, in frequency space
		\[ \cF' T P_{\leq N} u = \cF' P_{\leq N} u (\xi', 0) = \int_\R \cF P_{\leq N} u \, d\xi_d. \]
	As $\cF P_{\leq N} u$ is supported in the ball $|\xi| \lesssim N$, we see that $T P_N u$ has Fourier support in the ball $|\xi'| \lesssim N$. 	
\end{proof}

\begin{lemma}
	Given a tempered distribution $u \in \cS' (\R^d)$, the trace of the Littlewood-Paley projection $P_{\leq N} u$ satisfies
		\[ || TP_{\leq N} u ||_{L^p (\R^{d - 1}_{x'})} \lesssim N^{\frac1p} ||P_{\leq N} u||_{L^p (\R^d_x)} \]
\end{lemma}

\begin{proof}
	Since $\cF P_{\leq N} u$ is supported in $|\xi| \lesssim N$, we see that $\cF_{x_d} P_{\leq N} u$ is supported in $|\xi_d| \lesssim N$ for each fixed $x' \in \R^{d - 1}$. Thus by Minkowski's integral inequality and Bernstein's inequality in one-dimension, 
		\begin{align*}
			||T P_{\leq N} u||_{L^p (\R^{d - 1}_{x'})}
				&\leq \left|\left| || P_{\leq N} u||_{L^\infty (\R_{x_d})} \right|\right|_{L^p (\R^{d - 1}_{x'})} \\
				&\lesssim  \left|\left| N^{\frac1p} || P_{\leq N} u||_{L^p (\R_{x_d})} \right|\right|_{L^p (\R^{d - 1}_{x'})} = N^{\frac1p} ||P_{\leq N} u||_{L^p (\R^d_x)}.
		\end{align*}
	This completes the proof. 	
\end{proof}
	
\begin{theorem}[Besov trace theorem]
	Let $1 < p < \infty$ and $1 \leq q \leq \infty$ and $s > \tfrac1p$, then the trace map satisfies
		\[ ||T u||_{\dot B^{s - \frac1p, p}_q (\R^{d - 1})} \lesssim ||u||_{\dot B^{s, p}_q (\R^d)}. \]
	In particular, it extends to a bounded linear operator $T : \dot B^{s, p}_q (\R^d) \to \dot B^{s - \frac1p, p}_q (\R^{d - 1})$. The analogous results also hold for the inhomogeneous Besov spaces. 
\end{theorem}

\begin{proof}
	We perform a Littlewood-Paley decomposition $u = \sum_N P_N u$. By linearity, we can write $P'_M T u = \sum_N P'_M T P_N u$. Applying the triangle inequality, the previous two lemmas, and boundedness of the projections, we estimate
		\begin{align*}
			||P'_M Tu||_{L^p (\R^{d-1}_{x'})}
				&\lesssim \sum_{N : M \lesssim N} ||P'_M T P_N u||_{L^p (\R^{d - 1}_{x'})}\\
				&\lesssim \sum_{N : M \lesssim N} N^{1/p} ||P_N u||_{L^p (\R^d_x)}.
		\end{align*}		
	Multiplying both sides by $M^{s- 1/p}$ and taking $\ell^q$-norms, this reduces the problem to showing boundedness of the linear operator $A : \ell^q (2^\Z) \to \ell^q (2^\Z)$ defined as
		\[ A (\{ c_N\}_{N \in 2^\Z})_M := \sum_{N : M \lesssim N} N^{\frac1p - s} M^{s - \frac1p} c_N. \]
	Taking $c_N := N^s ||P_N u||_{L^p}$ completes the proof. By Schur's test, we simply need to show that the corresponding kernel of the operator $K(N, M) := \mathbb 1_{N \lesssim M} N^{1/p - s} M^{s - 1/p}$ is uniformly summable in $N$ and $M$. Indeed
	\begin{align*}
		\sum_{N \in 2^\Z}K (N, M) \sim  \sum_{N : N \lesssim M} N^{\frac1p - s} M^{s - \frac1p} \sim 1, \\
		\sum_{M \in 2^\Z}K (N, M) \sim  \sum_{M : N \lesssim M} N^{\frac1p - s} M^{s - \frac1p} \sim 1.
	\end{align*}
	The proof in the inhomogeneous spaces is the same modulo trivial modifications for low frequencies. 
\end{proof}

\begin{remark}
	This proof does not extend to the endpoints $p = 1, \infty$ due to the failure of the Littlewood-Paley decomposition $u = \sum_N P_N u$ to converge in $L^p (\R^d)$. Nonetheless, the result continues to hold at the endpoints, c.f. the maximal function argument of Triebel \cite[Theorem 2.7.2]{Triebel1983}. 
\end{remark}

\begin{theorem}[Besov trace extension theorem]
\label{thm:besovext}
	Let $1 \leq p, q \leq \infty$ and $s > \frac1p$, then there exists a trace extension operator, i.e. $TE = \operatorname{Id}$, satisfying 
		\[ ||Ev||_{\dot B^{s, p}_q (\R^{d})} \lesssim ||v||_{\dot B^{s - \frac1p, p}_q (\R^{d - 1})}. \]
	In particular, it extends to a bounded linear operator $E : \dot B^{s - \frac1p, p}_q (\R^{d - 1}) \to \dot B^{s, p}_q (\R^d)$ and the trace operator is surjective. The analogous results hold for the inhomogeneous Besov spaces. 
\end{theorem}

\begin{proof}
	Let $\psi \in C^\infty_c ((1, 2))$ such that $\int \psi = 1$, and set $\psi_M (\xi_d) := \psi (M \xi_d)$. Notice then that the inverse Fourier transforms satisfy
		\[ \widecheck{\psi_M} (0) = \frac1M, \qquad || \widecheck{\psi_M} ||_{L^p (\R)} \sim M^{\frac1p - 1}. \]
	Define the extension operator by 	
		\[ Ev (x', x_d) := \sum_{M \in 2^\Z} M P'_M v(x') \widecheck{\psi_M} (x_d) . \]
	It is clear from construction that $TE = \operatorname{Id}$, so it remains to show boundedness of the extension operator. Choosing projections $P_N$ appropriately, we have $P_N (P'_M v \widecheck{\psi_M}) = 0$ whenever $N \neq M$. Combining this observation with Fubini's theorem, we obtain 
		\begin{align*}
			N^{s - \frac1p} ||P_N Ev||_{L^p (\R^{d})} 
			 	&\lesssim N^{s - \frac1p + 1} ||  P'_N v ||_{L^p (\R^{d - 1})} ||\widecheck{\psi_N}||_{L^p (\R)} \lesssim  N^{s} ||P'_N v ||_{L^p (\R^{d - 1})}.
		\end{align*}	 
	Summing in $\ell^q_N$ completes the proof. Minor modifications furnishes the inhomogeneous case. 	
\end{proof}

\subsection{Example: Holder spaces}

The most familiar examples of Besov spaces are the \emph{Holder spaces} $C^{k, \alpha} (\R^d)$ for $k \in \N_0$ and $0 < \alpha < 1$. As the proofs carry through the same, we will in fact introduce a slightly more general class of Holder spaces; denote the translates $u^h (x) := u(x - h)$, then the \emph{homogeneous Holder space} $\dot \Lambda^{k + \alpha, p} (\R^d)$ is the completion of the homogeneous Schwartz space $\dot \cS (\R^d)$ with respect to the norm
	\[
		||u||_{\dot \Lambda^{k + \alpha, p}} := \sup_{h \neq 0} \frac{||\nabla^k u^h - \nabla^k u||_{L^p}}{|h|^\alpha}.
	\]
We analogously define the \emph{inhomogeneous Holder space} $\Lambda^{k + \alpha, p} (\R^d)$ as the completion of Schwartz space $\cS (\R^d)$ with respect to the norm 
	\[ ||u||_{\Lambda^{k + \alpha, p}} := ||u||_{L^p} +  \sup_{0 < |h| < 1} \frac{||\nabla^k u^h - \nabla^k u||_{L^p}}{|h|^\alpha} .\]
When $p = \infty$, this coincides with the familiar Holder spaces $C^{k, \alpha} (\R^d)$ of functions with $\alpha$-Holder continuous derivatives up to order $k$. 
	
\begin{proposition}
	Let $1 \leq p \leq \infty$ and $k \in \N_0$ and $0 < \alpha < 1$, then 
		\begin{align*}
			||u||_{\dot \Lambda^{k + \alpha, p}} 
				&\sim ||u||_{\dot B^{k + \alpha, p}_\infty}, \\
			||u||_{\Lambda^{k + \alpha, p}} 
				&\sim ||u||_{B^{k + \alpha, p}_\infty}.	
		\end{align*}	
	In particular, $\dot \Lambda^{k + \alpha, p} (\R^d) = \dot B^{k + \alpha, p}_\infty (\R^d)$ and $\Lambda^{k + \alpha, p} (\R^d) = B^{k + \alpha, p}_\infty (\R^d)$. 
\end{proposition}

\begin{proof}
	As usual, we prove the homogeneous case, the inhomogeneous case is a trivial exercise. For simplicity, let us only consider the case $k = 0$. Using $\int \widecheck{\psi_N} = \psi_N (0) = 0$ and a change of variables $Ny = z$, we can write
		\begin{align*}
			u_N (x) = (u * \widecheck{\psi_N})(x) 
				&= N^d \int_{\R^d} u(x - y) \widecheck\psi (Ny) dy \\
				&= \int_{\R^d} u(x - z/N) \widecheck \psi (z) dz = \int_{\R^d} \Big( u(x - z/N) - u(x)\Big) \widecheck \psi (z) dz.
		\end{align*}	
	Taking the $L^p$-norm of the above, applying Minkowski's inequality and the Holder condition gives
		\begin{align*}
			||u_N||_{L^p} \leq  \int_{\R^d} ||u^{z/N} - u||_{L^q} |\widecheck \psi (z)| dz \lesssim ||u||_{\dot \Lambda^{k + \alpha}} N^{-\alpha} . 
		\end{align*}
	This proves the Holder norm controls the Besov norm. To prove the converse inequality, we decompose the difference $u^h - u$ into high and low frequencies; let $M \in 2^\Z$ satisfy $ |h|^{-1} \sim M$, it follows from the triangle inequality that
		\[ || u^h - u||_{L^p} \leq ||u^h_{\leq M} - u_{\leq M} ||_{L^p} + ||u^h_{\geq M} + u_{\geq M} ||_{L^p}.  \]
	For the high frequency terms, we crudely estimate by the triangle inequality, noting $||u^h_{K}||_{L^p} = ||u_{K} ||_{L^p}$,
		\[ || u^h_{\geq M} - u_{\geq M} ||_{L^p} \lesssim \sum_{N\geq M} ||u_N||_{L^p} \lesssim \Big|\Big| ||N^{\alpha} u_N||_{L^p} \Big|\Big|_{\ell^\infty_N} \sum_{N \geq M} N^{-\alpha} \sim ||u||_{\dot B^{\alpha, p}_\infty} |h|^{\alpha} .\]	
	This furnishes the desired bound for the high frequency terms. For the low frequency terms, we can write using the fundamental theorem of calculus
		\[  u^h_{\leq M} (x) -  u_{\leq M} (x)  = h \cdot \int_0^1 \nabla u_{\leq M} (x - \theta h) d \theta. \]	
	Taking the $L^p$-norm of the above and applying the triangle, Minkowski, Sobolev-Bernstein inequalities, we obtain
		\[ ||u^h_{\leq M} - u_{\leq M} ||_{L^q} \leq |h| \sum_{N \leq M} ||\nabla u_{N}||_{L^q_x} \lesssim ||u||_{\dot B^{\alpha, p}_\infty} |h| \sum_{N \leq M} N^{1-\alpha} \sim   ||u||_{\dot B^{\alpha, p}_\infty} |h|^\alpha . \]	
	This furnishes the desired bound for the low frequency terms, completing the proof. 		
\end{proof}


