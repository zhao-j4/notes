When studying function spaces, such as Lorentz spaces or Sobolev spaces, it is useful to decompose a generic function into simpler pieces, and attempt to prove the desired results for each of those pieces. For example, functions in Lorentz spaces can be decomposed in \textit{physical space} into the sum of \textit{quasi-step functions}. Our approach in these notes will be to decompose into \textit{frequency-localised} pieces and  study the various ways these pieces sum. 

To that end, we construct a dyadic partition of unity as follows; let $\phi  \in C^\infty_c (\R^d)$ satisfy $0 \leq \phi \leq 1$ and 
\begin{align*}
	\phi(x) 
		:= 
		\begin{cases}
			1 , 				&|x| \leq 1.4, \\
			0, 				&|x| > 1.42. 
		\end{cases}
\end{align*}
Denote the dyadics by $2^\Z := \{ 2^n : n \in \Z \}$. For $N \in 2^\Z$, define $\psi, \psi_N, \phi_{\leq N} \in C^\infty_c (\R^d)$ to be 
	\[ \psi(x) := \phi(x) - \phi(2x), \qquad \psi_N (x) := \psi(x/N), \qquad \phi_{\leq N} (x) := \phi(x/N).  \]
Observe that $\sum_N \psi_N \equiv 1$ since pointwise it forms a telescoping sum. Given a tempered distribution $f \in \cS' (\R^d)$, we define its \emph{Littlewood-Paley projections} to frequencies $|\xi| \sim N$ and $|\xi| \lesssim N$ respectively by
	\begin{align*}
		\widehat{f_N} &= \widehat{P_N f}  = \psi_N \widehat f , \qquad
		\widehat{f_{\leq N}} = \widehat{P_{\leq N} f} = \phi_{\leq N} \widehat f.
	\end{align*}	
Define the Littlewood-Paley projections to frequencies $|\xi| \gtrsim N$ and $N \lesssim |\xi| \lesssim M$ respectively by 
	\[ f_{\geq N} = P_{\geq N} f = (1 - P_{\leq N}) f, \qquad f_{N \leq - \leq M} = P_{N \leq - \leq M} f = \sum_{N \leq K \leq M} P_K f. \]
The name ``projection'' is a bit of a misnomer; the multipliers $P_N$ fail to be true projections in the sense that by choosing smooth cutoffs in frequency space rather than sharp cutoffs, we have $P_N P_N \neq P_N$. Nevertheless, a slightly modified statement holds; define the \emph{fattened Littlewood-Paley projections} to frequencies $|\xi| \sim N$ and their corresponding multipliers by
	\[ \widetilde{P_N} := P_{\frac{N}{2}} + P_{N} + P_{2N},\qquad  \widetilde{\psi_N} := \psi_{\frac{N}{2}} + \psi_N + \psi_{2N}. \]
Since $\widetilde{\psi_N} \equiv 1$ on the support of $\psi_N$, it follows that $\widetilde{P_N} P_N = P_N$. Similarly, we can define the fattened projections to frequencies $|\xi| \lesssim N$ by 
	\[ \widetilde{P_{\leq N}} = P_{\leq 2N}, \qquad \widetilde{\phi_{\leq N}} := \phi_{\leq 2N}.  \]	

\begin{remark}
	By the Paley-Wiener theorem, the projections are analytic functions in physical space. Thus we can study the Littlewood-Paley projections pointwise without any philosophical consternation.
\end{remark}


