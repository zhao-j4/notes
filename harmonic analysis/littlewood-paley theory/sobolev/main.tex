\documentclass[reqno]{amsart}

\usepackage{external/takodachi}
\renewcommand{\emph}{\textsc}


% Title: change problem set number as needed
\title
{
	\emph{Sobolev spaces}
} 

\author{Jason Zhao}
\date{\today}

\begin{document}
\maketitle

\begin{abstract}
	These notes exposit the theory of Sobolev spaces $W^{s, p} (\R^d)$ from an harmonic analysis perspective via the Riesz and Bessel potentials. This definition is robust in that it allows us to define fractional regularity spaces and prove the classical inequalities using Littlewood-Paley theory, which clarifies the underlying role of the uncertainty principle. We draw from \cite[Chapter III]{Stein16} and \cite[Appendix A]{Tao2006}.
\end{abstract}

\tableofcontents

\section{Preliminaries}
The starting point for the study of fractional differential and integral operators is the observation that the Fourier transform maps differentiation to multiplication by a monomial, namely 
	\[\widehat{\nabla u} (\xi) = 2\pi i \xi \widehat u (\xi).\]
We see from the formula above that differentiation \textit{accentuates} high frequencies and \textit{diminishes} low frequencies. Conversely, we expect integration to diminish high frequencies and accentuate low frequencies. The natural way to generalise the formula above is to replace the monomial $2\pi i \xi$ with fractional powers $|2\pi i \xi|^s$, where positive powers $s > 0$ correspond then to differential operators, and negative powers $s < 0$ to integrals. 


\subsection{Riesz potential}

For $s \in \R$ and $u \in \cS (\R^d)$, define the \emph{Riesz potential} to be the Fourier multiplier
	\[ \widehat{(|\nabla|^s u)} (\xi) := |2\pi i \xi|^s \widehat u(\xi).  \]	
The multiplier $|2\pi i \xi|$ is smooth, non-vanishing, and grows linearly at infinity on $\R^d \setminus 0$, however it is non-differentiable and vanishes at the origin. Thus it does not quite map the Schwartz space onto itself, so to remedy this, define the \emph{homogeneous Schwartz space} $\dot \cS (\R^d)$ as the space of Schwartz functions whose Fourier transforms vanish to infinite order at the origin,  
	\[ \dot \cS (\R^d) := \{ u \in \cS (\R^d) : \nabla^k \widehat u (0) = 0 \text{ for all $k$} \}. \]
The Riesz potentials map the homogeneous Schwartz space onto itself isomorphically, $|\nabla|^s : \dot \cS (\R^d) \to \dot \cS (\R^d)$, with the obvious inverse $|\nabla|^s |\nabla|^{-s} = \operatorname{Id}$. Furthermore, this subspace is still generic enough for us to apply density arguments, 

\begin{lemma}[Genericity of $\dot \cS (\R^d)$]
	Let $f \in \cS(\R^d)$ and $1 \leq p < \infty$, then there exists $\{ g_\epsilon\}_{\epsilon > 0} \subseteq \dot \cS (\R^d)$ such that $||f - g_\epsilon||_{L^p} \to 0$. In particular, $\dot \cS (\R^d)$ is dense in $L^p (\R^d)$. 
\end{lemma}

\begin{proof}
	Let $\phi \in C^\infty_c (\R^d)$ be a bump function satisfying $\phi \equiv 1$ for $|\xi| \leq 1$. Define 
		\[ \widehat{g_\epsilon} (\xi) := \widehat g (\xi) (1 - \phi(\xi/\epsilon)). \]
	By construction the Fourier transform of $g_\epsilon$ vanishes in a neighborhood of the origin, so in particular $g_\epsilon \in \dot \cS  (\R^d)$. Writing $g - g_\epsilon = g * \epsilon^d \phi(\epsilon - )$, it follows from Young's inequality and a change of variables $\epsilon x = y$ that 
		\[ ||g - g_\epsilon||_{L^p} \leq ||g||_{L^1} ||\epsilon^d \phi(\epsilon x)||_{L^p_x} \lesssim \epsilon^{d - \frac{d}{p}} ||\phi(y)||_{L^p_y} \overset{\epsilon \to 0}{\longrightarrow} 0,  \]
	as desired. 	
\end{proof}

For $-d < s < 0$, the multiplier $|2\pi i \xi|^s$ decays at infinity and is locally integrable at the origin, so $|\nabla|^s$ can be well-defined as an integral operator on $\cS (\R^d)$. To compute its kernel, we remark that $|2\pi i \xi|^s$ is homogeneous of order $s$, so in view of the Fourier transform, the corresponding kernel has homogeneity of order $-d - s$. 
	
\begin{proposition}[Kernel of Riesz potential]
	Let $0 < \alpha < d$, then 
		\[ \widecheck{|2\pi i \xi|^{-\alpha}} = 2^{-\alpha} \pi^{- d/2} \frac{\Gamma (\frac{d - \alpha}{2})}{\Gamma(\frac{\alpha}{2})} |x|^{\alpha - d}. \]
\end{proposition}

\begin{proof}
	The key observation is that $|\xi|$ can be expressed as a weighted integral in $t$ of Gaussians $\{  e^{-\pi t |\xi|^2}\}_{t > 0}$. Indeed, making the change of variables $u = \pi t|\xi|^2$, we compute
		\begin{align*}
			\int_0^\infty e^{-\pi t |\xi|^2} t^{\frac{\alpha}{2}} \frac{dt}{t}
				&= \int_0^\infty e^{-u} \left( \frac{u}{\pi |\xi|^2} \right)^{\frac{\alpha}{2}} \frac{du}{u}\\
				&= \frac{|\xi|^{-\alpha}}{\pi^{\frac{\alpha}{2}}} \int_0^\infty e^{-u} u^{\frac{\alpha}{2}} \frac{du}{u} = \Gamma \left(\frac{\alpha}{2}\right) \frac{|\xi|^{-\alpha}}{\pi^{\frac{d - \alpha}{2}}}.
		\end{align*}	
	Taking the inverse Fourier transform of the left-hand side formally, noting that the computation can be made rigorous in the sense of tempered distributions using Fubini's theorem, 
		\begin{align*}
			\widecheck{\int_0^\infty e^{-\pi t |\xi|^2} t^{\frac{\alpha}{2}} \frac{dt}{t}}
				&= \int_0^\infty e^{-\pi |x|^2/t} t^{\frac{\alpha - d}{2}} \frac{dt}{t} = \Gamma \left(\frac{d - \alpha}{2}\right) \frac{|x|^{\alpha - d}}{\pi^{\frac{\alpha}{2}}},
		\end{align*}
	where the first equality is the Fourier transform of Gaussians $\widecheck{e^{-\pi t |\xi|^2}} = e^{-\pi |x|^2/t} t^{-d/2}$ and the second equality is an application of our initial computation, replacing $\alpha$ with $d - \alpha$. Rearranging completes the proof. 
\end{proof}

\begin{remark}
	This computation furnishes the fundamental solution of the Laplacian for dimension $d \geq 3$. Indeed, Laplace's equation takes the form, in physical space and frequency space respectively, 
	\begin{align*}
		\Delta u
			&= f, \\
		4\pi^2 |\xi|^2 \widehat u
			&= \widehat f.	
	\end{align*}
	Thus, for example, in dimension $d = 3$, the solution is given by 
		\[ \widehat u (\xi) = \frac{1}{4\pi |\xi|^2} \widehat f(\xi) = \widehat{ \frac{1}{4\pi |x|} * f} (\xi). \]
\end{remark}

\begin{remark}
	The integral operator $\mathcal I_\alpha u := \widecheck{|2\pi i \xi|^{-\alpha}} * u$ is traditionally known as the \textit{Riesz fractional integration} operator. Inversely, we can view the positive order operator $|\nabla|^s$ for $s > 0$ as \textit{fractional differentiation} operator. 
\end{remark}



\subsection{Bessel potential}

For $s \in \R$ and $u \in \cS (\R^d)$, define the \emph{Bessel potential} to be the Fourier multiplier
	\[ \widehat{(\langle \nabla \rangle^s u)} (\xi) := \langle 2\pi i \xi \rangle^s \widehat u (\xi).  \]
The multiplier $\langle 2\pi i \xi \rangle = (1 + 4\pi^2 |\xi|^2)^{1/2}$ is smooth, non-vanishing, and grows linearly at infinity, so the Bessel potential maps Schwartz space onto itself isomorphically, $\langle \nabla \rangle^s : \cS (\R^d) \to \cS (\R^d)$ with the obvious inverse $\langle \nabla \rangle^s \langle \nabla \rangle^{-s} = \operatorname{Id}$. By construction, the Bessel potential treats high frequencies the same as the Riesz potential, however it has the advantage that low frequencies are better treated. 

\begin{proposition}[Kernel of Bessel potential]
	Let $\alpha > 0$, then 
		\[\widecheck{\langle 2\pi i \xi \rangle^{-\alpha}}  = \frac{1}{(4\pi)^{\frac\alpha2} \Gamma(\frac\alpha2)} \int_0^\infty  e^{- \frac{\pi |x|^2}{t}} e^{-\frac{t}{4\pi}}  t^{\frac{\alpha - d}{2}} \, \frac{dt}{t}. \]
\end{proposition}

\begin{proof}
	We follow our computation of the Riesz potential kernel, where recall we represented $|\xi|$ as an integral of Gaussians. Replacing $|\xi|$ with $(1 + 4\pi^2 |\xi|^2)^{1/2}$ in that computation gives
		\begin{align*}
			 \int_0^\infty e^{-\frac{t}{4\pi} (1 + 4\pi^2 |\xi|^2)} t^{\frac\alpha2} \frac{dt}{t} = \Gamma\left(\frac\alpha2\right)\frac{(1 + 4\pi^2 |\xi|^2)^{-\frac\alpha2} }{(4\pi)^{\frac\alpha2}} . 
		\end{align*}
	Taking the inverse Fourier transform of the left-hand side and rearranging gives the result. 
\end{proof}

\begin{proposition}[Asymptotics of Bessel potential kernel]
	Let $\alpha > 0$, then the kernel of the Bessel potential satisfies the following properties: 
	\begin{enumerate}
		\item It is non-negative, integrable, and unit mass,  
			\[ ||\widecheck{\langle 2\pi i \xi \rangle^{-\alpha}}||_{L^1} = \int_{\R^d} \widecheck{\langle 2\pi i \xi \rangle^{-\alpha}} \, dx = 1.  \]
			
		\item The growth at infinity is at most exponential, 
			\[ \widecheck{\langle 2\pi i \xi \rangle^{-\alpha}} (x) \lesssim e^{-|x|/2}, \qquad \text{uniformly in $|x| \geq 1$.} \]
		
		\item For $0 < \alpha < d$, the growth at zero is that of the Riesz potential kernel to leading order, 
			\[
				 \widecheck{\langle 2\pi i \xi \rangle^{-\alpha}} (x)= 2^{-\alpha} \pi^{- d/2} \frac{\Gamma (\frac{d - \alpha}{2})}{\Gamma(\frac{\alpha}{2})} |x|^{\alpha - d} + o(|x|^{\alpha - d}), \qquad \text{as $|x| \to 0$.}
			\] 
	\end{enumerate}
\end{proposition}

\begin{proof}
\leavevmode
\begin{enumerate}
	\item Non-negativity is obvious, $||\widecheck{\langle 2\pi i \xi \rangle^{-\alpha}}||_{L^1} = 1$ follows from Fubini's theorem, or remarking that its Fourier transform is precisely $1$ at the origin. 
	
	\item Observe that $\tfrac{\pi |x|^2}{t} + \tfrac{t}{4\pi} \geq \tfrac{t}{4\pi} + \tfrac\pi t$ and $\tfrac{\pi |x|^2}{t} + \tfrac{t}{4\pi} \geq |x|$ whenever $|x| \geq 1$. Averaging between the two inequalities furnishes 
		\[
			\widecheck{\langle 2\pi i \xi \rangle^{-\alpha}} (x)  \leq \left( \frac{1}{(4\pi)^{\frac\alpha2} \Gamma(\frac\alpha2)} \int_0^\infty  e^{-\frac{t}{8\pi} - \frac{\pi}{2t}}   t^{\frac{\alpha - d}{2}} \, \frac{dt}{t}\right) e^{-|x|/2}.
		\]
	\item Substituting the asymptotic expansion $e^{-t/4\pi} = 1 + o(e^{-t/4\pi})$ into the kernel and recalling our computation of the Riesz potential kernel gives the result. 
\end{enumerate}
\end{proof}



\section{Sobolev spaces}
Let $1 < p < \infty$ and $s \in \R$, we define the \emph{inhomogeneous Sobolev space}, also known as the Bessel potential space, $W^{s, p} (\R^d)$, as the closure of Schwartz space $\cS (\R^d)$ with respect to the norm 
	\[ ||u||_{W^{k, p}} := ||\langle \nabla \rangle^s u||_{L^p}. \]
Analogously, we define the \emph{homogeneous Sobolev space}, also known as the Riesz potential space, $\dot W^{s, p} (\R^d)$, as the closure of the homogeneous Schwartz space $\dot\cS (\R^d)$ with respect to the norm 
	\[ ||u||_{\dot W^{s, p}} := |||\nabla|^s u||_{L^p}. \]

\begin{proposition}[Basic embeddings]
	Let $1 < p < \infty$, then 
	\begin{enumerate}
		\item The inhomogeneous Sobolev space is monotonic in $s$. That is, if $s_1 < s_2$, then 
			\[ || \langle \nabla \rangle^{s_1} u||_{L^p} \lesssim ||\langle \nabla \rangle^{s_2} u||_{L^{p}}. \]
			In particular, we have the continuous embedding $W^{s_2, p} (\R^d) \subseteq W^{s_1, p} (\R^d)$.
			
		\item The inhomogeneous Sobolev space embeds into the homogeneous space. That is, if $s > 0$, then 
			\[ |||\nabla^s u||_{L^p} \lesssim   ||\langle \nabla \rangle^s u||_{L^p}. \]
			In particular, we have the continuous embedding $W^{s, p} (\R^d) \subseteq \dot W^{s, p} (\R^d)$.
	\end{enumerate}
\end{proposition}

\begin{proof}
	An elementary calculation shows that $\langle \nabla \rangle^{s_1 - s_2}$ and $|\nabla|^s \langle \nabla \rangle^{-s}$ are Hormander-Mikhlin multipliers, so they form bounded operators on $L^p (\R^d)$.
\end{proof}

When $s \in \N$ is integer, we would like for the Sobolev spaces defined using multipliers to coincide with the classical Sobolev spaces defined using classical derivatives. Indeed, 

\begin{proposition}[Characterisation using derivatives]
	Let $1 < p < \infty$, and suppose $k \in \N$ and $s \in \R$, then 
		\[ |||\nabla|^{s + k} u||_{L^p} \sim || |\nabla|^s \nabla^k u||_{L^p}, \]
	and		
		\[ || \langle \nabla \rangle^{s + k} u||_{L^p} \sim || \langle \nabla \rangle^s u||_{L^p} + || \langle \nabla \rangle^s \nabla^k u ||_{L^p} \sim \sum_{j = 0}^k ||\langle \nabla \rangle^s \nabla^j u||_{L^p}.\]
\end{proposition}

\begin{proof}
	The characterisation of the homogeneous Sobolev space by derivatives follows from boundedness of the Riesz transforms $\partial_j/|\nabla|$ on $L^p (\R^d)$, writing 
		\[ |\nabla| = \sum_{j = 1}^d \frac{\partial_j}{|\nabla|} \partial_j , \qquad \partial_j = \frac{\partial_j}{|\nabla|} |\nabla|. \]
	For the inhomogeneous Sobolev spaces, the second quantity controls the first via the Hormander-Mikhlin multiplier $\langle \nabla \rangle^k /(1 + |\nabla|^k)$, the third quantity trivially controls the second, the first quantity controls the third via the Hormander-Mikhlin multipliers $|\nabla|^j/\langle \nabla \rangle^k$ for each $j = 0, \dots, k$.  
\end{proof}

The expected duality relations hold for the Sobolev spaces; the proofs are essentially identical to that for Lebesgue spaces, relying on Holder's inequality. Let $1 \leq p, p' \leq \infty$ be dual exponents, i.e.  $\tfrac1p + \tfrac{1}{p'} = 1$, then 

\begin{theorem}[Duality]
	Let $1 < p < \infty$ and $s \in \R$, then 
		\begin{align*}
			\dot W^{s, p} (\R^d)^* 
				&\cong \dot W^{-s, p'} (\R^d), \\
			W^{s, p} (\R^d)^* 
				&\cong  W^{-s, p'} (\R^d).
		\end{align*}
\end{theorem}





\section{Embeddings}
The Sobolev embedding inequalities trade regularity for integrability. The general strategy will be to prove the embedding inequality for each Littlewood-Paley projection, and then efficiently sum by using Bernstein's inequality to recover the original result,
	\begin{align*}
		||u||_{L^q}
			\leq \sum_{N \in 2^\Z} ||u_N||_{L^q} \lesssim \sum_{N \in 2^\Z} N^{\frac{d}{p} - \frac{d}{q}} ||u_N||_{L^p},
	\end{align*}
for $1 \leq p < q \leq \infty$. The factor $N^{\frac{d}{p} - \frac{d}{q}}$ decays for $N \lesssim 1$, so we can control low frequencies by a lower order term. The factor instead grows for $N \gtrsim 1$, so we need to control high frequencies by a higher order term via the \textit{Sobolev-Bernstein} inequality to recover decay. 

\subsection{$L^p$-inequalities}

When localising to frequencies $|\xi| \sim N$, the Riesz potentials behave like $|\nabla|^s \sim N^s$, while the Bessel potentials behave like $\langle \nabla \rangle^s \sim \langle N \rangle^s$. The heuristic is that positive-order differential operators $s > 0$ accentuate high frequencies and diminish low frequencies, while negative-order operators have the opposite effect. More precisely, we have the following family of inequalities, 


\begin{lemma}[Sobolev-Bernstein inequalities] For $f \in L^p (\R^d)$, 
				\begin{alignat*}{2}
					|| |\nabla|^s u_N||_{L^p} 
						&\sim N^s ||u_N||_{L^p}, &&\qquad \text{if }s \in \R \text{ and } 1 \leq p \leq \infty, \\
					|| |\nabla|^s u_{\leq N}||_{L^p} 
						&\lesssim N^s ||u_{\leq N}||_{L^p}, &&\qquad\text{if }s > 0 \text{ and } 1 \leq p < \infty, \\
					|| |\nabla|^s u_{\geq N}||_{L^p} 
						&\lesssim N^s ||u_{\geq N}||_{L^p}, &&\qquad\text{if }s < 0 \text{ and } 1 \leq p < \infty.
				\end{alignat*}			
		The endpoint cases $p = \infty$ for the latter two inequalities hold for $u \in \cS (\R^d)$. 
\end{lemma}



\begin{theorem}[Gagliardo-Nirenberg inequality]
	Let $1 < p < q \leq \infty$ and $s > 0$, then 
		\[ ||u||_{L^q} \lesssim ||u||_{L^p}^{1 - \theta} || |\nabla|^s u ||_{L^p}^\theta \]
	where $0 < \theta < 1$ satisfies $\tfrac1p - \tfrac1q = \tfrac{\theta s}{d}$. 	
\end{theorem}

\begin{proof}
	Taking a Littlewood-Paley decomposition $u = \sum_N u_N$, we apply the triangle inequality and Bernstein's inequality to obtain
	\begin{align*}
		||u||_{L^q}
			\leq \sum_{N \in 2^\Z} ||u_N||_{L^q} \lesssim \sum_{N \in 2^\Z} N^{\theta s} ||u_N||_{L^p}.
	\end{align*}
	Boundedness of the Littlewood-Paley projections and the Sobolev-Bernstein inequality give the respective bounds
		\[ ||u_N||_{L^p} \lesssim ||u||_{L^p}, \qquad ||u_N||_{L^p} \lesssim N^{-s} || |\nabla|^s u||_{L^p}.\]
	Hence
		\[ ||u_N||_{L^p} \lesssim \min \Big\{ ||u||_{L^p}, N^{-s} || |\nabla|^s u||_{L^p} \Big\}. \]
	Decomposing the sum at the transition $N \sim (|||\nabla|^s u||_{L^p}/||u||_{L^p})^{1/s}$ into high and low frequencies, we conclude
		\begin{align*}
			||u||_{L^q}
				&\lesssim \Big( \sum_{N \lesssim (|||\nabla|^s u||_{L^p}/||u||_{L^p})^{1/s}} + \sum_{N \gtrsim (|||\nabla|^s u||_{L^p}/||u||_{L^p})^{1/s}} \Big) N^{\theta s} ||u_N||_{L^p} \\
				&\lesssim \sum_{N \lesssim (|||\nabla|^s u||_{L^p}/||u||_{L^p})^{1/s}} N^{\theta s} ||u||_{L^p} + \sum_{N \gtrsim (|||\nabla|^s u||_{L^p}/||u||_{L^p})^{1/s}} N^{(\theta - 1) s}  |||\nabla|^s u||_{L^p} \lesssim ||u||_{L^p}^{1 - \theta} || |\nabla|^s u||_{L^p}^\theta 
		\end{align*}	
	as desired. 	
\end{proof}

\begin{theorem}[Non-endpoint Sobolev inequality]
	Let $1 \leq p < q \leq \infty$, and suppose $u \in \cS (\R^d)$ then 
		\[ ||u||_{L^q} \lesssim || \langle \nabla \rangle^s u ||_{L^p} \]
	whenever $\tfrac1p - \tfrac1q < \tfrac{s}{d}$ and $s > 0$. In particular, $W^{s, p} (\R^d) \subseteq L^q (\R^d)$ and, in the endpoint case $q = \infty$ and $\tfrac1p < \tfrac{s}{d}$, we have $W^{s, p} (\R^d) \subseteq C^0 (\R^d)$. 
\end{theorem}

\begin{proof}
	Taking a Littlewood-Paley decomposition $u = \sum_N u_N$, we apply the triangle inequality and Bernstein's inequality to obtain
		\[ ||u||_{L^q} \leq \sum_{N \in 2^\Z} ||u_N||_{L^q} \lesssim \sum_{N \in 2^\Z} N^{\frac{d}{p} - \frac{d}{q}} ||u_N||_{L^p}. \]
	Boundedness of the Littlewood-Paley projections and the Sobolev-Bernstein inequality give the respective bounds	
		\[ ||u_N||_{L^p} \lesssim ||u||_{L^p} \lesssim || \langle \nabla \rangle^s u||_{L^p}, \qquad ||u_N||_{L^p} \sim N^{-s} |||\nabla|^s u||_{L^p}\lesssim N^{-s} || \langle \nabla \rangle^s u||_{L^p}. \]
	Hence
		\[ ||u_N||_{L^p} \lesssim \min\{ 1, N^{-s} \} || \langle \nabla \rangle^s u||_{L^p}.\]
	Decomposing the sum at the transition $N = 1$ into high and low frequencies, we conclude
		\begin{align*}
			 ||u||_{L^q} 
			 	&\lesssim \left( \sum_{N < 1} + \sum_{N \geq 1} \right) N^{\frac{d}{p} - \frac{d}{q}} ||u_N||_{L^p} \\
			 	&\lesssim \sum_{N \leq 1} N^{\frac{d}{p} - \frac{d}{q} -s} || \langle \nabla \rangle^s u||_{L^p} +  \sum_{N < 1} N^{\frac{d}{p} - \frac{d}{q}} || \langle \nabla \rangle^s u||_{L^p} \sim || \langle \nabla \rangle^s u||_{L^p},
		\end{align*}	 
	as desired. 	
\end{proof}

\subsection{$C^{0, \alpha}$-inequalities}

In the non-endpoint inequality, surplus regularity $s > \tfrac{d}{p}$ allows us to go past integrability $q = \infty$ and recover continuity. We can be more precise and show that in fact Holder continuity holds. Let $0 < \alpha < 1$, define the \emph{inhomogeneous Holder space}, $C^{0, \alpha} (\R^d)$, as the closure of Schwartz space $\cS (\R^d)$ with respect to the norm 
	\[ ||u||_{C^{0, \alpha}} := ||u||_{C^0} + \sup_{0 < |x - y| < 1} \frac{|u(x) - u(y)|}{|x - y|} ,\]
and the \emph{homogeneous Holder space}, $\dot C^{0, \alpha} (\R^d)$, as the closure of homogeneous Schwartz space $\dot \cS (\R^d)$ with respect to the norm 
	\[ ||u||_{\dot C^{0, \alpha}} := \sup_{x \neq y} \frac{|u(x) - u(y)|}{|x - y|} .\]
The Holder spaces admit the following characterisation by Littlewood-Paley projections,


\begin{proposition}[Besov characterisation of Holder norm]
	Let $0 < \alpha < 1$, then 
		\begin{align*}
			 ||u||_{C^{0, \alpha}}
			 	&\sim ||u_{\leq 1}||_{L^\infty} + \sup_{N \in 2^\N} N^\alpha ||u_N||_{L^\infty},
			 \\
			 ||u||_{\dot C^{0, \alpha}}
			 	&\sim \sup_{N \in 2^\Z} N^\alpha ||u_N||_{L^\infty}.
		\end{align*}
	The quantities on the right are known as \emph{Besov norms}. 
\end{proposition}

\begin{proof}
	As usual, we prove the homogeneous case, the inhomogeneous case is a trivial exercise. For simplicity, let us only consider the case $k = 0$. Using $\int \widecheck{\psi_N} = \psi_N (0) = 0$ and a change of variables $Ny = z$, we can write
		\begin{align*}
			u_N (x) = (u * \widecheck{\psi_N})(x) 
				&= N^d \int_{\R^d} u(x - y) \widecheck\psi (Ny) dy \\
				&= \int_{\R^d} u(x - z/N) \widecheck \psi (z) dz = \int_{\R^d} \Big( u(x - z/N) - u(x)\Big) \widecheck \psi (z) dz.
		\end{align*}	
	Taking the $L^\infty$-norm of the above, applying Minkowski's inequality and the Holder condition gives
		\begin{align*}
			||u_N||_{L^\infty} \lesssim ||u||_{\dot C^{0, \alpha}} N^{-\alpha} . 
		\end{align*}
	This proves the Holder norm controls the Besov norm. To prove the converse inequality, we decompose the difference $u(x) - u(y)$ into high and low frequencies; let $M \in 2^\Z$ satisfy $ |x - y|^{-1} \sim M$, it follows from the triangle inequality that
		\[ | u(x) - u(y)| \leq | u_{\leq M}(x) - u_{\leq M} (y)|+ | u_{\geq M}(x) - u_{\geq M} (y)|.  \]
	For the high frequency terms, we crudely estimate by the triangle inequality, 
		\[ | u_{\geq M} (x) - u_{\geq M}(y) | \lesssim \sum_{N\geq M} ||u_N||_{L^\infty} \lesssim \sup_{N \in 2^\Z} N^{\alpha} || u_N||_{L^\infty}  \sum_{N \geq M} N^{-\alpha} \sim  |x - y|^{\alpha} \sup_{N \in 2^\Z} N^\alpha || u_N||_{L^\infty}.\]	
	This furnishes the desired bound for the high frequency terms. For the low frequency terms, we can write using the fundamental theorem of calculus
		\[  u_{\leq M} (x) -  u_{\leq M} (y)  = (x - y) \cdot \int_0^1 \nabla u_{\leq M} (\theta x - (1 - \theta) y) d \theta. \]	
	Applying the triangle, Minkowski, Sobolev-Bernstein inequalities, we obtain
		\[ |u_{\leq M} (x) -  u_{\leq M} (y)| \leq |x - y| \sum_{N \leq M} ||\nabla u_{N}||_{L^\infty_x} \lesssim   |x - y|^{\alpha} \sup_{N \in 2^\Z} N^\alpha || u_N||_{L^\infty} . \]	
	This furnishes the desired bound for the low frequency terms, completing the proof. 		
\end{proof}



\begin{theorem}[Morrey's inequality]
	Let $1 < p < \infty$, then 
		\[ || u ||_{\dot C^{0, \alpha}} \lesssim |||\nabla|^s u||_{L^p}  \]
	for $0 < \alpha < 1$ satisfying $s - \tfrac{d}{p} = \alpha$. In particular, $\dot W^{s, p} (\R^d) \subseteq \dot C^{0, \alpha} (\R^d)$. Similarly for the inhomogeneous case. 
\end{theorem}

\begin{proof}
	By Sobolev-Bernstein and Bernstein's inequality, we obtain 
		\begin{align*}
			||u_N||_{L^\infty}
				&\leq N^{-s} || |\nabla|^s u||_{L^\infty} \lesssim N^{-\alpha} |||\nabla|^s u||_{L^p}.
		\end{align*}
	Rearranging and taking the supremum over $N \in 2^\Z$ completes the proof. 
\end{proof}

\subsection{Hardy-Littlewood-Sobolev inequality}

The proof of the endpoint Sobolev inequality in the spirit of that above is far more technical since we do not access to the lower order terms on the right. Furthermore, the triangle inequality and boundedness of the projections are too inefficient to be effective in the critical case. We instead follow the classical approach of \cite{Hedberg} using an inverse Sobolev inequality, namely

\begin{lemma}[Hardy-Littlewood-Sobolev inequality]
	For $u \in L^p (\R^d)$ and $1 < p, r < \infty$, we have
		\[ || |\nabla|^{-\alpha} u||_{L^r} \lesssim ||u||_{L^p} \]
	whenever $0 < \alpha < d$ and $\frac{\alpha}{d} + \tfrac1r = \tfrac1p$. At the endpoint $p = 1$, we also have the weak-type estimate
		\[ || |\nabla|^{-\alpha} u||_{L^{r, \infty}} \lesssim ||u||_{L^1}. \]
	The analogous results hold replacing $|\nabla|^{-\alpha}$ with $\langle \nabla \rangle^{-\alpha}$. 
\end{lemma}

\begin{proof}
	Without loss of generality take $||u||_{L^p} = 1$. We aim to control the convolution pointwise by the maximal function of $f$ and the $L^p$-norm of $u$. For $R > 0$ to be chosen later, 	
		\[ |(u * |x|^{\alpha - d})(x)| \leq \int_{\R^d} |x - y|^{\alpha - d} |u(y)| dy = \Big( \int_{|x - y| < R} + \int_{|x - y| \geq R} \Big)  |x - y|^{\alpha - d} u(y)| dy. \]
	Dividing the integral over the singularity into integrals over annular regions with dyadic radii, 
		\begin{align*}
			\int_{|x - y| < R}  |x - y|^{\alpha - d} |u(y)| dy 
				&\leq \sum_{r \in 2^\Z \ : \ r \lesssim R} \int_{r \leq |x - y| \leq 2r}  |x - y|^{d - \alpha} |u(y)| dy \\
				&\leq \sum_{r \in 2^\Z \ : \ r \lesssim R} r^{\alpha - d} \int_{|x - y| \leq 2r} |u(y)| dy \lesssim  \sum_{r \in 2^\Z \ : \ r \lesssim R} r^{\alpha} Mu (x) \lesssim R^{\alpha} Mu(x).
		\end{align*}	
	Applying Holder's inequality, the integral away from the singularity is controlled pointwise by
		\[  \int_{|x - y| \geq R}  |x - y|^{\alpha - d} |u(y)| dy \leq ||u||_{L^p} || |x|^{\alpha - d} ||_{L^{p'} (|x| \geq R)} \sim R^{-\frac{d}{r}}. \]
	 Choosing the optimal radius $R = Mu(x)^{-p/d}$, we obtain the pointwise bound $|(u * |x|^{\alpha - d} ) (x)| \lesssim Mu (x)^{p/r}$. Since the maximal operator is strong-type $(p, p)$, we conclude
		\[ ||u * |x|^{\alpha - d} ||_{L^r_x} \lesssim ||(Mu)^{p/r}||_{L^r} = ||Mu||^{p/r}_{L^p} \lesssim ||u||_{L^p}^{p /r} =1,  \]
	and analogously at the endpoint $p = 1$ using the weak-type $(1, 1)$ inequality. This completes the proof. 
\end{proof}

\begin{theorem}[Endpoint Sobolev inequality]
	Let $1  < p < q < \infty$, then 
		\[ ||u||_{L^q} \lesssim |||\nabla|^s u||_{L^p} \]
	whenever $\tfrac1p - \tfrac1q = \tfrac{s}{d}$ and $0 < s < \tfrac{d}{p}$. In particular, $\dot W^{s, p} (\R^d) \subseteq L^q (\R^d)$. The analogous inhomogeneous results also hold.
\end{theorem}

\begin{proof}
	By duality, Plancharel's theorem, and Holder's inequality, 
		\begin{align*}
			 ||u||_{L^q} 
			 	&= \sup_{v \in \dot \cS (\R^d) \, ; \, ||v||_{L^{q'}} = 1} \langle u, v \rangle =  \sup_{v \in \dot \cS (\R^d) \, ; \, ||v||_{L^{q'}} = 1} \langle \widehat u, \widehat v \rangle=  \sup_{v \in \dot \cS (\R^d) \, ; \, ||v||_{L^{q'}} = 1} \langle |\xi|^s \widehat u, |\xi|^{-s} \widehat v \rangle \\
			 	&\sim  \sup_{v \in \dot \cS (\R^d) \, ; \, ||v||_{L^{q'}} = 1} \langle |\nabla|^s u, |x|^{s - d} * v \rangle \leq  || |\nabla|^s u||_{L^p} \sup_{v \in \dot \cS (\R^d) \, ; \, ||v||_{L^{q'}} = 1} |||x|^{s - d} * v||_{L^{p'}}. 
		\end{align*}	 
	It follows from Hardy-Littlewood-Sobolev that
		\[ || |x|^{d - s} * v||_{L^{p'}} \lesssim ||v||_{L^{q'}} \]
	for $1 + \tfrac{1}{p'} = \tfrac{1}{q'} + \tfrac{d - s}{d}$, which, upon rearranging, is equivalent to $\tfrac1p = \tfrac1q + \tfrac{s}{d}$, completing the proof. 
\end{proof}


\begin{theorem}
	Let $1 \leq p \leq \infty$ and $\alpha > 0$, then the Bessel potential maps $L^p (\R^d)$ onto itself with operator norm $||\langle \nabla \rangle^{-\alpha}||_{L^p \to L^p} \leq 1$. In particular, for $s > 0$, we have
		\[ ||u||_{L^p} \leq || \langle \nabla \rangle^s u ||_{L^p}. \]
\end{theorem}

\begin{proof}
	The Bessel potential kernel has unit mass, so the result follows from Young's convolution inequality. The corresponding Sobolev inequality follows from duality.  
\end{proof}





\section{Localisation}
The classical trace, Poincare, and Hardy inequalities concern trading \textit{regularity} for \textit{spatial localisation} via the uncertainty principle. As a concrete example, consider the Heisenberg uncertainty principle
	\begin{equation} ||u||_{L^2_x}^2 \lesssim ||x u||_{L^2_x} ||\nabla u||_{L^2_\xi}. \tag{*}\label{eq:heisen}
	\end{equation}
Normalising $||u||_{L^2_x} = 1$, if $u$ is localised in physical space in that $||xu||_{L^2_x} \ll 1$, then the inequality above forces a loss in regularity $||\nabla u||_{L^2_\xi}\gg 1$. Conversely, we can use regularity to control spatial localisation. 

\subsection{Traces}

In the interest of solving boundary value problems, we study the boundedness properties of the \emph{trace} operator $T$ which sends a function $u : \R^d \to \C$ to its restriction on a hyperplane $u_{|x_d = 0} : \R^{d - 1} \to \C$, 
	\[ T u (x') := u(x', 0) = \langle u, \delta_{x_d = 0} \rangle \]
where we use the notation $x = (x_1, \dots, x_d)$ and $x' = (x_1, \dots, x_{d - 1})$. We similarly denote $\cF'$ the Fourier transform and $P_N'$ the Littlewood-Paley projections on the hyperplane $\R^{d - 1}$. The main ingredient for studying the trace operator on Sobolev spaces is the uncertainty principle. This is encapsulated by the following two lemmas: 

\begin{lemma}
	Given a tempered distribution $u \in \cS' (\R^d)$, the trace of the Littlewood-Paley projection $P_{\leq N} u$ has Fourier support in $|\xi'| \lesssim N$. In particular, 
		\[ P'_M T P_{\leq N} u = 0 \]
	for $M \gtrsim N$. 
\end{lemma}

\begin{proof}
	From the uncertainty principle, restricting to a point in physical space is equivalent to integrating in frequency space, that is, $\langle \delta_0, \phi \rangle = \langle 1, \widehat \phi \rangle$. Moreover, the composition of the Fourier transform on $\R^{d - 1}_{x'}$ with the Fourier transform on $\R_{x_d}$ gives the Fourier transform on $\R^d_x$. Hence, in frequency space
		\[ \cF' T P_{\leq N} u = \cF' P_{\leq N} u (\xi', 0) = \int_\R \cF P_{\leq N} u \, d\xi_d. \]
	As $\cF P_{\leq N} u$ is supported in the ball $|\xi| \lesssim N$, we see that $T P_N u$ has Fourier support in the ball $|\xi'| \lesssim N$. 	
\end{proof}

\begin{lemma}
	Given a tempered distribution $u \in \cS' (\R^d)$, the trace of the Littlewood-Paley projection $P_{\leq N} u$ satisfies
		\[ || TP_{\leq N} u ||_{L^p (\R^{d - 1}_{x'})} \lesssim N^{\frac1p} ||P_{\leq N} u||_{L^p (\R^d_x)} \]
\end{lemma}

\begin{proof}
	Since $\cF P_{\leq N} u$ is supported in $|\xi| \lesssim N$, we see that $\cF_{x_d} P_{\leq N} u$ is supported in $|\xi_d| \lesssim N$ for each fixed $x' \in \R^{d - 1}$. Thus by Minkowski's integral inequality and Bernstein's inequality in one-dimension, 
		\begin{align*}
			||T P_{\leq N} u||_{L^p (\R^{d - 1}_{x'})}
				&\leq \left|\left| || P_{\leq N} u||_{L^\infty (\R_{x_d})} \right|\right|_{L^p (\R^{d - 1}_{x'})} \\
				&\lesssim  \left|\left| N^{\frac1p} || P_{\leq N} u||_{L^p (\R_{x_d})} \right|\right|_{L^p (\R^{d - 1}_{x'})} = N^{\frac1p} ||P_{\leq N} u||_{L^p (\R^d_x)}.
		\end{align*}
	This completes the proof. 	
\end{proof}
	
\begin{theorem}[Sobolev trace theorem]
	Let $1 < p < \infty$ and $s > \sigma> \tfrac1p$, then the trace map satisfies
		\[ ||T u||_{W^{s - \sigma, p} (\R^{d - 1})} \lesssim ||u||_{W^{s, p} (\R^d)}. \]
	In particular, it extends to a bounded linear operator $T : W^{s, p} (\R^d) \to W^{s - \sigma, p} (\R^{d - 1})$.
\end{theorem}

\begin{proof}
	We perform a Littlewood-Paley decomposition $u = \sum_N P_N u$. By linearity, we can write $P'_M T u = \sum_N P'_M T P_N u$. Applying the triangle inequality and Sobolev-Bernstein gives
		\[ || \langle \nabla' \rangle^{s - \sigma} Tu||_{L^p (\R^{d - 1}_{x'})} \lesssim ||P'_{\leq 1} Tu||_{L^p (\R^{d - 1}_{x'})} +  \sum_{M \in 2^\N} M^{s - \sigma} ||P'_M T u||_{L^p (\R^{d - 1}_{x'})} .\]
	Applying the previous two lemmas and Sobolev-Bernstein gives, for low frequencies, 
		\begin{align*}
			||P'_{\leq 1} Tu||_{L^p (\R^{d - 1}_{x'})}
				&\leq\sum_{N \gtrsim 1}
			||P'_{\leq 1} T P_N u||_{L^p (\R^{d - 1}_{x'})}\\
				&\lesssim \sum_{N \gtrsim 1} N^{1/p} ||P_{N} u||_{L^p (\R^d_x)} \\
				&\lesssim  \sum_{N \gtrsim 1} N^{1/p - s} ||\langle \nabla \rangle^s u||_{L^p (\R^d_x)} \sim   ||\langle \nabla \rangle^s u||_{L^p (\R^d_x)}
		\end{align*}
	and for high frequencies, 
		\begin{align*}
			\sum_{M \in 2^\N} M^{s - \sigma} ||P'_M T u||_{L^p (\R^{d - 1}_{x'})}
				&\lesssim \sum_{M \in 2^\N} \sum_{N : M \lesssim N} M^{s - \sigma} ||P'_M T P_N u||_{L^p (\R^{d - 1}_{x'})}\\
				&\lesssim \sum_{M \in 2^\N} \sum_{N : M \lesssim N} M^{s - \sigma} N^{1/p} || P_N u||_{L^p (\R^{d}_{x})} \\
				&\lesssim \sum_{M \in 2^\N} \sum_{N : M \lesssim N} M^{s - \sigma} N^{1/p - s} || \langle \nabla \rangle^s u||_{L^p (\R^{d}_x)} \sim ||\langle \nabla \rangle^s u||_{L^p (\R^d_x)}.
		\end{align*}
	This completes the proof. 
\end{proof}

\begin{remark}
	This proof does not extend to the endpoints $p = 1, \infty$ due to the failure of the Littlewood-Paley decomposition $u = \sum_N P_N u$ to converge in $L^p (\R^d)$. The endpoint $s = \sigma$ holds in the Hilbert space case $p = 2$; we leave this as an exercise. 
\end{remark}

\subsection{Poincare inequality}

The uncertainty principle tells us that functions localised to frequency scale $|\xi| \lesssim N$ are approximately constant at physical scales $\sim 1/N$. Combined with the uniform boundedness of the Littlewood-Paley projections, this tells us that the support of $u \in C^\infty_c (\R^d)$ sees very little of the mass of its projection $P_N u$ for large $N\gg 1$, as $P_N$ spreads out the mass to increasingly large scales. Thus, high frequencies $P_{\geq M} u$ are the dominant contributions to $u$ for $M \ll 1$. 

\begin{theorem}[Poincare inequality]
	Let $1 \leq p \leq \infty$, and suppose $u \in C^\infty_c (|x| \leq R)$ for some $R > 0$. Then 
		\[ ||u||_{L^p} \lesssim R ||\nabla u||_{L^p}. \]
\end{theorem}

\begin{proof}
	Let $\chi \in C^\infty_c (|x| \leq 2R)$ such that $\chi \equiv 1$ on the ball $|x| \leq R$ and $0 \leq \chi \leq 1$. We decompose
		\[u = \chi P_{> C/R} u + \sum_{N < C/R} \chi P_N u \]
	for some small constant $C \ll 1$ to be chosen later. It follows from Holder's inequality and Bernstein's inequality
		\[  \sum_{N \leq C/R} ||\chi P_N u||_{L^p} \lesssim R^{d/p} \sum_{N \leq C/R} ||P_N u||_{L^\infty} \lesssim R^{d/p} \sum_{N \leq C/R} N^{d/p} ||P_N u||_{L^p} \lesssim C^{d/p} ||u||_{L^p}.   \]	
	Choosing $C \ll 1$, this shows that the high frequencies are the dominant contribution to the decomposition. We conclude from Sobolev-Bernstein
		\[ ||u||_{L^p} \lesssim ||P_{> C/R} u||_{L^p} \lesssim R ||\nabla u||_{L^p}. \]
	This completes the proof. 
\end{proof}

\begin{remark}
	One can also prove the $L^2$-Poincare inequality using the Heisenberg uncertainty principle (\ref{eq:heisen}). We leave this as an exercise. 
\end{remark}

\subsection{Hardy's inequality}

The uncertainty principle roughly states that $\triangle x \cdot \triangle \xi \gtrsim 1$. Formally rearranging, one arrives at an inequality of the form $\tfrac{1}{x} \lesssim \tfrac{d}{dx}$. This heuristic calculation is made precise by Hardy's inequality, 

\begin{theorem}[Hardy's inequality]
	Let $0 \leq s < d/2$, then 
		\[ \left|\left| \frac{u}{|x|^s} \right|\right|_{L^2} \lesssim |||\nabla|^s u||_{L^2}. \]
\end{theorem}

\begin{proof}
	We decompose dyadically in both physical and frequency space, writing 
		\begin{align*}
			\int_{\R^d} \frac{|u(x)|^2}{|x|^{2s}} dx 
				&\lesssim \sum_R R^{-2s} \int_{|x| \leq R} |u|^2 \, dx \\
				&\lesssim \sum_R \sum_N R^{-2s} ||P_N u||_{L^2 (|x| \leq R)}^2 \leq  \sum_R  R^{-2s} \left(\sum_N \min\{ 1, (NR)^{d/2} \} ||P_N u||_{L^2}\right)^2 ,
		\end{align*}
	where the last inequality follows from the bounds
		\begin{align*}
			||P_N u||_{L^2 (|x| \leq R)} 
				&\leq ||P_N u||_{L^2} ,\\
			||P_N u||_{L^2 (|x| \leq R)}
				&\lesssim R^{d/2} ||P_N u||_{L^\infty} \lesssim (NR)^{d/2} ||P_N u||_{L^2}.
		\end{align*}	
	This reduces the problem to showing boundedness of the linear operator $A : \ell^2 (2^\Z) \to \ell^2 (2^\Z)$ defined as
		\[ A(\{c_N\}_{N \in 2^\Z})_R := \sum_N \min \{ (NR)^{-s}, (NR)^{d/2 - s} \} c_N. \]
	Taking $c_N := N^s ||P_N u||_{L^2}$ completes the proof. By Schur's test, we simply need to show that the corresponding kernel of the operator $K(R, N) :=  \min \{ (NR)^{-s}, (NR)^{-d/2 - s} \}$ is uniformly summable in $N$ and $R$. Indeed, 
		\begin{align*}
			\sum_{R} K(R, N) 
				&\sim \sum_{R \geq 1/N} N^{-s} R^{-s} + \sum_{R < 1/N} N^{-d/2 - s} R^{d/2- s} \sim 1, \\
			\sum_{N} K(R, N) 
				&\sim \sum_{N \geq 1/R} N^{-s} R^{-s} + \sum_{N < 1/R} N^{-d/2 - s} R^{d/2- s} \sim 1	,
		\end{align*}
	as desired. 
\end{proof}

\bibliographystyle{alpha}
\bibliography{external/biblio}

\end{document}