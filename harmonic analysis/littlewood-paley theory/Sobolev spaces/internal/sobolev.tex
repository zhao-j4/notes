Let $1 < p < \infty$ and $s \in \R$, we define the \emph{inhomogeneous Sobolev space}, also known as the Bessel potential space, $W^{s, p} (\R^d)$, as the closure of Schwartz space $\cS (\R^d)$ with respect to the norm 
	\[ ||u||_{W^{k, p}} := ||\langle \nabla \rangle^s u||_{L^p}. \]
Analogously, we define the \emph{homogeneous Sobolev space}, also known as the Riesz potential space, $\dot W^{s, p} (\R^d)$, as the closure of the homogeneous Schwartz space $\dot\cS (\R^d)$ with respect to the norm 
	\[ ||u||_{\dot W^{s, p}} := |||\nabla|^s u||_{L^p}. \]

\begin{proposition}[Basic embeddings]
	Let $1 < p < \infty$, then 
	\begin{enumerate}
		\item The inhomogeneous Sobolev space is monotonic in $s$. That is, if $s_1 < s_2$, then 
			\[ || \langle \nabla \rangle^{s_1} u||_{L^p} \lesssim ||\langle \nabla \rangle^{s_2} u||_{L^{p}}. \]
			In particular, we have the continuous embedding $W^{s_2, p} (\R^d) \subseteq W^{s_1, p} (\R^d)$.
			
		\item The inhomogeneous Sobolev space embeds into the homogeneous space. That is, if $s > 0$, then 
			\[ |||\nabla^s u||_{L^p} \lesssim   ||\langle \nabla \rangle^s u||_{L^p}. \]
			In particular, we have the continuous embedding $W^{s, p} (\R^d) \subseteq \dot W^{s, p} (\R^d)$.
	\end{enumerate}
\end{proposition}

\begin{proof}
	An elementary calculation shows that $\langle \nabla \rangle^{s_1 - s_2}$ and $|\nabla|^s \langle \nabla \rangle^{-s}$ are Hormander-Mikhlin multipliers, so they form bounded operators on $L^p (\R^d)$.
\end{proof}

When $s \in \N$ is integer, we would like for the Sobolev spaces defined using multipliers to coincide with the classical Sobolev spaces defined using classical derivatives. Indeed, 

\begin{proposition}[Characterisation using derivatives]
	Let $1 < p < \infty$, and suppose $k \in \N$ and $s \in \R$, then 
		\[ |||\nabla|^{s + k} u||_{L^p} \sim || |\nabla|^s \nabla^k u||_{L^p}, \]
	and		
		\[ || \langle \nabla \rangle^{s + k} u||_{L^p} \sim || \langle \nabla \rangle^s u||_{L^p} + || \langle \nabla \rangle^s \nabla^k u ||_{L^p} \sim \sum_{j = 0}^k ||\langle \nabla \rangle^s \nabla^j u||_{L^p}.\]
\end{proposition}

\begin{proof}
	The characterisation of the homogeneous Sobolev space by derivatives follows from boundedness of the Riesz transforms $\partial_j/|\nabla|$ on $L^p (\R^d)$, writing 
		\[ |\nabla| = \sum_{j = 1}^d \frac{\partial_j}{|\nabla|} \partial_j , \qquad \partial_j = \frac{\partial_j}{|\nabla|} |\nabla|. \]
	For the inhomogeneous Sobolev spaces, the second quantity controls the first via the Hormander-Mikhlin multiplier $\langle \nabla \rangle^k /(1 + |\nabla|^k)$, the third quantity trivially controls the second, the first quantity controls the third via the Hormander-Mikhlin multipliers $|\nabla|^j/\langle \nabla \rangle^k$ for each $j = 0, \dots, k$.  
\end{proof}

The expected duality relations hold for the Sobolev spaces; the proofs are essentially identical to that for Lebesgue spaces, relying on Holder's inequality. Let $1 \leq p, p' \leq \infty$ be dual exponents, i.e.  $\tfrac1p + \tfrac{1}{p'} = 1$, then 

\begin{theorem}[Duality]
	Let $1 < p < \infty$ and $s \in \R$, then 
		\begin{align*}
			\dot W^{s, p} (\R^d)^* 
				&\cong \dot W^{-s, p'} (\R^d), \\
			W^{s, p} (\R^d)^* 
				&\cong  W^{-s, p'} (\R^d).
		\end{align*}
\end{theorem}



