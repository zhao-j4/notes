\documentclass[reqno]{amsart}

\usepackage{external/takodachi}

% To show labels in the margin:
\usepackage[notref, notcite]{showkeys}

% kill subsections in ToC
%\setcounter{tocdepth}{1} 

\title
{
	The energy method for non-linear wave equations
} 
\author{Jason Zhao}
%\address{Department of Mathematics, University of California, Berkeley, 94720}
%\email{zhao.j@berkeley.edu}
\date{\today}


\begin{document}

\begin{abstract}
	We provide an introduction to the classical local well-posedness theory for non-linear wave equations via the energy. The equations we consider in furthest generality take the form 
		\begin{align*}
			\Box_{\bfg(\phi, \partial\phi)} \phi 
				&= \cN (\phi, \partial \phi), \\
			(\phi, \partial_t \phi)_{|t = 0}
				&= (\phi_0, \phi_1),
		\end{align*}
	with initial data posed on the scale of $L^2_x$-based Sobolev spaces $(\phi_0, \phi_1) \in (H^s_x \times H^{s - 1}_x) (\R^d)$. Following \cite{Sogge1995}, we present a proof for sufficiently regular data $s \gg 1$ using physical space methods, i.e. integration-by-parts. To reach the classical exponent $s > \tfrac{d}{2} + 1$ due to \cite{FischerMarsden1972, HughesEtAl1977}, we introduce the paradifferential formulation of the equation, drawing from \cite{BahouriEtAl2011,Taylor2011b,IfrimTataru2022b}.
\end{abstract}
\maketitle

\tableofcontents

\section{Introduction}
Before getting into any multi-linear algebra, it is important to get a grasp of coordinates in linear algebra and how these coordinates respond to change of bases. Let $V$ be an $n$-dimensional real vector space, and denote $V^*$ its dual space. Given a basis $\{e_i\}_i \subseteq V$, there exists a dual basis $\{\epsilon^j\}_j \subseteq V^*$ satisfying 
	\[ \langle e_i, \epsilon^j \rangle = \delta^j_i. \]
Choose another basis $\{\widetilde e_i\}_i \subseteq V$ and denote its dual basis by $\{\widetilde \epsilon^j\}_j \subseteq V^*$. There exists a change of basis matrix $C^k_i \in \mathsf{GL} (\R^n)$ sending the original basis to the new basis,
	\[ \widetilde e_i = C^k_i e_k. \]	
On the other hand, the inverse change of basis matrix $(C^{-1})^j_k \in \mathsf{GL} (\R^n)$, i.e. $(C^{-1})^j_k C^k_i = \delta^j_i$, transforms the original dual basis to the new dual basis 	
	\[ \widetilde \epsilon^j = (C^{-1})^j_k \epsilon^k. \]
Throughout these notes, we will use these \emph{Einstein summation notation}, where repeated indices are summed over, e.g. $a^i b_i := \sum_i a^i b_i$.


\subsection{{Contravariance}}

We say an object is \emph{contravariant} if the coordinate representation \textit{contra-varies} with respect to change of basis, that is, transforms by the inverse matrix $(C^{-1})^j_i$. Such coordinates are indexed by \textit{upper indices}. The prototypical example of a contravariant object is a \emph{vector} $v \in V$. Every vector admits a unique coordinate representation $\{v^i\}_i \subseteq \R$ with respect to the basis $\{e_i\}_i$, i.e.
	\[ v = v^i e_i . \]
Let $\{ \widetilde v^j \}_j \subseteq \R$ be the unique coordinates with respect to the basis $\{\widetilde e_j\}_j$, then the change of coordinates from $\{v^i\}_i$ to $\{\widetilde v^j\}_j$ is given by the inverse change of basis matrix,
	\[ \widetilde v^j = {(C^{-1})}^j_i v^i. \]
Indeed, 	
	\[ v = \widetilde v^j \widetilde e_j  = \left( {(C^{-1})}^j_i v^i \right) \left( C^k_j e_k \right) = \delta^k_i v^i e_k = v^i e_i . \]
We can interpret a choice of basis $\{e_i\}_i$ as endowing $V$ with a ``measuring tool'', where the coordinates $\{v^i\}_i$ representing the resulting ``measurement''. A change of basis corresponds to changing the choice of ``measuring tool'', e.g. we can view a change of basis $\widetilde e_i = \tfrac{1}{100} e_i$ as changing from ``meters'' $e_i$ to ``centimeters'' $\widetilde e_i$, so the corresponding change of coordinates is 
	\[ \widetilde v^i \text{ meters } = 100 v^i \text{ centimeters}. \]




\subsection{Covariance}

We say that an object is \emph{covariant} if the coordinate representation \textit{co-varies} with respect to change of basis, that is, transforms by the matrix $C^k_i$.  Such coordinates are indexed by \textit{lower indices}. The prototypical example of a covariant object is a \emph{covector} $\omega \in V^*$. Every covector admits a unique coordinate representation $\{\omega_i \}_i \subseteq \R$ with respect to the basis $\{\epsilon^i\}_i$, i.e.
	\[ \omega = \omega_i \epsilon^i = \widetilde \omega_j \widetilde \epsilon^j. \]
Let $\{\widetilde \omega_j \}_j \subseteq \R$ be the unique coordinates with respect to the basis $\{\widetilde \epsilon^j\}_j$, then the change of coordinates from $\{\omega_i\}_i$ to $\{\widetilde \omega_j\}_j$ is given by the change of basis matrix,
	\[ \widetilde \omega_j = C^i_j \omega_i \]
Indeed, 
	\[ \omega = \widetilde \omega_j \widetilde \epsilon^j = \left( C^i_j \omega_i \right) \left( (C^{-1})_k^j \epsilon^k \right) =  \delta^i_k \omega_i \epsilon^k = \omega_i \epsilon^i. \]
Scalars are regarded as ``dimensionless'' quantities, so since a covector acting on a vector produces a scalar, they have inverse dimensions. For example, we can view a change of basis $\widetilde \epsilon^j = 100 \epsilon^j$ as changing from  ``meters$^{-1}$'' $\epsilon^j$ to ``centimeters$^{-1}$'' $\widetilde \epsilon^j$, so the corresponding change of coordinates is 
	\[ \widetilde \omega_j \text{ meters$^{-1}$} = \frac{1}{100} \omega_j  \text{ centimeters$^{-1}$}. \]



\section{Linear wave equations}

In the linear setting, energy estimates are essentially equivalent to well-posedness for the initial data problem. To illustrate the argument in a simplified setting, let $\mathsf L : X \to Y$ be a linear map between finite-dimensional vector spaces, and denote $\mathsf L^* : Y^* \to X^*$ its adjoint map. Then the existence for the original problem, 
    \begin{quote}
        \it 
        for every $f \in Y$, there exists a solution $\phi \in X$ to the equation 
            \[
                \mathsf L \phi 
                    = f,
            \]
    \end{quote}
is related to uniqueness for the dual problem
    \begin{quote}
        \it 
        for every $f \in X^*$, there is at most one solution $\phi \in Y^*$ to the equation 
            \[
                \mathsf L^* \phi 
                    = f,
            \]
    \end{quote}
in that the image of $\mathsf L$ is equal to the annihilator of the kernel of $\mathsf L^*$,
    \[
        \Im \mathsf L = (\ker \mathsf L^*)^\perp.
    \]
It follows that showing existence for the original problem, i.e. $\mathsf L$ is surjective, is equivalent to showing uniqueness for the dual problem, i.e. $\mathsf L^*$ is injective. 


    \begin{align*}
        \mathsf L 
            := \bfg^{\mu\nu} \partial_\mu \partial_\nu + \bfb^\mu \partial_\nu + \bfa 
    \end{align*}

\subsection{A priori estimates}

Let us begin with the simplest energy estimate, namely the conservation of energy for the linear wave equation on Minkowski space, 
\begin{equation}\tag{W}\label{eq:linear}
    \begin{split} 
    \Box \phi 
        &= f.
    \end{split}
\end{equation}
This equation is invariant under time-translation, so by Noether's theorem, we can produce a conservation law for solutions to the equation by multiplying the equation by $\partial_t \phi$. Differentiating-by-parts appropriately, we obtain the divergence identity
    \begin{align*}
        f \partial_t \phi = \Box \phi \partial_t \phi 
            &= \Big( - \partial_t^2 + \sum_{j = 1}^d\partial_j^2 \phi \Big) \partial_t \phi \\
            &= \partial_t \Big( - \frac12 |\partial_t \phi|^2\Big) + \sum_{j = 1}^d  \partial_j (\partial_j \phi \partial_t \phi) - \partial_j \phi \partial_t \partial_j \phi \\
            &= \partial_t \Big( - \frac12 |\partial_t \phi|^2 - \frac12 \sum_{j = 1}^d |\partial_j \phi|^2 \Big) + \nabla_x \cdot (\partial_t \phi \nabla_x \phi). 
    \end{align*}
Integrating on the space-time region $[0, T] \times \R^d$ and applying the divergence theorem furnishes 

\begin{proposition}[Energy identity]
    Let $f \in L^1_t L^2_x ([0, T] \times \R^d)$ and suppose $\phi \in C^0_t  H^1_x \cap C^1_t L^2_x ([0, T] \times \R^d)$ is a solution to the linear wave equation $\Box \phi = f$. Then 
        \begin{equation}\label{eq:identity}
            \int_{t = T} \frac12 |\nabla_{t, x} \phi|^2 \, dx 
                = \int_{t = 0} \frac12 |\nabla_{t, x} \phi|^2 \, dx + \int_0^T \int_{\R^d} f \, \partial_t \phi \, dx dt. 
        \end{equation}
    The solution also satisfies the energy estimates
        \begin{align}
            || (\phi, \partial_t \phi) ||_{C^0_t (\dot H^1_x \times L^2_x)} 
                &\lesssim   || (\phi_0, \phi_1) ||_{\dot H^1_x \times L^2_x} + ||f||_{L^1_t L^2_x},\label{eq:linest1}
            \\
            || (\phi, \partial_t \phi) ||_{C^0_t (H^1_x \times L^2_x)} 
            &\lesssim \langle T \rangle  \Big( || (\phi_0, \phi_1) ||_{H^1_x \times L^2_x} + ||f||_{L^1_t L^2_x}\Big)\label{eq:linest2}. 
        \end{align}
\end{proposition}

\begin{proof}
    To prove \eqref{linest1}, we simply apply Cauchy-Schwartz and Cauchy's inequality to the right-hand side of the energy identity \eqref{identity} to obtain 
        \begin{align*}
           \frac12 || (\phi, \partial_t \phi) ||_{C^0_t (\dot H^1_x \times L^2_x)}^2
                &\leq \frac12 || (\phi_0, \phi_1) ||_{\dot H^1_x \times L^2_x}^2 + ||f||_{L^1_t L^2_x} ||\partial_t \phi||_{C^0_t L^2_x} \\
                &\leq \frac12 || (\phi_0, \phi_1) ||_{\dot H^1_x \times L^2_x}^2 +\frac{\epsilon^{-1}}{2} ||f||_{L^1_t L^2_x}^2 + \frac{\epsilon}{2} ||\partial_t \phi||_{C^0_t L^2_x}^2,
        \end{align*}
    for any choice of $\epsilon > 0$. In particular, choosing $\epsilon \ll 1$ allows us to absorb the last term in the second line into the left-hand side, completing the proof of \eqref{linest1}. 

    To prove \eqref{linest2}, we apply the fundamental theorem of calculus in time to the estimates on the top-order terms in \eqref{linest1} to recover control over the lower-order terms, at the price of linear growth in $T$. Indeed, it suffices to bound the $L^2_x$-norm of the solution by the right-hand side. Writing,
        \[
            \phi(T) = \phi_0 + \int_0^T \partial_t \phi (t) \, dt, 
        \]
    and applying the $L^2_x$-norm to both sides, it follows from Minkowski's integral inequality and the triangle inequality that 
        \[
            ||\phi||_{C^0_t L^2_x} 
                \leq ||\phi_0||_{L^2_x} + T ||\partial_t \phi||_{C^0_t L^2_x}. 
        \]
    Inserting the first linear energy estimate \eqref{linest1} into the right-hand side completes the proof of \eqref{linest2}. 
\end{proof}

\begin{remark}
    The first energy estimate \eqref{linest1} states that the top-order terms stay uniformly bounded for all time, while the second energy estimate \eqref{linest2} allows the lower-order terms to grow linearly in time. 
\end{remark}

As a corollary, one arrives at the energy estimate, 

\begin{theorem}[Energy estimate for constant-coefficient wave equation]
    Let $f \in L^1_t H^{s -1}_x ([0, T] \times \R^d)$ and suppose $\phi \in C^0_t H^s_x \cap C^1_t H^{s - 1}_x ([0, T] \times \R^d)$ is a solution to the wave equation $\Box \phi = f$. Then 
        \begin{align}
            ||\phi||_{C^0_t H^s_x}
                &\lesssim \langle T \rangle \Big( ||\phi (0)||_{H^s_x} + ||\nabla_{t, x} \phi (0)||_{H^{s - 1}_x} + ||f||_{L^1_t H^{s - 1}_x }\Big).\label{eq:linest4}
        \end{align}
\end{theorem}

\begin{proof}
    The Fourier multiplier $\langle \nabla \rangle^s$ commutes with $\Box$, so the result follows from \eqref{linest2}. 
\end{proof}

\subsection{Existence-uniqueness duality}





\begin{lemma}[Existence-uniqueness duality]
    Let $\mathsf L : X \to Y$ be a linear operator between Banach spaces, and denote $\mathsf L^* : Y^* \to X^*$ its adjoint. The following statements hold:
        \begin{itemize}
            \item uniqueness furnishes existence for the dual problem, i.e. the energy estimate for $\mathsf L$ 
                \[ 
                    ||u||_X \lesssim ||\mathsf L u||_Y
                \]
            implies the adjoint operator is surjective, $\Im \mathsf L^* = X^*$,

            \item existence furnishes uniqueness for the dual problem, i.e. if $\mathsf L$ is surjective, $\Im \mathsf L = Y$, then the adjoint satisfies the energy estimate,
                \[
                    ||v||_{Y^*} 
                        \lesssim ||\mathsf L^* ||_{X^*}. 
                \]
        \end{itemize}
    In particular, if $X$ is reflexive, then the energy estimate furnishes existence and uniqueness for the problem $\mathsf L u = f$. 
\end{lemma}



\begin{lemma}[Hahn-Banach theorem]
    Let $X$ be a normed vector space and suppose $Y \hookrightarrow X$ is a linear subspace. If $f \in Y^*$ is a bounded linear functional on the subspace $Y$, then there exists an extension $\widetilde{f} \in X^*$ to a bounded linear functional on the entire space $X$ such that 
        \[
            ||\widetilde f||_{X^*} = ||f||_{Y^*}. 
        \]
\end{lemma}




    \[
        \mathsf L := g^{\mu \nu} \partial_\mu \partial_\nu + b^\mu \partial_\mu + a
    \]

\begin{equation}\label{eq:varlinear}\tag{VW}
    \begin{split}
        \mathsf L \phi 
        &= f, \\
    (\phi, \partial_t \phi)_{|t = 0} 
        &= (\phi_0, \phi_1). 
    \end{split}
\end{equation}


\begin{theorem}[Existence and uniqueness]
    Let $s \in \R$, then for every forcing term $f \in L^1_t H^{s - 1} ([0, T] \times \R^d)$, there exists a unique solution $\phi \in (C^0_t H^s_x \cap C^1_t H^{s - 1}_x) ([0, T] \times \R^d)$ to the initial data problem 
        \begin{align*}
            \mathsf L \phi 
                &= f, \\
            (\phi, \partial_t \phi)_{|t = 0}
                &= (0, 0). 
        \end{align*}
\end{theorem}

\section{Energy methods}

With our discussion of linear wave equations at hand, we are ready to begin our study of non-linear wave equations. 
    \[
        \Box_{\bfg(\phi)} := \bfg^{\mu \nu} (\phi) \partial_\mu \partial_\nu
    \]

    \begin{equation}\tag{P}\label{eq:perturb}
        | \bfg^{\mu \nu} - \bfm^{\mu \nu} |
            \ll 1. 
    \end{equation}


\begin{equation}\label{eq:QLW}\tag{QLW}
	\begin{split}
		\Box_{\bfg(\phi)} \phi 
			&= \mathcal N(\phi, \partial\phi), \\
		(\phi, \partial_t \phi)_{|t = 0}
			&= (\phi_0, \phi_1),
	\end{split}
\end{equation}


The simplest model to consider is the semi-linear wave equation with power-type non-linearity,
    \[
        \Box \phi 
            = \phi^3
    \]




\section{Paradifferential calculus}
\input{internal/paradiff}

\bibliographystyle{alpha}
\bibliography{external/biblio}

\end{document}
