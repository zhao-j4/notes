\documentclass[reqno]{amsart}

\usepackage{external/takodachi}

% To show labels in the margin:
\usepackage[notref, notcite]{showkeys}

% kill subsections in ToC
\setcounter{tocdepth}{1} 
\newcommand{\timestep}{\mathtt{dt}}

\title
{
	Dispersive quantisation in KdV
} 
\author{Jason Zhao}
%\address{Department of Mathematics, University of California, Berkeley, 94720}
%\email{zhao.j@berkeley.edu}
\date{\today}


\begin{document}

\begin{abstract}
	It has been observed both experimentally and mathematically that solutions to linear dispersive equations, such as the Schr\"odinger and Airy equations, posed on the torus exhibit dramatically different behaviors between rational and irrational times. For example, the evolution of piecewise constant data remains so at rational times, while it becomes continuous and fractalised at irrational times. A natural question to ask is whether this \textit{Talbot effect}, as it broadly known, persists under non-linear dispersive flows. Focusing on the KdV equation, we will present two perspectives, each drawing from the Fourier restriction norm estimates due to
	\cite{Bourgain1993} and the normal form transformation of \cite{BabinEtAl2011}: the first is the non-linear smoothing effect observed by \cite{ErdoganTzirakis2013}, and the second is the numerical work of \cite{HofmanovaSchratz2017,RoussetSchratz2022}. 
\end{abstract}
\maketitle

\tableofcontents

\section{Introduction}
When studying function spaces, such as Lorentz spaces or Sobolev spaces, it is useful to decompose a generic function into simpler pieces, and attempt to prove the desired results for each of those pieces. For example, functions in Lorentz spaces can be decomposed in \textit{physical space} into the sum of \textit{quasi-step functions}. Our approach in these notes will be to decompose into \textit{frequency-localised} pieces and  study the various ways these pieces sum. 

To that end, we construct a dyadic partition of unity as follows; let $\phi  \in C^\infty_c (\R^d)$ satisfy $0 \leq \phi \leq 1$ and 
\begin{align*}
	\phi(x) 
		:= 
		\begin{cases}
			1 , 				&|x| \leq 1.4, \\
			0, 				&|x| > 1.42. 
		\end{cases}
\end{align*}
Denote the dyadics by $2^\Z := \{ 2^n : n \in \Z \}$. For $N \in 2^\Z$, define $\psi, \psi_N, \phi_{\leq N} \in C^\infty_c (\R^d)$ to be 
	\[ \psi(x) := \phi(x) - \phi(2x), \qquad \psi_N (x) := \psi(x/N), \qquad \phi_{\leq N} (x) := \phi(x/N).  \]
Observe that $\sum_N \psi_N \equiv 1$ since pointwise it forms a telescoping sum. Given a tempered distribution $f \in \cS' (\R^d)$, we define its \emph{Littlewood-Paley projections} to frequencies $|\xi| \sim N$ and $|\xi| \lesssim N$ respectively by
	\begin{align*}
		\widehat{f_N} &= \widehat{P_N f}  = \psi_N \widehat f , \qquad
		\widehat{f_{\leq N}} = \widehat{P_{\leq N} f} = \phi_{\leq N} \widehat f.
	\end{align*}	
Define the Littlewood-Paley projections to frequencies $|\xi| \gtrsim N$ and $N \lesssim |\xi| \lesssim M$ respectively by 
	\[ f_{\geq N} = P_{\geq N} f = (1 - P_{\leq N}) f, \qquad f_{N \leq - \leq M} = P_{N \leq - \leq M} f = \sum_{N \leq K \leq M} P_K f. \]
The name ``projection'' is a bit of a misnomer; the multipliers $P_N$ fail to be true projections in the sense that by choosing smooth cutoffs in frequency space rather than sharp cutoffs, we have $P_N P_N \neq P_N$. Nevertheless, a slightly modified statement holds; define the \emph{fattened Littlewood-Paley projections} to frequencies $|\xi| \sim N$ and their corresponding multipliers by
	\[ \widetilde{P_N} := P_{\frac{N}{2}} + P_{N} + P_{2N},\qquad  \widetilde{\psi_N} := \psi_{\frac{N}{2}} + \psi_N + \psi_{2N}. \]
Since $\widetilde{\psi_N} \equiv 1$ on the support of $\psi_N$, it follows that $\widetilde{P_N} P_N = P_N$. Similarly, we can define the fattened projections to frequencies $|\xi| \lesssim N$ by 
	\[ \widetilde{P_{\leq N}} = P_{\leq 2N}, \qquad \widetilde{\phi_{\leq N}} := \phi_{\leq 2N}.  \]	

\begin{remark}
	By the Paley-Wiener theorem, the projections are analytic functions in physical space. Thus we can study the Littlewood-Paley projections pointwise without any philosophical consternation.
\end{remark}




\section{Preliminaries}

\subsection{Littlewood-Paley theory}



\subsection{Fourier restriction norm}

Denoting the space-time Fourier transform of $u: \R \times \TT \to \R$ by $\widetilde u(\tau, k)$, the \textit{Fourier restriction norm} for the Airy flow is defined by 
    \[
        ||u ||_{X^{s, b}}
            := || \langle k \rangle^s \langle \tau - k^3\rangle^b \widetilde u(\tau, k)||_{L^2_{\tau, k}}. 
    \]
The index $s \in \R$ simply captures the spatial Sobolev regularity, while the index $b> -\tfrac12$ measures the degree to which $u$ fails to solve the linear Airy equation \eqref{Airy}, since solutions to the equation necessarily have Fourier support on the cubic $\tau = k^3$. We also define the local-in-time norm 
    \[
        ||u||_{X^{s, b}(I)}
            := \inf_{U(t)\equiv u(t) \text{ for $t \in I$}} ||U||_{X^{s, b}}
    \]


\begin{lemma}[Local-in-time inhomogeneous $X^{s,b}$-estimate]
    Let $-\tfrac12 < b' \leq 0$ and $0 \leq b \leq b' + 1$, then for any time interval $I \subseteq \R$, 
        \begin{equation}\label{eq:Xsbsmooth}
            \left|\left| \int_0^t e^{- (t - t') \partial_x^3} f(t') \, dt' \right|\right|_{X^{s, b} (I) } 
                \lesssim |I|^{1 - b + b'} ||f||_{X^{s, b} (I)}.
        \end{equation}
\end{lemma}

\subsection{Local well-posedness theory}

\cite{CollianderEtAl2003}

\begin{theorem}
    For $s \geq -\tfrac12$, \eqref{KdV} is globally well-posed in $H^s (\TT)$, and satisfies, 
    \begin{enumerate}
        \item the local-in-time bound
            \begin{align}\label{eq:LWP}
                ||u||_{X^{s, \frac12} (I)} 
                    \lesssim ||u_0||_{H^s}, 
            \end{align}
            on time-scale 
                \[|I| = \frac{1}{ ||u_0||_{H^s}^{O_s(1)}.}\]

        \item the polynomial-growth bound 
            \begin{align}\label{eq:high}
                ||u||_{H^s} 
                    \leq C(||u_0||_{H^s}) \langle t \rangle^{O_s(1)}. 
            \end{align}
    \end{enumerate}
\end{theorem}

\section{Non-linear smoothing effect}\label{sec:smooth}

The first and foremost enemy to defeat when studying the KdV equation is the derivative in the non-linearity. On the real line, one can avoid this loss of derivative in the estimates by leveraging the local smoothing effect from the linear flow, see \cite[Chapter 4.1]{Tao2006} for a survey of this literature. This manifestation of dispersion relies on high frequency portions of the solution moving to spatial infinity, and thus fails in the setting of the torus. Nonetheless, there is a \textit{non-linear} smoothing effect via exploiting time-averaging of the non-resonant interactions, see \cite{Bourgain1993}, \cite{KenigEtAl1996a}, \cite{CollianderEtAl2003}, \cite{BabinEtAl2011}. 

The particular smoothing effect which we want to show is that the difference between the non-linear flow \eqref{KdV} and the linear flow \eqref{Airy} is smoother than either of the two. We can already see this at the level of the Duhamel formula, 
    \begin{equation}
        u(t) - e^{-t \partial_x^3} u_0
            =  6 \int_0^t e^{-(t - t') \partial_x^3} \Big(u(t')\partial_x u(t') \Big) \, dt'.
    \end{equation}
Solving the equation by Picard iteration, we compute the $k$-th Fourier mode of the difference between the first iterate and the linear flow, 
    \begin{align*}
        \widehat{\left(6 \int_0^t e^{-(t - t') \partial_x^3} \Big(e^{-t' \partial_x^3}u_0 \partial_x e^{-t' \partial_x^3} u_0 \Big) \, dt'\right)} (k) 
            &=  3e^{i t k^3} \sum_{\substack{k_1, k_2 \in \Z \\ k_1 + k_2 = k}}  \int_0^t e^{it' (-k^3 + k_1^3 + k_2^3)}  \widehat{u_0} (k_1) \cdot i k_2 \widehat{u_0} (k_2) dt' \\
            &= \frac{e^{i t k^3}}{k} \sum_{\substack{k_1, k_2 \in \Z \\ k_1 + k_2 = k}} \frac{\widehat{u_0}(k_1) \widehat{u_0} (k_2)}{k_1} \left(  1 - e^{-3i k_1 k_2 k t}\right),
    \end{align*}
where we use the following identity to perform the integration in time, allowing us to ameliorate the derivative in the non-linearity, 
    \begin{equation}\label{eq:identity}
        (k_1 + k_2)^3 - k_1^3 - k_2^3 = 3(k_1 + k_2) k_1 k_2.
    \end{equation}
Suppose, for example, $u_0 \in L^2 (\TT)$, then, from either an application of Young's convolution inequality in frequency space or Holder's inequality in physical space, we see that the difference between the first Picard iterate and the linear flow is a full derivative smoother,
    \begin{align*}
        ||u^{(1)} (t) - e^{-t \partial_x^3} u_0||_{H^1}
            &\lesssim || u_0 \partial_x^{-1} u_0 ||_{L^2} \\
            &\lesssim || u_0||_{L^2} ||\partial_x^{-1} u_0||_{L^\infty_x} \lesssim ||u_0||_{L^2_x}^2. 
    \end{align*}

The gain in regularity follows from exploiting the non-resonant interactions in the non-linearity. With this motivating example at hand, we are ready to state the main result of \cite{ErdoganTzirakis2013};

\begin{theorem}[Non-linear smoothing for KdV]
    Let $s > -\tfrac12$ and $\sigma < \min (3s + 1, s + 1)$. Given initial data $u_0 \in H^s (\TT)$ with zero mean, suppose the solution to \eqref{KdV} satisfies the polynomial-growth bound,
        \begin{equation}
            ||u(t)||_{H^s}
                \leq C(||u_0||_{H^s}) \langle t \rangle^{O(1)},
        \end{equation}
    then the difference between the non-linear flow and linear flows satisfies the smoothing estimate 
        \begin{equation}
            ||u(t) - e^{-t \partial_x^3} u_0 ||_{H^\sigma}
                \leq C(s, \sigma, ||u_0||_{H^s})  \langle t \rangle^{O(1)}. 
        \end{equation}    
\end{theorem}

\begin{remark}
    In fact, the difference of the flows 
        \[
            t \mapsto u(t) - e^{-t \partial_x^3} u_0
        \]
    is continuous with respect to the $H^\sigma$-topology. We leave this as an exercise. 
\end{remark}

In particular, the difference between the non-linear flow and linear flow is \textit{continuous}, in view of Sobolev embedding, for $s > -\tfrac16$. Since $\mathrm{BV} (\TT) \subseteq \dot W^{1, 1} (\TT) \subseteq L^2 (\TT)$, it follows that the qualitative properties of the linear flow for bounded variation data, Theorem \ref{thm:talbot}, persist for the non-linear flow,

\begin{corollary}[Non-linear Talbot effect]
    Let $u_0 \in \mathrm{BV} (\TT)$ be a $2\pi$-periodic function of bounded variation. Then the global \eqref{KdV} evolution $u \in C^0_t L^2_x (\R \times \TT)$ satisfies the following properties: 
    \begin{enumerate}
        \item The uniform bound 
            \[
                ||u||_{L^\infty_{t, x}} 
                    \lesssim C(s, \sigma, ||u_0||_{L^2_x}) \langle t \rangle^{O(1)} + ||u_0||_{\mathrm{BV}},
            \]
        \item For each $t$ such that $\tfrac{t}{2\pi} = \tfrac{p}{q}$ for $p, q \in \Z$ coprime, $u$ has $Nq$-many discontinuities in $x$, where $N \in \overline \N$ is the number of discontinuities of $u_0$. 
        
        \item For each $t$ such that $\tfrac{t}{2\pi}$ is irrational, $u$ is a continuous function of $x$.
    \end{enumerate}
    In particular, if $u_0$ is a continuous function, then $u \in C^0_{t, \loc} C^0_x (\R \times \TT)$. 
\end{corollary}


\subsection{Normal form transformation}


We appeal to the normal form approach of Babin-Ilyin-Titi \cite{BabinEtAl2011}, which, in a nutshell, boils down to integration-by-parts and Cauchy-Schwartz. 

Our derivation of the transformation is primarily physical space-based\footnote{This argument is due to Sung-Jin Oh.}; for the integration-by-parts approach starting from frequency space, the interested reader may find a textbook treatment in \cite[Chapter 3.4, Chapter 4.1.2]{ErdoganTzirakis2016}. The non-linearity has the structure of a symmetric bilinear form, so we introduce such a form $\cB(f, g)$, to be chosen later, and make the following \textit{ansatz} for the change of variables, 
    \begin{equation}\label{eq:normalform}
        e^{-t \partial_x^3} w 
            := u + \cB(u, u).
    \end{equation}
Then, using the KdV equation \eqref{KdV} for $u: I \times \TT \to \R$, symmetry and bilinearity of $\cB(f, g)$, the evolution equation for $w: I \times \TT \to \R$ takes the form 
    \begin{align*}
        \partial_t w
            &= e^{t \partial_x^3} \left( \partial_t u + \partial_x^3 u + 2 \cB(\partial_t u, u) + \partial_x^3 \cB(u, u)  \right)\\
            &= e^{t\partial_x^3} \left( - 6 u \partial_x u + 2 \cB(-\partial_x^3 u + 6u \partial_x u, u) + 2 \cB(\partial_x^3 u, u) + 6 \cB(\partial_x^2 u, \partial_x u) \right)\\
            &= e^{t \partial_x^3} \left( -6 u \partial_x u + 6\cB( \partial_x (u^2), u ) + 6 \cB(\partial_x^2 u, \partial_x u) \right). 
    \end{align*}
We can kill the first and third terms on the last line by taking 
    \[
        \cB (f, g) 
            := \partial_x^{-1} f \cdot \partial_x^{-1} g,
    \]
which furnishes the evolution equation for the renormalised variable 
    \begin{equation}\label{eq:renorm}
        \partial_t w 
            = 6\cT(u, u, u),
    \end{equation}
where the trilinear term on the right-hand side is given by 
    \[
        \cT(f, g, h) 
            := e^{t \partial_x^3} \left(f \cdot g \cdot \partial_x^{-1} h\right). 
    \]
Thus, we have eliminated the derivative in the non-linearity at the cost of increasing the order of the non-linearity from quadratic to cubic and performing a non-linear transformation in the unknowns. 

To estimate the non-linearity on the right-hand side of \eqref{renorm}, we separate the resonant and non-resonant terms. For the non-resonant terms, we can leverage time-oscillation to prove a trilinear smoothing estimate after averaging-in-time. The resonant terms then are the main enemy after the normal form reductions, though since the KdV non-linearity is a total derivative, it does not admit any zero modes, so we can exploit a cancellation in the resonant interactions.  Towards this resonant/non-resonant decomposition, we write equation \eqref{renorm} in frequency space, 
\begin{align}\label{eq:fourier}
    \partial_t \widehat w(k)
        &= 6 \sum_{\substack{k_1, k_2, k_3 \in \Z\\ k_1 + k_2 + k_3 = k\\ k_1 + k_2 \neq 0}} \frac{e^{-3it(k_1 + k_2)(k_2 + k_3)(k_3 + k_1)}}{k_3}\widehat{v} (k_1) \widehat v (k_2) \widehat v (k_3).
\end{align}
where 
    \[
        v = e^{t \partial_x^3} u,
    \]
owing to the identity 
    \[
        (k_1 + k_2 + k_3)^3 = 3(k_1 + k_2)(k_2 + k_3)(k_3 + k_1) + k_1^2 + k_2^2 + k_3^3. 
    \]
We decompose the right-hand side of \eqref{fourier} into resonant and non-resonant interactions, 
    \begin{align*}
        \widehat{\cR(f, g, h)} (k)
            &:= \sum_{\substack{k_1 + k_2 + k_3 = k \\|k_1| = |k_2| = |k_3| = |k|}} + \sum_{\substack{k_1 + k_2 + k_3 = k \\ k_n + k_m = 0 \text{ for distinct $n, m \in \{1,2, 3\}$}}} \frac{1}{k_3}\widehat{f} (k_1) \widehat g (k_2) \widehat h (k_3),\\
        \widehat{\cN(f, g, h)} (k)
            &:= \sum_{\substack{k_1 + k_2 + k_3 = k \\ (k_1 + k_2)(k_1 + k_3) (k_2 + k_3) \neq 0}}  \frac{e^{-3it(k_1 + k_2)(k_2 + k_3)(k_3 + k_1)}}{k_3}\widehat{f} (k_1) \widehat g (k_2) \widehat h (k_3).
    \end{align*}
Observe however that off-diagonal terms in the resonant sum cancel out, owing to preservation of reality. Indeed, since for real-valued $f: \TT \to \R$ the Fourier coefficients satisfy $\widehat{f}(k) = \overline{\widehat f(-k)}$, we see that 
    \begin{align*}
        \widehat{\cR_{\text{off}}(v, v, v)} (k)
            &= \overline{\widehat \cR_{\text{off}}(v, v, v)} (-k) \\
            &= \sum_{\substack{k_1 + k_2 + k_3 = -k \\ k_n + k_m = 0 \text{ for distinct $n, m \in \{1,2, 3\}$}}} \frac{1}{k_3}\overline{\widehat{v} (k_1) \widehat v (k_2) \widehat v (k_3)} \\
            &= \sum_{\substack{k_1 + k_2 + k_3 = k \\ k_n + k_m = 0 \text{ for distinct $n, m \in \{1,2, 3\}$}}} \frac{1}{-k_3}\overline{\widehat{v} (-k_1) \widehat v (-k_2) \widehat v (-k_3)} \\
            &= \sum_{\substack{k_1 + k_2 + k_3 = k \\ k_n + k_m = 0 \text{ for distinct $n, m \in \{1,2, 3\}$}}} \frac{1}{-k_3} \widehat{v} (k_1) \widehat v(k_2) \widehat v(k_3) = -  \widehat{\cR_{\text{off}}(v, v, v)} (k),
    \end{align*}
using the change of variables $k_n \mapsto -k_n$ in the third line. Thus we conclude 
    \begin{align}\label{eq:nonres}
        \partial_t w 
            = 6 \cN(v, v, v) + 12 \cR_{\text{diag}} (v, v, v),
    \end{align}
where 
    \[
        \widehat{\cR_{\text{diag}}(f, g, h)} (k)
            := \frac1k \widehat f(k) \overline{\widehat{g}(k)} \widehat h(k). 
    \]
Integrating \eqref{nonres} in time and inverting the normal form transformation \eqref{normalform}, we obtain the following equation for the difference between the non-linear and linear flows, 
    \begin{equation}\label{eq:integral}
        \begin{split}
        u(t) - e^{-t \partial_x^3} u_0 
            &= e^{-t \partial_x^3} \cB(u_0, u_0) - \cB(u(t), u(t)) + 6 \int_0^t e^{-(t - t')\partial_x^3} \cN(u(t'), u(t'), u(t')) \, dt'.\\
            &\qquad + 12 \int_0^t e^{-(t - t')\partial_x^3} \cR_{\mathrm{diag}}(u(t'), u(t'), u(t')) \, dt'
        \end{split}
    \end{equation}
Taking $H^\sigma$-norms on both sides, we have reduces the problem to controlling the $H^\sigma$-norm of the right-hand side using the $H^s$-norm of the initial data. 

\subsection{Bilinear and trilinear estimates}

The bilinear estimates are a straightforward application of Cauchy-Schwartz in frequency space, 

\begin{proposition}[Bilinear estimates]
    Let $s > -\tfrac12$ and $\sigma \leq s + 1$, then the following bilinear estimate holds
        \begin{equation}\label{eq:bilinear}
            ||\cB(f, g)||_{H^\sigma} 
                \lesssim ||f||_{H^s} ||g||_{H^s},
        \end{equation}
    whenever $f, g \in H^s (\TT)$ are mean zero, $\widehat f(0) = \widehat g(0) = 0$. 
\end{proposition}

\begin{proof}
    The Fourier modes of the bilinear form are 
        \[
            \widehat{\cB(f, g)} (k)
                = \sum_{\substack{k_1, k_2 \in \Z \\ k_1 + k_2 = k}} \frac{\widehat f(k_1) \widehat g(k_2)}{k_1 k_2}.
        \]
    By symmetry, it suffices to estimate the terms in the upper diagonal $|k_1| \leq |k_2|$, in which case $|k| \leq 2 |k_2|$. Using the triangle inequality, 
        \begin{align*}
            |\widehat{\cB(f, g)} (k)|
                &\leq \sum_{\substack{k_1, k_2 \in \Z \\ k_1 + k_2 = k}} \frac{|\widehat f(k_1)| \cdot |\widehat g(k_2)|}{|k_1| \cdot |k_2|}\\
                &\leq \sum_{\substack{k_1, k_2 \in \Z \\ k_1 + k_2 = k}} \Big( |k_1|^s |\widehat f(k_1)| \Big) \Big(  |k_2|^s |\widehat g(k_2)| \Big) \frac{1}{|k_1|^{s + 1} |k_2|^{s + 1}} \\
                &\leq \frac{2^\sigma}{|k|^{\sigma}} \sum_{\substack{k_1, k_2 \in \Z \\ k_1 + k_2 = k}} \Big( |k_1|^s |\widehat f(k_1)| \Big) \Big(  |k_2|^s |\widehat g(k_2)| \Big) \frac{1}{|k_1|^{s + 1}} .
        \end{align*}
    Multiplying by $|k|^\sigma$, summing in $\ell^2_k$, and applying Young's convolution inequality and Cauchy-Schwartz concludes the proof, 
        \[
            ||\cB(f, g)||_{H^\sigma}
                \lesssim \left|\left| \frac{|k_1|^s}{|k_1|^{s + 1}} \widehat f(k_1) \right|\right|_{\ell^1} \cdot || |k_2|^s \widehat g(k_2)||_{\ell^2} \lesssim ||f||_{H^s} ||g||_{H^s}.
        \]
\end{proof}

The resonant interactions can be handled by a simple energy estimate; 

\begin{proposition}[Resonance estimate]
    Let $s > -\tfrac12$ and suppose $s < \sigma < 3s + 1$, then 
        \begin{equation}\label{eq:resonantest}
            || \cR_{\operatorname{diag}}(f, g, h) ||_{H^\sigma}
                \leq ||f||_{H^s} ||g||_{H^s} ||h||_{H^s}. 
        \end{equation}
\end{proposition}

\begin{proof}
    In frequency space, 
        \begin{align*}
            || \langle k \rangle^{\sigma} \widehat{\cR_{\operatorname{diag}}(f, g, h)} (k) ||_{\ell^2_k}
                &\leq  \left|\left| \langle k \rangle^s \widehat f(k) \cdot \langle k \rangle^s \widehat g(k) \cdot \langle k \rangle^s \widehat h(k) \cdot \langle k\rangle^{\sigma - 3s - 1}\right|\right|_{\ell^2k}\\ 
                &\leq ||f||_{H^s} ||g||_{H^s} ||h||_{H^s},
        \end{align*}
    estimating two of the terms pointwise, and the weight $\langle k \rangle^{\sigma - 3s - 1} \leq 1$ for $s < \sigma < 3s + 1$.  
\end{proof}

The main technical piece to estimate is the trilinear non-resonant interactions. It is here where we need to rely on some linear smoothing effects rather than naive energy arguments. To estimate in the $L^\infty_t H^\sigma (I \times \TT)$-norm, Sobolev embedding in time allows us to control this term in $X^{\sigma, b} (I \times \TT)$-norm at the cost of losing derivatives $b > \tfrac12$. To gain a derivative back, we use the smoothing estimate \eqref{Xsbsmooth},
    \begin{align*}
        \left|\left| \int_0^t e^{- (t - t') \partial_x^3} \cN(u(t'), u(t'), u(t')) dt'\right|\right|_{L^\infty_t H^\sigma}
            &\lesssim      \left|\left| \int_0^t e^{- (t - t') \partial_x^3} \cN(u(t'), u(t'), u(t')) dt'\right|\right|_{X^{s, \frac12+}} \\
            &\lesssim ||\cN(u, u, u)||_{X^{s, -\frac12 +}}. 
    \end{align*}
It remains then to prove the following trilinear estimate, 

\begin{proposition}[Trilinear estimate]
    Let $s > -\tfrac12$, and suppose $\sigma < \min (s + 1, 3s + 1)$. Then 
        \begin{equation}\label{eq:trilinear}
            ||\cR(f, g, h) ||_{X^{\sigma, - \frac12 +}} 
                \lesssim ||f||_{X^{s, \frac12}} ||g||_{X^{s, \frac12}} ||h||_{X^{s, \frac12}}. 
        \end{equation}
\end{proposition}

\begin{proof}
    Let $\phi \in X^{-\sigma, \frac12-} (I \times \TT)$, by duality it suffices to prove 
        \[
            \left| \int_{\R} \int_\TT  \cR(f, g, h) \, \phi dx dt\right| \lesssim ||f||_{X^{s, \frac12}} ||g||_{X^{s, \frac12}} ||h||_{X^{s, \frac12}} ||\phi||_{X^{-\sigma, \frac12-}}. 
        \]
    Passing to frequency space, we write 
        \begin{align*}
            F (\tau, k)
                &= |\widehat f(\tau, k)| \cdot |k|^s \langle \tau - k^3 \rangle^{1/2}, \\
            G (\tau, k)
                &= |\widehat g(\tau, k)| \cdot |k|^s \langle \tau - k^3 \rangle^{1/2}, \\
            H (\tau, k) 
                &= |\widehat h(\tau, k)| \cdot |k|^s \langle \tau - k^3 \rangle^{1/2},\\
            \Phi (\tau, k)
                &= |\widehat \phi(\tau, k)| \cdot |k|^{-\sigma} \langle \tau - k^3 \rangle^{1/2 - \epsilon},
        \end{align*}
    Then we need to prove 
        \begin{align*}
            &\sum_{\substack{k_1 + k_2 + k_3 + k_4 = 0 \\ (k_2 + k_3)(k_1 + k_2) (k_1 + k_3) \neq 0}} \int_{\tau_1 + \tau_2 + \tau_3 + \tau_4 = 0} K(\vec\tau, \vec k) F(\tau_1, k_1) G(\tau_2, k_2) H(\tau_3, k_3) \Phi(\tau_4, k_4) \\ 
                &\qquad \lesssim ||F||_{L^2_\tau \ell^2_k}  ||G||_{L^2_\tau \ell^2_k}   ||H||_{L^2_\tau \ell^2_k}   ||\Phi||_{L^2_\tau \ell^2_k} 
        \end{align*}
    with kernel 
        \[
            K(\vec\tau, \vec k) := \frac{|k_1 k_2 k_3|^{-s} |k_4|^\sigma}{|k_1| \langle \tau_1 - k_1^3 \rangle^{1/2 -} \langle \tau_2 - k_2^3 \rangle^{1/2 -} \langle \tau_3 - k_3^3 \rangle^{1/2 -} \langle \tau_4 - k_4^3 \rangle^{1/2 -}} 
        \]
    We have the dispersive estimate 
        \begin{align}
            \left|\left| \widecheck{\left( \frac{|k|^{-\epsilon}}{\langle \tau - k^3 \rangle^{\frac12 + \epsilon}} \right) F } \right| \right|_{L^6_{t, x} (\R \times \TT)}
                \lesssim ||F||_{\ell^2_k}.
        \end{align}
    We also have that, for $\tau_1 + \tau_2 + \tau_3 + \tau_4 = k_1 + k_2 + k_3 + k_4 = 0$, 
        \begin{equation}
            (\tau_1 - k_1^3) + (\tau_2 - k_2^3) + (\tau_3 - k_3^3) + (\tau_4 - k_4^3) = 3(k_1 + k_2) (k_1 + k_2)(k_2 + k_3). 
        \end{equation}
    Therefore, by pigeonholing, 
        \begin{equation}
            \max_{i = 1, 2, 3, 4} \langle \tau_i - k_i^3 \rangle \gtrsim |k_1 + k_2| \cdot |k_1 + k_3| \cdot |k_2 + k_3| \gtrsim |k_i|.
        \end{equation}
    By careful analysis,     
        \[
            \frac{|k_1 k_2 k_3|^{-s} |k_4|^{-\sigma} }{|k_1| \cdot (|k_1 + k_2| \cdot |k_1 + k_3| \cdot |k_2 + k_3|)^{1/2 - 7\epsilon}} 
                \lesssim |k_1 k_2 k_3 k_4|^{-\epsilon}
        \]
    Collecting these inequalities gives the result. 
   \end{proof}






\subsection{Conclusion of the proof}

Continuing from the identity \eqref{integral} obtained from inverting the normal form transformation \eqref{normalform}, the difference between the non-linear flow \eqref{KdV} and the linear flow \eqref{Airy} satisfies 
    \[
        ||u(t) - e^{- t \partial_x^3} u_0||_{H^\sigma}
            \leq || \cB(u_0, u_0)||_{H^\sigma} + ||\cB(u(t), u(t))||_{H^\sigma} + 6 \left|\left| \int_0^t e^{-(t - t') \partial_x^3} \cM(u(t'), u(t'), u(t')) \, dt' \right|\right|_{H^\sigma}. 
    \]
The first two terms can be handled by a simple application of the bilinear estimates \eqref{bilinear}, 
    \[
        || \cB(u_0, u_0)||_{H^\sigma} + ||\cB(u(t), u(t))||_{H^\sigma}
            \lesssim \langle t \rangle^{O(1)} ||u_0||_{H^s}^2. 
    \]
The last term can be controlled by Sobolev embedding in $t$, the smoothing estimate \eqref{Xsbsmooth}, and the trilinear estimate \eqref{trilinear}, and local well-posedness \eqref{LWP}
    \begin{align*}
        \left|\left| \int_0^t e^{-(t - t') \partial_x^3} \cR(u(t'), u(t'), u(t')) \, dt' \right|\right|_{L^\infty_t H^\sigma_x (I)}
            &\lesssim  \left|\left| \int_0^t e^{-(t - t') \partial_x^3} \cR(u(t'), u(t'), u(t')) \, dt' \right|\right|_{X^{\sigma, \frac12+} (I)}\\
            &\lesssim |I|^{0+} ||\cR(u(t), u(t), u(t))||_{X^{\sigma, -\frac12+} (I)} \\
            &\lesssim  |I|^{0+} ||u||_{X^{s, \frac12} (I)}^3\\
            &\lesssim |I|^{0+} ||u_0||_{H^s}^3.
    \end{align*}
Iterating the polynomial bounds from the local well-posedness theory, we conclude the result. 

\section{Convergence of numerical schemes}\label{sec:numerics}
\begin{flushright}
    \begin{quote}
        {\hfill\it Kids these days don't do any experiments.}\\
        \hfill ---Maciej Zworski
    \end{quote}    
\end{flushright}

The non-linear smoothing effect gives credence to numerical simulations of \eqref{KdV} which seem to demonstrate the Talbot effect, see for example \cite{ChenOlver2013}. However, typically one needs to start with smooth data to be confident that the numerics are accurately telling the story, as most schemes rely on \textit{a priori} bounds at high regularity to guarantee convergence, ruling out convergence for data such as square waves. Nevertheless, in the same vein as the local well-posedness theory, we can exploit non-linear smoothing to construct numerical schemes well-adapted to \eqref{KdV} and dispersive estimates to handle convergence for low regularity data. 

We will present the resonance-based scheme of \cite{HofmanovaSchratz2017} and the proof of convergence for $H^{0+}$-data from \cite{RoussetSchratz2022}. Discretisation time by introducing a \textit{time step} $\timestep \ll 1$ and setting $t_n := n \timestep$, we can use the Duhamel formula for the variable $v = e^{t \partial_x^3} u$ to write the difference between the flow at time $t = t_n$ and $t = t_{n + 1}$ as 
    \begin{equation}
        v(t_{n + 1}) 
            = v(t_n) - 3 \int_0^{\timestep} e^{(\timestep + s) \partial_x^3} \partial_x \left( e^{-(t_n + s) \partial_x^3} v(t_n + s) \right)^2 \, ds.
    \end{equation}
Approximating $v(t_n + s)$ by $v(t_n)$, we expand the non-linear forcing term and use the identity \eqref{identity},
    \begin{align*}
        -3\int_0^\timestep e^{(t_n + s) \partial_x^3} \partial_x \left( e^{-(t_n + s) \partial_x^3} v(t_n) \right)^2 \, ds
            &= - 3 \sum_{k_1, k_2 \in \Z} \int_0^\timestep e^{-3i (t_n + s)(k_1 + k_2) k_1 k_2} i(k_1 + k_2) \\
            &\qquad\qquad \qquad \times \widehat v(t_n, k_1) \widehat v(t_n, k_2) e^{i (k_1 + k_2)x} ds \\
            &=  \sum_{k_1, k_2 \in \Z} \left( e^{-3i (k_1 + k_2)k_1k_2 t_{n + 1}} - e^{-3i (k_1 + k_2)k_1k_2 t_{n}} \right) \\ 
            &\qquad\qquad\qquad\times\frac{\widehat v(t_n, k_1) \widehat v(t_n, k_2)}{k_1 k_2} e^{i(k_1 + k_2)x} \\
            &= e^{t_{n + 1}\partial_x^3} \left( e^{-t_{n + 1} \partial_x^3} \partial_x^{-1} v(t_n) \right)^2 - e^{t_{n + 1}\partial_x^3} \left( e^{-t_{n} \partial_x^3} \partial_x^{-1} v(t_n) \right)^2.
    \end{align*}
Returning to the original variable $u$, the Duhamel formula suggests the following numerical scheme 
    \begin{equation}\label{eq:numerical}
        \mathtt{u_{n  +1}} 
            := e^{- \timestep \partial_x^3} \mathtt{u_n} - \left(e^{-\timestep \partial_x^3} \partial_x^{-1} \mathtt{u_n}\right)^2 - e^{-\timestep \partial_x^3} \left(\partial_x^{-1} \mathtt{u_n}\right)^2.
    \end{equation}

\begin{remark}
    The numerical scheme is essentially a time-discretised version of the normal form transformation \eqref{integral} which omits the cubic terms. Interestingly, the authors of \cite{HofmanovaSchratz2017} appeared to be unaware of this at the time. 
\end{remark}

Following \cite{RoussetSchratz2022}, we introduce a filtered version of the numerical scheme which models the Galerkin approximate equation. Truncating to frequencies below $N = \timestep^{-1/3}$, we define 
\begin{equation}\label{eq:numerical2}
    \mathtt{u_{n  +1}} 
        := e^{- \timestep \partial_x^3} P_{\leq N} \mathtt{u^{(n)}} - P_{\leq N} \left(e^{-\timestep \partial_x^3} \partial_x^{-1} P_{\leq N} \mathtt{u_n}\right)^2 - e^{-\timestep \partial_x^3} \left(\partial_x^{-1} P_{\leq N} \mathtt{u_n}\right)^2.
\end{equation}

    \begin{theorem}[Convergence of numerical scheme]
        Let $s > 0$ and fix initial data $u_0 \in H^s (\TT)$ with zero mean. Given a time $T > 0$ and a time step $\timestep \ll 1$, the error between the non-linear flow \eqref{KdV} and the numerical scheme \eqref{numerical2} satisfies the bound 
            \begin{equation}
                ||\mathtt{u_n} - u(t_n)||_{L^2} \lesssim_T \max \left((\timestep)^{\frac{s}{3}}, \timestep \right),
            \end{equation}
        for $t_n \leq T$. 
    \end{theorem}

We will deviate from the original proof from \cite{RoussetSchratz2022}, which is more robust in the sense that it handles several different schemes, but more technical as it relies on time-discrete analogues of the Fourier restriction norm estimates developed in \cite{CollianderEtAl2003}. Our presentation instead draws from the observations of \cite{BabinEtAl2011} concerning the normal form transformation. 

\subsection{Galerkin method}

The filtered numerical scheme for \eqref{KdV} can be viewed as the unfiltered numerical scheme for the Galerkin approximation of the equation. Fix a frequency $N > 0$, then we denote the approximate solution by $u^{(N)} : \R \times \TT \to \R$, which solves
    \begin{equation}\label{eq:galerkin}
        \begin{split}
        \partial_t u^{(N)} + \partial_x^3 u^{(N)} + 3 P_{\leq N} \partial_x \left(  {(u^{(N)})}_{\leq N}\right)^2 
            &= 0, \\
        {u^{(N)}}_{|t = 0}
            &= (u_0)_{\leq N}. 
        \end{split}
    \end{equation}
Observe that in frequency space, this reduces to a $2N$-dimensional ODE for the Fourier modes up to frequency $|k| \leq N$. The solutions to \eqref{galerkin} may be regarded as low frequency approximate solutions to the full non-linear flow \eqref{KdV}. Indeed, this follows from the following folklore result; 

\begin{proposition}[Convergence of Galerkin approximates]
    Let $s \geq 0$, and fix initial data $u_0 \in H^s (\TT)$ with zero mean. Given a time $T > 0$, the error between the full \eqref{KdV} flow and the Galerkin approximate flow \eqref{galerkin} satisfies the bound 
        \begin{equation}
            ||u - u^{(N)} ||_{L^2_x} 
                \lesssim_T N^{-s},
        \end{equation}
    for all $t \in [0, T]$. 
\end{proposition}

In a nutshell, we can define the 
    \begin{equation}\label{eq:normalformG}
        e^{-t \partial_x^3} w^{(N)}
            := u^{(N)} + \cB^{(N)} (u^{(N)}, u^{(N)}),
    \end{equation}
where, in the same spirit as our derivation of the normal form transformation \eqref{normalform} for \eqref{KdV}, we should make the following judicious choice of bilinear form,
    \[
        \cB^{(N)} (f, g) 
            := P_{\leq N} \left( \partial_x^{-1} f_{\leq N}\cdot \partial_x^{-1} g_{\leq N} \right). 
    \]
It follows that the approximate solution satisfies the functional equation
    \begin{equation}\label{eq:integral2}
        \begin{split}
        u^{(N)}(t) - e^{-t \partial_x^3} (u_0)_{\leq N}
            &= e^{-t \partial_x^3} \cB^{(N)} (u_0, u_0) - \cB^{(N)} (u^{(N)} (t), u^{(N)}(t)) \\
            &\qquad+ 6 \int_0^t e^{-(t - t') \partial_x^3} \cT^{(N)} (u^{(N)}(t'), u^{(N)}(t'), u^{(N)}(t'))\, dt' ,
        \end{split}
    \end{equation}
where the trilinear form in the non-linear forcing is 
    \[
        \cT^{(N)}( f, g, h) 
            := P_{\leq N} \left( f_{\leq N} \cdot g_{\leq N} \cdot \partial_x^{-1} h_{\leq N} \right) . 
    \]

To prove the error bound, we consider the equation for the difference of the flows $\phi := u - u^{(N)}$. Subtracting \eqref{integral2} from \eqref{integral}, we write 
    \begin{equation}
        \phi (t)
            = \text{I} + \text{II} + \text{III}, 
    \end{equation}
where we distinguish the linear, bilinear, and trilinear terms, 
    \begin{align*}
        \text{I} 
            &:= e^{-t \partial_x^3} \phi_0 \\
        \text{II}
            &:= e^{-t \partial_x^3} \Big( \cB(u_0, u_0) - \cB^{(N)} (u_0, u_0) \Big) \\
            &\qquad - \Big(  \cB (u, u)  -  \cB^{(N)} (u^{(N)}, u^{(N)}) \Big)\\
            &=  e^{-t \partial_x^3} \Big( P_{> N} \cB(u_0, u_0) + P_{\leq N}\left( \cB((u_0)_{> N}, (u_0)_{> N}) + 2 \cB((u_0)_{> N}, (u_0)_{\leq N})\right) + \cB^{(N)} (\phi_0, (u_0) + (u_0)^{(N)})\Big)\\
            &\qquad - \Big( P_{> N} \cB(u, u) + P_{\leq N}\left( \cB(u_{> N}, u_{> N}) + 2 \cB(u_{> N}, u_{\leq N})\right) + \cB^{(N)} (\phi, u + u^{(N)}) \Big)\\
            &=: \text{II}_{\text{high}} + \text{II}_{\text{low}}\phi\\
        \text{III}
            &:=  6\int_0^t e^{-(t - t') \partial_x^3} \Big(\cT (u^{(N)}(t'), u^{(N)}(t'), u^{(N)}(t')) -  \cT^{(N)} (u(t'), u(t'), u^(t')) \Big) dt'.
    \end{align*}
By the $H^s$-well-posedness theory, high frequencies decay a l\'a \eqref{high}, so we can estimate these terms via Bernstein's inequality. For example, the linear term obeys
    \[
        ||\mathrm{I} ||_{L^2_x} 
            \lesssim || (u_0)_{>N}||_{L^2_x} \lesssim N^{-s} ||u_0||_{H^s_x}. 
    \]
Clearly the bilinear estimate \eqref{bilinear} for $\cB(f, g)$ also holds for $\cB^{(N)} (f, g)$. A similar application of Bernstein gives the bound for the terms with high frequencies, while for low frequencies 
    \[
        ||\mathrm{II}_{\text{high}}||_{L^2_x} 
            \lesssim N^{-s} \langle T \rangle^{O(1)} ||u_0||_{H^s_x}^2,
    \]
while for 
    \[
        ||\mathrm{II}_{\text{low}}||_{L^2_x}
            \lesssim N^{-s} ||P_N (u - u^(N))
    \]


\subsection{Discrete Fourier restriction spaces}

The convergence of the numerical scheme to the Galerkin approximate solutions follows the same lines as previous proof. The only modification will be the trilinear estimates \eqref{trilinear}, for which we need a time-discrete analogue. Denoting by $\vec u := (\mathtt{u^{(n)}})_n$ a sequence of functions on $\TT$, which heuristically we view as snapshots of our flow at various timesteps, 
    \[
        u(t_n, x) \approx \mathtt{u^{(n)}} (x), 
    \]
define the spacetime Fourier transform with respect to time step $\timestep$ by 
    \[
        \widetilde{\vec u} (\tau, k) 
            := \timestep \sum_{n \in \Z} \widehat{\mathtt{u^{(n)}}} (k) e^{i n \timestep \tau}.  
    \]
Then the time-discretised Fourier restriction norm is given by 
    \[
        ||\vec u||_{X^{s, b}_\timestep} 
            := || \langle k \rangle^s \langle d_\tau ( \tau + k^3) \rangle^b \widetilde{\vec u} (\tau, k)  ||_{L^2_\tau \ell^2_k}
    \]
where 
    \[
        d_{\timestep} (\sigma) 
            := \frac{e^{i \timestep \sigma} - 1}{\timestep}. 
    \]
Observe that in the physical variables this corresponds to the finite-difference 
    \[
        D_{\timestep} (\vec u) 
            := \left\{ \frac{u_{n - 1} - u_n}{\tau} \right\}_n. 
    \]
The weight $d_\timestep (\tau + k^3)$ vanishes if $\timestep(\tau + k^3) = 2\pi m$ for $m \in \Z$. For data localised to frequencies $|k| \lesssim \timestep^{-\frac13}$. this will behave like the continuous case with only a cancellation at $\tau + k^3 = 0$. For larger frequencies there are additional cancelations which cause problems with the product estimates. 

\begin{lemma}[Discretised trilinear estimate]
    Let $s \geq 0$, then for every $\vec u, \vec v \in Y^s_{\timestep} (\TT)$, 
        \begin{equation}
            \left|\left|\partial_x P_{N} \left(P_{N} \vec u P_{N} \vec v\right) \right|\right|_{Y^s_\timestep} 
                \lesssim ||\vec u||_{X^{s, \frac12}_\timestep} || \vec v||_{X^{s, \frac13}_\timestep} +  ||\vec v||_{X^{s, \frac12}_\timestep} || \vec u||_{X^{s, \frac13}_\timestep},
        \end{equation}
    where $N \sim \timestep^{-\frac13}$ is the semi-classical timescale. 
\end{lemma}

\begin{lemma}
    Let $s > -\tfrac12$ and suppose $\sigma < \min(s + 1, 3s + 1)$. Then 
        \begin{equation}
            ||\cN (P_{\leq N} f, P_{\leq N} g, P_{\leq N} h)||_{X^{\sigma, -\frac12+}_\timestep} 
                \lesssim ||f||_{X^{\sigma, \frac12}_\timestep}  ||g||_{X^{\sigma, \frac12}_\timestep}  ||h||_{X^{\sigma, \frac12}_\timestep} 
        \end{equation}
    for $N \sim \timestep^{-\frac13}$. 
\end{lemma}

\subsection{Difference equations}

By the Duhamel formula, 


    \begin{align}
        u^{(N)} (t_{n + 1}) 
            &=  e^{-\timestep \partial_x^3}  P_{\leq N} u^{(N)}(t_n) + e^{-\timestep \partial_x^3} \cB^{(N)} (u^{(N)}(t_n), u^{(N)} (t_n)) - \cB^{(N)} (u^{(N)} (t_{n + 1}), u^{(N)}(t_{n + 1})) \\
            &\qquad+ 6 \int_0^{\timestep} e^{-(\timestep - t') \partial_x^3} \cT^{(N)} (u^{(N)}(t'), u^{(N)}(t'), u^{(N)}(t'))\, dt' ,  \\
        \mathtt{u_{n  +1}} 
            &:= e^{- \timestep \partial_x^3} P_{\leq N} \mathtt{u^{(n)}} + e^{-\timestep \partial_x^3} \cB^{(N)} (e^{-\timestep \partial_x^3} \mathtt{u_n}, e^{-\timestep \partial_x^3} \mathtt{u_n}) - \cB^{(N)} (\mathtt{u_n},\mathtt{u_n})
    \end{align}



\bibliographystyle{alpha}
\bibliography{external/biblio}

\end{document}
