

In view of the Bogomoln'yi identity \eqref{B} and the Hamiltonian structure of the equation (see Appendix \ref{appendix:hasimoto}), the equation can be put in self-dual form,
    \begin{equation}
        \partial_t u 
            = \bfJ \bfD_z \partial_{\overline z} u,\label{eq:selfdual}
    \end{equation}
where 
    \begin{align*}
        \partial_{\overline z} u  
            &:= \partial_1 u - \bfJ \partial_2 u ,\\
        \bfD_z v 
            &:= \bfD_1 v + \bfJ \bfD_2 v. 
    \end{align*}
Then, applying the covariant Cauchy-Riemann operator to the self-dual formuation \eqref{selfdual}, we obtain the generalised Hasimoto-transformed Schr\"odinger maps equation, which is an elliptic-dispersive system,
    \begin{equation}
        \begin{split}
        \bfD_t \epsilon'
            &= \bfJ \bfD_{\overline z} \bfD_z \epsilon',\\
        \epsilon'
            &= \partial_{\overline z} u . 
        \end{split}
    \end{equation}
    

    \begin{equation}
        (i \partial_t - \widetilde{\mathsf H}_Q) \psi = \text{non-linear}
    \end{equation}
where 
    \begin{align*}
        \widetilde{\mathsf H}_Q 
            &:= - \Delta + \widetilde{V}_Q,\\
        \widetilde{V}_Q (r)
            &:= \frac{2m(1 - h_3(r))}{r^2} = \frac{4m}{r^2 (r^2 + 1)}
    \end{align*}