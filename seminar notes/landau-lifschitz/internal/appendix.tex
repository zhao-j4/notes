\subsection{Generalised Hasimoto transform}\label{appendix:hasimoto}

It is convenient to write the Schr\"odinger maps equation \eqref{schrodinger} in geometric formulation, 
    \[
        \partial_t u 
            = \bfJ (\bfD_1 \partial_1 u + \bfD_2 \partial_2 u). 
    \]  
Then 
    \begin{align*}
        \bfJ \bfD_z \partial_{\overline z} u 
            &= \bfJ (\bfD_1 + \bfJ \bfD_2) (\partial_1 u - \bfJ \partial_2 u) \\
            &= \bfJ (\bfD_1 \partial_1 u - \bfJ \bfJ \bfD_2 \partial_2 u) + \bfJ\bfJ (\bfD_1 \partial_2 u - \bfD_2 \partial_1u)\\
            &= \bfJ (\bfD_1 \partial_1 u + \bfD_2 \partial_2 u).
    \end{align*}


\subsection{Bogomoln'yi identity}\label{appendix:bogo}

We compute
    \begin{align*}
        |\partial_1 u - u \times \partial_2 u |^2 
            &= (\partial_1 u - u \times \partial_2 u) \cdot (\partial_1 u - u \times \partial_2 u)\\
            &= |\partial_1 u|^2 + |u \times \partial_2 u|^2 - 2 (\partial_1 u) \cdot (u \times \partial_2 u) .
    \end{align*}
In the last line, the first two terms are exactly the Dirichlet energy density, while the last term is the pull-back of the volume form on $\SS^2$ to $\R^2$ under $u$. Indeed, since the almost complex structure acts isometrically on the tangent space,    
    \[
        |\partial_1 u|^2 + |u \times \partial_2 u|^2 = |\partial_1 u|^2 + |\partial_2 u|^2. 
    \]
To see the pull-back of the volume form, recall that 
    \[
        d\operatorname{Vol}_{\SS^2}
            := u^1 du^2 \wedge du^3 - u^2 du^1 \wedge du^3 + u^3 du^1 \wedge du^2.
    \]
Then, pulling back by $u : \R^2 \to \SS^2$, we obtain 
    \begin{align*}
        u^* d \operatorname{Vol}_{\SS^2}
            &= u^1 (\partial_1 u^2 dx^1 + \partial_2 u^2 dx^2) \wedge  (\partial_1 u^3 dx^1 + \partial_2 u^3 dx^2) \\
            &\qquad - u^2 (\partial_1 u^1 dx^1 + \partial_2 u^1 dx^2)  \wedge (\partial_1 u^3 dx^1 + \partial_2 u^3 dx^2)  \\
            &\qquad + u^3 (\partial_1 u^1 dx^1 + \partial_2 u^1 dx^2)  \wedge  (\partial_1 u^2 dx^1 + \partial_2 u^2 dx^2) \\
            &= u \cdot (\partial_1 u \times \partial_2 u) \, dx^1 \wedge dx^2\\
            &=  (\partial_1 u) \cdot (u \times \partial_2 u) \, dx^1 \wedge dx^2, 
    \end{align*}
where the last line we have used the scalar triple product identity. By the degree theorem, 
    \begin{align*}
        \int_{\R^2} (\partial_1 u) \cdot (u \times \partial_2 u) \, dx
            &= \int_{\R^2} u^* d\operatorname{Vol} = \deg(u) \int_{\SS^2} d\operatorname{Vol} = 4\pi  \deg(u)
    \end{align*}
This completes the proof of the Bogomoln'yi identity \eqref{B}. 


\subsection{Harmonic maps from \eqref{LL}}

Here we show that 
    \[
        \partial_t u 
            = \alpha( \Delta u + |\nabla u|^2 u) + \beta (u \times \Delta u)
    \]
is equivalent to 
    \[
        \partial_t u 
            = -\alpha (u \times (u \times \Delta u)) + \beta (u \times \Delta u ).
    \]
Consider the vector cross product formula 
    \[
        a \times (b \times c) = (a \cdot c) b - (a \cdot b) c. 
    \]
Then 
    \[
        u \times (u \times \Delta u) = (u \cdot \Delta u) u - |u|^2 \Delta u. 
    \]
Observe that 
    \[
        u \cdot \nabla u = 0. 
    \]
Thus 
    \[
        u \cdot \Delta u = - |\nabla u|^2. 
    \]