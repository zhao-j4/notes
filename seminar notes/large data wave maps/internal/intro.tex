Let $(\R^{1 + 2}, m)$ be the $(1 + 2)$-dimensional Minkowski space-time, and suppose $(\MM, g)$ is a compact Riemannian manifold. By Nash's theorem, we can view the target manifold extrinsically via an isometric embedding $\MM \hookrightarrow \R^N$, denoting the second fundamental form by $\bfS  : T\MM \times T\MM \to T \MM^\perp$. We say that $\phi : \R^{1 + 2} \to \MM$ solves the \emph{wave maps equation} if 
\begin{equation}
	\begin{split}
		\Box \phi^a 
			&= - \bfS ^a_{bc} (\phi)  \partial^\alpha \phi^b \partial_\alpha \phi^c, \\
		\phi_{|t = 0}
			&= \phi_0 ,\\
		\partial_t \phi_{|t = 0}
			&= \phi_1,	
	\end{split}
	\label{eq:wave}
	\tag{WM}
\end{equation}
for initial data $(\phi_0, \phi_1)$ satisfying the constraints $\phi_0 (x) \in \MM$ and $\phi_1 (x) \in T_{\phi_0 (x)} \MM$. Formally, wave maps are critical points of the Lagrangian
	\[ \cL [\phi] := \int_{\R^{1 + 2}} \langle\partial^\alpha \phi, \partial_\alpha \phi \rangle_g \, dt dx \]
of which the wave maps equation \eqref{wave} is the Euler-Lagrange equation. There is a corresponding stress-energy tensor 
	\[ 
		T_{\alpha\beta} [\phi] := \langle \partial_\alpha \phi, \partial_\beta \phi \rangle_g  - \frac12 m_{\alpha \beta} \langle \partial^\gamma \phi, \partial_\gamma \phi \rangle_g
	 \]
which is divergence-free
	\[ \partial^\alpha T_{\alpha \beta} [\phi] = 0. \]
Then, contracting the stress-energy tensor with the vector field $\partial_t$, applying the divergence-free condition and Stokes' theorem on the space-time slab $I \times \R^2$ furnishes conservation of the \emph{energy} for solutions to the wave maps equation, 
	\[ \cE[\phi (t)] := ||\vec \phi||_{\dot H^1 \times L^2}^2 (t) = \int_{\R^2} |\partial_t \phi|^2 + |\nabla_x \phi|^2 \, dx , \]
where we have denoted $\phi[t]= (\phi, \partial_t \phi) (t)$. In view of Noether's theorem, conservation of energy corresponds to the time-translation symmetry of the Lagrangian $\cL [\phi]$. Note that the wave maps equation and its Lagrangian are invariant with respect to the scaling 
	\[ \phi(t, x) \mapsto \phi(\lambda t, \lambda x). \]
The Dirichlet energy is also invariant with respect to this scaling in $(1 + 2)$-dimensions and coincides precisely with the \emph{energy space} $\phi [t] \in \dot H^1 \times L^2$, thus, we refer to the wave maps equation \eqref{wave} on $\R^{1 + 2}$ as \emph{energy-critical}. 

\subsection{Main results}

We consider the initial data problem for the wave maps equations \eqref{wave} on $\R^{1 + 2}$ with finite energy data $\vec \phi_0 \in \dot H^1_x \times L^2_x$. Global well-posedness where the target manifold is a sphere $\MM = \mathbb S^k$ was established in \cite{Tao2001} and for general targets in \cite{Tataru2005}. In this note we turn towards addressing the following questions for large initial data:
	\begin{itemize}
		\item global well-posedness, 
		\item scattering.
	\end{itemize}
Using the symmetries of the equation and the finite speed of propagation, we can reduce the study of the wave map to the forward light cone $C$. For blow-up, we use time-reversibility of the equation and finite speed of propagation. For scattering, we choose a ball $B$ large so that energy is small outside of this ball, so the small data theory applies. Hence it remains to study the influence cone of $B$. 


\begin{theorem}[Bubbling theorem]
	Let $\phi$ be a finite energy solution to the wave maps equation \eqref{wave} which either admits time-like energy concentration at the tip of the light cone $(t, x) = (0, 0)$ (respectively at infinity $t = \infty$), 
		\[ \limsup_{t} \cE_{C_\gamma \cap S_t} [\phi] > 0,  \]
	where we write the limit $t \searrow 0$ (resp. $t \nearrow \infty$). Then there exists a sequence of concentration points $(t_n, x_n) \in C$ such that $(t_n, x_n) \to (0, 0)$ (resp. $t_n \nearrow \infty$), and scales $r_n > 0$ with the following properties:
		\begin{enumerate}
			\item time-like concentration, 
				\[ 
					\limsup_{n \to \infty} \frac{x_n}{t_n} = v,
				\]
				for some velocity $v \in \R^2$ with $|v| < 1$, 
			
			\item below self-similar scale, 
				\[ 
					\limsup_{n \to \infty} \frac{r_n}{t_n} = 0,  
				\]	
				
			\item convergence to a soliton, 
				\[
					\lim_{n \to \infty} \phi(t_n + r_n t, x_n + r_n x) = L_v Q (t, x)
				\]	
				strongly in $H^1_{\loc} ([-\tfrac12, \tfrac12] \times \R^2)$ to a Lorentz transformation with velocity $v$ of a non-trivial harmonic map $Q : \R^2 \to \MM$ which contains some of the energy concentration, 
					\[ 0 < ||Q||_{\dot H^1} \leq \lim_{t} \cE_{S_t} [\phi].   \]
		\end{enumerate}	
\end{theorem}

\begin{remark}
	The soliton resolution conjecture asks whether the nature of blow-up or non-scattering can be completely characterised by a superposition of solitons. The question of blow-up was partially answered by the thesis of Grinis \cite{Grinis2016}, using the bubble tree argument. In this case, the energy concentration can be completely decomposed into the sum of energies of solitons,
		\[
			\lim_{t} \cE_{C_\gamma \cap S_t} [\phi] = \sum_{Q \text{ bubble}} ||Q||_{\dot H^1}, 
		\]
	 sometimes known as the \emph{energy identity}. The remarkable heart of the analysis is showing that no energy is lost between the scales at which the solitons appear.  
\end{remark}

\begin{theorem}[Threshold theorem]
	The wave maps equation is globally well-posed for all initial data below the energy threshold and the corresponding solutions scatter in the following sense:
	\begin{enumerate}
		\item (regular data) For regular data $\vec \phi_0$, then there exists a unique global regular solution which has Lipschitz dependence on the initial data locally in time in the $\dot H^1 \times L^2$ topology. 
		
		\item (rough data) The flow map admits an extension to rough data. 
		
		\item (weak Lipschitz dependence) The flow map is globally Lipschitz in the $\dot H^{\sigma}$ topology for $\sigma < 1$ close to $1$. 
		
		\item (scattering) The $\sfS$-norm is finite.
	\end{enumerate}	
\end{theorem}

\begin{theorem}[Dichotomy theorem]
	The wave maps equation \eqref{wave} is locally well-posed for arbitrary finite energy data. Further, one of the following two properties must hold for the forward maximal solution:
	\begin{enumerate}
		\item the solution is global, scatters at infinity,
		\item bubbling off a soliton. 
	\end{enumerate}
\end{theorem}

