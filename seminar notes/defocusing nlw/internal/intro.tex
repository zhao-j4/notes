Given initial data in the energy-space $(\phi_0, \phi_1) \in \dot H^1_x (\R^d) \times L^2_x (\R^d)$, we are interested in the evolution under the energy-critical defocusing non-linear wave equation, 
	\begin{equation}\label{eq:NLW}\tag{NLW}
		\begin{split}
		\Box \phi
			&= + |\phi|^{\frac{4}{d - 2}} \phi,\\
		\phi_{|t = 0}
			&= \phi_0, \\
		\partial_t \phi_{|t = 0}
			&= \phi_1.
		\end{split}			
	\end{equation}
This equation is energy-critical in the sense that the conserved energy, 
	\[
		\cE [\phi[t]] := \int_{\R^d} \frac12 |\partial_t \phi|^2 + \frac12 |\nabla \phi|^2 + \frac{d - 2}{2d} |\phi|^{\frac{2d}{d - 2}} \, dx ,
	\]
is invariant under the associated scaling symmetry to the equation,
	\[
		\phi(t, x) \mapsto \lambda^{-\frac{d-2}{2}} \phi(t/\lambda, x/\lambda).
	\]
We are interested in establishing global well-posedness and scattering, that is, the existence of a strong solution $\phi \in C^0_{t, \loc} \dot H^1_x \cap \dot C^1_{t, \loc} L^2_x (\R \times \R^d)$ which depends continuously on the initial data and is bounded in an appropriate Strichartz norm $||\phi||_{\sfS_{t, x} (\R \times \R^d)} < + \infty$. For non-linear wave equations, the main obstruction is the existence of non-trivial steady-states $Q$, which solve the ground state equation
	\begin{equation}
		\Delta Q = + |Q|^{\frac{4}{d - 2}} Q.
		\label{eq:GS}
		\tag{GS}
	\end{equation}
In general, defocusing equations do not admit ground states. For the non-linear wave equation, one can see this by multiplying the equation by $Q$ and integrating by parts. We quickly see from the discrepancy in sign that the only ground state solution is the trivial solution $Q \equiv 0$. The \emph{defocusing conjecture} states that, in view of the lack of ground states, defocusing equations should be globally well-posed and scatter. This conjecture holds \eqref{NLW}, see \cite[Chapter 5.1]{Tao2006} and references therein,


\begin{theorem}[Defocusing theorem]\label{thm:defocusing}
	For each finite energy initial data $\phi[0] \in \dot H^1_x (\R^d) \times L^2_x (\R^d)$, there exists a unique global solution $\phi \in C^0_{t} \dot H^1_x \cap \dot C^1_{t} L^2_x (\R \times \R^d)$ to the defocusing \eqref{NLW} depending continuously on the initial data and scatters. 
\end{theorem}

The subject of this note will be to establish the conjecture using the energy-dispersion + bubbling argument of Sterbenz and Tataru \cite{SterbenzTataru2010a, SterbenzTataru2010}. 


