
We derive the key monotonicity formula associated to the scaling vector field. We follow a computation similar to that for Maxwell-Klein-Gordon \cite[Section 5]{OhTataru2016}. In view of the degeneracy of the light cone, we make a translation in time $t \mapsto t + \epsilon$, defining 
	\begin{align*}
		X_\epsilon
			&= \frac{1}{\rho_\epsilon^k} ((t + \epsilon) \partial_t + r \partial_r), \\
		w_\epsilon
			&= \frac{d - k - 1}{2 \rho^k_\epsilon},	
	\end{align*}
where $\rho_\epsilon := \sqrt{(t + \epsilon)^2 - r^2}$. In view of scaling, $k = d - 1 - \tfrac{4}{p - 1}$; we are interested in the NLW energy critical exponents, so $k = 1$. Observe that $X_0 = \tfrac1\rho S = \partial_\rho$, where $S$ is the scaling vector field. To simplify the discussion, we first restrict to the case $\epsilon = 0$. Translating in time, we will conclude


\begin{theorem}[Monotonicity formula]
	Let $\phi$ be a smooth solution to \eqref{NLW} on an open subset $\cO \subseteq C_{(0, \infty)}$ of the forward light cone. Then the $1$-current defined by 
		\begin{equation}
			^{(X_\epsilon)}P_\mu [\phi]
				= {^{(X_\epsilon)}} J_\mu [\phi] + {^{(w_\epsilon)}} J_\mu [\phi] + {^{(\cH_\epsilon)}} J_\mu [\phi] +  {^{(\cN_\epsilon)}} J_\mu [\phi]
		\end{equation}	
	satisfies the divergence identity
		\begin{equation}
			\partial^\mu \left( {^{(X_\epsilon)}} P_\mu [\phi] \right) = \frac{1}{\rho_\epsilon} \left| \left( \partial_{\rho_\epsilon} + \frac1{\rho_\epsilon} \right)\phi \right|^2.
		\end{equation}	
	In null coordinates,
		\begin{equation}
		\begin{split}
			{^{(X_\epsilon)}} P_L [\phi]
				&= \frac12 \left( \frac{v_\epsilon}{u_\epsilon} \right)^{1/2} \left| r^{-\frac{d - 2}{2}} L \left( r^{\frac{d - 2}{2}} \phi \right)\right|^2 + \frac12 \left( \frac{u_\epsilon}{v_\epsilon} \right)^\frac12 \left( |\slashed \nabla \phi|^2 + \frac{(d - 2)^2}{4} \frac1{r^2} |\phi|^2 + \frac{2\kappa}{p + 1} |\phi|^{p + 1} \right)\\
			{^{(X_\epsilon)}} P_{\underline L} [\phi]
				&= \frac12 \left( \frac{u_\epsilon}{v_\epsilon} \right)^{1/2} \left| r^{-\frac{d - 2}{2}} \underline L \left( r^{\frac{d - 2}{2}} \phi \right)\right|^2 + \frac12 \left( \frac{v_\epsilon}{u_\epsilon} \right)^\frac12 \left(  |\slashed \nabla \phi|^2 + \frac{(d - 2)^2}{4} \frac1{r^2} |\phi|^2 + \frac{2\kappa}{p + 1} |\phi|^{p + 1} \right)	.
		\end{split}	
		\end{equation}	
\end{theorem}

\subsubsection{$0$-current}

We first work in hyperbolic coordinates. Writing the metric in these coordinates, the Lie derivative of the metric along the coordinate vector field $\partial_\rho$ is given by $\cL_{\partial_\rho} \bfm = 2 \rho (\d y^2 + \sinh^2 (y) g_{\SS^{d - 1}})$. Thus, we can write the deformation tensor along $X_0$ as 
	\begin{align*}
		\frac12 {^{(X_0)}} \pi
			&= \rho (\d y^2 + \sinh^2 (y) g_{\SS^{d - 1}} \\
			&= \frac1\rho \d \rho^2 + \frac1\rho \bfm,	
	\end{align*}
with metric dual
	\begin{align*}
		\frac12 {^{(X_0)}} \pi^\# 
			&= \rho (\partial_y \odot \partial_y + \sinh^2 (y) g_{\SS^{d - 1}}^{-1} ) \\
			&= \frac1\rho \partial_\rho \odot \partial_\rho + \frac1\rho \bfm^{-1}. 	
	\end{align*}
The $0$-current is given by 
	\begin{align*}
		{^{(X_0)}} K[\phi]
			&= T[\phi] \left( \frac12 {^{(X_0)}} \pi^\#\right) \\
			&= \frac1\rho \left( T_{\rho \rho} + \operatorname{tr} T \right) \\
			&= \frac1\rho \left( |\partial_\rho \phi|^2 + \frac{2 - d}{2} |\nabla \phi|^2 - \kappa \frac{d - 2}{2} |\phi|^{\frac{2d}{d - 2}} \right).
	\end{align*}
For $d \geq 3$, the second term has bad sign. Furthermore, we want to rewrite the first term in terms of the infinitesimal generator of the scaling symmetry $\Lambda = \partial_\rho + \tfrac1\rho$. To remove the term with bad sign, we introduce the generalised current. The d'Alembertian of our weight is $-\tfrac12 \Box \tfrac1\rho = - \frac{d - 2}{2} \tfrac1{\rho^3}$, so the generalised $0$-current takes the form 
	\begin{align*}
		{^{(w_0)}} K[\phi]
			&= \frac{d - 2}{2\rho} |\nabla \phi|^2 - \frac{(d - 2)^2}{4} \frac{1}{\rho^3} |\phi|^2 + \kappa \frac{d - 2}{2 \rho} |\phi|^{\frac{2d}{d - 2}}\\
			&= \frac1\rho \left( \frac{d - 2}{2} |\nabla \phi|^2 - \frac{(d - 2)^2}{4} \left| \frac1\rho \phi\right|^2 + \kappa \frac{d - 2}{2} |\phi|^{\frac{2d}{d - 2}} \right).
	\end{align*}
It remains to handle the first term of $^{(X_0)} K[\phi]$ and the second term coming from $^{(w_0)} K[\phi]$. This will be done by Hardy's inequality, or, more descriptively, completing-the-square with integration-by-parts. Define the Hardy $0$- and $1$-currents 
	\begin{equation}
	\begin{split}
		{^{(\cH_0}} J_\rho [\phi]
			&:= - \frac{d - 2}{2\rho^2} |\phi|^2, \\
		{^{(\cH_0)}} K[\phi]
			&:= - \frac{(d - 2)^2}{2\rho^3} |\phi|^2 + \frac{d - 2}{2\rho^2} |\phi|^2\\
			&= \frac1\rho \left( \frac{(d -2)^2}{2} \left| \frac1\rho \phi \right|^2 + \frac{d - 2}{\rho^2} \phi \partial_\rho \phi \right)	.
	\end{split}		
	\end{equation}
We set the other components of $^{(\cH_0)} J[\phi$ to be zero. Then, using $\nabla^\mu ({^{(\cH_0)}} J_\mu  = - \rho^{-d} \partial_\rho (\rho^d ({^{(\cH_0)}} J_\rho  ))$, we obtain the divergence identity 
	\begin{equation}
		\nabla^\mu \left( {^{(\cH_0)}} J_\mu [\phi] \right) = {^{(\cH_0)}} K[\phi].
	\end{equation}
Collecting the $0$-curents, 
	\begin{align}
		{^{(X_0)}} K[\phi]
			&= \frac1\rho \left( |\partial_\rho \phi|^2 + \frac{2 - d}{2} |\nabla \phi|^2 - \kappa \frac{d - 2}{2} |\phi|^{\frac{2d}{d - 2}} \right),\\
		{^{(w_0)}} K[\phi]
			&= \frac1\rho \left( \frac{d - 2}{2} |\nabla \phi|^2 - \frac{(d - 2)^2}{4} \left| \frac1\rho \phi\right|^2 + \kappa \frac{d - 2}{2} |\phi|^{\frac{2d}{d - 2}} \right),\\
		{^{(\cH_0)}} K[\phi]
			&= \frac1\rho \left( \frac{(d - 2)^2}2 \left| \frac1\rho \phi\right|^2 + \frac{d - 2}{\rho^2} \phi \partial_\rho \phi \right),
	\end{align}
we write
	\begin{equation}
		 {^{(X_0)}} K[\phi] + {^{(w_0)}} K[\phi] + {^{(\cH_0)}} K[\phi] = \frac1\rho \left| \left( \partial_\rho + \frac{d - 2}{2 \rho} \right) \phi \right|^2.
	\end{equation}
\subsubsection{$1$-current}

Ideally we wold like for the flux term to have a sign. It is convenient here to work in null coordinates, so we perform a null decomposition. At each point $p = (t_0, x_0)$, consider an orthonormal frame $\{ e_\fra \}_{\fra = 1, \dots, d - 1}$ for the sphere $\{t_0\} \times \partial B_{\rho_0} (0)$. Then $\{L, \underline L, e_1, \dots, e_{d - 1}\}$ forms a null frame at $p$. We decompose $\nabla \phi$ with respect to the null frame into $L\phi$, $\underline L \phi$ and $\slashed \nabla_\fra \phi$,
	\begin{equation}
		\begin{split}
			T[\phi](L, \underline L)
				&= |L\phi|^2, \\
			T[\phi]	(\underline L, \underline L)
				&= |\underline L \phi|^2, \\
			T[\phi] (L, \underline L)
				&= |\slashed \nabla \phi|^2 + \kappa \frac{d - 2}{d} |\phi|^{\frac{2d}{d - 2}}.
		\end{split}
	\end{equation}
Furthermore, recall that 
	\[
		\rho^2 = uv, \qquad X_0 = \frac12 \left( \frac{v}{u} \right)^{1/2} L + \frac12 \left( \frac{u}{v} \right)^{1/2} \underline L. 
	\]
Then the $1$-currents in these coordinates take the form 
	\begin{equation}
	\begin{split}
		{^{(X_0)}} J_L [\phi]
			&= \frac12 \left( \frac{v}{u} \right)^{1/2}  |L \phi|^2 + \frac12 \left( \frac{u}{v}\right)^{1/2} \left( |\slashed \nabla \phi|^2 + \kappa \frac{d - 2}{d} |\phi|^{\frac{2d}{d - 2}}  \right),\\
		{^{(X_0)}} J_{\underline L} [\phi]
			&= \frac12 \left( \frac{u}{v} \right)^{1/2}  |\underline L \phi|^2 + \frac12 \left( \frac{v}{u}\right)^{1/2} \left( |\slashed \nabla \phi|^2 + \kappa \frac{d - 2}{d} |\phi|^{\frac{2d}{d - 2}}  \right)	
	\end{split}
	\end{equation}	
and
	\begin{equation}
	\begin{split}
		{^{(w_0)}} J_L [\phi]
			&= \frac{d - 2}{2\rho} \phi L\phi + \frac{d - 2}{4} \left( \frac{u}{v} \right)^{1/2} \frac1{\rho^2} |\phi|^2,\\ 
		{^{(w_0)}} J_{\underline L} [\phi]
			&= \frac{d - 2}{2\rho} \phi \underline L\phi + \frac{d - 2}{4} \left( \frac{v}{u} \right)^{1/2} \frac1{\rho^2} |\phi|^2	
	\end{split}
	\end{equation}		
and, writing $\d \rho = \tfrac12 \left(\tfrac{u}{v}\right)^{1/2} \d v + \tfrac12 \left( \tfrac{v}{u} \right)^{1/2} \d u$, 
	\begin{equation}
	\begin{split}
		{^{(\cH_0)}} J_L [\phi]
			&= - \frac{d -2}{2} \left( \frac{u}{v} \right)^{1/2} \frac1{\rho^2} |\phi|^2, \\
		{^{(\cH_0)}} J_L [\phi]
			&= - \frac{d -2}{2} \left( \frac{v}{u} \right)^{1/2} \frac1{\rho^2} |\phi|^2.	
	\end{split}
	\end{equation}
Evidently, adding these three expressions does not give a good sign. We complete the square again, computing 
	\begin{align*}
		\left| r^{- \frac{d - 2}{2}} L \left( r^{\frac{d - 2}{2}} \phi \right) \right|^2	
			&= |L\phi|^2 + \frac{d - 2}{r} \phi L \phi + \frac{(d - 2)^2}{4} \frac1{r^2} |\phi|^2, \\
		\left| r^{- \frac{d - 2}{2}}\underline L \left( r^{\frac{d - 2}{2}} \phi \right) \right|^2	
			&= |\underline L\phi|^2 - \frac{d - 2}{r} \phi L \phi + \frac{(d - 2)^2}{4} \frac1{r^2} |\phi|^2.
	\end{align*}	
The game then is to rewrite	
	\begin{align*}
		{\color{green} \frac{d - 2}{2\rho} \phi L \phi - \frac12 \left(\frac{v}{u} \right)^{1/2} \frac{d - 2}{r}  \phi L \phi}
			&= {\color{green} -\frac{d - 2}{2} \frac{t}{r \rho} \phi L \phi}, 
			\\
		{\color{green} \frac{d - 2}{2\rho}  \phi \underline L  \phi  + \frac12 \left(\frac{u}{v} \right)^{1/2} \frac{d - 2}{r}  \phi \underline L  \phi }
			&={\color{green} +\frac{d - 2}{2} \frac{t}{r \rho} \phi \underline L  \phi },
			\\
		{\color{red} -\frac{d - 2}{4} \left(  \frac{u}{v}\right)^{1/2} \frac{1}{\rho^2} |\phi|^2 - \frac12 \left(  \frac{v}{u}\right)^{1/2} \frac{(d - 2)^2}{4} \frac{1}{r^2} |\phi|^2 }	
			&= {\color{red} - \left(\frac{d - 2}{4}    \frac{1}{\rho v}  + \frac12  \frac{(d - 2)^2}{4} \frac{v}{r^2 \rho} \right) |\phi|^2},
			\\
		{\color{red} -\frac{d - 2}{4} \left(  \frac{v}{u}\right)^{1/2} \frac{1}{\rho^2} |\phi|^2 - \frac12 \left(  \frac{u}{v}\right)^{1/2} \frac{(d - 2)^2}{4} \frac{1}{r^2} |\phi|^2 }
			&= {\color{red}-\left( \frac{d - 2}{4}  \frac{1}{\rho u}  + \frac12 \frac{(d - 2)^2}{4} \frac{u}{r^2 \rho} \right) |\phi|^2}	.
	\end{align*}
To kill these terms and replace them with ones with non-negative sign, we introduce a divergence-free $1$-current known as the \emph{null current} (see for example \cite{OhTataru2016} for the Maxwell-Klein-Gordon case). Define 
	\begin{align*}
		{^{(N_0)}} J_L [\phi]
			&:= + \frac{d - 2}{4} \frac{1}{r^{d - 1}} L \left( r^{d - 1} \frac{t}{\rho r} |\phi|^2 \right) ,
			\\
		{^{(N_0)}} J_{\underline L} [\phi]
			&:=  -\frac{d - 2}{4} \frac{1}{r^{d - 1}} \underline L \left( r^{d - 1}\frac{t}{\rho r} |\phi|^2 \right)	,
	\end{align*}
and the other components are set to zero. This is divergence-free since mixed partials commute, $L \underline L = \underline L L$. Expanding via the product rule, we obtain
	\begin{align*}
		{^{(N_0)}} J_L [\phi]
			&= {\color{green}+\frac{d - 2}{2} \frac{t}{r \rho}  \phi L \phi } 
			{\color{red}+
			\left(\frac{d - 2}{4} \frac{1}{\rho v} + \frac{(d - 2)^2}{4} \frac{v}{r^2 \rho} - \frac{(d - 2)^2}{4} \frac{1}{r \rho}\right) |\phi|^2},
			\\
		{^{(N_0)}} J_{\underline L} [\phi]
			&= {\color{green} -\frac{d - 2}{2} \frac{t}{r \rho}  \phi \underline L  \phi } {\color{red}+
			\left(\frac{d - 2}{4} \frac{1}{\rho u} + \frac{(d - 2)^2}{4} \frac{u}{r^2 \rho}  + \frac{(d - 2)^2}{4} \frac{1}{r \rho}\right) |\phi|^2}.
	\end{align*}
Adding the currents together, we see that the $ \phi L \phi $ terms cancel. It remains to examine the coefficient in front of $|\phi|^2$. Factoring out $\tfrac{d - 2}{4}$, we consider
	\begin{align*}
		\frac{v}{r^2 \rho} -\frac12 \frac{v}{r^2 \rho} -  \frac{1}{r \rho}
			&= \frac12 \frac{v}{r^2 \rho} -  \frac{r}{r^2 \rho} = \frac12 \frac{u}{r^2 \rho} = \frac12 \left( \frac{u}{v} \right)^{1/2} \frac{1}{r^2}, \\
		\frac{u}{r^2 \rho} -\frac12 \frac{u}{r^2 \rho} +  \frac{1}{r \rho}
			&= \frac12 \frac{u}{r^2 \rho} +  \frac{r}{r^2 \rho} = \frac12 \frac{v}{r^2 \rho} = \frac12 \left( \frac{v}{u} \right)^{1/2} \frac{1}{r^2}.	
	\end{align*}
This completes the derivation of the monotonicity formula.

