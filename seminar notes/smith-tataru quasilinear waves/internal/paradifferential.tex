
Decomposing the left-hand side via the Littlewood-Paley trichotomy, we schematically write the low-high, high-low, and high-high interactions of $\Box_{\bfg(\phi)}$ as 
        \[
                \Box_{\bfg(\phi)} \phi 
                    \approx \Box_{\bfg(\phi)_{< \lambda}} \phi_\lambda
                        +  \bfg(\phi)_\lambda \cdot \partial\partial \phi_{< \lambda}
                        + \sum_{\mu \geq \lambda} \bfg(\phi)_\mu \cdot \partial\partial \phi_{\mu} .
        \]
Placing the latter two on the right-hand side, we obtain the paradiagonal formulation of the equation \eqref{QNLW}
        \begin{equation}
               \Box_{\bfg(\phi)_{< \lambda}} \phi_\lambda
                   \approx \bfg(\phi)_\lambda \cdot \partial\partial \phi_{< \lambda}
                        + \bfg(\phi)_\lambda \cdot \partial\partial \phi_{< \lambda} +  \sum_{\mu \geq \lambda} \bfg(\phi)_\mu \cdot \partial\partial \phi_{\mu} + \cN(\phi)(\partial \phi, \partial \phi)_\lambda.
        \end{equation}
A similar decomposition can be made for the linear wave counterpart, 
	\begin{equation}\tag{LW} \label{eq:linearwave}
	\begin{alignedat}{2}
		\Box_{\bfg(\phi)} \psi 
			&= 0,\\
		(\psi, \partial_t \psi)_{|t = 0} 
			&= (\psi_0, \psi_1).
	\end{alignedat}
	\end{equation}
Thus, the main non-perturbative part of the argument resides in analysing the linear paradifferential equation, 
	\begin{equation}\tag{PLW} \label{eq:paralinear}
	\begin{alignedat}{2}
		\Box_{\bfg(\phi)_{< \lambda}} \sfP_\lambda \psi 
			&= 0,\\
		(\psi, \partial_t \psi)_{|t = 0}
			&= (\sfP_{\lambda}\psi_0, \sfP_\lambda \psi_1).
	\end{alignedat}
	\end{equation} 

Our strategy will consist of constructing a suitable approximate solution to this initial data problem, and proving dispersive estimates hold for the approximation. 

\begin{proposition}[Existence of a parametrix]
    Assuming the bootstrap, for each $\phi[0] \in (H^1 \times L^2)_x (\R^n)$, there exists a family of smooth functions $\psi^\lambda \in C^\infty ([-2, 2] \times \R^n)$ which satisfy 
        \begin{enumerate}
                \item frequency localisation, 
                        \begin{equation}
                                \supp \widehat{\psi^\lambda (t)}
                                        \subseteq \{ \xi \in \R^n : |\xi| \sim \lambda \},
                        \end{equation}
                \item initial data matching,    
                        \begin{equation}
                                \psi^\lambda [-2] = \sfP_\lambda \phi[0],
                        \end{equation}
                \item small error, 
                        \begin{equation}
                                \| \Box_{\bfg_{< \lambda}} \psi^\lambda \|_{L^1_t L^2_x} 
                                        \lesssim \epsilon_0 \|\phi[0]\|_{H^1_x \times L^2_x},
                        \end{equation}
                \item Strichartz-type estimates for $\sigma > \frac{n - 1}{2}$, 
                 \begin{equation}\label{eq:parametrixstrichartz}
                \| \psi^\lambda \|_{L^2_t L^\infty_x} 
                        \lesssim \epsilon_0^{-\frac12} \lambda^{\sigma - 1}\| \phi[0]\|_{H^1_x \times L^2_x},
        \end{equation}
        \end{enumerate}
\end{proposition}

With the approximate solution at hand, we can analytically set-up the iteration scheme for solving \eqref{linearwave} as follows, 
        \begin{itemize}
                \item approximate solution to the homogeneous problem \eqref{paralinear}, 
                        \[
                                \mathcal S^{\mathrm{appr}}(t) \phi[0]
                                        := \sum_\lambda \psi^{\lambda}
                        \]
                \item approximate solution to the inhomogeneous problem with zero initial data, 
                        \[
                                \big(\Box_{\bfg}^{-1}\big)^{\mathrm{appr}} F
                                        :=  \int_{0}^t \mathcal S^{\mathrm{appr}}(t) \begin{pmatrix} 0 \\ F(s) \end{pmatrix} \, ds,
                        \]
        \end{itemize}
It is a standard argument to replace the $L^1_t L^2_x$-error bound to an $L^2_{t, x}$-error bound. This will be convenient for estimating terms $\partial \bfg \partial \psi$ in $L^1_t L^2_x$, placing the metric in Strichartz and the solution in energy norm. Then the parametrix error bounds implies that the linear map
        \begin{equation}
        \begin{split}
                L^2_{t, x} ([0, 1] \times \R^n)
                    &\longrightarrow L^2_{t, x} ([0, 1] \times \R^n), \\
                F 
                        &\longmapsto\left( \Box_{\bfg} \,\big(\Box_{\bfg}^{-1}\big)^{\mathrm{appr}} - \mathrm{Id}\right) F
        \end{split}
        \end{equation}
is a contraction. In particular, each time we feed in an approximate solution to \eqref{linearwave}, and solve away the error on the right, we gain. Thus, we can write the exact solution to \eqref{linearwave} as 
        \begin{equation}
                \phi 
                        = \cS^{\mathrm{appr}}(t) \phi[0] + \big(\Box_{\bfg}^{-1}\big)^{\mathrm{appr}} F,
        \end{equation}
where $F$ is some power series expansion which kills the error completely, and can easily be checked to satisfy
        \[
                \| F\|_{L^2_{t, x}} 
                        \lesssim \epsilon_0 \| \phi[0] \|_{H^1_x \times L^2_x}.
        \]
In this form, it is easy to pass the Strichartz bounds on the parametrix to the exact solution, with energy norm on the right-hand side. After suitable commutations, one can do the same for higher regularity norms.
