
The analysis in the previous sections tell us that the geometry of slabs is approximately that of Minkowski space. Thus, one can expect that the same harmonic analysis counting arguments used to prove Strichartz estimate hold in our setting. 

\begin{proposition}[Strichartz estimate for wave packet parametrix]
    Let $\cT$ be the collection of $\lambda$-scale wave packets, then 
        \begin{equation}\label{eq:strichartzWP}
            \Big\|\sum_{\mathtt T \in \cT} a_{\mathtt T} \mathfrak w^{\mathtt T}\Big\|_{L^2_t L^\infty_x}
                \lessapprox \lambda^{\frac{n - 1}{2}} \lambda^{-1} \| a_{\mathtt T} \|_{\ell^2_{\mathtt T}}.
        \end{equation}
\end{proposition}

Here we use the standard harmonic analysis asymptotic notation $A \lessapprox B$, i.e. the inequality holds up to logarithmic losses in $\lambda$. For our purposes, it will suffice $A \lesssim_\epsilon \lambda^{\epsilon} B$ for any $\epsilon > 0$. Our proof will proceed in two steps. First, we count the number of slabs containing any pair of points. This can be thought of as a dispersive decay-type estimate, considering data localised around a singular point, and estimating pointwise how the ensuing wave packets spread. Second, we discretise the Strichartz-type estimate and reduce it to counting. 

\subsection{Dispersive decay}

Fix two points $\mathtt P_1 = (t_1, x_1)$ and $\mathtt P_2 = (t_2, x_2)$ in $[-2, 2] \times \R^n$. Our goal is to estimate 
    \[
        \#_\lambda (\mathtt P_1, \mathtt P_2) 
                := \text{$\#$ of slabs at scale $\lambda$ containing $\mathtt P_1$ and $\mathtt P_2$.}
    \]
\begin{proposition}[Dispersive decay]\label{prop:dispersive-decay}
    The number of slabs at scale $\lambda$ containing a pair of points $\mathtt P_1$ and $\mathtt P_2$ is bounded by 
        \begin{equation}\label{eq:dispersive-decay}
            \#_\lambda (\mathtt P_1, \mathtt P_2) 
                \lesssim \left(\frac{\lambda}{\epsilon_0} \right)^{\frac{n - 1}{2}} \big(\lambda |t_1 - t_2|\big)^{-1}.
        \end{equation}
\end{proposition}

To this end, we introduce $C_{\mathtt P_1} \subseteq [-2, 2] \times \R^n$ the forward light cone starting from $\mathtt P_1$, and let $\gamma_\theta$ be the null geodesic contained in $\Sigma_\theta$ starting from $\gamma_\theta (t_1) = x_1$. Denote $q_\theta := \gamma_\theta (t_2)$. Define
    \[
        \dist(x_2, C_{\mathtt P_1}^{t_2}) 
            = \inf_{\theta \in \SS^{n - 1}} |x_2 - \gamma_\theta (t_2)| 
    \]
which is the distance of $x_2$ to the $t_2$-slice of the cone, and set 
    \[
        \delta u (\mathtt P_1, \mathtt P_2) 
            := \sup_{\theta \in \SS^{n - 1}} |u_\theta (\mathtt P_2) - u_\theta (\mathtt P_1)| .
    \]
These two are related as follows, 

\begin{lemma}[$\delta u$ is signed distance to cone]
    The parameter $\delta u$ is negative if $\mathtt P_2$ is inside the cone, positive in the exterior. Furthermore, $\delta u \approx \dist(x_2, C_1^{t_2})$.
\end{lemma}

\begin{proof}
    See \cite[Lemma 9.1]{SmithTataru2005}.
\end{proof}


\begin{lemma}[Characterisation of slabs containing two points]
    Define then the set of angles 
    \[
            A_\lambda 
                := \Big\{ \theta \in \SS^{n - 1} : |u_\theta (\mathtt P_2) - u_\theta (\mathtt P_1)| \leq \lambda^{-1} \text{ and } |\gamma_\theta(t_2) - x_2| \leq (\epsilon_0 \lambda)^{-\frac12} \Big\} ,
    \]
Then
    \begin{equation}\label{eq:slabcounting}
        \#_\lambda (\mathtt P_1, \mathtt P_2) 
            \lesssim \# \text{ of $(\tfrac{\epsilon_0}{\lambda})^{\frac12}$-balls covering $A_\lambda$}.
    \end{equation}
\end{lemma}

\begin{proof}
    Recall that if a slab at scale $\lambda$ in the direction $\theta$ contains both points $\mathtt P_1$ and $\mathtt P_2$, then the slab centered on $\gamma_\theta$ of scale $\lambda/4$ must also contain $\mathtt P_1$ and $\mathtt P_2$. The result follows. 
\end{proof}


    \begin{equation}
        u_\omega(t_2, q_2) - u_\omega(t_1, x_1) 
            = - \frac12|t_1 - t_2| \cdot |\omega - \theta|^2 + (\omega - \theta) \cdot \overrightarrow{q_\theta q_2} + \text{error}.
    \end{equation}

We now turn to the proof of Proposition \ref{prop:dispersive-decay}.

\begin{proof}[Proof of $|t_1 - t_2| \leq 2 \lambda^{-1}$ case]
     In this case, the wave packets emanating from $\mathtt P_1$ have not had time to spread, so we are free to use the trivial bound
        \[
            \#_\lambda (\mathtt P_1, \mathtt P_2) 
                \lesssim \left(\frac{\lambda}{\epsilon_0} \right)^{\frac{n - 1}{2}} \lesssim \left(\frac{\lambda}{\epsilon_0} \right)^{\frac{n - 1}{2}} \big(\lambda |t_1 - t_2|\big)^{-1}.
        \]  
\end{proof} 

\begin{proof}[Proof of $\delta u < - 4 \lambda^{-1}$ case]
    In this case, 
        \[
             \#_\lambda (\mathtt P_1, \mathtt P_2)  = 0.
        \]
\end{proof}

\begin{proof}[Proof of $|\delta u| \leq 4 \lambda^{-1}$ and $|t_1 - t_2| > 2 \lambda^{-1}$ case]
    
    
        \[
             \# \text{ of $(\tfrac{\epsilon_0}{\lambda})^{\frac12}$-balls covering $A_\lambda$}
                \lesssim ( \epsilon_0 |t_1 - t_2| )^{-\frac{n - 1}{2}},
        \]
    which for $n \geq 3$ is stronger than the desired bound. The result follows from \eqref{slabcounting}. 
\end{proof}

\begin{proof}[Proof of $\delta u \geq 4 \lambda^{-1}$ and $|t_1 - t_2| > 2 \lambda^{-1}$ and $2|t_1 - t_2| \geq \delta u$ case]
    
    
\end{proof}

\begin{proof}[Proof of $\delta u \geq 4 \lambda^{-1}$ and $|t_1 - t_2| > 2 \lambda^{-1}$ and $2|t_1 - t_2| \leq \delta u$ case]
    
    
\end{proof}




\subsection{Strichartz estimates}

Our proof of the dispersive estimate for the parametrix relies only on pointwise bounds on the wave packets, not their oscillation. That is, it will suffice to bound a superposition of wave packets pointwise by 
    \begin{align*}
        \Big|\sum_{\mathtt T \in \cT} a_{\mathtt T} \mathfrak w^{\mathtt T}\Big|
            \lesssim (\epsilon_0 \lambda)^{\frac12 \frac{n - 1}{2}} \lambda^{-\frac12} \sum_{\mathtt T \in \cT} |a_{\mathtt T}| \cdot \mathbb 1_{\mathtt T},
    \end{align*}
which holds since wave packets have amplitude $O((\epsilon_0 \lambda)^{\frac12\frac{n - 1}{2}} \lambda^{-\frac12})$. Thus, to prove Strichartz for the wave packet parametrix \eqref{strichartzWP}, it suffices to prove that, for functions of the form 
    \[
            \phi 
                := \sum_{\mathtt T \in \cT} a_{\mathtt T} \mathbb 1_{\mathtt T}  
        \]
we have the Strichartz-type bound 
    \begin{equation}\label{eq:strichartz-reduced}
            \| \phi \|_{L^2_t L^\infty_x} 
                \lessapprox \Big( \frac{\lambda}{\epsilon_0} \Big)^{\frac12 \frac{n - 1}{2}} \lambda^{-\frac12} \| a_{\mathtt T} \|_{\ell^2_{\mathtt T}}
        \end{equation}
By scaling, we may assume $\| a_{\mathtt T} \| \leq 1$. 

\subsubsection*{Step 1: discretising the problem}

Dividing $[0, 1]$ into $O(\lambda)$-many sub-intervals $I_j$ of length $2\lambda^{-1}$, we can find points $\mathtt P_j = (t_j, x_j)$ nearly maximising $|\phi|$ on each space-time region $I_j \times \R^n$. It follows that
    \begin{align*}
        \| \phi \|_{L^2_t L^\infty_x}
            &\lesssim \Big( \sum_j \int_{I_j} \|\phi(t) \|_{L^\infty_x}^2 \, dt \Big)^{\frac12} \lesssim \Big(\sum_j \lambda^{-1} |\phi(t_j, x_j)|^2 \Big)^{\frac12}.
    \end{align*}
After passing to an $O(\lambda)$ subset of points, we can choose $t_j$ to be $\lambda^{-1}$-separated and such that the inequality above continues to hold. Next, we dyadically decompose the sum over slabs $\mathtt T$ with respect to the size $N^{-\frac12}$ of the coefficients $a_T$, and similarly the sum over points $\mathtt P_j$ with respect to $L$ the number of slabs containing them; we denote the number of such points by $m(L)$. It follows that
    \begin{align*}
        \|\phi \|_{L^2_t L^\infty_x}
            &\lesssim \lambda^{-\frac12} \sum_{\substack{L \in 2^\N \\ L \lesssim (\frac{\lambda}{\epsilon_0})^{\frac{n - 1}2}}} \Big( \sum_{\substack{j \\ \text{$\# \{ \mathtt T \in \cT : \mathtt P_j \in \mathtt T \} \sim L$}}} |\phi(t_j, x_j)|^2 \Big)^{\frac12}\\
            &\lesssim \lambda^{-\frac12} \sum_{\substack{L \in 2^\N \\ L \lesssim (\frac{\lambda}{\epsilon_0})^{\frac{n - 1}2}}} \Big( \sum_{\substack{j \\ \text{$\# \{ \mathtt T \in \cT : \mathtt P_j \in \mathtt T \} \sim L$}}} \Big| \sum_{\substack{\substack{N \in 2^{\N} \\ N \lesssim (\frac{\lambda}{\epsilon_0})^{\frac{n - 1}2}} }} \quad \sum_{\substack{\mathtt T \in \cT \\ |a_{\mathtt T}| \sim N^{-\frac12}}} a_{\mathbb T} \mathbb 1_{\mathbb T} (t_j, x_j) \Big|^2 \Big)^{\frac12}\\
            &\lesssim \lambda^{-\frac12} \sum_{\substack{L \in 2^\N \\ L \lesssim (\frac{\lambda}{\epsilon_0})^{\frac{n - 1}2}}} \Big( \sum_{\substack{j \\ \text{$\# \{ \mathtt T \in \cT : \mathtt P_j \in \mathtt T \} \sim L$}}} \Big| \sum_{\substack{\substack{N \in 2^{\N} \\ N \lesssim (\frac{\lambda}{\epsilon_0})^{\frac{n - 1}2}} }} L N^{-\frac12}  \Big|^2 \Big)^{\frac12}\\
            &\lesssim \lambda^{-\frac12} \sum_{\substack{L \in 2^\N \\ L \lesssim (\frac{\lambda}{\epsilon_0})^{\frac{n - 1}2}}} \sum_{\substack{\substack{N \in 2^{\N} \\ N \lesssim (\frac{\lambda}{\epsilon_0})^{\frac{n - 1}2}} }} \Big( m(L) L^2 N^{-1} \Big)^{\frac12}
    \end{align*}
In the above, we needed only to sum over a finite range of scales, since each point lies in $O((\tfrac{\lambda}{\epsilon_0})^{\frac{n - 1}{2}})$-many slabs. This restricts us to a finite range of scales for $L$, and allows us to regard the contribution of slabs with small coefficients $|a_T| \lesssim (\tfrac{\epsilon_0}\lambda)^{\frac{n - 1}{2}}$ as $O(1)$. The dyadic sums are thus harmless, as they only contribute a logarithm. Therefore the Strichartz-type estimate \eqref{strichartz-reduced} is reduced to 
    \begin{equation}\label{eq:counting}
        m(L) N^{-1} L^2
            \lessapprox \Big(\frac{\lambda}{\epsilon_0} \Big)^{\frac{n - 1}{2}},
    \end{equation}
fixing $N$ and $L$ for the remainder of the argument. 

\subsubsection*{Step 2: counting argument (or Fubini's theorem)}

We conclude with a counting argument. To summarise notation and introduce new ones, 
    \begin{align*}
        N^{-\frac12} 
            &:= \text{size of coefficient},\\
        L 
            &:= \text{$\#$ of slabs containing a point},\\
        m
            &:= \text{$\#$ of points intersecting $L$-many slabs},\\
        k
            &:= \text{$\#$ of pairs $(i,j)$ for which $\mathtt P_i, \mathtt P_j$ lie in a common slab w/ multiplicity},\\
        n(\mathtt T)
            &:= \text{$\#$ of points in slab $\mathtt T$},\\
        \cT_N 
            &:= \text{set of slabs with coefficients of size $a_{\mathtt T} \sim N^{-\frac12}$}.
    \end{align*}

We count $k$ in two different ways\footnote{Otherwise known as Fubini's theorem.}: we could either fix a slab $\mathtt T$ with at least two points, and count the approximately $\binom{n(\mathtt T)}{2} \approx |n(\mathtt T)|^2$ pairs; or we could fix a pair of points $(\mathtt P_i, \mathtt P_j)$ and count the number of slabs containing both points, i.e. $\#_\lambda (\mathtt P_i, \mathtt P_j)$, 
    \begin{equation}\label{eq:Kcounting}
        \begin{split}
        k
            &\approx \sum_{\substack{\mathtt T \in \cT_N \\ n(\mathtt T) \geq 2}} |n(\mathtt T)|^2 \\
            &\approx \sum_{\substack{1 \leq i, j \leq m \\ i \neq j  }} \#_\lambda (\mathtt P_i, \mathtt P_j).
        \end{split}
    \end{equation}
On the other hand, by counting in two different ways, namely fixing a slab and counting the points, or fixing a point and counting the slabs, 
    \begin{align}\label{eq:totalpoints}
        \sum_{\substack{\mathtt T \in \cT_N}} n(\mathtt T)
            &\sim m \cdot L.
    \end{align}

\subsubsection*{Step 3: establishing a dichotomy}

Partitioning the collection of slabs based on the size of their coefficients, $\sum_{\mathtt T} |a_{\mathtt T}|^2 \sim \sum_N \sum_{\mathtt T \in \cT_N} N^{-1}$, we see from the pigeonhole principle that the number of slabs with coefficients of size $N^{-\frac12}$ is of size 
    \begin{equation}\label{eq:slabcount}
    |\cT_N| \lesssim N.
    \end{equation} 
It follows from Cauchy-Schwartz in the slabs $\mathtt T$ and counting pairs \eqref{Kcounting} that 
    \begin{equation}\label{eq:CS-slabs}
        \sum_{\substack{\mathtt T \in \cT_N \\ n(\mathtt T) \geq 2}} n(\mathtt T) 
            \lesssim N^{\frac12} \Big( \sum_{\substack{\mathtt T \in \cT_N \\ n(\mathtt T) \geq 2}} |n(\mathtt T)|^2 \Big)^{\frac12} \approx N^{\frac12} k^{\frac12}. 
    \end{equation}
Our dichotomy will follow by considering two cases in the sum \eqref{totalpoints}: one where the contribution from slabs with only a single point dominates, which allows us to estimate using the naive estimate on the number of slabs \eqref{slabcount}, or when the contribution from slabs with at least two points dominates, in which Cauchy-Schwartz \eqref{CS-slabs} is more judicious,
    \begin{enumerate}
        \item Either 
            \[
                \sum_{\substack{\mathtt T \in \cT_N \\ n(\mathtt T) \geq 2}} n(\mathtt T) 
                    \leq \sum_{\substack{\mathtt T \in \cT_N\\ n(\mathtt T) = 1}} n(\mathtt T) 
            \]
           which, combined with \eqref{totalpoints} and counting the number of slabs \eqref{slabcount}, implies 
            \begin{align*}
                 m \cdot L
                    &\approx \sum_{\substack{\mathtt T \in \cT_N}} n(\mathtt T)\\
                    &\approx \sum_{\substack{\mathtt T \in \cT_N \\ n(\mathtt T) = 1}} n(\mathtt T) \lesssim \sum_{\substack{\mathtt T \in \cT_N \\ n(\mathtt T) = 1}} 1 \lesssim |\cT_N| \lesssim N. 
            \end{align*}
        \item Or
            \[
                \sum_{\substack{\mathtt T \in \cT_N \\ n(\mathtt T) \geq 2}} n(\mathtt T) 
                    \geq \sum_{\substack{\mathtt T \in \cT_N\\ n(\mathtt T) = 1}} n(\mathtt T) ,
            \]
            so by \eqref{totalpoints}, Cauchy-Schwartz \eqref{CS-slabs}, we have 
            \begin{align*}  
                m \cdot L 
                    &\approx \sum_{\substack{\mathtt T \in \cT_N}} n(\mathtt T)\\
                    &\approx \sum_{\substack{\mathtt T \in \cT_N \\ n(\mathtt T) \geq 2}} n(\mathtt T) \lesssim N^{\frac12} k^{\frac12}.
            \end{align*}
            Rearranging gives 
                \[
                    k \gtrsim m^2 N^{-1} L^2.
                \]
    \end{enumerate}

\subsubsection*{Step 4: concluding the argument} Proceeding from the conclusions of the dichotomy, 
    \begin{enumerate}
        \item $N \gtrsim m L$; in this case, 
            \[m N^{-1} L^2 \lesssim L,\]
           so \eqref{counting} follows from the trivial bound $O((\frac{\lambda}{\epsilon_0})^{\frac{n - 1}{2}})$ on the number of slabs.  
        
        \item $k \gtrsim m^2 N^{-1} L^2$; then, counting $k$ by the number of slabs intersecting two points \eqref{Kcounting}, using the ``dispersive decay" estimate \eqref{dispersive-decay}, and also the $\lambda^{-1}$-separation of $t_j$, we have
            \begin{align*}
                m N^{-1} L^2 
                    &\lesssim  m^{-1} k \\
                    &\lesssim m^{-1} \sum_{\substack{1 \leq i, j \leq m \\ i \neq j  }} \#_\lambda (\mathtt P_i, \mathtt P_j) \\
                    &\lesssim \left(\frac{\lambda}{\epsilon_0} \right)^{\frac{n - 1}{2}} m^{-1}\sum_{\substack{1 \leq i, j \leq m \\ i \neq j  }}  (\lambda |t_i - t_j|)^{-1} \lessapprox \left(\frac{\lambda}{\epsilon_0} \right)^{\frac{n - 1}{2}} ,
            \end{align*}
        which concludes \eqref{strichartz-reduced}. 
    \end{enumerate}
