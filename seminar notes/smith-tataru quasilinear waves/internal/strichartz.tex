
The analysis in the previous sections tell us that the geometry of slabs is approximately that of Minkowski space. Thus, one can expect that the same harmonic analysis counting arguments used to prove Strichartz estimate (or, alternatively, Fourier restriction estimates) hold in this setting. 

\begin{proposition}
    Let 
        \[
            \phi 
                := \sum_{T \in \cT} a_T \mathbb 1_{T}  
        \]
    then 
        \begin{equation}
            \| \phi \|_{L^2_t L^\infty_x} 
                \lesssim (\epsilon_0 \lambda)^{- \frac12 \frac{n - 1}{2}} \lambda^{\frac12}
        \end{equation}
\end{proposition}


\begin{proposition}[Dispersive decay, I]
    For all pairs of points $P_1, P_2$ in space-time, the number of slabs at scale $\lambda$ containing both is 
        \begin{equation}
            \#_\lambda (P_1, P_2) 
                \lesssim 
                \begin{cases} 
                    \epsilon_0 
                        & \text{if $m \in I_1$},\\
                    \epsilon_0 
                        & \text{if $m \in I_2$},\\
                    0 
                        & \text{otherwise},
                \end{cases}
        \end{equation}
    where 
        \begin{align*}
            I_1 
                &:= \\
            I_2
                &:=
        \end{align*}
\end{proposition}


\begin{corollary}[Dispersive decay, II]
    For all pairs of points $P_1, P_2$ in space-time, the number of slabs at scale $\lambda$ containing both is 
        \begin{equation}
            \#_\lambda (P_1, P_2) 
                \lesssim \epsilon_0^{-\frac{n - 1}{2}} \lambda^{\frac{n - 3}{2}} |t_1 - t_2|^{-1}
        \end{equation}
\end{corollary}