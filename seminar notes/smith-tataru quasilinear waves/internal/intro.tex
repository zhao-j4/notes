
In this note, we consider the local well-posedness of \textit{quasilinear wave equations} of the form
	\begin{equation}\tag{QNLW} \label{eq:QNLW}
	\begin{alignedat}{2}
		\Box_{\bfg(\phi)} \phi 
			&= \cN(\phi) (\partial \phi, \partial \phi), 
			&& \qquad \text{on $[0, T] \times \R^n$},\\
		(\phi, \partial_t \phi) 
			&= (\phi_0, \phi_1),
			&& \qquad\text{on $t = 0$},
	\end{alignedat}
	\end{equation}
where $\bfg_{\mu\nu} (\phi)$ is a symmetric matrix with signature $(-, +, \dots +)$, using the convention\footnote{One can equivalently consider the divergence form of the equation, i.e. using $\partial_\mu \bfg^{\mu\nu} \partial_\nu$ instead of $\bfg^{\mu\nu} \partial_\mu \partial_\nu$ on the left-hand side, as the lower-order terms are encapsulated by the right-hand side.} $\Box_{\bfg} := \bfg^{\mu\nu} \partial_\mu \partial_\nu$ for its associated wave operator, and $\cN(\phi)(\partial \phi, \partial \phi) := \cN^{\alpha\beta} (\phi) \partial_\alpha \phi \partial_\beta \phi$ is a bilinear form. Without loss of generality, we can take $t = \text{const}$ to be space-like hypersurfaces by reducing to metrics of the form 
	\[\bfg_{\mu\nu} dx^\mu dx^\nu = - dt^2 + \bfg_{ij} dx^i dx^j.\]
We shall also assume sufficient smoothness and boundedness of the metric $\bfg^{\mu\nu}(\phi)$, its inverse $\bfg_{\mu\nu}(\phi)$, and of the bilinear form $\cN^{\alpha\beta} (\phi)$ as functions of $\phi$. 

\begin{example}
The following can be recast in the form \eqref{QNLW}, 
	\begin{itemize}
		\item the Einstein vacuum equations in wave coordinates,
		\item the irrotational compressible Euler equations.
	\end{itemize}
For the former, this was observed by Choquet-Bruhat \cite{Foures-Bruhat1952}, while the later is due to Hughes-Kato-Marsden \cite{HughesEtAl1977}. The reader may find the lecture notes \cite{Luk} as a more modern reference. 
\end{example}

Following the standard set by Hadamard, we say that the initial data problem for the quasi-linear wave equation \eqref{QNLW} is \textit{locally well-posed} in $(H^s_x \times H^{s - 1}_x) (\R^n)$ if the following hold: 
	\begin{enumerate}
		\myitem{(a)}\label{item:exist1} \textit{Existence}: for each initial data $\phi[0] \in (H^s_x \times H^{s - 1}_x) (\R^n)$, there exists a time $T > 0$ and a solution $\phi[t] \in C^0_t (H^s_x \times H^{s - 1}_x) ([0, T] \times \R^n)$ to \eqref{QNLW}. 
		
		\myitem{(b)} \textit{(Unconditional) uniqueness}: for each initial data $\phi[0] \in (H^s_x \times H^{s - 1}_x) (\R^n)$, the solution $\phi[t]$ to \eqref{QNLW} is unique in the space $C^0_t (H^s_x \times H^{s - 1}_x) ([0, T] \times \R^n)$.
		
		\myitem{(c)}\label{item:cty}\textit{Continuity of data-to-solution map}: if $\{\phi_k[0]\}_k$ is a sequence of data converging in the $(H^s_x \times H^{s - 1}_x)$-topology to $\phi[0]$, then there exists a common time of existence\footnote{To be more precise, one can introduce the notion of the \textit{maximal lifespan} $T \equiv T(\phi[0])$ of a solution, and require it to be lower semi-continuous as a function of initial data $\phi[0] \in (H^s_x \times H^{s - 1}_x)(\R^n)$.} on which the corresponding sequence of solutions $\{\phi_k[t]\}_k$ to \eqref{QNLW} converges to $\phi[t]$ in the $L^\infty_t (H^s_x \times H^{s - 1}_x)$-topology,
			\begin{align*}
				\phi_k[0] 
					&\overset{k \to \infty}{\longrightarrow} \phi[0] \qquad \text{in $H^s_x \times H^{s - 1}_x$} \\
				\text{implies} \qquad 
				\phi_k [t] 
					&\overset{k \to \infty}{\longrightarrow} \phi[t] \qquad \, \text{in $L^\infty_t (H^s_x \times H^{s - 1}_x)$}.
			\end{align*}
	\end{enumerate}
For the working definition, we will need to slightly modify the existence and uniqueness statements, strengthening the former while weakening the latter, and require an additional property of the data-to-solution map:
	\begin{enumerate}
		\myitem{(a+)} \label{item:exist}\textit{(Sub-critical) existence}: the time of existence can be taken to depend only on the size of the data 
			\[
				T \equiv T(\|\phi[0]\|_{H^s_x \times H^{s - 1}_x}).
			\]
	
		\myitem{(b')}\label{item:unique} \textit{(Conditional) uniqueness}: uniqueness holds only in the smaller Strichartz space, 
			\[ 
				\left\{ \phi[0] \in C^0_t (H^s_x \times H^{s - 1}_x) : \partial \phi \in L^2_t L^\infty_x  \right\}.
			\]

		\myitem{(c+)}\label{item:lipschitz} \textit{Weak Lipschitz continuity of data-to-solution map}: there exists a regularity $s_{\text{Lip}} < s$ such that the data-to-solution map is Lipschitz continuous on bounded sets in $(H^s \times H^{s - 1})_x(\R^n)$ with respect to the weaker $(H^{s_{\text{Lip}}} \times H^{s_{\text{Lip}} - 1})_x(\R^n)$-topology, i.e. for solutions $\phi[t], \psi[t] \in C^0_t (H^s_x \times H^{s - 1}_x)([0, T] \times \R^n)$ to \eqref{QNLW} satisfying 
			\begin{align*}
				\| \phi[0] \|_{H^s \times H^{s - 1}} , \, \| \psi[0] \|_{H^s \times H^{s - 1}} 
					&\leq R,
			\end{align*}
		the following stability estimate holds:
			\begin{align*}
				\| \phi[t] - \psi[t] \|_{L^\infty_t (H^{s_{\text{Lip}}} \times H^{s_{\text{Lip}} - 1})_x}
					&\leq C(R) \cdot  \| \phi[0] - \psi[0] \|_{(H^{s_{\text{Lip}}} \times H^{s_{\text{Lip}} - 1})_x}.
			\end{align*}
	\end{enumerate}

In sum, we say that the initial data problem for the quasi-linear wave equation \eqref{QNLW} is \textit{locally well-posed} in $(H^s_x \times H^{s - 1}_x)(\R^n)$ if \ref{item:exist1}, \ref{item:exist}, \ref{item:unique}, \ref{item:cty}, \ref{item:lipschitz} hold. This leads us to the following natural question
\begin{quote}
	\it For which values of $s \in \R$ is the initial data problem for the quasi-linear wave equation \eqref{QNLW} locally well-posed in $(H^s_x \times H^{s - 1}_x)(\R^n)$?
\end{quote}

To simplify the presentation, we will only consider high spatial dimensions, i.e. $n \geq 3$; in dimension $n = 2$, the dispersion of waves is weaker, though analogues of various statements made in this note continue to hold with suitable modifications. In this setting, the sharp result, due to Smith-Tataru, may be stated as follows, 

\begin{theorem}[Sharp local well-posedness for \eqref{QNLW} \cite{SmithTataru2005}]\label{thm:main}
	In dimensions $n = 3, 4, 5$, the initial data problem for the quasi-linear wave equation \eqref{QNLW} is locally well-posed in $(H^s \times H^{s - 1})_x(\R^n)$ for $s > \tfrac{n}{2} + \tfrac12$, with weak Lipschitz continuity of the data-to-solution map with respect to the $(H^1 \times L^2)_x$-topology. 
\end{theorem}

\begin{remark}
	The equation \eqref{QNLW} is invariant under the scaling symmetry
	\[
		\phi(x^\mu) \mapsto \phi(\tfrac{x^\mu}{\lambda})
	\]
	which also preserves the homogeneous Sobolev norm $\dot H^{s_{\text{crit}}}_x \times \dot H^{s_{\text{crit}} - 1}_x(\R^n)$, where $s_{\text{crit}} := \tfrac{n}{2}$. It is natural to ask whether one can prove well-posedness up to the critical regularity. Unfortunately, Theorem \ref{thm:main} is sharp for generic \eqref{QNLW} in dimensions $n = 2, 3$ due to counterexamples of Lindblad \cite{Lindblad1993,Lindblad1996a}. On the other hand, it is not difficult to show that the \textit{Nirenberg example}
		\[
			\Box \phi = \partial^\alpha \phi \partial_\alpha \phi ,
		\]
	is locally well-posed for $s > s_{\text{crit}}$, thanks to the \textit{null structure} of the non-linearity. 
\end{remark}

\begin{remark}
	The proof contained in \cite{SmithTataru2005} breaks down in higher dimensions $n \geq 6$ due to a technical failure in the orthogonality argument for the wave packet decomposition; see Section \ref{sec:parametrix}. 
\end{remark}

The basic starting point is the energy estimate, and local well-posedness result for sufficienly smooth initial data,

\begin{theorem}[Local well-posedness for \eqref{QNLW} with smooth data \cite{HughesEtAl1977}]
	The initial data problem for the quasi-linear wave equation \eqref{QNLW} is (unconditionally) locally well-posed in $(H^s_x \times H^{s - 1}_x)(\R^n)$ for $s > \frac{n}{2} + 1$. Furthermore, for all $s \geq 0$, there exists $C \gg 1$ such that any smooth solution $\phi$ obeys the a priori estimate
		\begin{equation}
			\| \partial \phi \|_{L^\infty_t H^{s - 1}_x} 
				\lesssim \exp \left( C \int_0^T \| \partial \phi \|_{L^\infty_x} \, dt \right) \|\partial \phi (0)\|_{H^{s - 1}_x}.
		\end{equation}
\end{theorem}

\begin{proof}
	See \cite{IfrimTataru2022} for a modern treatment. 
\end{proof}

To prove Theorem \ref{thm:main}, one essentially needs to close the energy estimate. This is accomplished by proving Strichartz estimates for the linearised equation\footnote{Strictly speaking, this is not quite the linearisation of \eqref{QNLW}, as there is an extra order-zero term $(\partial_\phi \bfg)^{\mu\nu}(\phi)\partial_\mu\partial_\nu \phi \, \psi$, however this can be easily treated as perturbative on short time-scales. }
	\begin{equation}\tag{LW} \label{eq:LW}
	\begin{alignedat}{2}
		\Box_{\bfg(\phi)} \psi 
			&= 0,\\
		(\psi, \partial_t \psi)_{|t = 0} 
			&= (\psi_0, \psi_1),
	\end{alignedat}
	\end{equation}
and its paradifferential counterpart 
	\begin{equation}\tag{PLW} \label{eq:LWpara}
	\begin{alignedat}{2}
		\Box_{\bfg(\phi)_{< \lambda}} \psi_\lambda 
			&= 0,\\
		(\psi, \partial_t \psi)_{|t = 0}
			&= (\psi_0, \psi_1),
	\end{alignedat}
	\end{equation} 
where $\phi$ is a solution to \eqref{QNLW}. For flat backgrounds $\bfg(\phi) \equiv \bfm$, i.e. when \eqref{LW} reduces to the classical linear wave equation, one has the classical Strichartz estimates 
	\begin{align*}
		\| \partial \phi \|_{L^\infty_t H^{s - 1}_x} + \| \partial \phi \|_{L^2_t L^\infty_x} 
					\lesssim \|\phi[0]\|_{H^s_x \times H^{s - 1}_x}, \qquad s > \tfrac{n}{2} + \tfrac12,
	\end{align*}
with the $L^2_t L^\infty_x$-control allowing us to close the argument. If one only assumes $\partial \bfg \in L^2_t L^\infty_x$ in \eqref{LW}, then the best one can do is to obtain the above Strichartz estimate but with a $\tfrac16$-derivative loss \cite{Tataru2001c,Tataru2001d}. The major breakthrough of Smith-Tataru was that actually there is no derivative-loss in Strichartz for \eqref{LW} when one uses that $\phi$ is a solution to \eqref{QNLW}, which in turn roughly implies $\Box_{\bfg(\phi)} \bfg(\phi) \approx 0$, 

\begin{theorem}[Loss-less Strichartz estimates for \eqref{QNLW} \cite{SmithTataru2005}] \label{thm:strichartz}
	Let $\phi[t] \in C^0_t (H^s_x \times H^{s - 1}_x) ([0, T] \times \R^n)$ be a solution to \eqref{QNLW} as in Theorem \ref{thm:main}. Then 
		\begin{enumerate}
		\item \textup{Strichartz bounds for non-linear evolution}: the solution satisfies
			\[
				\| \partial \phi \|_{L^\infty_t H^{s - 1}_x} + \| \partial \phi \|_{L^2_t L^\infty_x} 
					\lesssim \|\phi[0]\|_{H^s_x \times H^{s - 1}_x}.
			\]
		\item \textup{Strichartz bounds for linear evolution}: for $1 \leq \sigma \leq s + 1$ and each $t_0 \in [0, T]$, the initial data problem for the linear equation \eqref{LW} is well-posed in $(H^\sigma_x \times H^{\sigma - 1}_x)(\R^n)$. Furthermore, the solution satisfies
			\begin{equation}
				\| \partial \psi \|_{L^\infty_t H^{\sigma - 1}_x} + \| \langle \nabla_x \rangle^{\sigma - \frac{n}{2} - \frac12 - 1 - } \partial \psi \|_{L^2_t L^\infty_x} 
					\lesssim \| \psi[t_0] \|_{H^\sigma_x \times H^{\sigma - 1}_x}.
			\end{equation}
	\end{enumerate}
\end{theorem}


\begin{remark}
	The dimension of the $L^p_t L^\infty_x$-norm of $\partial \phi$ under the scaling symmetry reads
		\begin{align*}
			\|\partial \phi\|_{L^p_t L^\infty_x} 
				&\approx [t]^{\frac1p} [x]^{-1} \approx [\partial]^{1 - \frac1p}.
		\end{align*}
	Thus, the continuation criterion $L^1_t L^\infty_x$ is scale-invariant, controlling  $L^\infty_{t, x}$ via Sobolev embedding incurs a full derivative difference from scaling $1 - \tfrac1\infty = 1$, while control of $L^2_t L^\infty_x$ in $n \geq 3$ via Strichartz leads to half-derivative from scaling $1 - \tfrac12 = \tfrac12$, and similarly $L^4_t L^\infty_x$ in $n = 2$ leads to three-quarters $1 - \tfrac14 = \tfrac12 + \tfrac14$. 
\end{remark}

\begin{table}[ht]
	\centering
	\renewcommand{\arraystretch}{1.4}
	{%
	\begin{tabular}{|c|c|c|c|}
		\hline
		 & \bf $\bfg$ in \eqref{LW} & \bf Strichartz & \bf Regularity \\
		\hline 
		Hughes-Kato-Marsden \cite{HughesEtAl1977} & $\partial\bfg \in L^\infty_{t,x}$ & N/A & $s > \tfrac{n}{2} + 1$\\
		\hline 
		Bahouri-Chemin \cite{BahouriChemin1999} & $\partial\bfg \in L^2_t L^\infty_x$ & $\tfrac14$ loss & $s > \tfrac{n}{2} + \tfrac12 + \tfrac14$ \\
		\hline
		Tataru \cite{Tataru2001c,Tataru2001d} & $\partial\bfg \in L^2_t L^\infty_x$ & sharp $\tfrac16$ loss & $s > \tfrac{n}{2} + \tfrac12 + \tfrac16$ \\
		\hline 
		Klainerman-Rodnianski \cite{KlainermanRodnianski2003} & $\Box_{\bfg} \bfg \approx 0$ & $\tfrac{2 - \sqrt 3}{2}$ loss & $s > \tfrac{n}{2} + \tfrac12 + \tfrac{2 - \sqrt 3}{2}$ \\
		\hline 
		Smith-Tataru \cite{SmithTataru2005} & $\Box_{\bfg} \bfg \approx 0$ & lossless & $s > \tfrac{n}{2} + \tfrac12$ \\
		\hline
	\end{tabular}
	}
\vspace{1em}

    \caption{A non-exhaustive historical overview of the local well-posedness of quasi-linear wave equations for $n \geq 3$, though one can find results concerning $n = 2$ among the references, and the result of Klainerman-Rodnianski \cite{KlainermanRodnianski2003} works only with $n = 3$.  }\label{table:history}
\end{table}


