
Let $u_\theta$ be the solution the eikonal equation initialised at $t = -2$, 
    \begin{equation}
        \begin{split}
        \bfg^{\mu\nu} \partial_\mu u \partial_\nu u 
            &= 0,\\
         u_{|t = -2}
            &= \theta \cdot x + 2. 
        \end{split}
    \end{equation}
We denote the level sets $u_\theta = r$, which are null hypersurfaces with respect to the metric $\bfg$, by $\Sigma_{\theta, r}$. Write $x_\theta := x \cdot \theta$ and let $x'_\theta$ be coordinates on the orthogonal complement $\theta^\perp$, so that $(x'_\theta, x_\theta)$ form an orthonormal coordinate system on $\R^n$. By the implicit function theorem, we can write these level sets as the graphs of functions $\tau_{\theta, r} \equiv \tau_{\theta, r} (t, x'_\theta)$. Thus we may parametrise $\Sigma_{\theta, r}$ by $(t, x'_\theta) \in [-2, 2] \times \R^{n - 1}$, and we can write 
    \[
        \Sigma_{\theta, r} 
            := \{ (t, x) \in [-2, 2] \times \R^n : x_\theta - \tau_{\theta, r} (t, x'_\theta) = 0 \}.
    \]

\begin{example}
    In the flat case $\bfg = \bfm$, we have the explicit solution 
        \[
            u_\theta (t, x) 
                = x \cdot \theta - t,
        \]
    so that the null hypersurfaces are hyperplanes given by 
        \[
            \Sigma_{\theta, r} 
                = \{ (t, x) : x_\theta - t = r \},
        \]
    where $r \in \R$ is a fixed constant (not radius! confusingly enough), and $\tau := t + r$. 
\end{example}

\subsection{Regularity of null hypersurfaces}

It is convenient to define the Sobolev-type norms on the null hypersurfaces $\Sigma_{\theta, r}$ by
    \[
        \| f \|_{\mathcal H^s_{t, x'_\theta} (\Sigma_{\theta, r})}
            :=
                \sup_{\theta, r} \big\| f \big\|_{L^2_t H^s_{x'_\theta} (\Sigma_{\theta, r})} + \big\| \partial_t f \big\|_{L^2_t H^{s - 1}_{x'_\theta} (\Sigma_{\theta, r})}. 
    \]

\begin{proposition}[Characteristic energy estimates]
    
        \begin{equation}
            \|\bfg - \bfm \|_{\cH^s (\Sigma_{\theta, r})}  + \| \partial \bfg_{< \lambda} \|_{(\Sigma_{\theta, r})} + \lambda^{-1} \| \partial_x \partial \bfg_{< \lambda} \|_{\cH^s (\Sigma_{\theta, r})}
                \lesssim \epsilon_2. 
        \end{equation}
\end{proposition}

\begin{proposition}[Regularity of null hypersurfaces]

        \begin{equation}\label{eq:nullregularity}
            \| d \tau_{\theta, r} (t) - dt \|_{C^{1, 0+}_{x'_\theta}(\R^{n - 1})}
                \lesssim \epsilon_2 + \| \partial \bfg (t) \|_{C^{0, 0+}_x(\R^n)}.
        \end{equation}
\end{proposition}

It will be convenient to introduce some notation on $\Sigma$. Let $\bfV$ be the vector field obtained by raising the indices of $du$, 
    \[
        \bfV^\alpha := \bfg^{\alpha \beta} \partial_\beta u.
    \]
Set 
    \[
        \sigma := \langle dt, \bfV \rangle = \bfV^0,
    \]
and 
    \[
        \bfL := \sigma^{-1} \bfV.
    \]
Thus $\bfL$ is the $\bfg$-normal field to $\Sigma$ normalised so that $\bfL^0 = 1$. Set 
    \[
        \underline{\bfL}^\alpha
            := \bfL^\alpha + 2 \bfg^{\alpha 0} \partial_0.
    \]
Then $\{ \bfL, \underline \bfL\}$ form a null frame. 


\subsection{Geometry of light cones}

\begin{proposition}[Angle of null generators]
    Let $\theta, \omega \in \SS^{n - 1}$, then 
        \begin{equation}\label{eq:differenceL}
            \bfL_\theta - \bfL_\omega 
                = (\theta - \omega) + o(|\theta - \omega|).
        \end{equation}
    Also, 
        \begin{equation}\label{eq:angleL}
            \langle \bfL_\theta, \bfL_\omega \rangle_{\bfg}
                = -\frac12|\theta - \omega|^2 + o(|\theta - \omega|^2).
        \end{equation}
\end{proposition}

\begin{proposition}[Separation of null geodesics]
    Let $\theta, \omega \in \SS^{n - 1}$, and fix $(t_1, x_1) \in [-2, 2] \times \R^n$. Denote $\gamma_\theta$ and $\gamma_\omega$ the null geodesics with data 
        \[
            \gamma_\theta(t_1) = \gamma_\omega(t_1) = x_1, \qquad \dot \gamma_\theta(t_1) \parallel \theta, \qquad \dot \gamma_\omega(t_1) \parallel \omega.
        \]
    Then 
        \begin{equation}
            \gamma_\theta (t) - \gamma_\omega(t) 
                = (t - t_1) (\theta - \omega) + o(|t - t_1| \cdot |\theta - \omega|).
        \end{equation}
\end{proposition}