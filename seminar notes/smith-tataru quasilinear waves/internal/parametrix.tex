
To prove Strichartz estimates for the linear wave equation, we construct a \textit{wave packet parametrix}, that is, a superposition of wave packets which form an approximate solution the initial data problem. Given a null geodesic $\gamma(t)$ contained in a null hypersurface $\Sigma_{\theta, r} = \{ x_\theta - \tau_{\theta, r} = 0 \}$, we define a \textit{wave packet} $\mathfrak w$ localised around $\gamma$ at scale $\lambda \in 2^\N$ to be a function of the form  
    \[
        \mathfrak w
            := (\epsilon_0 \lambda)^{\frac12\frac{n - 1}{2}} \lambda^{\frac12 - 1 - 1}\sfT_{< \lambda} \left( w \, \delta(u_{\theta, r}) \right),
    \]
where $\delta$ is the Dirac mass at zero, and thus $\delta(u_{\theta, r})$ is a measure\footnote{Alternatively, one can think of this as the pushforward of the Lebesgue measure on $[-2, 2] \times \R^{n - 1}$ under the embedding $(t, x') \mapsto (t, x', \tau)$. This convention is common in harmonic analysis in the context of the Fourier extension and restriction problems. Strictly speaking, this is not the same as \textit{surface measure} as it is referred in \cite{SmithTataru2005}. }   supported on the null hypersurface $\Sigma_{\theta, r}$ which takes the form  
    \begin{align*}
        \langle \delta (u_\theta), \varphi \rangle 
            &:= \int_{[-2, 2]} \int_{\R^{n - 1}} \varphi(t, x'_\theta, \tau_{\theta, r}) \, dx_\theta' \, dt \qquad \text{for $\varphi \in C^\infty_c ((-2, 2) \times \R^n)$},
    \end{align*}   
and $w$ is a smooth bump function on $\Sigma_{\theta, r}$ localised at scale $(\epsilon_0 \lambda)^{-\frac12}$ about the null geodesic $\gamma$, i.e.  
    \[
        w(t, x_\theta') = w_0 \big( (\epsilon_0 \lambda)^{\frac12} (x_\theta' - \gamma_\theta' (t)) \big), \qquad w_0 \in C^\infty_c (|x'| \leq 1),
    \]
and finally $\sfT_{< \lambda}$ is a mollification to spatial scales $\triangle x \sim \tfrac1\lambda$, taking it with kernel $\chi_{< \lambda} (x) := \lambda^n \chi(\lambda x)$ for some compactly-supported cut-off $\chi \in C^\infty_c (|x| \leq \tfrac{1}{2000})$; morally, one should think of this as a localisation to frequency $|\xi| \lesssim \lambda$. 

The parameter $\epsilon_0 \ll 1$ shall be chosen such that the error as an approximate solution $\Box_{\bfg} \mathfrak w$ is small in $L^1_t L^2_x$, while the choice of amplitude will normalise the wave packet to be of size $O(1)$ in $(H^1 \times L^2)_x$; see Lemmas \ref{lem:errorWP} and \ref{lem:energyWP} respectively.

\begin{figure}[h]
    \centering
    \includegraphics[scale = 0.2]{graphics/WP.jpg}
    \caption{Given a null geodesic $\gamma$ residing in a null hypersurface $\Sigma_{\theta, u}$, the corresponding wave packet is supported in a $1 \times \lambda^{-1} \times (\epsilon_0 \lambda)^{-\frac{1}{2} \times (n - 1)}$-slab. }
\end{figure}

Given a function restricted to frequency $\lambda$ for all $t$, we want to construct a wave packet resolution, i.e. show that it arises as a superposition of wave packets. To that end, decompose $\R^n$ into a parallel tiling of rectangles of dimensions $(4\epsilon_0 \lambda)^{-\frac12 \times (n - 1)} \times (8\lambda)^{-1}$ in $(x_\theta', x_\theta)$-coordinates. Then define 
    \begin{align*}
        R_{\theta, j} 
            &:= \text{doubles of these rectangles at $t = -2$},\\
        \Sigma_{\theta, j}
            &:= \text{null hypersurface centered on $R_{\theta, j}$},\\
        \gamma_{\theta, j}
            &:= \text{null geodesic in $\Sigma_{\theta, j}$ through center of $R_{\theta, j}$},\\
        \mathtt T_{\theta, j}
            &:= \text{$(32\lambda)^{-1}$-neighborhood of $\Sigma_{\theta, j} \cap \{ |x'_\theta - \gamma_{\theta, j}(t)| \leq (\epsilon_0 \lambda)^{-\frac12} \}$}. 
    \end{align*}
We shall refer to the space-time regions $\mathtt T_{\theta, j}$ as \text{slabs}; by construction, wave packets are supported in slabs. These slabs satisfy a finite-overlap condition; indeed, those associated to different null hypersurfaces are disjoint, while those associated to the same null hypersurface have finite overlap in the $x_\theta'$-variable. Furthermore, we consider slabs with angles $\theta$ taken from 
    \[
        \Omega 
            := \text{maximal collection of $\big(\tfrac{\lambda}{\epsilon_0}\big)^{\frac{n - 1}{2}}$-many unit vectors $\theta \in \SS^{n - 1}$ separated by at least $\big(\tfrac{\epsilon_0}{\lambda}\big)^{\frac{1}{2}}$}.
    \]

\begin{figure}[h]
    \centering
    \includegraphics[scale = 0.2]{graphics/superWP.jpg}
    \caption{Given a fixed angle $\theta$, we cover $[-2,2 ] \times \R^n$ by slabs localised along null geodesics $\gamma_{\theta, j}$ with dimensions $(\epsilon_0 \lambda)^{-\frac12 \times (n - 1)} \times \lambda^{-1}$. }\label{fig:physicalWP}
\end{figure}


\begin{figure}[h]
    \centering
    \includegraphics[scale = 0.2]{graphics/fourier.jpg}
    
    \caption{In frequency space, the wave packets at time $t = -2$ are effectively localised to $({\epsilon_0}\lambda)^{\frac12}$-neighborhoods of the angles $\theta$. These angles are separated by at least $(\tfrac{\epsilon_0}\lambda)^{\frac12}$, so when restricted to the dyadic shell $|\xi| \sim \lambda$, there is $O(1)$-overlap between wave packets.  }\label{fig:frequencyWP}
\end{figure}
    


\begin{proposition}[Existence of wave packet parametrix]\label{prop:parametrix}
    Let $(\phi_0, \phi_1) \in (H^1 \times L^2)_x (\R^n)$ be initial data. Then, in dimensions $n= 2, 3, 4, 5$, there exists a superposition of wave packets 
        \[
            \phi 
                := \sum_{\theta, j} a_{\theta, j}  \, \mathfrak w^{\theta, j}
        \]
    which is an approximate solution to the paralinearised initial data problem \eqref{paralinear} in the sense that 
    \begin{enumerate}
        \item \label{item:WPdata}it matches the initial data at $t = -2$, 
            \begin{equation}
                \sfP_\lambda \phi[-2] = (\sfP_\lambda \phi_0, \sfP_\lambda \phi_1)
            \end{equation}
        \item \label{item:WPsize}the size of the coefficients is comparable to the size of the initial data, 
            \begin{equation}
                \Bigl(\sum_{\theta, j} |a_{\theta, j}|^2 \Bigr)^\frac12
                \lesssim \| \phi[0] \|_{(H^1 \times L^2)_x}.
            \end{equation}
        \item \label{item:WPenergy}the energy estimate holds, 
            \begin{equation}\label{eq:parametrixenergy}
                \| \partial \sfP_\lambda \phi \|_{L^\infty_t L^2_x}
                \lesssim \Bigl( \sum_{\theta, j} |a_{\theta, j}|^2 \Bigr)^\frac12
        \end{equation}
            
        \item \label{item:WPerror}the error on the right-hand side is small, 
             \begin{equation}\label{eq:parametrixerror}
                \| \Box_{\bfg_{< \lambda}} \phi_\lambda||_{L^1_t L^2_x} 
                \lesssim \epsilon_0 \, \Bigl( \sum_{\theta, j} |a_{\theta, j}|^2 \Bigr)^\frac12. 
            \end{equation}
    \end{enumerate}

    
\end{proposition}


\subsection{Properties of wave packets}

    As a first step, we prove that a singular wave packet is an approximate solution, in the sense that it satisfies the energy bound and error estimate. 

    \begin{lemma}[Energy estimate for $\mathfrak w$]\label{lem:energyWP}
        Wave packets have $O(1)$-energy, 
            \begin{equation}\label{eq:WPenergy}
                \| \partial \sfP_\lambda \mathfrak w \|_{L^\infty_t L^2_x} 
                    \lesssim 1. 
            \end{equation}
    \end{lemma}

    \begin{proof}[Proof for $\partial = \partial_x$]
        The estimate for the full space-time gradient will follow from the usual energy estimate once one has the error estimate \eqref{L2error}, though it is instructive to see how to read off the result for the spatial derivatives from the construction. Roughly speaking\footnote{Here, we write $\lessapprox$ and $\approx$ to mean that we aren't making any rigorous claims. }, 
            \begin{align*}
                \text{support of $\sfT_{< \lambda}(w \delta(u_{\theta, r}))$}
                    &\approx 1 \times \lambda^{-1} \times (\epsilon_0 \lambda)^{-\frac12 \times (n - 1)},\\
                \text{amplitude of $\sfT_{< \lambda}(w \delta(u_{\theta, r}))$}
                    &\approx \lambda.
            \end{align*}
        The former is fairly clear; to say a word about the latter, $\delta(x_\theta - \tau)$ is a unit point mass, while the mollification spreads it out in the $\theta$-direction to scale $\lambda^{-1}$, so dimensional analysis tells us the resulting amplitude is $\lambda$. Using the usual heuristic that $\partial_x \approx \lambda$ at frequencies comparable to $\lambda$, we arrive at 
            \[
                \| \partial_x \sfP_\lambda \mathfrak w \|_{L^\infty_t L^2_x} 
                    \lessapprox \lambda \cdot (\epsilon_0\lambda)^{\frac12 \frac{n - 1}{2}} \lambda^{\frac12 - 1 -1} \big|\text{amplitude}\big| \cdot \big|\text{support}\big|^{\frac12} \lessapprox 1. 
            \]
        This gives \eqref{WPenergy}, modulo time-derivative control. 
    \end{proof}

    \begin{lemma}[Wave packet error decomposition]
        Let $\mathfrak w$ be a wave packet, then the error decomposes as 
            \begin{equation}\label{eq:errordecomp}
                \Box_{\bfg_{< \lambda}} \sfP_\lambda \mathfrak w 
                    = \mathcal L \big( \partial \bfg, \partial \widetilde {\sfP_\lambda} \widetilde{\mathfrak w} \big) + (\epsilon_0 \lambda)^{\frac12 \frac{n - 1}{2}} \lambda^{\frac12 -1 - 1} \sfP_\lambda \sfT_{< \lambda} \sum_{j = 0, 1, 2} \psi_j \delta^{(j)} (x_\theta - \tau_{\theta, r}),
            \end{equation}
        where $\mathcal L$ is a bilinear form which obeys H\"older's inequality, $\widetilde{\sfP_\lambda}$ is a fattened frequency projection, $\widetilde{\mathfrak w}$ is another wave packet, $\delta^{(j)}$ are derivatives of the Dirac mass, and $\psi_j(t, x')$ obey 
            \begin{equation}\label{eq:errorcoeff}
                \big\| \psi_j ((\epsilon_0 \lambda)^{-\frac12} x'_\theta) \big\|_{L^2_t H^{s - 1}_{x'_\theta}}
                    \lesssim \epsilon_0 \lambda^{1 - j}.
            \end{equation}
    \end{lemma}

    \begin{corollary}[Error estimate for $\mathfrak w$]\label{lem:errorWP}
        Each wave packet has small error, 
            \begin{align}
                \| \Box_{\bfg_{< \lambda}} \sfP_\lambda \mathfrak w \|_{L^1_t L^2_x} 
                    &\lesssim \epsilon_0,\label{eq:L1error}\\
                \| \Box_{\bfg_{< \lambda}} \sfP_\lambda \mathfrak w \|_{L^2_{t, x}} 
                    &\lesssim \epsilon_0.\label{eq:L2error}
            \end{align}
    \end{corollary}

\begin{proof}
    Obviously \eqref{L2error} is stronger than \eqref{L1error}, so we focus on proving an $L^2_{t, x}$-error estimate. For the first term on the right-hand side of \eqref{errordecomp}, we place $\partial \bfg$ in $L^2_t L^\infty_x$, gaining smallness from our bootstrap assumption, and $\partial \widetilde{\mathfrak w}$ in $L^\infty_t L^2_x$, in which it is unit size by\footnote{To be perfectly rigorous, the energy estimate and the error estimate should be proved in conjunction. } \eqref{WPenergy}, yielding
        \[
            \| \mathcal L(\partial \bfg, \partial{\widetilde{\sfP_\lambda}} \widetilde{\mathfrak w}) \|_{L^2_{t, x}}
                \lesssim \| \partial \bfg \|_{L^2_t L^\infty_x} \| \partial \widetilde{\mathfrak w} \|_{L^\infty_t L^2_x} \lesssim \epsilon_2. 
        \]
    Taking $\epsilon_2 \ll \epsilon_0$ is an acceptable contribution towards \eqref{L2error}. 

    For the second term on the right-hand side of \eqref{errordecomp}, dimensional analysis yields 
        \begin{align*}
            \text{support of $\sfP_\lambda \sfT_{< \lambda} (\psi_j \, \delta^{(j)})$}
                &\approx 1 \times \lambda^{-1} \times (\epsilon_0 \lambda)^{-\frac12 \times (n - 1)},\\
            \|\text{amplitude of $\sfP_\lambda \sfT_{< \lambda} (\psi_j \, \delta^{(j)})$}\|_{L^2_t}
                &\lessapprox \|\text{amplitude of $\psi_j$}\|_{L^2_t} \cdot \lambda^{1 + j} \lessapprox \epsilon_0 \lambda^2.
        \end{align*}
    Roughly speaking, $s - 1 > \tfrac{n - 1}{2}$ under the assumptions of Theorem \ref{thm:main}, we can use Sobolev embedding on $\R^{n - 1}$ to read off the amplitude bounds using the scaled Sobolev estimate \eqref{errorcoeff}, while derivatives of the Dirac mass contribute amplitude $\lambda$. By dimensional analysis, we arrive at \eqref{L2error}. 
\end{proof}


\begin{proof}[Proof of error decomposition \eqref{errordecomp}]
    Suppressing subscripts for clarity $u \equiv u_{\theta, r}$ and $\tau \equiv \tau_{\theta, r}$, we compute
    \begin{align*}
        \Box_{\bfg_{< \lambda}} \mathsf P_\lambda \mathfrak w 
            &= (\epsilon_0 \lambda)^{\frac12 \frac{n - 1}{2}} \lambda^{-\frac12 + 1} \Bigl( [\Box_{\bfg_{< \lambda}}, \sfP_\lambda \sfT_{<\lambda}] + \sfP_{\lambda} \sfT_{< \lambda} \Box_{\bfg_{<\lambda}} \Bigr) w \, \delta(u) \\
            &= (\epsilon_0 \lambda)^{\frac12 \frac{n - 1}{2}} \lambda^{-\frac12 + 1}  [\Box_{\bfg_{< \lambda}}, \sfP_\lambda \sfT_{<\lambda}] w \, \delta(u) \\
            &\qquad + (\epsilon_0 \lambda)^{\frac12 \frac{n - 1}{2}} \lambda^{-\frac12 + 1}\sfP_{\lambda} \sfT_{< \lambda} \left( \Box_{\bfg_{<\lambda}} w \cdot \delta(u) + 2\overline{\bfg}_{< \lambda}^{\alpha \beta} \partial_\alpha w \cdot \partial_\beta \delta(u) + w \cdot \, \Box_{\bfg_{< \lambda}} \delta(u)\right)\\ 
            &=: \mathrm{I} + \mathrm{II} + \mathrm{III} + \mathrm{IV},
    \end{align*}
    commuting $\Box$ with the frequency projections and then applying the product rule. Here we have denoted $\overline \bfg^{\alpha \beta}_{< \lambda} = \tfrac12 (\bfg^{\alpha \beta}_{< \lambda} + \bfg^{\beta \alpha}_{< \lambda})$. 

    We analyse each term. The main game one has the play is that derivatives of the metric $\partial \bfg_{< \lambda}$ are small thanks to the bootstrap assumption, while the low frequencies of the metric itself $\bfg_{< \lambda}$ is \textit{a priori} $O(1)$, but we can replace it by the high frequencies $\bfg_{> \lambda} \lessapprox \tfrac1\lambda \partial \bfg_{> \lambda}$ plus the metric itself $\bfg$, which will be contracted by terms such that there is favourable cancellation. 

    \subsubsection*{\underline{Term $\mathrm{I}$}} Since the metric is cut-off to frequencies much lower than $\lambda$, the commutator clearly projects to frequencies $|\xi| \sim \lambda$. Thus, one can harmlessly insert fattened projections $\widetilde \sfP_{\lambda} \widetilde \sfT_{< \lambda}$ in front of the commutator. Furthermore, while two derivatives fall on the wave packet, standard commutator arguments\footnote{In a word, the principal symbol of the commutator is given by the Poisson bracket, so one can, to leading order, write $[\bfg(x), \chi(\nabla/\lambda)] \approx \{ \bfg(x), \chi(\xi/\lambda) \} \approx \partial_x \bfg \cdot \partial_\xi \chi(\xi/\lambda) \approx \tfrac1\lambda \partial_x \bfg$.} allow us to move one derivative onto the metric. In total, we can rewrite
        \[
            \mathrm{I} = [\bfg_{< \lambda}^{\alpha \beta}, \sfP_{\lambda} \sfT_{< \lambda}]\partial_\alpha \partial_\beta \widetilde{\mathfrak w} = \mathcal L (\partial \bfg, \partial \widetilde{\mathfrak w})
        \]
    for another wave packet $\widetilde{\mathfrak w}$ and some translation-invariant bilinear operator $\mathcal L(-, -)$. 

    \subsubsection*{\underline{Term $\mathrm{II}$}}
    
    
    We compute two derivatives of the bump function on $\R^{n - 1}$ localised to the null geodesic $\gamma$, 
        \begin{align*}
            \partial_\alpha \partial_\beta w
                = 
                \begin{cases}
                    O(\epsilon_0 \lambda) 
                        &\text{if two spatial derivatives},\\
                    O((\epsilon_0 \lambda)^{\frac12} \ddot \gamma)
                        &\text{if two time derivatives}, \\
                    O(\epsilon_0 \lambda \dot \gamma)
                        &
                \end{cases}
        \end{align*}
    Since $\|\ddot \gamma\|_{L^2_t} \lesssim \epsilon_1$, this is acceptable for \eqref{errorcoeff} when placed into $\psi_0$. 

    \subsubsection*{\underline{Term $\mathrm{III}$}}
        We compute, schematically, 
            \begin{align*}
                \overline{\bfg}_{< \lambda} \cdot \partial w \cdot \partial \delta(u) 
                    &= \overline{\bfg_{< \lambda}} \cdot \partial w \cdot \partial u \cdot \delta^{(1)} (u) \\
                    &= (\overline{\bfg}_{< \lambda})_{|\Sigma}\cdot \partial w \cdot \partial u \cdot \delta^{(1)} (u) + (\partial \overline{\bfg}_{< \lambda})_{|\Sigma}\cdot \partial w \cdot \partial u \cdot \delta (u).
            \end{align*}
        Putting the coefficients then in their respective boxes, 
            \begin{align*}
                \psi_0 
                    &:= \partial u \cdot \partial \overline{\bfg}_{< \lambda} \cdot \partial w, \\
                \psi_1
                    &:= \partial u \cdot \overline{\bfg}_{< \lambda} \cdot \partial w.
            \end{align*}
        It is easy to see that the first coefficient is acceptable for \eqref{errorcoeff}. To see that the second is also acceptable, we decompose 
            \begin{align*}
                \psi_1 = \overline{\bfg}_{< \lambda}^{\alpha \beta} \partial_\alpha u \partial_\beta w 
                    &=  \overline{\bfg}^{\alpha \beta} \partial_\alpha u \partial_\beta w - \overline{\bfg}_{> \lambda}^{\alpha \beta} \partial_\alpha u \partial_\beta w \\
                    &= \left(\overline{\bfg}^{\alpha \beta} - \overline{\bfg}^{\alpha\beta}_{|x'= \gamma(t)}\right) \partial_\alpha u \partial_\beta w - \overline{\bfg}_{> \lambda}^{\alpha \beta} \partial_\alpha u \partial_\beta w.
            \end{align*}
        The first line is obvious, for the second line, we use the fact that $\bfg \cdot \partial u \cdot \partial w = 0$ when restricted to the null geodesic $\gamma$, since $(1, \dot \gamma) \propto \bfg^{\alpha\beta} \partial_\alpha u$. These terms can then easily be estimated by $\partial \bfg$. 
        

    \subsubsection*{\underline{Term $\mathrm{IV}$}}
        We compute
        \begin{align*}
            \Box_{\bfg_{< \lambda}} \delta(u)
                &= \bfg_{< \lambda}^{\alpha \beta} \left( \partial_\alpha u \, \partial_\beta u \, \delta^{(2)} (u) + \partial_\alpha \partial_\beta \tau \, \delta^{(1)} (u) \right).
        \end{align*}
    Applying the distributional product rule, we can rewrite the terms above schematically as 
        \begin{align*}
            \bfg_{< \lambda}\cdot \partial\partial \tau \cdot \delta^{(1)} (x_\theta - \tau)
                &= (\partial \bfg)_{|\Sigma} \cdot \partial \partial \tau \cdot \delta(x_\theta - \tau)\\
            \bfg_{< \lambda} \cdot (\partial u)^2 \cdot \delta^{(2)} (x_\theta - \tau) 
                &= (\partial\partial \bfg_{< \lambda})_{|\Sigma} \cdot (\partial u)^2 \cdot \delta (x_\theta - \tau) \\
                &\qquad + 2(\partial \bfg_{< \lambda})_{|\Sigma} \cdot (\partial u)^2 \cdot \delta^{(1)}(x_\theta - \tau)\\
                &\qquad + (\bfg_{> \lambda})_{|\Sigma} (\partial u)^2 \cdot \delta^{(2)} (x_\theta - \tau)
        \end{align*}
    Here, the derivatives on the metric are in the $\theta$-direction; observe that $\partial u$ depends only on $t$ and $x_\theta'$ so no $x^\theta$-derivatives fall on these terms. In the last line, we have freely replaced $\bfg_{< \lambda}(\partial u)^2$ by $\bfg_{> \lambda} (\partial u)^2$, since $u$ is optical, i.e. $\bfg^{\alpha\beta}\partial_\alpha u \partial_\beta u = 0$. 
    
    Putting the coefficients then in their respective boxes, 
        \begin{align*}
            \psi_0
                &:= w \cdot \left( \partial\partial \bfg_{< \lambda} \cdot (\partial u)^2 + \partial \bfg_{< \lambda} \cdot \partial\partial \tau \right), \\
            \psi_1 
                &:= w \cdot \left(2 \partial \bfg_{< \lambda} \cdot (\partial u)^2 - \bfg_{< \lambda} \cdot \partial\partial \tau \right),\\
            \psi_2 
                &:= w \cdot  (\bfg_{> \lambda})_{|\Sigma} \cdot (\partial u)^2, 
        \end{align*}
    the error bound \eqref{errorcoeff} follow from characteristic energy estimates. 
    
\end{proof}

\begin{remark}
    It is instructive to compare the computation above against the flat case. Writing $u := x_\theta - t$ and $\underline u_\theta:= x_\theta + t$, we write in coordinates $\Box = -4 \partial \underline \partial + \Delta_{x'_\theta}$, so 
        \[
            \Box \delta(u) 
                = \partial_\alpha u \partial^\alpha u \, \delta^{(2)} (u) + \Box u \, \delta^{(1)} (u) = 0. 
        \]
    Thus, the measures $\delta(u)$ on null hypersurfaces are exact solutions to the linear wave equation on flat backgrounds. 
\end{remark}







    



\subsection{Orthogonality of wave packets}

We want to show that, given any superposition of wave packets, the energy estimate \eqref{parametrixenergy} and the error estimate \eqref{parametrixerror} hold. In view of the corresponding estimate for a single wave packet, we want to show that the interactions between wave packets are negligible, i.e. they are almost orthogonal. In place of the $L^\infty_t L^2_x$-bound \eqref{parametrixenergy}, it is convenient to prove the weaker $L^2_{t, x}$-bound, 
\begin{equation}\label{eq:L2L2orthogonal}
    \| \partial \sfP_\lambda \phi \|_{L^2_{t, x}} 
            \lesssim \Big( \sum_{\theta, j} |a_{\theta, j}|^2 \Big)^{\frac12}. 
\end{equation}
Then \eqref{L2L2orthogonal} and \eqref{parametrixerror} imply \eqref{parametrixenergy} by the energy estimate. 

We prove \eqref{L2L2orthogonal} by showing a fixed-time $t = \text{const}$ orthogonality estimate. This will be a consequence of the following two ``properties'' of wave packets which are ``morally" true,
    \begin{itemize}
        \item Parallel wave packets, i.e. fixing $\theta$, have finitely-overlapping supports $\mathtt T_{\theta, j}^t$ in physical space, see Figure \ref{fig:physicalWP}
        \item Non-parallel wave packets, i.e.  distinct $\theta, \omega \in \Omega$, at scale $\lambda$ are effectively localised in frequency space to the dual rectangles $(\mathtt T_{\theta, j}^t)^*$ and $(\mathtt T_{\omega, k}^t)^*$, which are contained in $(\epsilon_0 \lambda)^{\frac12}$-neighborhoods of $\theta$ and $\omega$ respectively. Since we have chosen angles separated by at least $(\frac{\epsilon_0}\lambda)^{\frac12}$, the supports are effectively disjoint on the dyadic shell $|\xi| \sim \lambda$, see Figure \ref{fig:frequencyWP}. 
    \end{itemize}
Evidently these hold in the flat case, we want to show these persist for the variable background. The picture one should keep in mind is as follows:   


\begin{figure}[h]
    \centering
    \includegraphics[scale = 0.25]{graphics/WPbad.jpg}
    \caption{On a ``bad'' time slice, the conormal direction $du_\theta$ to a single wave packet can vary by a large amount on the total support, so to ensure that we are looking at appropriately localised Fourier supports, we partition the support in physical space into smaller cubes of length $\ell$ where the conormal direction does not vary much. }\label{fig:badtime}
\end{figure}

    \begin{lemma}[Dual tube frequency localisation]
         Let $n \leq 5$ and suppose $t\in [-2, 2]$ is a time such that $\| \partial \bfg(t) \|_{C^{0, 0+}_x} \leq \epsilon_0$, then there exists coordinates $(y_\theta, y'_\theta)$ such that 
            \begin{equation}\label{eq:trace}
                \| \sfT_{< \lambda} \big( \psi^{\theta, j} \cdot \delta(x_\theta - \tau_\theta) \big) \|_{L^2 H^{\frac{n - 1}{2}_{a, y'_\theta} +} (\R^n)} 
                    \lesssim \lambda^{\frac12} \| \psi^{\theta, j} \|_{H^{\frac{n - 1}{2} +}_{a ,x'_\theta} (\R^{n - 1})}.
            \end{equation}
        Here the subscript $a = (\epsilon_0 \lambda)^{-\frac12}$ denotes the scaled-Sobolev space 
            \[
                \| f (y) \|_{H^{s}_{a, y} (\R^{n - 1}) } 
                    := \| f(ay) \|_{H^s_y (\R^{n - 1})}.
            \]
    \end{lemma}

    \begin{proof}
        In the flat case, this is true by direct application of Bernstein's inequality in the $\theta$-direction, 
            \begin{align*}
                \Big\| \sfT_{< \lambda} \Big(\psi(x'_\theta) \, \delta(x_\theta - t) \Big) \Big\|_{L^2_{x_\theta} H^{\frac{n - 1}{2} + }_{x'_\theta} (\R^n)}
                    &\lesssim \lambda^{\frac12} \| \psi  \|_{H^{\frac{n - 1}{2} +} (\R^{n - 1})}. 
            \end{align*}

        In the variable-coefficient case, since the derivative of the metric is $O(1)$, we know that the level hyper-surfaces are $C^2$-regular, and, roughly speaking, thanks to \eqref{nullregularity}, $x_\theta \approx \tau$, so the time $t$-sections of the tubes are contained in a rectangle of dimensions $(\epsilon_0 \lambda)^{-\frac12 \times (n - 1)} \times \lambda^{-1}$. In particular, 
            \[
                \supp \cF_x \sfT_{< \lambda} \Big(\psi(x'_\theta) \, \delta(x_\theta - \tau) \Big) \subseteq (\epsilon_0 \lambda)^{\frac12 \times (n - 1)} \times \lambda \text{ dual rectangle}.
            \]
        We can choose coordinates adapted to this rectangle, and take the Fourier transform in the $\lambda \mapsto \lambda^{-1}$-direction. By Plancharel's theorem and suitable rescaling, it will suffice to prove 
            \[
                \| \psi e^{i \eta \tau} \|_{H^{\frac{n - 1}{2}+} (\R^{n - 1})}
                    \lesssim (1 + |\eta|)^{100000} \| \psi \|_{H^{\frac{n - 1}{2}+} (\R^{n - 1})},
            \]
        where, denoting $\eta$ the dual Fourier variable to $x_\theta$, we have used the standard identity
            \[
                e^{i \eta \tau} = \cF_{x_\theta} \left(  \delta(x_\theta - \tau) \right).
            \]
        When both derivatives fall on $\psi$, we are happy, however when both derivatives fall on the oscillation, we need to control 
            \[
                \| e^{i \eta \tau} \|_{C^{2, 0+}} \lesssim \| \tau \|_{C^{2, 0+}} \lesssim (1 + |\eta|)^{2 + }.
            \]
        This is acceptable since we have $C^{1, 0+}$-control of the metric. 
    \end{proof}

    \begin{remark}
        Here we used the famous inequality 
            \[
                \frac{n - 1}{2} \leq 2 \qquad \text{ if and only if } n \leq 5.
            \]
    \end{remark}


    \begin{lemma}[Orthogonality at ``good" $t$]
        Set
            \[
                \Phi
                    := (\epsilon_0 \lambda)^{\frac12 \frac{n - 1}{2}} \lambda^{\frac12 - 1} \sfP_\lambda \sfT_{< \lambda} \sum_{\theta, j} \psi^{\theta, j} \, \delta(u_{\theta, j}).
            \]
        Let $n \leq 5$, and suppose $t\in [-2, 2]$ is a time such that $\| \partial \bfg(t) \|_{C^{0, 0+}_x} \leq \epsilon_0$, then 
            \begin{equation}\label{eq:goodorthogonal}
                \| \Phi \|_{L^2_x}^2
                    \lesssim \sum_{\theta, j} \left\| \psi^{\theta, j} ((\epsilon_0 \lambda)^{-\frac12} x ) \right\|_{H^{\frac{n - 1}{2} +} (\R^{n - 1})}^2
            \end{equation}
    \end{lemma}

    \begin{proof}
        Decompose into packets with the same angle, 
            \begin{align*}
                \Phi
                    &:= \sum_{\theta \in \Omega} \Phi_\theta, \\
                \Phi_{\theta} 
                    &:= \sfP_\lambda \sum_j \mathfrak v^{\theta, j},\\
                \mathfrak v^{\theta, j} 
                    &:= (\epsilon_0 \lambda)^{\frac12 \frac{n - 1}{2}} \lambda^{\frac12 - 1} \sfT_{< \lambda} \big( \psi^{\theta, j} \delta(u_{\theta, j}) \big).
            \end{align*}
        Thanks to the finite-overlap property in physical space and the estimate \eqref{trace}, we have that the packets with the same angle are essentially orthogonal,
            \[
                \| \Phi_\theta \|_{L^2_{y_\theta} H^{\frac{n - 1}{2} + }_{y'_\theta} (\R^n)}^2 
                    \lesssim (\epsilon_0 \lambda)^{\frac{n - 1}{2}} \sum_j \| \psi^{\theta, j} ((\epsilon_0 \lambda)^{-\frac12} x'_\theta)\|_{H^{\frac{n - 1}{2} +} (\R^{n - 1})}^2.
            \]
        after rescaling. By Plancherel's theorem and frequency localisation, super-positions of packets with different angles $\Phi_\theta$ are also effectively orthogonal, so we can conclude the result. 
    \end{proof}

    \begin{lemma}[Orthogonality at ``bad'' $t$]
        Set
            \[
                \Phi
                    := (\epsilon_0 \lambda)^{\frac12 \frac{n - 1}{2}} \lambda^{\frac12 - 1} \sfP_\lambda \sfT_{< \lambda} \sum_{\theta, j} \psi^{\theta, j} \, \delta(u_{\theta, j}).
            \]
        Suppose $t\in [-2, 2]$ is a time such that $\| \partial \bfg(t) \|_{C^{0, 0+}_x} \geq \epsilon_0$. Then  
            \begin{equation}\label{eq:badorthogonal}
                \| \Phi(t) \|_{L^2_x}^2 
                    \lesssim \Big( \frac{1}{\epsilon_0} \| \partial \bfg (t)\|_{C^{0, 0+}_x} \Big)^{\frac{n - 1}{2}} \sum_{\theta, j} \| \psi^{\theta, j} \|_{H^{\frac{n - 1}{2} +} (\R^{n - 1})}^2.
            \end{equation}
    \end{lemma}

    \begin{proof}
        Suppose $\| \partial \bfg(t) \|_{C^{0, 0+}_x} \geq \lambda$, then the bound follows from Cauchy-Schwartz in $\theta$. Indeed, since there are $O((\frac{\lambda}{\epsilon_0})^{\frac{n - 1}{2}})$-many angles,
            \begin{align*}
                 |\Phi|^2
                    \lesssim (\epsilon_0 \lambda)^{\frac{n - 1}{2}}\lambda^{-1} \cdot  \sum_\theta \Big|\lambda^{\frac{n - 1}{2} -\frac12} \sfP_\lambda \sum_j  \sfT_{< \lambda} \big( \psi^{\theta, j} \, \delta(u_{\theta, j}) \big) \Big|^2,
            \end{align*}
        then integrating, using physical space separation, and applying Bernstein gives the result. Morally this should be the easy case, since the equation we are effectively studying is with the truncated metric $\bfg_{< \lambda}$, so contributions when the metric is larger than $\lambda$ should be negligible. 

        Suppose then $\| \partial \bfg(t)\|_{C^{0, 0+}_x} \leq \lambda$. At the end of the day, we want to use the fact that the metric at fixed $t$ can be ``bad'', but it is only bad on average. The issue with large metric is that the conormal direction $du_\theta$ can vary a lot on the total support of each wave packet. By the null regularity estimate \eqref{nullregularity}, we know that the derivative of the conormal vector satisfies $\| \partial_x du \|_{C^{0+}} \lesssim \| \partial \bfg \|_{C^{0+}}$, so this Lipschitz control implies 
            \[
                \text{variation of conormal on region diamater $\ell$} \lessapprox \| \partial \bfg \|_{C^{0+}} \cdot \ell .
            \]  
        Moving in $\ell$-units in physical space corresponds to $\ell^{-1}$-units in frequency space. On the other hand, the angle change can lead to frequencies curving in bad directions as one moves along the long direction in frequency space, see Figure \ref{fig:badtime}. To ensure consistency, we set 
            \begin{align*}
                \text{short length} 
                    &\approx\text{long length} \cdot \text{angle} \approx \lambda \cdot \| \partial \bfg \|_{C^{0+}} \cdot \ell \\
                    &\approx \ell^{-1}.
            \end{align*}
        This gives 
            \[
                \ell \approx \big( \lambda \| \partial \bfg \|_{C^{0+}} \big)^{-\frac12}
            \]
        as the optimal scale to partition the support of the wave packet in physical space such that the angle does not change as much. Accordingly, we form a partition of unity $\{\chi_m\}_m$ of $\Sigma^t_{\theta, j}$ adapted to cubes of length $\ell$, and decompose
            \[
                \psi^{\theta, j} 
                    = \sum_m \chi_m \psi^{\theta, j}.
            \]
        Then the uncertainty principle analogous to \eqref{trace} yields
            \begin{align}
                \| \sfT_{< \lambda} \big( \chi_m \psi^{\theta, j} \cdot \delta(x_\theta - \tau) \big) \|_{L^2 H^{\frac{n - 1}{2}_{a, y'_\theta} +} (\R^n)} 
                    &\lesssim \lambda^{\frac12} \| \psi^{\theta, j} \|_{H^{\frac{n - 1}{2} +}_{a ,x'_\theta} (\R^{n - 1})}, \label{eq:badtrace}
            \end{align}
        where, by construction, $\theta$ is such that 
            \[
                |\theta - du_\theta | \lesssim \| \partial \bfg \|_{C^{0+}} \cdot \ell \lesssim (\lambda \ell)^{-1} \qquad \text{uniformly on $\chi_m$}.
            \]
        
        From here the proof proceeds similarly to the good time case, replacing $\epsilon_0$ with $\| \partial \bfg(t) \|_{C^{0, 0+}_x}$ in appropriate places. We leave this to the reader. 
    \end{proof}

    \begin{proof}[Proof of Proposition \ref{prop:parametrix} (\ref{item:WPenergy}) (energy bounds)]
        It immediately follows from the fixed-time orthogonality estimates \eqref{goodorthogonal}-\eqref{badorthogonal}  
        \begin{align*}
            \| \partial_x \sfP_\lambda \phi (t) \|_{L^2_x}^2 
                &\lesssim \Big( 1 + \Big( \frac1{\epsilon_0} \| \partial \bfg(t)\|_{C^{0, 0+}_x} \Big)^{\frac{n - 1}{2}}  \Big) \sum_{\theta, j} |a_{\theta, j}|^2.
        \end{align*}
        Integrating in time, we can control the right-hand side by the bootstrap $\| \partial \bfg \|_{L^2_t C^{0, 0+}_x} \lesssim \epsilon_0$ in dimensions $n = 3, 4, 5$, which yields the energy bound \eqref{parametrixenergy} for the spatial derivatives. We leave the time derivatives as an exercise. 
    \end{proof}

     \begin{proof}[Proof of Proposition \ref{prop:parametrix} (\ref{item:WPerror}) (error bounds)]
        Summing the error decompositions \eqref{errordecomp} of each wave packet, 
            \[
                \Box_{\bfg_{< \lambda}} \sfP_\lambda \phi 
                    = \mathcal L \big( \partial \bfg, \partial \widetilde{\sfP_\lambda} \widetilde \phi \big) + (\epsilon_0 \lambda)^{\frac12 \frac{n - 1}{2}} \lambda^{\frac12 - 1 - 1} \sfP_\lambda \sfT_{< \lambda} \sum_{\theta, j}  a_{\theta, j}\sum_{k = 0, 1, 2} \psi_k^{\theta, j} \delta^{(k)} (x_\theta - \tau_{\theta, j}),
            \]
        where $\widetilde \phi$ is another superposition of wave packets with the same coefficients. H\"older's inequality and the energy bound \eqref{parametrixenergy} tell us that the first term on the right is harmless, 
            \[
                \| \mathcal L (\partial \bfg, \partial \widetilde{\sfP_\lambda} \widetilde \phi) \|_{L^1_t L^2_x} 
                    \lesssim \| \partial \bfg \|_{L^2_t L^\infty_x} \| \partial \widetilde \phi \|_{L^\infty_t L^2_x} 
                    \lesssim \epsilon_0 \Big( \sum_{\theta, j} |a_{\theta, j}|^2 \Big)^{\frac12}.
            \]
        Applying the fixed-time orthogonality estimates \eqref{goodorthogonal}-\eqref{badorthogonal} to the remaining terms, setting 
            \[
                f(t) 
                    := (\epsilon_0 \lambda)^{\frac12 \frac{n - 1}{2}} \lambda^{\frac12 - 1 - 1} \sfP_\lambda \sfT_{< \lambda} \sum_{\theta, j}  a_{\theta, j}\sum_{k = 0, 1, 2} \psi_k^{\theta, j} \delta^{(k)} (x_\theta - \tau_{\theta, j})
            \]
        then
            \begin{align*}
                \Big\| f(t) \Big\|_{L^2_x}^2
                    \lesssim \Big( 1 + \Big( \frac1{\epsilon_0} \| \partial \bfg(t)\|_{C^{0, 0+}_x} \Big)^{\frac{n - 1}{2}}  \Big) \sum_{\theta, j} |a_{\theta, j}|^2 \sum_{k = 0, 1, 2} \lambda^{k - 1} \| \psi_k^{\theta, j} ((\epsilon_0 \lambda)^{-\frac12} x'_\theta) \|_{H^{\frac{n - 1}{2}+} (\R^{n - 1})}^2.
            \end{align*}
        Applying Cauchy-Schwartz in time, 
            \begin{align*}
                \| f\|_{L^1_{t} L^2_x}^2 
                    &\lesssim \Big( \int_{-2}^2 1 + \Big( \frac1{\epsilon_0} \| \partial \bfg(t)\|_{C^{0, 0+}_x} \, \Big)^{\frac{n - 1}{2}} \,dt  \Big) \\
                    &\qquad \times\left( \int_{-2}^2 \sum_{\theta, j} |a_{\theta, j}|^2 \sum_{k = 0, 1, 2} \lambda^{2(k - 1)}  \| \psi_k^{\theta, j} ((\epsilon_0 \lambda)^{-\frac12} x'_\theta) \|_{H^{\frac{n - 1}{2}+} (\R^{n - 1})}^2 \, dt \right)\\
                    &\lesssim \epsilon_0 \sum_{\theta, j} |a_{\theta, j}|^2,
            \end{align*}
        again using the $L^2_t C^{0, 0+}$-smallness of the metric and the error bounds \eqref{errorcoeff}. 
    \end{proof}

    \begin{remark}
        Again we needed to use
            \[
                \frac{n - 1}{2} \leq 2 \qquad \text{ if and only if } n \leq 5,
            \]
        to ensure that the ``bad'' times are good on average. 
    \end{remark}

\subsection{Matching wave packets to initial data}

It remains to show Proposition \ref{prop:parametrix} (\ref{item:WPdata})-(\ref{item:WPsize}); put loosely, any initial data $(\phi_0, \phi_1) \in (H^1 \times L^2)_x (\R^n)$ can be matched at time $t = -2$ to a superposition of wave packets. 




Maximal collection of $\theta$. We decompose 
    \[
        \phi[0] 
            = \sum_{\theta \in \Omega} \phi^\theta [0], 
    \]
where 
    \[
        \phi^\theta 
            := \frac12 \left( \phi_0^\theta (x + t \theta) + u_0 \right)
    \]



Fourier transform trick, $u_0^\omega$ compact support in frequency, take Fourier transform in $x_\theta$, then extend periodically the Fourier transform with period $\lambda \theta$, 
    \[
        \widehat{\phi^\theta} 
            = \sum_{k \in \Z} 
    \]

\begin{remark}
    Here we did not need to use the restriction on dimenions $n \leq 5$. 
\end{remark}