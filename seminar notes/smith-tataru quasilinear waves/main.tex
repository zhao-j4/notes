\documentclass[reqno]{amsart}

\usepackage{external/takodachi}

% To show labels in the margin:
\usepackage[notref, notcite]{showkeys}

\newcommand{\dS}{\mathrm{dS}}

% Custom tag item reference
\makeatletter
\newcommand{\myitem}[1]{%
\item[#1]\protected@edef\@currentlabel{#1}%
}
\makeatother

% kill subsections in ToC
\setcounter{tocdepth}{1} 

\title[Local well-posedness for quasilinear wave equations]
{
	Local well-posedness for quasilinear wave equations \\
	(d'apr\'es Smith-Tataru)
} 
\author{Jason Zhao}
%\address{Department of Mathematics, University of California, Berkeley, 94720}
%\email{zhao.j@berkeley.edu}
\date{\today}


\begin{document}

\begin{abstract}
	In this note, we outline the work by Smith-Tataru \cite{SmithTataru2005} concerning the sharp local well-posedness for generic quasi-linear wave equations. That is, given sufficiently regular Lorentzian metrics $\bfg_{\mu\nu} (\phi)$ and semi-linear terms $\cN(\phi)(\partial \phi, \partial \phi)$, we prove that the initial data problem 
		\begin{equation*}
			\begin{split}
				\Box_{\bfg(\phi)} \phi &= \cN(\phi)(\partial \phi, \partial \phi),\\
				(\phi, \partial_t \phi)_{|t = 0} 
					&= (\phi_0, \phi_1), 
			\end{split}
		\end{equation*}
	is locally well-posed in $H^s_x \times H^{s - 1}_x(\R^n)$ for $s > \frac{n}{2} + \frac12$ when $n = 3, 4, 5$. 
\end{abstract}
\maketitle

\tableofcontents

\section{Introduction}
When studying function spaces, such as Lorentz spaces or Sobolev spaces, it is useful to decompose a generic function into simpler pieces, and attempt to prove the desired results for each of those pieces. For example, functions in Lorentz spaces can be decomposed in \textit{physical space} into the sum of \textit{quasi-step functions}. Our approach in these notes will be to decompose into \textit{frequency-localised} pieces and  study the various ways these pieces sum. 

To that end, we construct a dyadic partition of unity as follows; let $\phi  \in C^\infty_c (\R^d)$ satisfy $0 \leq \phi \leq 1$ and 
\begin{align*}
	\phi(x) 
		:= 
		\begin{cases}
			1 , 				&|x| \leq 1.4, \\
			0, 				&|x| > 1.42. 
		\end{cases}
\end{align*}
Denote the dyadics by $2^\Z := \{ 2^n : n \in \Z \}$. For $N \in 2^\Z$, define $\psi, \psi_N, \phi_{\leq N} \in C^\infty_c (\R^d)$ to be 
	\[ \psi(x) := \phi(x) - \phi(2x), \qquad \psi_N (x) := \psi(x/N), \qquad \phi_{\leq N} (x) := \phi(x/N).  \]
Observe that $\sum_N \psi_N \equiv 1$ since pointwise it forms a telescoping sum. Given a tempered distribution $f \in \cS' (\R^d)$, we define its \emph{Littlewood-Paley projections} to frequencies $|\xi| \sim N$ and $|\xi| \lesssim N$ respectively by
	\begin{align*}
		\widehat{f_N} &= \widehat{P_N f}  = \psi_N \widehat f , \qquad
		\widehat{f_{\leq N}} = \widehat{P_{\leq N} f} = \phi_{\leq N} \widehat f.
	\end{align*}	
Define the Littlewood-Paley projections to frequencies $|\xi| \gtrsim N$ and $N \lesssim |\xi| \lesssim M$ respectively by 
	\[ f_{\geq N} = P_{\geq N} f = (1 - P_{\leq N}) f, \qquad f_{N \leq - \leq M} = P_{N \leq - \leq M} f = \sum_{N \leq K \leq M} P_K f. \]
The name ``projection'' is a bit of a misnomer; the multipliers $P_N$ fail to be true projections in the sense that by choosing smooth cutoffs in frequency space rather than sharp cutoffs, we have $P_N P_N \neq P_N$. Nevertheless, a slightly modified statement holds; define the \emph{fattened Littlewood-Paley projections} to frequencies $|\xi| \sim N$ and their corresponding multipliers by
	\[ \widetilde{P_N} := P_{\frac{N}{2}} + P_{N} + P_{2N},\qquad  \widetilde{\psi_N} := \psi_{\frac{N}{2}} + \psi_N + \psi_{2N}. \]
Since $\widetilde{\psi_N} \equiv 1$ on the support of $\psi_N$, it follows that $\widetilde{P_N} P_N = P_N$. Similarly, we can define the fattened projections to frequencies $|\xi| \lesssim N$ by 
	\[ \widetilde{P_{\leq N}} = P_{\leq 2N}, \qquad \widetilde{\phi_{\leq N}} := \phi_{\leq 2N}.  \]	

\begin{remark}
	By the Paley-Wiener theorem, the projections are analytic functions in physical space. Thus we can study the Littlewood-Paley projections pointwise without any philosophical consternation.
\end{remark}




\section{The bootstrap argument}

\cite{IfrimTataru2022}

\begin{theorem}
    Let $\phi^{(\lambda)}$
        \begin{enumerate}
            \item uniform bounds 
                \begin{equation}
                    \| \phi^{(\lambda)} [t]\|_{L^\infty_t (H^s \times H^{s -1 })_x} 
                        \lesssim \epsilon, 
                \end{equation}
            \item difference bounds 
                
        \end{enumerate}
\end{theorem}

Bootstrap assumptions, 
    \begin{enumerate}
        \item Uniform $(H^s \times H^{s - 1})_x$-bounds, 
            \begin{equation}
                \| \phi^{(\lambda)} [t]\|_{L^\infty_t (H^s \times H^{s -1 })_x} 
                    \leq 1, 
            \end{equation}
        \item Difference bounds
            \begin{equation}
                \| \psi^{(\lambda)} [t] \|_{L^\infty_t (H^1 \times L^2)_x} 
                    \leq N^{-s},
            \end{equation}
    \end{enumerate}

    \begin{equation}
        \| (\phi^{(\lambda)} - \phi^{(\mu)}) [t] \|_{L^\infty_t (H^1 \times L^2)_x} 
            \leq \epsilon c_\lambda
    \end{equation}

\section{Geometry and regularity of null hypersurfaces}

Let $u_\theta$ be the solution the eikonal equation initialised at $t = -2$, 
    \begin{equation}
        \begin{split}
        \bfg^{\mu\nu} \partial_\mu u \partial_\nu u 
            &= 0,\\
         u_{|t = -2}
            &= \theta \cdot x + 2. 
        \end{split}
    \end{equation}
We denote the level sets $u_\theta = r$, which are null hypersurfaces with respect to the metric $\bfg$, by $\Sigma_{\theta, r}$. Write $x_\theta := x \cdot \theta$ and let $x'_\theta$ be coordinates on the orthogonal complement $\theta^\perp$, so that $(x'_\theta, x_\theta)$ form an orthonormal coordinate system on $\R^n$. By the implicit function theorem, we can write these level sets as the graphs of functions $\tau_{\theta, r} \equiv \tau_{\theta, r} (t, x'_\theta)$. Thus we may parametrise $\Sigma_{\theta, r}$ by $(t, x'_\theta) \in [-2, 2] \times \R^{n - 1}$, and we can write 
    \[
        \Sigma_{\theta, r} 
            := \{ (t, x) \in [-2, 2] \times \R^n : x_\theta - \tau_{\theta, r} (t, x'_\theta) = 0 \}.
    \]

\begin{example}
    In the flat case $\bfg = \bfm$, we have the explicit solution 
        \[
            u_\theta (t, x) 
                = x \cdot \theta - t,
        \]
    so that the null hypersurfaces are hyperplanes given by 
        \[
            \Sigma_{\theta, r} 
                = \{ (t, x) : x_\theta - t = r \},
        \]
    where $r \in \R$ is a fixed constant (not radius! confusingly enough), and $\tau := t + r$. 
\end{example}

\subsection{Regularity of null hypersurfaces}

\begin{proposition}[Characteristic energy estimates]
    
        \begin{equation}
            \|\bfg - \bfm \|_{(\Sigma_{\theta, r})}  + \| \partial \bfg_{< \lambda} \|_{(\Sigma_{\theta, r})} + \lambda^{-1} \| \partial_x \partial \bfg_{< \lambda} \|_{(\Sigma_{\theta, r})}
                \lesssim \epsilon_2. 
        \end{equation}
\end{proposition}

\begin{proposition}[]
    
        \begin{equation}
            \| d \tau_{\theta, r} (t) - dt \|_{C^{1, 0+}_{x'_\theta}(\R^{n - 1})}
                \lesssim \epsilon_2 + \| \partial \bfg (t) \|_{C^{0, 0+}_x(\R^n)}.
        \end{equation}
\end{proposition}

\subsection{Geometry of light cones}

\begin{proposition}[Angle of null generators]
    Let $\theta, \omega \in \SS^{n - 1}$, then 
        \begin{equation}
            \bfL_\theta - \bfL_\omega 
                = (\theta - \omega) + o(|\theta - \omega|).
        \end{equation}
    Also, 
        \begin{equation}
            \langle \bfL_\theta, \bfL_\omega \rangle_{\bfg}
                = -\frac12|\theta - \omega|^2 + o(|\theta - \omega|^2).
        \end{equation}
\end{proposition}

\begin{proposition}[Separation of null geodesics]
    Let $\theta, \omega \in \SS^{n - 1}$, and fix $(t_1, x_1) \in [-2, 2] \times \R^n$. Denote $\gamma_\theta$ and $\gamma_\omega$ the null geodesics with data 
        \[
            \gamma_\theta(t_1) = \gamma_\omega(t_1) = x_1, \qquad \dot \gamma_\theta(t_1) \parallel \theta, \qquad \dot \gamma_\omega(t_1) \parallel \omega.
        \]
    Then 
        \begin{equation}
            \gamma_\theta (t) - \gamma_\omega(t) 
                = (t - t_1) (\theta - \omega) + o(|t - t_1| \cdot |\theta - \omega|).
        \end{equation}
\end{proposition}

\section{Paradifferential decomposition}

Decomposing the left-hand side via the Littlewood-Paley trichotomy, we schematically write the low-high, high-low, and high-high interactions of $\Box_{\bfg(\phi)}$ as 
        \[
                \Box_{\bfg(\phi)} \phi 
                    \approx \Box_{\bfg(\phi)_{< \lambda}} \phi_\lambda
                        +  \bfg(\phi)_\lambda \cdot \partial\partial \phi_{< \lambda}
                        + \sum_{\mu \geq \lambda} \bfg(\phi)_\mu \cdot \partial\partial \phi_{\mu} .
        \]
Placing the latter two on the right-hand side, we obtain the paradiagonal formulation of the equation \eqref{QNLW}
        \begin{equation}
               \Box_{\bfg(\phi)_{< \lambda}} \phi_\lambda
                   \approx \bfg(\phi)_\lambda \cdot \partial\partial \phi_{< \lambda}
                        + \bfg(\phi)_\lambda \cdot \partial\partial \phi_{< \lambda} +  \sum_{\mu \geq \lambda} \bfg(\phi)_\mu \cdot \partial\partial \phi_{\mu} + \cN(\phi)(\partial \phi, \partial \phi)_\lambda.
        \end{equation}
The right-hand side is perturbative, so 
	\begin{equation}\tag{LW} \label{eq:linearwave}
	\begin{alignedat}{2}
		\Box_{\bfg(\phi)} \psi 
			&= 0,\\
		(\psi, \partial_t \psi)_{|t = 0} 
			&= (\psi_0, \psi_1),
	\end{alignedat}
	\end{equation}
and its paradifferential counterpart 
	\begin{equation}\tag{PLW} \label{eq:paralinear}
	\begin{alignedat}{2}
		\Box_{\bfg(\phi)_{< \lambda}} \sfP_\lambda \psi 
			&= 0,\\
		(\psi, \partial_t \psi)_{|t = 0}
			&= (\sfP_{\lambda}\psi_0, \sfP_\lambda \psi_1),
	\end{alignedat}
	\end{equation} 


\begin{proposition}
    
        \begin{equation}
            \| \psi \|_{L^2_t L^\infty_x}   
                \lesssim \epsilon_0^{-\frac12} \lambda^{\sigma - 1} \|\psi[-2]\|_{H^1_x \times L^2_x},
        \end{equation}
\end{proposition}

\section{Wave packet parametrix}\label{sec:parametrix}


Let $\gamma \equiv \gamma(t)$ be a null geodesic, and let $\Sigma_{\theta, u}$ be the null surface containing $\gamma$,
    \[
        \Sigma_{\theta, u} 
            := \{ (t, x) \in [-2, 2] \times \R^n : x_\theta - \tau_{\theta, u} (t, x'_\theta) = 0  \}.
    \]

    \[
        \mathfrak w
            := (\epsilon_0 \lambda)^{\frac12\frac{n - 1}{2}} \lambda^{\frac12 - 1} \sfT_{< \lambda} \left( w \, \mathrm{dS}_{\Sigma_{\theta, u}} \right) 
    \]
where $w$ is a smooth bump function on $\Sigma_{\theta, u}$ localised at scale $(\epsilon_0 \lambda)^{-\frac12}$ about the null geodesic $\gamma$, i.e.  
    \[
        w = w_0 \big( (\epsilon_0 \lambda)^{\frac12} (x_\theta' - \gamma_\theta' (t)) \big)
    \]
for some $w_0 \in C^\infty_c (|x'| \leq 1)$. 

The direction $\theta \in \SS^{n - 1}$ varies over maximal collection of approximately $\epsilon_0^{\frac{n - 1}{2}} \lambda^{\frac{n - 1}{2}}$ unit vectors separated by at least $\epsilon_0^{\frac12} \lambda^{-\frac12}$

Decompose $\R^n$ into parallel tiling of rectangles, length $\lambda^{-1}$ in parallel $x_\theta$ and $(\epsilon_0 \lambda)^{-\frac12}$ in the remainin $x'_\theta$ directions. 

\begin{proposition}[Existence of wave packet parametrix]\label{prop:parametrix}
    Let $(\phi_0, \phi_1) \in (H^1 \times L^2)_x (\R^n)$ be initial data. Then, in dimensions $n= 2, 3, 4, 5$, there exists a superposition of wave packets 
        \[
            \phi 
                := \sum_{\theta, j} a_{\theta, j}  \, \mathfrak w^{\theta, j}
        \]
    which is an approximate solution to the paralinearised initial data problem in the sense that 
    \begin{enumerate}
        \item \label{item:WPdata}it matches the initial data at $t = -2$, 
            \begin{equation}
                \sfP_\lambda \phi[-2] = (\phi_0, \phi_1)
            \end{equation}
        \item \label{item:WPsize}the size of the coefficients is comparable to the size of the initial data, 
            \begin{equation}
                \Bigl(\sum_{\theta, j} |a_{\theta, j}|^2 \Bigr)^\frac12
                \lesssim \| \phi[0] \|_{(H^1 \times L^2)_x}.
            \end{equation}
        \item \label{item:WPenergy}the energy estimate holds, 
            \begin{equation}
                \| \partial \sfP_\lambda \phi \|_{L^\infty_t L^2_x}
                \lesssim \Bigl( \sum_{\theta, j} |a_{\theta, j}|^2 \Bigr)^\frac12
        \end{equation}
            
        \item \label{item:WPerror}the error on the right-hand side is small, 
             \begin{equation}
                \| \Box_{\bfg_{< \lambda}} \phi_\lambda||_{L^1_t L^2_x} 
                \lesssim \epsilon_0 \, \Bigl( \sum_{\theta, j} |a_{\theta, j}|^2 \Bigr)^\frac12. 
            \end{equation}
    \end{enumerate}

    
\end{proposition}


\subsection{Wave packets as approximate solutions}


    \begin{lemma}[Wave packet error decomposition]
        
    \end{lemma}

    \begin{lemma}[Energy estimate for $\mathfrak w$]
        Wave packets have $O(1)$-energy, 
            \begin{equation}\label{eq:WPenergy}
                \| \partial \sfP_\lambda \mathfrak w \|_{L^\infty_t L^2_x} 
                    \lesssim 1. 
            \end{equation}
    \end{lemma}

    \begin{proof}
        By construction, the wave packet has amplitude 
            \[
                \| \partial \sfP_\lambda \mathfrak w \|_{L^\infty_t L^2_x} 
                    \lesssim \lambda \cdot |\text{amplitude}| \cdot |\text{support}|^{\frac12} \lesssim 1 
            \]
    \end{proof}

    \begin{lemma}[Error estimate for $\mathfrak w$]
        Each wave packet has small error, 
            \begin{align}
                \| \Box_{\bfg_{< \lambda}} \sfP_\lambda \mathfrak w \|_{L^1_t L^2_x} 
                    &\lesssim \epsilon_0,\label{eq:L1error}\\
                \| \Box_{\bfg_{< \lambda}} \sfP_\lambda \mathfrak w \|_{L^2_{t, x}} 
                    &\lesssim \epsilon_0.\label{eq:L2error}
            \end{align}
    \end{lemma}
Obviously \eqref{L2error} is stronger than \eqref{L1error}, so we focus on proving an $L^2_{t, x}$-error estimate. We write 
    \begin{align*}
        \Box_{\bfg_{< \lambda}} \mathsf P_\lambda \mathfrak w 
            &= (\epsilon_0 \lambda)^{\frac12 \frac{n - 1}{2}} \lambda^{-\frac12 + 1} \Bigl( [\Box_{\bfg_{< \lambda}}, \sfP_\lambda \sfT_{<\lambda}] + \sfP_{\lambda} \sfT_{< \lambda} \Box_{\bfg_{<\lambda}} \Bigr) w \, \dS_{\Sigma} \\
            &= (\epsilon_0 \lambda)^{\frac12 \frac{n - 1}{2}} \lambda^{-\frac12 + 1}  [\Box_{\bfg_{< \lambda}}, \sfP_\lambda \sfT_{<\lambda}] w \, \dS_{\Sigma_{\theta, u}} \\
            &\qquad + (\epsilon_0 \lambda)^{\frac12 \frac{n - 1}{2}} \lambda^{-\frac12 + 1}\sfP_{\lambda} \sfT_{< \lambda} \left( \Box_{\bfg_{<\lambda}} w \cdot \dS_\Sigma + 2\overline{\bfg}_{< \lambda}^{\alpha \beta} \partial_\alpha w \cdot \partial_\beta \dS_{\Sigma} + w \cdot \, \Box_{\bfg_{< \lambda}} \dS_{\Sigma}\right)\\ 
            &=: \mathrm{I} + \mathrm{II} + \mathrm{III} + \mathrm{IV},
    \end{align*}
commuting $\Box$ with the frequency projections and then applying the product rule. Here we have denoted $\overline \bfg^{\alpha \beta}_{< \lambda} = \tfrac12 (\bfg^{\alpha \beta}_{< \lambda} + \bfg^{\beta \alpha}_{< \lambda})$. 

\begin{proof}[Estimating the term $\mathrm{I}$]
    Since the metric is cut-off to frequencies much lower than $\lambda$, the commutator clearly projects to frequencies $|\xi| \sim \lambda$. Thus, one can harmlessly insert fattened projections $\widetilde \sfP_{\lambda} \widetilde \sfT_{< \lambda}$ in front of the commutator. Furthermore, while two derivatives fall on the wave packet, standard commutator arguments\footnote{In a word, the principal symbol of the commutator is given by the Poisson bracket, so one can, to leading order, write $[\bfg(x), \chi(\nabla/\lambda)] \approx \{ \bfg(x), \chi(\xi/\lambda) \} \approx \partial_x \bfg \cdot \partial_\xi \chi(\xi/\lambda) \approx \tfrac1\lambda \partial_x \bfg$.} allow us to move one derivative onto the metric. In total, we can rewrite
        \[
            \mathrm{I} = [\bfg_{< \lambda}^{\alpha \beta}, \sfP_{\lambda} \sfT_{< \lambda}]\partial_\alpha \partial_\beta \widetilde{\mathfrak w} = \mathcal L (\partial \bfg, \partial \widetilde{\mathfrak w})
        \]
    for another wave packet $\widetilde{\mathfrak w}$ and some translation-invariant bilinear operator $\mathcal L(-, -)$ with finite-measure kernel. To estimate in $L^2_{t, x}$, we place $\partial \bfg$ in $L^2_t L^\infty_x$, gaining smallness from our bootstrap assumption, and $\partial \widetilde{\mathfrak w}$ in $L^\infty_t L^2_x$, in which it is unit size by construction \eqref{WPenergy}, yielding
        \[
            \| \mathrm I \|_{L^2_{t, x}}
                \lesssim \| \partial \bfg \|_{L^2_t L^\infty_x} \| \partial \widetilde{\mathfrak w} \|_{L^\infty_t L^2_x} \lesssim \epsilon_2. 
        \]
    Taking $\epsilon_2 \leq \epsilon_0$ is an acceptable contribution towards \eqref{L2error}. 
\end{proof}

\begin{proof}[Estimating the term $\mathrm{II}$]
    We are left to compute two derivatives of the bump function on $\R^{n - 1}$ localised to the null geodesic $\gamma$, 
        \begin{align*}
            \partial_\alpha \partial_\beta w
                = 
                \begin{cases}
                    O(\epsilon_0 \lambda) 
                        &\text{if two spatial derivatives},\\
                    O((\epsilon_0 \lambda)^{\frac12} \ddot \gamma)
                        &\text{if two time derivatives}, \\
                    O(\epsilon_0 \lambda \dot \gamma)
                        &
                \end{cases}
        \end{align*}
    Since $\|\ddot \gamma\|_{L^2_t} \lesssim \epsilon_1$ this is acceptable.
\end{proof}

\begin{proof}[Estimating the term $\mathrm{III}$]

\end{proof}

\begin{proof}[Estimating the term $\mathrm{IV}$]
    Surface measure is an approximate solution. 
        \begin{align*}
            \langle \Box_{\bfg_{< \lambda}} \dS_{\Sigma}, \varphi \rangle 
                &= \int_\Sigma \partial_\alpha \partial_\beta \left(\bfg_{< \lambda}^{\alpha \beta} \cdot \varphi\right) \mathrm{dS}_\Sigma \\
                &= \int_\Sigma \partial_\alpha \partial_\beta \bfg^{\alpha\beta}_{< \lambda} \cdot \varphi + \left(\partial_\alpha \bfg^{\alpha\beta}_{< \lambda} \cdot \partial_\beta \varphi + \partial_\beta \bfg^{\alpha \beta}_{< \lambda} \cdot \partial_\alpha \varphi\right) + \bfg^{\alpha \beta}_{< \lambda} \cdot \partial_\alpha \partial_\beta \varphi \, \dS_\Sigma\\
                &= \int_\Sigma \left(\bfg - \bfg_{> \lambda}\right)^{\alpha \beta} \partial_\alpha \partial_\beta \varphi \, \dS_\Sigma\\
                &= \int_{\R^n} \bfg_{< \lambda}^{\alpha \beta} \partial_\alpha \partial_\beta \varphi \, \sqrt{1 + \partial^\bfa \partial_\bfa \tau} \, dx' dt
        \end{align*}
    $(t, x') \mapsto (t, x', \tau(t, x'))$
\end{proof}




\subsection{Almost orthogonality of wave packets}


    \begin{equation}
        \| \partial \sfP_\lambda \phi \|_{L^2_{t, x}} 
            \lesssim \Big( \sum_{\theta, j} |a_{\theta, j}|^2 \Big)^{\frac12}. 
    \end{equation}

    \begin{lemma}[Orthogonality at ``good" $t$]
        
            \begin{equation}
                \| \phi(t) \|_{L^2_x}^2
                    \lesssim \sum_{\theta, j} \| \psi^{\theta, j}  \|_{H^{\frac{n - 1}{2} +} (\R^{n - 1})}^2
            \end{equation}
    \end{lemma}

    \begin{lemma}[Orthogonality at ``bad'' $t$]
        
            \begin{equation}
                \| \Phi(t) \|_{L^2_x}^2 
                    \lesssim \Big( \frac{1}{\epsilon_0} \| \partial \bfg (t)\|_{C^{0, \delta}_x} \Big)^{\frac{n - 1}{2}} \sum_{\theta, j} \| \psi^{\theta, j} \|_{H^{\frac{n - 1}{2} +} (\R^{n - 1})}^2
            \end{equation}
    \end{lemma}

    It would follow that 
        \begin{align*}
            \| \partial_x \sfP_\lambda \phi (t) \|_{L^2_x}^2 
                &\lesssim \Big( 1 + \Big( \frac1{\epsilon_0} \| \bfg(t)\|_{C^{0, \delta}_x} \Big)^{\frac{n - 1}{2}}  \Big) \sum_{\theta, j} |a_{\theta, j}|^2 \\
        \end{align*}
     

\subsection{Matching wave packets to initial data}

It remains to show Proposition \ref{prop:parametrix} (\ref{item:WPdata})-(\ref{item:WPsize}); put loosely, any initial data $(\phi_0, \phi_1) \in (H^1 \times L^2)_x (\R^n)$ can be matched at time $t = -2$ to a superposition of wave packets. 

\subsubsection*{Approximate solution for $\Box$}

Maximal collection of $\theta$. We decompose 
    \[
        \phi[0] 
            = \sum_{\theta \in \Omega} \phi^\theta [0], 
    \]
where 
    \[
        \phi^\theta 
            := \frac12 \left( \phi_0^\theta (x + t \theta) + u_0 \right)
    \]

\subsubsection*{Approximate solution for $\Box_{\bfg_{< \lambda}}$}

Fourier transform trick, $u_0^\omega$ compact support in frequency, take Fourier transform in $x_\theta$, then extend periodically the Fourier transform with period $\lambda \theta$, 
    \[
        \widehat{\phi^\theta} 
            = \sum_{k \in \Z} 
    \]

\section{Dispersive estimates}

The analysis in the previous sections tell us that the geometry of slabs is approximately that of Minkowski space. Thus, one can expect that the same harmonic analysis counting arguments used to prove Strichartz estimate (or, alternatively, Fourier restriction estimates) hold in this setting. 




\subsection{Dispersive decay}


    \[
        \dist(x_2, C_{P_1}^t) 
            = \inf_{\theta \in \SS^{n - 1}} |x_2 - \gamma_\theta (t_2)| 
    \]

    \[
        \delta u (P_1, P_2) 
            := \sup_{\theta \in \SS^{n - 1}} |u_\theta (P_2) - u_\theta (P_1)| 
    \]

\begin{lemma}[Properties of $\delta u$]
    The parameter $\delta u$ is negative if $P_2$ is inside the cone, positive in the exterior. Furthermore, $\delta u \approx \dist(x_2, C_1^t)$
\end{lemma}

    Let 
        \[
            \#_\lambda (\mathtt P_1, \mathtt P_2) 
                := \text{$\#$ of slabs at scale $\lambda$ containing $\mathtt P_1$ and $\mathtt P_2$.}
        \]

\begin{proposition}[Dispersive decay, I]
    The number of slabs at scale $\lambda$ containing a pair of points $\mathtt P_1 = (t_1, x_1)$ and $\mathtt P_2 = (t_2, x_2)$ is bounded by 
        \begin{equation}
            \#_\lambda (\mathtt P_1, \mathtt P_2) 
                \lesssim 
                \begin{cases} 
                    \left( \frac{\lambda}{\epsilon_0} \right)^{\frac{n - 1}{2}} \left(1 + \lambda \dist(x_2, C_1^{t_2})\right)^{\frac{n - 3}{2}} \left( 1 + \lambda |t_1 - t_2|\right)^{-\frac{n - 1}{2}}
                        & \text{if $\delta u \in I_1$},\\
                    \left( \frac{\lambda}{\epsilon_0} \right)^{\frac{n - 1}{2}} \left(1 + \lambda \dist(x_2, C_1^{t_2})\right)^{-1} 
                        & \text{if $\delta u \in I_2$},\\
                    0 
                        & \text{otherwise},
                \end{cases}
        \end{equation}
    where 
        \begin{align*}
            I_1 
                &:= \left\{   -4 \lambda^{-1} \leq \delta u(\mathtt P_1, \mathtt P_2) \leq \min (2 |t_1 - t_2| , C (\lambda\epsilon_0)^{-1} |t_1 - t_2|^{-1}) \right\}  \\
            I_2
                &:= \left\{ 2 |t_1 - t_2| \leq \delta u (\mathtt P_1, \mathtt P_2) \leq C (\epsilon_0 \lambda)^{-\frac12} \right\}. 
        \end{align*}
\end{proposition}

In actuality we will only need the worst case, i.e. $\delta u \in I_2$. 

\begin{corollary}[Dispersive decay, II]
    The number of slabs at scale $\lambda$ containing a pair of points $\mathtt P_1$ and $\mathtt P_2$ is bounded by 
        \begin{equation}
            \#_\lambda (P_1, P_2) 
                \lesssim \left(\frac{\lambda}{\epsilon_0} \right)^{\frac{n - 1}{2}} \big(\lambda |t_1 - t_2|\big)^{-1}
        \end{equation}
\end{corollary}

\begin{proof}[Proof for $\delta u < - 4 \lambda^{-1}$ case]
    There are no such slabs, 
        \[
            \#_\lambda (\mathtt P_1, \mathtt P_2) 
                = 0.
        \]
\end{proof}

\begin{proof}[Proof for $|\delta u| \leq 4 \lambda^{-1}$ and $|t_1 - t_2| \leq 2 \lambda^{-1}$ case]
    Here we use the trivial bound 
        \[
            \#_\lambda (\mathtt P_1, \mathtt P_2) 
                \lesssim \left(\frac{\lambda}{\epsilon_0} \right)^{\frac{n - 1}{2}}
        \]  
\end{proof}

\begin{proof}[Proof for $|\delta u| \leq 4 \lambda^{-1}$ and $|t_1 - t_2| > 2 \lambda^{-1}$ case]
    
\end{proof}

\begin{proof}[Proof for $4 \lambda^{-1} < \delta u \leq 2|t_1 - t_2|$ case]

\end{proof}


\subsection{Strichartz estimates}



Wave packets have size $(\epsilon_0 \lambda)^{\frac12\frac{n - 1}{2}} \lambda^{-\frac12}$

\begin{proposition}
    Let 
        \[
            \phi 
                := \sum_{\mathtt T \in \cT} a_{\mathtt T} \mathbb 1_{\mathtt T}  
        \]
    then 
        \begin{equation}
            \| \phi \|_{L^2_t L^\infty_x} 
                \lesssim (\epsilon_0 \lambda)^{- \frac12 \frac{n - 1}{2}} \lambda^{\frac12} \| a_{\mathtt T} \|_{\ell^2_{\mathtt T}}
        \end{equation}
\end{proposition}

We proceed by discretising the problem. Dividing $[0, 1]$ into $O(\lambda)$-many sub-intervals $I_j$ of length $2\lambda^{-1}$, we can find points $\mathtt P_j = (t_j, x_j)$ nearly maximising $|\phi|$ on each space-time region $I_j \times \R^n$. It follows that
    \begin{align*}
        \| \phi \|_{L^2_t L^\infty_x}
            &\lesssim \Big( \sum_j \int_{I_j} \|\phi(t) \|_{L^\infty_x}^2 \, dt \Big)^{\frac12} \lesssim \Big(\sum_j \lambda^{-1} |\phi(t_j, x_j)|^2 \Big)^{\frac12} \lesssim \lambda^{-\frac12}\sum_{T \in \cT} \Big( \sum_j |a_{\mathtt T}|^2 \cdot |\mathbb 1_{\mathtt T} (t_j, x_j)|^2 \Big)^{\frac12}.
    \end{align*}
After passing to an $O(\lambda)$ subset of points, we can choose $t_j$ to be $\lambda^{-1}$-separated and such that the inequality above continues to hold. Next, we dyadically decompose the sum over slabs $\mathtt T$ with respect to the size $N^{-\frac12}$ of the coefficients $a_T$, and similarly the sum over points $\mathtt P_j$ with respect to $L$ the number of slabs containing them; we denote the number of such points by $m(L)$. It follows that
    \begin{align*}
        \|\phi \|_{L^2_t L^\infty_x}
            &\lesssim \lambda^{-\frac12} \sum_{\substack{\substack{N \in 2^{\N} \\ N \lesssim (\frac{\lambda}{\epsilon_0})^{\frac{n - 1}2}} }} \quad \sum_{\substack{\mathtt T \in \cT \\ |a_{\mathtt T}| \sim N^{-\frac12}}}\Big(\sum_{\substack{L \in 2^\N \\ L \lesssim (\frac{\lambda}{\epsilon_0})^{\frac{n - 1}2}}}\quad \sum_{\substack{j \\ \text{$\# \{ \mathtt T \in \cT : \mathtt P_j \in \mathtt T \} \sim L$}}}  |a_{\mathtt T}|^2 \cdot |\mathbb 1_{\mathtt T} (t_j, x_j)|^2 \Big)^{\frac12} \\
            &\lesssim \lambda^{-\frac12}\Big(\sum_{L \lesssim (\frac{\lambda}{\epsilon_0})^{\frac{n - 1}2}}  m(L) \Big| \sum_{\substack{N \lesssim (\frac{\lambda}{\epsilon_0})^{\frac{n - 1}2} }} N^{-\frac12} \cdot L  \Big|^2\Big)^{\frac12} \\
            &\lesssim \lambda^{-\frac12} \sum_{N \lesssim (\frac{\lambda}{\epsilon_0})^{\frac{n - 1}2}} \Big(\sum_{L \lesssim (\frac{\lambda}{\epsilon_0})^{\frac{n - 1}2}} m(L)^2 N^{-1} L^2\Big)^{\frac12}
    \end{align*}
In the above, we needed only to sum over a finite range of scales, since each point lies in $O((\tfrac{\lambda}{\epsilon_0})^{\frac{n - 1}{2}})$-many slabs. This restricts us to a finite range of scales for $L$, and allows us to regard the contribution of slabs with small coefficients $|a_T| \lesssim (\tfrac{\epsilon_0}\lambda)^{\frac{n - 1}{2}}$ as $O(1)$. 

We conclude with a counting argument. To summarise notation and introduce new ones, 
    \begin{align*}
        N 
            &:= \text{size of coefficient},\\
        L 
            &:= \text{$\#$ of slabs},\\
        m(L)
            &:= \text{$\#$ of points intersecting $L$-many slabs},\\
        K
            &:= \text{$\#$ of pairs $(i,j)$ for which $\mathtt P_i, \mathtt P_j$ lie in a common slab w/ multiplicity},\\
        n(\mathtt T)
            &:= \text{$\#$ of points in slab $\mathtt T$}.
    \end{align*}
Observe that 
    \begin{align*}
        \sum_{\substack{\mathtt T \in \cT_N \\ n(\mathtt T) \geq 2}} |n(\mathtt T)|^2
            &\sim K,\\
        \sum_{\substack{\mathtt T \in \cT_N}} n(\mathtt T)
            &\sim m(L) \cdot L.
    \end{align*}
and Cauchy-Schwartz, assuming $|\cT_N| \sim N$, This is because $\sum |a_T|^2 \sim \sum_N \sum_{T \in \cT_N} N^{-1} \sim 1$ 
    \[
        \sum_{\substack{\mathtt T \in \cT_N \\ n(\mathtt T) \geq 2}} |n(\mathtt T)|^2 \gtrsim N^{-1} \Big(\sum_{\substack{\mathtt T \in \cT_N \\ n(\mathtt T) \geq 2}} n(\mathtt T) \Big)^2
    \]


    \[
        K   
            \lesssim \sum_{i ,j } \#_\lambda (\mathtt P_i, \mathtt P_j)
            \lesssim \left(\frac{\lambda}{\epsilon_0} \right)^{\frac{n - 1}{2}} \sum_{1 \leq i  < j \leq M} |t_i - t_j|^{-1} \lesssim m(L)  \left(\frac{\lambda}{\epsilon_0} \right)^{\frac{n - 1}{2}} \log \lambda
    \]

\bibliographystyle{alpha}
\bibliography{external/biblio}

\end{document}
