To identify the frequency interactions between the terms of the wave maps equation \eqref{wave}, we work instead with the paradifferential formulation, localising to frequencies $|\xi| \sim 2^k$. Decomposing each term in the non-linearity into Littlewood-Paley pieces $\bfS(\phi)_{k_1} \partial_\alpha \phi_{k_2}\partial^\alpha \phi_{k_3}$, we make the following dichotomy
	\begin{enumerate}
		\item high-high interactions, e.g. $k_2 \sim k_3 \gg k_1$, or the derivative terms have low frequencies, $k_1 \gg k_2, k_3$, 
		
		\item low-high-low interactions $k_2 \gg k_1, k_3$ or low-low-high interactions $k_3 \gg k_1, k_2$, i.e. one derivative term is high frequency.
	\end{enumerate}
The former are good interactions and can always be treated peturbatively, while for the latter one has to identify the non-peturbative part of the interaction. One also wants to use the geometry of the problem, namely $\bfS(\phi) \partial \phi = 0$. Paralinearising and adding to our paradifferential equation, we obtain 
	\[
		\Box \phi_k = - \bfA(\phi)^\alpha_{\ll k} \partial_\alpha \phi_k + G(\phi)
	\]
for some non-linear term $G(\phi)$ which we want to treat as perturbative and anti-symmetric matrices
	\[
		\bfA(\phi)^\alpha_{\ll k} := (\bfS(\phi) - \bfS(\phi)^\top)_{\ll k} \partial^\alpha \phi_{\ll k}.
	\]
	
\subsection{Renormalisation}	
Tao in his work \cite{Tao2001} on the small data problem for the case $\mathbb M = \mathbb S^k$ developed a renormalisation procedure which transforms the non-linearity into a perturbative non-linearity. One seeks a linear transformation $w_k = U_{< k} \psi_k$ to transform the linear paradifferential equation to the constant coefficient equation
	\[
		\Box w_k = \text{perturbative} .
	\]
Define the anti-symmetric matrix $\bfB$ by 
	\[
		B_k = (\bfS(\phi) - \bfS(\phi)^\top)_{< k - 10} \phi_k
	\]
then one wants to solve the ordinary differential equation
	\begin{equation}
		\begin{split}
			\frac{d}{dk} U_{< k}
				&= U_{< k} \bfB_k, \\
			\lim_{k \to - \infty} U_{< k}
				&= I.
		\end{split}
		\label{eq:ODE}
	\end{equation}	
We want the solution to have good $\sfS$ and $\sfN$-estimates, and approximately renormalise the equation $\bfA_\alpha = \nabla_\alpha \bfB$,
	\begin{equation}
			U_{< k}^\top \nabla_\alpha U_{< k} = \nabla_\alpha \bfB_{< k} - \int_{-\infty}^k [B_{k'}, U^\top_{< k'} , \nabla_\alpha U_{< k'}] dk'.
	\end{equation}
To get good estimates using this \emph{diffusion gauge}, one still needs smallness of the coefficients $\bfA(\phi)_{\ll k}$ in our paradifferential equation. In the case of the small data problem, this is no issue, however for large data we have to introduce a large \emph{frequency gap} $m$ to compensate.
	
\begin{proposition}[Gauge covariant $\sfS$-estimate]
	Let $\psi_k$ be a frequency-localised solution to the linear equation 
		\begin{equation}
			\Box \psi_k = 2 \bfA (\phi)_{< k - m}^\alpha \partial_\alpha \psi_k + G
			\label{eq:linwave}
		\end{equation} 
	where $\bfA(\phi)_{< k - m}^\alpha : I \times \R^2 \to \mathfrak{so} (\R^2)$ is the anti-symmetric matrix 
		\[
			\bfA(\phi)_{< k - m}^\alpha := \left(\bfS(\phi) - \bfS^\top (\phi)\right)_{< k - m}^\alpha \partial_\alpha \phi_{< k - m},
		\]
	and $\phi$ is a smooth wave map on $I$ with bounds 
		\begin{equation}
			||\phi||_{\underline{\mathsf E} [I]} + ||\phi||_{\underline{\mathsf X} [I]} + ||\phi||_{\sfS [I]} \leq \cF.
		\end{equation}
	There exists then a frequency gap $m \geq m(\cF) \geq 20$ of logarithmic growth $m(\cF) \sim \log \cF$ such that the following energy estimate holds
		\begin{equation}
			||\psi_k||_{\sfS[I]}
				\lesssim_\cF ||\psi_k [0]||_{\dot H^1 \times L^2} + ||G||_{\sfN}.\label{eq:renormest}
		\end{equation}	
\end{proposition}

\begin{proof}
	Standard energy estimates
		\begin{equation}
			||\psi_k||_{\underline{\mathsf E}[I]} \lesssim_\cF ||\psi_k [0]||_{\dot H^1 \times L^2} + 2^{\delta m} || G||_{\sfN [I]} + 2^{-\delta m} || \psi_k||_{\sfS[I]}.
		\end{equation}
\end{proof}