\documentclass[reqno]{amsart}

\usepackage{external/takodachi}

\usepackage{cancel}
\usepackage{tikz}
\usepackage{pgfplots}
\pgfplotsset{compat=1.18}
\usepackage{tensor}

% To show labels in the margin:
\usepackage[notref, notcite]{showkeys}

% kill subsections in ToC
\setcounter{tocdepth}{1} 

\title[Linear waves on extremal Reissner-Nordström spacetimes]
{
	Linear waves on extremal Reissner-Nordström spacetimes \\
	(in spherical symmetry)
} 
\author{Jason Zhao}
%\address{Department of Mathematics, University of California, Berkeley, 94720}
%\email{zhao.j@berkeley.edu}
\date{\today}


\begin{document}

\begin{abstract}
	In the series of works \cite{Aretakis2011a, Aretakis2011}, Aretakis showed that, for generic linear waves evolving on an extremal Reissner-Nordstr\"om background, ingoing null derivatives {do not decay} on the event horizon, or may even {grow polynomially}. The seed of the \textit{Aretakis instability} stems from the conservation of the \textit{Aretakis charges} along the event horizon. To pass to a growth mechanism, one must complement the conservation laws with a \textit{weak stability} in the form of suitable energy estimates and decay estimates. We provide a simplified account of these results by restricting to the spherically-symmetric setting. 
\end{abstract}
\maketitle

\tableofcontents

\section{Introduction}
When studying function spaces, such as Lorentz spaces or Sobolev spaces, it is useful to decompose a generic function into simpler pieces, and attempt to prove the desired results for each of those pieces. For example, functions in Lorentz spaces can be decomposed in \textit{physical space} into the sum of \textit{quasi-step functions}. Our approach in these notes will be to decompose into \textit{frequency-localised} pieces and  study the various ways these pieces sum. 

To that end, we construct a dyadic partition of unity as follows; let $\phi  \in C^\infty_c (\R^d)$ satisfy $0 \leq \phi \leq 1$ and 
\begin{align*}
	\phi(x) 
		:= 
		\begin{cases}
			1 , 				&|x| \leq 1.4, \\
			0, 				&|x| > 1.42. 
		\end{cases}
\end{align*}
Denote the dyadics by $2^\Z := \{ 2^n : n \in \Z \}$. For $N \in 2^\Z$, define $\psi, \psi_N, \phi_{\leq N} \in C^\infty_c (\R^d)$ to be 
	\[ \psi(x) := \phi(x) - \phi(2x), \qquad \psi_N (x) := \psi(x/N), \qquad \phi_{\leq N} (x) := \phi(x/N).  \]
Observe that $\sum_N \psi_N \equiv 1$ since pointwise it forms a telescoping sum. Given a tempered distribution $f \in \cS' (\R^d)$, we define its \emph{Littlewood-Paley projections} to frequencies $|\xi| \sim N$ and $|\xi| \lesssim N$ respectively by
	\begin{align*}
		\widehat{f_N} &= \widehat{P_N f}  = \psi_N \widehat f , \qquad
		\widehat{f_{\leq N}} = \widehat{P_{\leq N} f} = \phi_{\leq N} \widehat f.
	\end{align*}	
Define the Littlewood-Paley projections to frequencies $|\xi| \gtrsim N$ and $N \lesssim |\xi| \lesssim M$ respectively by 
	\[ f_{\geq N} = P_{\geq N} f = (1 - P_{\leq N}) f, \qquad f_{N \leq - \leq M} = P_{N \leq - \leq M} f = \sum_{N \leq K \leq M} P_K f. \]
The name ``projection'' is a bit of a misnomer; the multipliers $P_N$ fail to be true projections in the sense that by choosing smooth cutoffs in frequency space rather than sharp cutoffs, we have $P_N P_N \neq P_N$. Nevertheless, a slightly modified statement holds; define the \emph{fattened Littlewood-Paley projections} to frequencies $|\xi| \sim N$ and their corresponding multipliers by
	\[ \widetilde{P_N} := P_{\frac{N}{2}} + P_{N} + P_{2N},\qquad  \widetilde{\psi_N} := \psi_{\frac{N}{2}} + \psi_N + \psi_{2N}. \]
Since $\widetilde{\psi_N} \equiv 1$ on the support of $\psi_N$, it follows that $\widetilde{P_N} P_N = P_N$. Similarly, we can define the fattened projections to frequencies $|\xi| \lesssim N$ by 
	\[ \widetilde{P_{\leq N}} = P_{\leq 2N}, \qquad \widetilde{\phi_{\leq N}} := \phi_{\leq 2N}.  \]	

\begin{remark}
	By the Paley-Wiener theorem, the projections are analytic functions in physical space. Thus we can study the Littlewood-Paley projections pointwise without any philosophical consternation.
\end{remark}




\section{Preliminaries}
The starting point for the study of fractional differential and integral operators is the observation that the Fourier transform maps differentiation to multiplication by a monomial, namely 
	\[\widehat{\nabla u} (\xi) = 2\pi i \xi \widehat u (\xi).\]
We see from the formula above that differentiation \textit{accentuates} high frequencies and \textit{diminishes} low frequencies. Conversely, we expect integration to diminish high frequencies and accentuate low frequencies. The natural way to generalise the formula above is to replace the monomial $2\pi i \xi$ with fractional powers $|2\pi i \xi|^s$, where positive powers $s > 0$ correspond then to differential operators, and negative powers $s < 0$ to integrals. 


\subsection{Riesz potential}

For $s \in \R$ and $u \in \cS (\R^d)$, define the \emph{Riesz potential} to be the Fourier multiplier
	\[ \widehat{(|\nabla|^s u)} (\xi) := |2\pi i \xi|^s \widehat u(\xi).  \]	
The multiplier $|2\pi i \xi|$ is smooth, non-vanishing, and grows linearly at infinity on $\R^d \setminus 0$, however it is non-differentiable and vanishes at the origin. Thus it does not quite map the Schwartz space onto itself, so to remedy this, define the \emph{homogeneous Schwartz space} $\dot \cS (\R^d)$ as the space of Schwartz functions whose Fourier transforms vanish to infinite order at the origin,  
	\[ \dot \cS (\R^d) := \{ u \in \cS (\R^d) : \nabla^k \widehat u (0) = 0 \text{ for all $k$} \}. \]
The Riesz potentials map the homogeneous Schwartz space onto itself isomorphically, $|\nabla|^s : \dot \cS (\R^d) \to \dot \cS (\R^d)$, with the obvious inverse $|\nabla|^s |\nabla|^{-s} = \operatorname{Id}$. Furthermore, this subspace is still generic enough for us to apply density arguments, 

\begin{lemma}[Genericity of $\dot \cS (\R^d)$]
	Let $f \in \cS(\R^d)$ and $1 \leq p < \infty$, then there exists $\{ g_\epsilon\}_{\epsilon > 0} \subseteq \dot \cS (\R^d)$ such that $||f - g_\epsilon||_{L^p} \to 0$. In particular, $\dot \cS (\R^d)$ is dense in $L^p (\R^d)$. 
\end{lemma}

\begin{proof}
	Let $\phi \in C^\infty_c (\R^d)$ be a bump function satisfying $\phi \equiv 1$ for $|\xi| \leq 1$. Define 
		\[ \widehat{g_\epsilon} (\xi) := \widehat g (\xi) (1 - \phi(\xi/\epsilon)). \]
	By construction the Fourier transform of $g_\epsilon$ vanishes in a neighborhood of the origin, so in particular $g_\epsilon \in \dot \cS  (\R^d)$. Writing $g - g_\epsilon = g * \epsilon^d \phi(\epsilon - )$, it follows from Young's inequality and a change of variables $\epsilon x = y$ that 
		\[ ||g - g_\epsilon||_{L^p} \leq ||g||_{L^1} ||\epsilon^d \phi(\epsilon x)||_{L^p_x} \lesssim \epsilon^{d - \frac{d}{p}} ||\phi(y)||_{L^p_y} \overset{\epsilon \to 0}{\longrightarrow} 0,  \]
	as desired. 	
\end{proof}

For $-d < s < 0$, the multiplier $|2\pi i \xi|^s$ decays at infinity and is locally integrable at the origin, so $|\nabla|^s$ can be well-defined as an integral operator on $\cS (\R^d)$. To compute its kernel, we remark that $|2\pi i \xi|^s$ is homogeneous of order $s$, so in view of the Fourier transform, the corresponding kernel has homogeneity of order $-d - s$. 
	
\begin{proposition}[Kernel of Riesz potential]
	Let $0 < \alpha < d$, then 
		\[ \widecheck{|2\pi i \xi|^{-\alpha}} = 2^{-\alpha} \pi^{- d/2} \frac{\Gamma (\frac{d - \alpha}{2})}{\Gamma(\frac{\alpha}{2})} |x|^{\alpha - d}. \]
\end{proposition}

\begin{proof}
	The key observation is that $|\xi|$ can be expressed as a weighted integral in $t$ of Gaussians $\{  e^{-\pi t |\xi|^2}\}_{t > 0}$. Indeed, making the change of variables $u = \pi t|\xi|^2$, we compute
		\begin{align*}
			\int_0^\infty e^{-\pi t |\xi|^2} t^{\frac{\alpha}{2}} \frac{dt}{t}
				&= \int_0^\infty e^{-u} \left( \frac{u}{\pi |\xi|^2} \right)^{\frac{\alpha}{2}} \frac{du}{u}\\
				&= \frac{|\xi|^{-\alpha}}{\pi^{\frac{\alpha}{2}}} \int_0^\infty e^{-u} u^{\frac{\alpha}{2}} \frac{du}{u} = \Gamma \left(\frac{\alpha}{2}\right) \frac{|\xi|^{-\alpha}}{\pi^{\frac{d - \alpha}{2}}}.
		\end{align*}	
	Taking the inverse Fourier transform of the left-hand side formally, noting that the computation can be made rigorous in the sense of tempered distributions using Fubini's theorem, 
		\begin{align*}
			\widecheck{\int_0^\infty e^{-\pi t |\xi|^2} t^{\frac{\alpha}{2}} \frac{dt}{t}}
				&= \int_0^\infty e^{-\pi |x|^2/t} t^{\frac{\alpha - d}{2}} \frac{dt}{t} = \Gamma \left(\frac{d - \alpha}{2}\right) \frac{|x|^{\alpha - d}}{\pi^{\frac{\alpha}{2}}},
		\end{align*}
	where the first equality is the Fourier transform of Gaussians $\widecheck{e^{-\pi t |\xi|^2}} = e^{-\pi |x|^2/t} t^{-d/2}$ and the second equality is an application of our initial computation, replacing $\alpha$ with $d - \alpha$. Rearranging completes the proof. 
\end{proof}

\begin{remark}
	This computation furnishes the fundamental solution of the Laplacian for dimension $d \geq 3$. Indeed, Laplace's equation takes the form, in physical space and frequency space respectively, 
	\begin{align*}
		\Delta u
			&= f, \\
		4\pi^2 |\xi|^2 \widehat u
			&= \widehat f.	
	\end{align*}
	Thus, for example, in dimension $d = 3$, the solution is given by 
		\[ \widehat u (\xi) = \frac{1}{4\pi |\xi|^2} \widehat f(\xi) = \widehat{ \frac{1}{4\pi |x|} * f} (\xi). \]
\end{remark}

\begin{remark}
	The integral operator $\mathcal I_\alpha u := \widecheck{|2\pi i \xi|^{-\alpha}} * u$ is traditionally known as the \textit{Riesz fractional integration} operator. Inversely, we can view the positive order operator $|\nabla|^s$ for $s > 0$ as \textit{fractional differentiation} operator. 
\end{remark}



\subsection{Bessel potential}

For $s \in \R$ and $u \in \cS (\R^d)$, define the \emph{Bessel potential} to be the Fourier multiplier
	\[ \widehat{(\langle \nabla \rangle^s u)} (\xi) := \langle 2\pi i \xi \rangle^s \widehat u (\xi).  \]
The multiplier $\langle 2\pi i \xi \rangle = (1 + 4\pi^2 |\xi|^2)^{1/2}$ is smooth, non-vanishing, and grows linearly at infinity, so the Bessel potential maps Schwartz space onto itself isomorphically, $\langle \nabla \rangle^s : \cS (\R^d) \to \cS (\R^d)$ with the obvious inverse $\langle \nabla \rangle^s \langle \nabla \rangle^{-s} = \operatorname{Id}$. By construction, the Bessel potential treats high frequencies the same as the Riesz potential, however it has the advantage that low frequencies are better treated. 

\begin{proposition}[Kernel of Bessel potential]
	Let $\alpha > 0$, then 
		\[\widecheck{\langle 2\pi i \xi \rangle^{-\alpha}}  = \frac{1}{(4\pi)^{\frac\alpha2} \Gamma(\frac\alpha2)} \int_0^\infty  e^{- \frac{\pi |x|^2}{t}} e^{-\frac{t}{4\pi}}  t^{\frac{\alpha - d}{2}} \, \frac{dt}{t}. \]
\end{proposition}

\begin{proof}
	We follow our computation of the Riesz potential kernel, where recall we represented $|\xi|$ as an integral of Gaussians. Replacing $|\xi|$ with $(1 + 4\pi^2 |\xi|^2)^{1/2}$ in that computation gives
		\begin{align*}
			 \int_0^\infty e^{-\frac{t}{4\pi} (1 + 4\pi^2 |\xi|^2)} t^{\frac\alpha2} \frac{dt}{t} = \Gamma\left(\frac\alpha2\right)\frac{(1 + 4\pi^2 |\xi|^2)^{-\frac\alpha2} }{(4\pi)^{\frac\alpha2}} . 
		\end{align*}
	Taking the inverse Fourier transform of the left-hand side and rearranging gives the result. 
\end{proof}

\begin{proposition}[Asymptotics of Bessel potential kernel]
	Let $\alpha > 0$, then the kernel of the Bessel potential satisfies the following properties: 
	\begin{enumerate}
		\item It is non-negative, integrable, and unit mass,  
			\[ ||\widecheck{\langle 2\pi i \xi \rangle^{-\alpha}}||_{L^1} = \int_{\R^d} \widecheck{\langle 2\pi i \xi \rangle^{-\alpha}} \, dx = 1.  \]
			
		\item The growth at infinity is at most exponential, 
			\[ \widecheck{\langle 2\pi i \xi \rangle^{-\alpha}} (x) \lesssim e^{-|x|/2}, \qquad \text{uniformly in $|x| \geq 1$.} \]
		
		\item For $0 < \alpha < d$, the growth at zero is that of the Riesz potential kernel to leading order, 
			\[
				 \widecheck{\langle 2\pi i \xi \rangle^{-\alpha}} (x)= 2^{-\alpha} \pi^{- d/2} \frac{\Gamma (\frac{d - \alpha}{2})}{\Gamma(\frac{\alpha}{2})} |x|^{\alpha - d} + o(|x|^{\alpha - d}), \qquad \text{as $|x| \to 0$.}
			\] 
	\end{enumerate}
\end{proposition}

\begin{proof}
\leavevmode
\begin{enumerate}
	\item Non-negativity is obvious, $||\widecheck{\langle 2\pi i \xi \rangle^{-\alpha}}||_{L^1} = 1$ follows from Fubini's theorem, or remarking that its Fourier transform is precisely $1$ at the origin. 
	
	\item Observe that $\tfrac{\pi |x|^2}{t} + \tfrac{t}{4\pi} \geq \tfrac{t}{4\pi} + \tfrac\pi t$ and $\tfrac{\pi |x|^2}{t} + \tfrac{t}{4\pi} \geq |x|$ whenever $|x| \geq 1$. Averaging between the two inequalities furnishes 
		\[
			\widecheck{\langle 2\pi i \xi \rangle^{-\alpha}} (x)  \leq \left( \frac{1}{(4\pi)^{\frac\alpha2} \Gamma(\frac\alpha2)} \int_0^\infty  e^{-\frac{t}{8\pi} - \frac{\pi}{2t}}   t^{\frac{\alpha - d}{2}} \, \frac{dt}{t}\right) e^{-|x|/2}.
		\]
	\item Substituting the asymptotic expansion $e^{-t/4\pi} = 1 + o(e^{-t/4\pi})$ into the kernel and recalling our computation of the Riesz potential kernel gives the result. 
\end{enumerate}
\end{proof}



\section{Conservation laws on the event horizon}

The non-linear wave equation \eqref{NLW} is the Euler-Lagrange equation arising when one formally considers solutions as critical points of the Lagrangian
	\[
		\cS[\phi] 
			:= \int_{\R^{1 + d}} \frac12 \partial_\alpha \phi \partial^\alpha \phi + \frac{d - 2}{2d} |\phi|^{\frac{2d}{d - 2}} \, dt dx.
	\]
By Noether's principle, the continuous symmetries of the equation lead to conserved quantities. We present the energy-momentum tensor formalism, which is derived from the diffeomorphism-invariance of the Lagrangian. Define the \emph{energy-momentum tensor} by
	\[
		T_{\mu \nu} 
			= \partial_\mu \phi \partial_\nu \phi - m_{\mu \nu} \left( \frac12 \partial_\alpha \phi \partial^\alpha \phi + \frac{d - 2}{2d} |\phi|^{\frac{2d}{d - 2}}  \right).
	\]	
Observe that $T_{\mu \nu}$ is a symmetric $2$-tensor. Furthermore, if $\phi$ is a classical solution to \eqref{NLW}, the energy-momentum tensor is also divergence-free, 
	\[
		\nabla^\mu T_{\mu \nu} = 0.
	\]
We obtain energy identities for the non-linear wave equation by contracting the energy-momentum tensor with well-chosen vector fields, and then integrate over suitable space-time domains. Given a vector field $X$, we define its deformation tensor to be the Lie derivative of the metric with respect to $X$, i.e. $^{(X)} \pi = \cL_X \bfm$. In coordinates, 
	\[
		^{(X)} \pi_{\mu \nu} = \partial_\mu X_\nu + \partial_\nu X_\mu. 
	\]
Define the $1$- and $0$-currents
	\begin{align*}
		^{(X)} J_\mu [\phi]
			&:= T_{\mu \nu} [\phi] X^\nu, \\
		^{(X)} K[\phi]
			&:= T_{\mu \nu} [\phi] \left(\frac12 {^{(X)}} \pi^\# \right)^{\mu \nu}.
	\end{align*}	
Then, since $T$ is divergence-free,
	\begin{equation}\tag{$\nabla$}\label{eq:current}
		\partial^\mu \left({^{(X)}} J_\mu [\phi] \right) = {^{(X)}} K[\phi].
	\end{equation}	
Another way of deriving the divergence identity above would be to multiply the equation \eqref{NLW} by $X\phi$ and then integrating-by-parts.\footnote{Well, technically, ``differentiating-by-parts''} More generally, we can apply the same procedure after multiplying the equation by $w\phi$ for a scalar weight $w$. It follows that the \emph{generalised $0$- and $1$-currents}
	\begin{align*}
		^{(X)} J_\mu [\phi]
			&:=w \phi \partial_\mu \phi - \frac12 \partial_\mu w |\phi|^2, \\
		^{(X)} K[\phi]
			&:= w \partial_\mu \phi \partial^\mu \phi - \frac12 \Box w |\phi|^2 + \kappa |\phi|^{p + 1} w,
	\end{align*}	
satisfy the divergence identity
	\begin{equation}\tag{$\nabla'$}\label{eq:gcurrent}
		\partial^\mu \left({^{(w)}} J_\mu [\phi] \right) = {^{(w)}} K[\phi].
	\end{equation}	
	
	
\subsection{Energy conservation identities}
	
The simplest conservation law arises from the stationarity of the Minkowski metric, that is, the time-like vector field $T = \partial_t$ is a Killing vector field, $^{(T)} \pi = 0$. Contracting the energy-momentum tensor with $T$, the resulting $1$-current is precisely the energy density
	\[
		^{(T)} J_\mu [\phi]
			= \frac12 |\partial_t \phi|^2 + \frac12 |\nabla \phi|^2 + \frac{d - 2}{2d} |\phi|^{\frac{2d}{d - 2}} .
	\]
Integrating the divergence identity \eqref{current} with $X = T$ over the space-time slab $(t_0, t_1) \times \R^d$ and applying the divergence theorem yields the conservation of energy,	

\begin{proposition}[Conservation of energy]
	Let $\phi \in C^0_{t, \loc} \dot H^1_x \cap \dot C^1_{t, \loc} L^2_x (I \times \R^d)$ be a strong solution to \eqref{NLW}, then the energy is conserved, i.e. $\cE[\phi[t]] \equiv \cE$ for all $t \in I$. 
\end{proposition}
	
To make full use of the finite speed of propagation and small data theory, as we will soon see in Section \ref{sec:local}, we need to derive an energy conservation law which is local in space. Given an open subset $\Omega \subseteq \R^d$, define the local energy on $\Omega$ at time $t$ by
	\[
		\cE_\Omega [\phi[t]] 
			:= \int_\Omega \frac12 |\partial_t \phi|^2 + \frac12 |\nabla \phi|^2 +  \frac{d - 2}{2d} |\phi|^{\frac{2d}{d - 2}} \, dx.
	\]	
We also write $\cE_{S_t} [\phi] := \cE_{B_t} [\phi[t]]$. Then, integrating \eqref{current} over the slab of the light-cone $C_{[t_0, t_1]}$ and applying the divergence theorem, we relate the local energy on the time-slices of the light cone $S_{t_0}$ and $S_{t_1}$ modulo a flux through the null boundary $\partial C_{[t_0, t_1]}$. To compute this flux, we write the $1$-current in null coordinates, remarking that $T = \tfrac12 (L + \underline L)$, 
	\begin{align*}
		{^{(T)}} J_L [\phi]
			&= \frac12 |L\phi|^2 + \frac12 |\slashed \nabla \phi|^2 +\frac{d - 2}{2d} |\phi|^{\frac{2d}{d - 2}} ,\\
		{^{(T)}} J_{\underline L} [\phi]
			&= \frac12 |\underline L\phi|^2 + \frac12 |\slashed \nabla \phi|^2 +\frac{d - 2}{2d} |\phi|^{\frac{2d}{d - 2}} .	
	\end{align*}
Thus, the flux through the null boundary takes the form 
	\[
		\cF_{\partial C_{[t_0, t_1]}} [\phi]
			:= \int_{\partial C_{[t_0, t_1]}} \left( \frac12 |L \phi|^2 + \frac12 |\slashed \nabla \phi|^2 + \frac{d - 2}{2d} |\phi|^{\frac{2d}{d - 2}}  \right) dS.
	\]
In summary, we obtain the following local energy conservation law, 




\begin{proposition}[Energy-flux identity]\label{prop:energyflux}
	Let $\phi \in C^0_{t, \loc} \dot H^1_x \cap \dot C^1_{t, \loc} L^2_x (I \times \R^d)$ be a strong solution to \eqref{NLW}, then the local energy and flux obey the identity
		\[
			\cF_{\partial C_{[t_0, t_1]}} [\phi] := \cE_{S_{t_1}} [\phi] - \cE_{S_{t_0}} [\phi].
		\]
\end{proposition}

Observe that the flux through the null boundary is non-negative. This implies that the local energy $\cE_{S_t} [\phi]$ in the light cone is non-decreasing as $t \uparrow \infty$ and non-increasing as $t \downarrow 0$. Since we are working with finite energy solutions, it follows from monotone convergence that the limits
	\begin{align*}
		\cE_0
			&:=\lim_{t \downarrow 0} \cE_{S_t} [\phi],\\
		\cE_\infty
			&:= \lim_{t \uparrow \infty} \cE_{S_t} [\phi],
	\end{align*}
exist. The former will be relevant for our discussion of the blow-up scenario; we defer the discussion to Section \ref{sec:local}. For now, we observe that the existence of the limits implies that

\begin{corollary}[Flux decay property]\label{cor:fluxdecay}
	Let $\phi \in C^0_{t, \loc} \dot H^1_x \cap \dot C^1_{t, \loc} L^2_x (I \times \R^d)$ be a strong solution to \eqref{NLW}, then the flux through the light cone vanishes at the tip and at infinity, 
		\begin{align*}
			\lim_{t_0, t_1 \downarrow 0} \cF_{\partial C_{[t_0, t_1]}} [\phi]
				&= 0, \\
			\lim_{t_0, t_1 \uparrow \infty} \cF_{\partial C_{[t_0, t_1]}} [\phi] 
				&= 0.	
		\end{align*}
\end{corollary}

As a corollary, we can show that the local energy on the exterior annuli $B_{B_{3t} \setminus B_t}[\phi[t]]$ decays as $t\downarrow 0$, 

\begin{figure}[h]
	\begin{center}
		\includegraphics{graphics/exterior}
		\caption{We first choose, via flux decay, a time $t = t_*$ below which the flux is small, and then, via monotone convergence, $\sigma \ll 1$ such that the local energy in the annulus $B_{t_* + \sigma} \setminus B_{t_*}$ at time $t = t_*$ is small. The flux on the exterior cone has the correct sign, so we conclude from the energy-flux identity that the local energy in the annulus is small as $t \downarrow 0$. }
	\end{center}
\end{figure}

\begin{corollary}[Exterior energy decay]\label{cor:exterior}
	Let $\phi \in C^0_{t} \dot H^1_x \cap \dot C^1_{t} L^2_x ((0, 1] \times \R^d)$ be a finite energy local solution to \eqref{NLW}, then
		\[
			\lim_{\sigma \to 0} \limsup_{t \downarrow 0} \cE_{B_{t + \sigma} \setminus B_{t}} [\phi[t]] = 0.
		\]
	In particular, 
		\[
			\limsup_{t \downarrow 0} \cE_{B_{3t} \setminus B_{t}} [\phi[t]] = 0.
		\]	
\end{corollary}

\begin{proof}
	It is clear that for each $\sigma > 0$, we have $B_{3t} \setminus B_t \subseteq B_{t + \sigma} \setminus B_t$ for $t \ll \sigma$, which shows the latter decay statement is implied by the former. To prove the former, let $\epsilon > 0$, we choose from flux decay a time $t_* \ll 1$ such that
		\[
			\cF_{\partial C_{(0, t_*]}} [\phi] \ll \epsilon. 
		\]
	Then, considering the solution on the time slice $t = t_*$, it follows from monotone convergence that there exists $\sigma \ll 1$ such that 
		\[
			\cE_{B_{t_* + \sigma} \setminus B_{t_*}} [\phi[t_*]] \ll \epsilon.
		\]	
	It follows then from the energy-flux identity that, for $t \ll t_*$, 
		\begin{align*}
			 \cE_{B_{t + \sigma} \setminus B_{t}} [\phi[t]] 
			 	&= \cE_{B_{t + \sigma}} [\phi[t]] - \cE_{B_{t}} [\phi[t]] \\
			 	&= \left(\cE_{B_{t_* + \sigma}} [\phi[t_*]] - \cF_{- \sigma + \partial C_{[t, t_*]}} [\phi] \right) - \left( \cE_{B_{t_*}} [\phi[t_*]] - \cF_{\partial C_{[t, t_*]}} [\phi] \right).
		\end{align*}
	Since the flux is non-negative, we can throw away the first flux term, while by our choice of $t_* \ll 1$ and $\sigma \ll 1$ the exterior energy and flux at time $t = t_*$ is small,
		\[
			 \cE_{B_{t + \sigma} \setminus B_{t}} [\phi[t]] \leq  \cE_{B_{t_* + \sigma} \setminus B_{t_*}} [\phi[t_*]] + \cF_{\partial C_{(0, t_*]}} [\phi] \ll \epsilon.  
		\]
	This proves the result.
\end{proof}



\subsection{Monotonicity formula}

We derive a monotonicity formula for the non-linear wave equation arising from the scaling symmetry. The infinitesimal generator of this symmetry is given by $\Lambda = \partial_\rho + \tfrac1\rho$, so, after translating in time $t \mapsto t + \epsilon$ to handle the degeneracy at the light cone, we define 
	\begin{align*}
		X_\epsilon
			&= \frac{1}{\rho_\epsilon} ((t + \epsilon) \partial_t + r \partial_r), \\
		w_\epsilon
			&= \frac{d - 2}{2 \rho_\epsilon},	
	\end{align*}
where $\rho_\epsilon := \sqrt{(t + \epsilon)^2 - r^2}$. Observe that $X_0 = \tfrac1\rho S = \partial_\rho$, where $S$ is the scaling vector field. 
\begin{lemma}[Local monotonicity formula]
	Let $\phi$ be a smooth solution to \eqref{NLW} on an open subset $\cO \subseteq C_{(0, \infty)}$ of the forward light cone. Then the $1$-current defined in null coordinates by 
		\begin{align*}
			{^{(X_\epsilon)}} P_L [\phi]
				&= \frac12 \left( \frac{v_\epsilon}{u_\epsilon} \right)^{1/2} \left| r^{-\frac{d - 2}{2}} L \left( r^{\frac{d - 2}{2}} \phi \right)\right|^2 + \frac12 \left( \frac{u_\epsilon}{v_\epsilon} \right)^\frac12 \left( |\slashed \nabla \phi|^2 + \frac{(d - 2)^2}{4} \frac1{r^2} |\phi|^2 + \frac{d - 2}{d} |\phi|^{\frac{2d}{d - 2}} \right),\\
			{^{(X_\epsilon)}} P_{\underline L} [\phi]
				&= \frac12 \left( \frac{u_\epsilon}{v_\epsilon} \right)^{1/2} \left| r^{-\frac{d - 2}{2}} \underline L \left( r^{\frac{d - 2}{2}} \phi \right)\right|^2 + \frac12 \left( \frac{v_\epsilon}{u_\epsilon} \right)^\frac12 \left(  |\slashed \nabla \phi|^2 + \frac{(d - 2)^2}{4} \frac1{r^2} |\phi|^2 + \frac{d - 2}{d} |\phi|^{\frac{2d}{d - 2}}  \right),
		\end{align*}	
	setting the other components to zero, obeys the divergence identity
		\[
			\partial^\mu \left( {^{(X_\epsilon)}} P_\mu [\phi] \right) = \frac{1}{\rho_\epsilon} \left| \left( \partial_{\rho_\epsilon} + \frac1{\rho_\epsilon} \right)\phi \right|^2.
		\]	
\end{lemma}

\begin{proof}
	See Appendix \ref{app:A}
\end{proof}

Suppose the flux $\cF_{\partial C_{[t_0, t_1]}}[\phi]$ through the null boundary vanishes, then it follows that $\phi$ must also vanish along the null boundary. This kills the null boundary terms when we integrate the local monotonicity formula over the cone $C_{[t_0, t_1]}$ and apply the divergence theorem, thereby furnishing the identity
	\[
			\cM_{S_{t_1}} [\phi] + \iint_{C_{[t_0, t_1]}} \frac1\rho \left| \left( \partial_\rho + \frac1\rho \right) \phi\right|^2 \, dt dx = \cM_{S_{t_0}} [\phi],
	\]
for the weighted energy
	\begin{align*}
		\cM^\epsilon_{S_t} [\phi] 
			&:= \int_{S_t} {^{(X_\epsilon)}}P_T [\phi] \, dx \\
			&= \frac12 \int_{S_t} \left( \frac{v_\epsilon}{u_\epsilon} \right)^{1/2} \left| r^{-\frac{d - 2}{2}} \underline L \left( r^{\frac{d - 2}{2}} \phi \right)\right|^2 + \left( \frac{u_\epsilon}{v_\epsilon} \right)^{1/2} \left| r^{-\frac{d - 2}{2}} L \left( r^{\frac{d - 2}{2}} \phi \right)\right|^2 \\
			&\qquad + \left( \left( \frac{v_\epsilon}{u_\epsilon}\right)^\frac12  + \left( \frac{u_\epsilon}{v_\epsilon} \right)^\frac12 \right) \left(  |\slashed \nabla \phi|^2 + \frac{(d - 2)^2}{4} \frac1{r^2} |\phi|^2 + \frac{d - 2}{d} |\phi|^{\frac{2d}{d - 2}}  \right)	dx.
	\end{align*}
Since the second term is non-negative, this implies monotonocity of $\cM_{S_{t}}[\phi]$. In particular, the second term vanishes as $t_0, t_1 \to 0$, implying that rescalings of $\phi$ are asymptotically self-similar. On the other hand, the weights $(\tfrac{v}{u})^{1/2}$ blow-up at the null boundary, so there cannot be null concentration of energy. In the case where the flux does not vanish, we can still prove an "almost" monotonicity formula, leveraging the flux decay and a local Hardy's inequality to control the $\tfrac{|\phi|^2}{r^2}$ terms in the weighted energy. Before stating the almost monotonicity formula, we state and prove this Hardy decay. Define 
	\[
		\cG_{\partial S_t} [\phi]
			:= \frac1t \int_{\partial S_t} |\phi|^2.
	\]
This is well-defined for finite-energy solutions by the Sobolev trace theorem; in fact, $\phi \in H^{1/2}(\partial S_t)$.  

\begin{proposition}[Local Hardy's inequality]
	Let $\phi \in C^0_{t, \loc} \dot H^1_x \cap \dot C^1_{t, \loc} L^2_x (I \times \R^d)$, then
		\[
			\cG_{\partial S_{t_0}} [\phi] + \int_{t_0}^{t_1} \cG_{\partial S_t} [\phi] \frac{dt}{t} \leq \cG_{\partial S_{t_1}} [\phi] + \cF_{\partial C_{[t_0, t_1]}} [\phi].
		\]
	and 
		\[
			\cG_{\partial S_t} [\phi] \leq \cE_{\{t\} \times \R^d \setminus S_t} [\phi].
		\]		
\end{proposition}

\begin{proof}
	Standard proof of Hardy's inequality adapted to the null cone. 
\end{proof}

\begin{corollary}[Decay of Hardy term]
	Let $\phi \in C^0_{t, \loc} \dot H^1_x \cap \dot C^1_{t, \loc} L^2_x (I \times \R^d)$ be a strong solution to \eqref{NLW}, then $\cG_{\partial S_t} [\phi]$ vanishes as $t \to 0$ and $t \to \infty$,
		\begin{align*}
			\lim_{t \downarrow 0}\cG_{\partial S_t} [\phi]
				&= 0, \\
			\lim_{t \uparrow \infty}\cG_{\partial S_t} [\phi]
				&= 0.	
		\end{align*}	
\end{corollary}

\begin{proof}
	The local Hardy's inequality and finite energy implies that $\int \tfrac1t \cG_{\partial S_t} [\phi] < \infty$. We claim that if $\cG_{\partial S_t} [\phi]$ does not decay as $t\downarrow 0$ or $t\uparrow \infty$, then we can extract a uniform lower bound for either $t \ll 1$ or $t \gg 1$ respectively, a contradiction since $\int \tfrac1t$ diverges logarithmically. Suppose the $t \downarrow 0$ case fails; the case $t \uparrow \infty$ is similar, then local Hardy implies
		\[
			0 < \limsup_{t \downarrow 0} \cG_{\partial S_t} [\phi] \leq \cG_{S_t} [\phi] + \cF_{\partial C_{[0, t]}}.
		\]
	By flux decay, the flux term can be absorbed into the left-hand side for $t \ll 1$, proving the claim. 
\end{proof}

\begin{theorem}[Almost monotonicity formula I]\label{thm:monotone1}
	Let $\phi$ be a strong solution to \eqref{NLW} on $[\epsilon, 1] \times \R^d$ with negligible flux through the null boundary, finite energy and negligible Hardy term at $t = 1$, 
	\begin{align*}
		\cE_{S_1} [\phi]
			&\leq \cE, \\
		\cF_{\partial C_{[\epsilon, 1]}} [\phi]
			&\leq \epsilon^{1/2} \cE, \\
		\cG_{\partial S_1}[\phi]
			&\leq \epsilon^{1/2}\cE.
	\end{align*} 
	Then
		\[
			\cM_{S_{1}}^\epsilon [\phi] + \iint_{C_{[\epsilon, 1]}} \frac1{\rho_\epsilon} \left| \left( \partial_{\rho_\epsilon} + \frac1{\rho_\epsilon} \right) \phi\right|^2 \, dt dx \lesssim \cE.
		\]	
\end{theorem}

\begin{proof}
	Approximating a strong solution in the energy topology, it suffices to consider smooth solutions. We integrate the local monotonicity formula over the cone $C_{[\epsilon, 1]}$ and apply the divergence theorem, 
		\begin{align*}
			\cM_{S_{1}}^\epsilon [\phi] + \iint_{C_{[\epsilon, 1]}} \frac1{\rho_\epsilon} \left| \left( \partial_{\rho_\epsilon} + \frac1{\rho_\epsilon} \right) \phi\right|^2 \, dt dx 
				&=\cM_{S_{\epsilon}}^\epsilon [\phi] + \frac12 \int_{\partial C_{[\epsilon, 1]}} {^{(X_\epsilon)}}P_L [\phi] \, dS.
		\end{align*}
	We claim that the right-hand side is controlled by $\cE$.
	
	To control the weighted energy $\cM^\epsilon_{S_\epsilon} [\phi]$ on the time slice $S_\epsilon$, we first note that the weights can be treated as constants $|\tfrac{v_\epsilon}{u_\epsilon}| \sim |\tfrac{u_\epsilon}{v_\epsilon}| \sim 1$. It follows that we can estimate the integrand of the weighted energy $\cM^\epsilon_{S_\epsilon} [\phi]$ by the integrand of the local energy $\cE_{S_\epsilon} [\phi]$ and the Hardy term $\tfrac{|\phi|^2}{r^2}$, e.g.
		\begin{align*}
			\left(\frac{v_\epsilon}{u_\epsilon} \right)^{1/2} \left| r^{-\frac{d - 2}{2}} \underline L \left( r^{\frac{d - 2}{2}} \phi \right)\right|^2 + \left( \frac{u_\epsilon}{v_\epsilon} \right)^{1/2} \left| r^{-\frac{d - 2}{2}} L \left( r^{\frac{d - 2}{2}} \phi \right)\right|^2 
				&\lesssim  \left| r^{-\frac{d - 2}{2}} \underline L \left( r^{\frac{d - 2}{2}} \phi \right)\right|^2 + \left| r^{-\frac{d - 2}{2}} L \left( r^{\frac{d - 2}{2}} \phi \right)\right|^2 \\
				&\lesssim |\partial_t \phi|^2 + |\partial_r \phi|^2 + \frac{|\phi|^2}{r^2},
		\end{align*}
	and similarly for the other terms, so in total
		\[
			\cM^\epsilon_{S_\epsilon} [\phi] \lesssim \cE_{S_\epsilon} [\phi] + \int_{S_\epsilon} \frac{|\phi|^2}{r^2} \, dx \lesssim \cE
		\]	
	by finite energy, which controls $\cE_{S_\epsilon} [\phi]$, and negligible flux and Hardy term, which controls the right-hand side of the local Hardy's inequality. 

	To control the weighted flux term on the null boundary $\partial C_{[\epsilon, 1]}$, we observe that the weights are bounded above $|\tfrac{u_\epsilon}{v_\epsilon}| \leq 1$ and $|\tfrac{v_\epsilon}{u_\epsilon}| \lesssim \tfrac1\epsilon$ in this region. It follows that we can estimate the integrand of the weighted flux by the integrand of the usual flux $\cF_{[\epsilon, 1]}[\phi]$ and the Hardy term $\tfrac{|\phi|^2}{r^2}$, e.g.
		\begin{align*}
			^{(X_\epsilon)} P_L [\phi] \lesssim \epsilon^{-1/2} \left( |L\phi|^2 + \frac{|\phi|^2}{r^2} \right) + {^{(T)}} J_L [\phi],
		\end{align*}
	on the null boundary $\partial C_{[\epsilon, 1]}$, so integrating and applying the negligible flux and Hardy assumptions, 
		\[
			\int_{\partial C_{[\epsilon, 1]}} {^{(X_\epsilon)}} P_L [\phi] \, dS \lesssim \cE.
		\]	
	This completes the proof. 	
\end{proof}


\begin{corollary}[Almost monotonicity formula II]\label{thm:monotone2}
	Let $\phi$ be a strong solution to \eqref{NLW} on $[\epsilon, 1] \times \R^d$ satisfying the hypotheses of Theorem \ref{thm:monotone1}. Then
		\[
			\cM_{S^{\delta_1}_1} [\phi] \lesssim \cM_{S^{\delta_0}_{t_0}} [\phi] + \left( \left( \frac{\delta_1}{t_0}\right)^{1/2} + \frac{1}{|\log (\delta_1/\delta_0)|} \right) \cE
		\]	
	whenever $2\epsilon \leq \delta_0 \leq \delta_1 \leq t_0$. 	
\end{corollary}

\begin{proof}
	For any $\delta \in [\delta_0, \delta_1]$, we have that $S^{\delta_1}_1 \subseteq S^\delta_1$ and $S^{\delta_0}_{t_0} \subseteq S^{\delta}_{t_0}$. Thus, integrating the local monotonicity formula over the translated cone $C^\delta_{[t_0, 1]}$ and applying the divergence theorem, we see that it suffices to control the flux term 
		\[
			\int_{\partial C^\delta_{[t_0, 1]}} {^{(X)}} P_L [\phi] \, dS \lesssim  \left( \left( \frac{\delta_1}{t_0}\right)^{1/2} + \frac{1}{|\log (\delta_1/\delta_0)|} \right) \cE
		\]
	for appropriately chosen $\delta$. We control the term with $\tfrac{u}{v}$ weight by localised Hardy and local conservation of energy, 
		\begin{align*}
			\int_{\partial C^\delta_{[t_0, 1]}} &\left( \frac{u}{v} \right)^\frac12 \left( |\slashed \nabla \phi|^2 + \frac{(d - 2)^2}{4} \frac1{r^2} |\phi|^2 + \frac{d - 2}{d} |\phi|^{\frac{2d}{d - 2}} \right) dS\\
				 &\lesssim \left( \frac{\delta_1}{t_0} \right)^{1/2} \left( \cF_{\partial C^\delta_{[t_0, 1]}} [\phi] + \cE_{S_1 \setminus S_1^\delta} [\phi] + \cG_{S_1} [\phi]  \right) \lesssim  \left( \frac{\delta_1}{t_0} \right)^{1/2} \cE. 
		\end{align*}
	To treat the term with $\tfrac{v}{u}$ weight, we compute
		\[
			r^{-\frac{d - 2}{2}} L\left( r^{\frac{d- 2}{2}} \phi \right) = \left( L + \frac{d - 2}{2}\frac1r \right) \phi.
		\]	
	Finite energy plus pigeonhole principle. 	
\end{proof}

\section{Integrated local energy decay}


To motivate the local energy norm, let us consider the model case of the free wave equation $\Box \phi = 0$ with initial data given by a spherically symmetric Gaussian wave packet localised near the origin. From our heuristic discussion at the beginning of this note, we established the model estimate
	\[
		||  \nabla_{t, x} \phi||_{L^2_{t, x} (\R \times A_N)} \lesssim N^{1/2} ||\nabla_{t, x} \phi (0)||_{L^2_x},
	\]
where $A_N \subseteq \R^d$ denotes an annulus adapted to the dyadic scale $N \in 2^\N$, 
	\begin{align*}
		A_1
			&:= \{ x \in \R^d : |x| \leq 2\}, \\
		A_N
			&:= \{ x \in \R^d : N \leq |x| \leq 2N \}.
	\end{align*}
This motivates the introduction of the local energy norm and, to account the presence of forcing terms $\Box \phi = f$, its dual norm
	\begin{align*}
		||\psi||_{\LE_{t, x}}
			&:= \sup_{N \in 2^\N} ||\langle r \rangle^{-\frac12} \psi||_{L^2_{t, x} (\R \times A_N)},\\
		||f||_{\LE^*_{t, x}}
			&:= \sum_{N \in 2^\N} || \langle r \rangle^{\frac12} f||_{L^2_{t, x} (\R \times A_N)}.	
	\end{align*}
Throughout we will denote by $\langle -, - \rangle$ for the $L^2_x$-inner product. 	

\subsection{Morawetz estimate}

Our main tool for proving \eqref{iled} will be the positive commutator method. As a primer, we will use it to give a proof of the classical integrated local energy decay statement due to Morawetz \cite{Morawetz1997}. As a general principle, one generates estimates for the linear wave equation by making a judicious choice of multiplier, integrate-by-parts, and apply a duality argument to handle the inhomogeneous setting. For example, the \textit{Noether's theorem} tells us $\Box \phi X \phi$ can be written as a continuity equation for a conserved quantity when $X$ is an infinitesimal generator of a symmetry of $\Box$. For example, taking $X = \partial_t$ corresponding to the time-translation symmetry, 

\begin{proposition}[Energy identity]\label{lem:energy}
	For $\phi : [0, T] \times \R^d \to \C$ with sufficient regularity and decay, we have
		\[
			\frac12 ||\nabla_{t, x} \phi(T)||_{L^2_x}^2 = \frac12 ||\nabla_{t, x}\phi(0)||_{L^2_x}^2 - \int_0^T \int_{\R^d} \Box \phi \partial_t \phi \, dx dt. 
		\]
	In particular, $\phi$ obeys the energy estimate
		\[
			||\nabla_{t, x} \phi||_{L^\infty_t L^2_x} \lesssim ||\nabla_{t, x}\phi(0)||_{L^2_x} + ||\Box \phi||_{L^1_t L^2_x}.
		\]	
\end{proposition}

\begin{proof}
	We rewrite the product $\Box \phi \, \partial_t \phi$ in divergence form, 
		\begin{align*}
			\Box \phi \, \partial_t \phi 
				&= \left( - \partial_t^2 \phi + \sum_j \partial_j^2 \phi \right) \partial_t \phi \\
				&=	\partial_t \left( - \frac12 |\partial_t \phi|^2 \right) + \sum_j \partial_j (\partial_j \phi \partial_t \phi) - \partial_j \phi \partial_t \partial_j \phi \\
				&=\partial_t \left(-\frac12 |\partial_t \phi|^2 - \frac12 \sum_j |\partial_j \phi|^2 \right) + \nabla_x \cdot (\partial_t \phi \nabla_x \phi). 
		\end{align*}
	Integrating over the region $[0, T] \times \R^d$ and applying the divergence theorem, the boundary terms arising from the first term on the last line furnish the energy, while the second term vanishes provided sufficient decay at spatial infinity. This proves the energy identity. To prove the energy estimate, a duality argument and Cauchy's inequality imply that 
		\[
			\left|\int_0^t \int_{\R^d} \Box \phi \, \partial_t \phi \, dx dt\right| \leq ||\Box \phi||_{L^1_t L^2_x} ||\partial_t \phi||_{L^\infty_t L^2_x}  \leq 2||\Box \phi||_{L^1_t L^2_x}^2 + \frac14||\partial_t \phi||_{L^\infty_t L^2_x}^2,
		\]
	so the second term on the right can be absorbed into the left-hand side of the energy inequality. 
\end{proof}


For the \textit{positive commutator} argument, integration-by-parts is used to generate a commutator term, while one chooses the multiplier such that this commutator has positive sign. Suppose $X$ is stationary and anti-symmetric with respect to the $L^2_x$-inner product, then
	\[
		\langle X \phi, \Box \phi \rangle =  -\partial_t \langle X \phi, \partial_t \phi\rangle  + \frac12 \langle [X, \Delta] \phi, \phi \rangle,
	\]
since anti-symmetry allows us to write $\langle X\partial_t \phi, \partial_t \phi \rangle = 0$ and $2 \langle X \phi, \Delta \phi \rangle =\langle [\Delta, X] \phi, \phi \rangle$. We want to choose $X$ judiciously such that the commutator term has good sign. For example, one can choose the anti-symmetric operator $X = \partial_r + \tfrac{d - 1}{2r}$ and compute
	\[
		[\partial_r + \tfrac{d - 1}{2r}, \Delta] = - \frac2r \frac1{r^2} \slashed \Delta + \frac{(d - 1) (d - 3)}{2r} \frac{1}{r^2},
	\]
for $d \geq 4$; the modifications for the lower dimensional cases are left to the reader. Integrating-by-parts, we obtain

\begin{theorem}[Morawetz estimate]
	Let $\phi \in C^\infty_t \cS_x (\R \times \R^d)$ be a smooth solution to the free wave equation $\Box \phi = 0$, then 
		\[
			|| r^{-\frac32}  \slashed \nabla \phi ||_{L^2_{t, x}} + ||r^{-\frac32}\phi ||_{L^2_{t, x}}\lesssim ||\nabla_{t, x} \phi (0)||_{L^2_x}. 
		\]
\end{theorem}

\begin{proof}
	Since $X$ is anti-symmetric, multiplying the equation by $X\phi$ gives
		\[
			0 = \langle X \phi, \Box \phi \rangle = -\partial_t \langle X \phi, \partial_t \phi \rangle + \frac12 \langle [X, \Delta] \phi, \phi \rangle.
		\]
	Integrating the commutator term by parts gives control over $\slashed \nabla \phi$ and $\phi$ with the appropriate weights. Integrating in time and applying Cauchy-Schwartz, Hardy's inequality, and the energy identity, we conclude the estimate. 
\end{proof}

\begin{remark}
	Another proof using the energy-momentum tensor formalism can be found in the appendix of \cite{SterbenzRodnianski2005}.
\end{remark}

\begin{remark}
	When working with vector fields, recall that the commutator $[X, Y]$ is precisely the Lie derivative of $X$ along the flow of $Y$. In analogy with this picture, the principal symbol of the commutator $[X, \Box]$ is given by the derivative of the symbol of $X$ along the Hamilton flow of $\Box$. 
\end{remark}	



\subsection{\eqref{iled} for $\Box$}

Using the proof of the Morawetz estimate as inspiration, let us outline the strategy for proving the integrated local energy decay estimate. Again, we want to choose our multiplier $X$ to be stationary and anti-symmetric, in which case integration-by-parts leads us to 
	\[
		\langle X \phi, \Box \phi \rangle = - \partial_t \langle X \phi, \partial_t \phi \rangle + \frac12 \langle [X, \Delta] \phi, \phi \rangle.
	\]
Integrating in time and applying a duality argument,  
	\begin{equation}\tag{*}\label{eq:time}
		\begin{split}
		 \langle [X, \Delta] \phi, \phi \rangle 
		 	&\lesssim  ||X \phi (0)||_{L^2_x} ||\partial_t \phi (0)||_{L^2_x} +  ||X \phi (T)||_{L^2_x} ||\partial_t \phi (T)||_{L^2_x} \\
		 	&\qquad + ||X\phi||_{\LE_{t,x}\cap L^\infty_t L^2_x} ||\Box\phi||_{\LE_{t, x}^* + L^1_t L^2_x}.
		 \end{split}
	\end{equation}
Aiming towards \eqref{iled}, we want to choose $X$ such that it is bounded from $\dot H^1_x (\R^d)$ to $L^2_x (\R^d)$, bounded from $\nabla_x \LE_{t, x} \cap r^{-1} \LE_{t,x}$ to $\LE_{t, x}$, and has positive commutator with the Laplacian, 
	\begin{align}
		||X \phi||_{L^2_x}
			&\lesssim ||\nabla_x \phi||_{L^2_x}, \tag{L2}\label{eq:boundl2}\\
		||X \phi||_{\LE_{t, x}}
			&\lesssim ||\nabla_{t, x} \phi||_{\LE_{t, x}} + ||r^{-1} \phi||_{\LE_{t, x}},\tag{LE}\label{eq:boundle}\\	
		\langle [X, \Delta] \phi, \phi \rangle
			&\gtrsim ||\nabla_{t, x} \phi||_{\LE_{t, x}}^2 + ||r^{-1} \phi||_{\LE_{t, x}}^2\tag{C}\label{eq:boundpositive} .	
	\end{align}	
Indeed, inserting the inequalities above into \eqref{time}, we obtain
	\begin{align*}
		 ||\nabla_{t, x} \phi||_{\LE_{t, x}}^2 + ||r^{-1} \phi||_{\LE_{t, x}}^2 
		 	&\lesssim ||\nabla_{t, x}\phi(0)||_{L^2_x}^2 + ||\nabla_{t, x} \phi(T)||_{L^2_x}^2 \\
		 	&\qquad+ \left( ||\nabla_{t, x} \phi||_{\LE_{t, x} \cap L^\infty_tL^2_x} + ||r^{-1} \phi||_{\LE_{t, x} \cap L^\infty_t L^2_x}  \right) ||\Box \phi||_{\LE^*_{t, x} + L^1_t L^2_x}.
	\end{align*}
	In view of the energy identity from Lemma \ref{lem:energy} and duality, the $\LE_{t, x}$-norms on the left-hand side can be replaced by $\LE_{t, x} \cap L^\infty_t L^2_x$-norms, while on the right-hand side the energy at time $t = T$ can be controlled by the energy at time $t = 0$ and the last term on the right. Thus, we can write
	\begin{align*}	 
		 ||\nabla_{t, x} \phi||_{\LE_{t, x} \cap L^\infty_t L^2_x}^2 + ||r^{-1} \phi||_{\LE_{t, x} \cap L^\infty_t L^2_x}^2	
		 	&\lesssim  ||\nabla_{t, x} \phi(0)||_{ L^2_x}^2 \\
		 		&\qquad+ \left( ||\nabla_{t, x} \phi||_{\LE_{t, x} \cap L^\infty_tL^2_x} + ||r^{-1} \phi||_{\LE_{t, x} \cap L^\infty_t L^2_x}  \right) ||\Box \phi||_{\LE^*_{t, x} + L^1_t L^2_x}\\
		 	&\lesssim ||\nabla_{t, x}\phi (0)||_{L^2_x}^2 + \frac{2}{\delta} ||\Box \phi||_{\LE^*_{t, x} + L^1_t L^2_x}^2 \\
		 		&\qquad +  \delta \left( ||\nabla_{t, x} \phi||_{\LE_{t, x} \cap L^\infty_t L^2_x} + ||r^{-1} \phi||_{\LE_{t, x} \cap L^\infty_tL^2_x}  \right)^2 ,
	\end{align*}
using Cauchy's inequality. The choice of $\delta > 0$ here is arbitrary, so taking $\delta \ll 1$ we can absorb the final term on the second line into the left-hand side to conclude

\begin{theorem}[\eqref{iled} for $\Box$]
	The integrated local energy decay estimate \eqref{iled} holds for the d'Alembertian $P = \Box$. 
\end{theorem}

It remains to choose $X$ satisfying the boundedness properties \eqref{boundl2} and \eqref{boundle}, and the positivity of the commutator \eqref{boundpositive}. The multiplier $X = \partial_r + \tfrac{d - 1}{2r}$ is a good start for establishing \eqref{boundpositive}, however it does not furnish control over the radial derivatives or time derivatives, so some suitable modifications need to be made. In fact, the anti-symmetry of $X$ and the estimate \eqref{boundpositive} should not be taken too literally; as we detail the proof, some suitable substitutes will be introduced. Many computations will be omitted for brevity; to see details, we refer to \cite{MetcalfeSogge2006}.


\begin{proof}[``Proof'' of positive commutator estimates \eqref{boundpositive}]
	Our goal is to construct an anti-symmetric operator $X$ which has positive commutator with $\Delta$. Fix a smooth radial function $\alpha(r)$ to be chosen later, and set 
	\[
		X = \alpha \frac{x^j}{r} \partial_j + \partial_j \frac{x^j}{r} \alpha. 
	\]
The commutator is given by (exercise!)
	\[
		[X, \Delta] = -  \partial_k \frac{x^k}r \alpha' \frac{x^j}{r} \partial_j -  \partial_\ell \left( \delta^{\ell k} - \frac{x^k x^\ell}{r^2} \right) \frac{\alpha}{r} \left( \delta^{jk} - \frac{x^j x^k}{r^2} \right) \partial_j - (\Delta \partial_j) \left( \frac{x^j}{r} \alpha \right).
	\]
Integrating by parts, 
	\begin{equation}\label{eq:positive}\tag{$\dagger$}
		\langle [X, \Delta] \phi, \phi \rangle 
			=  \int_{\R^d} \left(   \alpha' |\partial_r \phi|^2 + \frac{\alpha}{r} \frac{1}{r^2} |\slashed \nabla \phi|^2 -  (\Delta \partial_j) \left( \frac{x^j}{r} \alpha \right) |\phi|^2 \right) dx. 
	\end{equation}
We want to choose $\alpha$ such that we have non-negative terms $\alpha, \alpha', (\Delta \partial_j) (\tfrac{x^j}{r} \alpha) > 0$ which behave like the weights $\tfrac{\alpha}{r}, \alpha' \approx r^{-1}$ and $(\Delta \partial_j) (\tfrac{x^j}{r} \alpha)  \approx r^{-3}$. Fix a dyadic integer $N > 0$, then set
	\[
		\alpha(r) := \frac{r}{r + N}. 
	\]
We compute	
	\begin{align*}
		\alpha'(r)
			&= \frac{N}{(r + N)^2},\\
		-(\Delta \partial_j) \left( \frac{x^j}{r} \alpha \right)	
			&= -\frac{1}{r^{d - 1}} \partial_r r^{d - 1} \partial_r \left( (d - 1) \frac{\alpha}{r} + \alpha' \right) \\
			&= \frac{1}{r (N + r)^3} \left(  (d - 3) r + 3(d - 3) \frac{Nr}{N + r} + \frac{3 N^2 (d - 1)}{N + r} \right).
	\end{align*}
We see that the weights are non-negative and have the desired size on the dyadic annulus $A_N$, 
	\begin{align*}
		\frac{\alpha(r)}{r}  \sim \alpha'(r)
			&\sim N^{-1}, \qquad \text{when $r \sim N$,}\\
		-(\Delta \partial_j) \left( \frac{x^j}{r} \alpha \right)	
			&\sim N^{-3},  \qquad \text{when $r \sim N$}.
	\end{align*}
The non-negativity allows us to restrict our attention to the dyadic annulus, in which case we can bound the commutator \eqref{positive} below by 
	\[
		\langle [X, \Delta] \phi, \phi \rangle \gtrsim || r^{-\frac12} \nabla_x \phi||_{L^2_x (A_N)}^2 +|| r^{-\frac12} r^{-1}  \phi||_{L^2_x (A_N)}^2.
	\]
	This estimate is uniform in $N$, so we can insert this into the left-hand side of \eqref{time} and take the supremum over $N$ to recover the full local energy norm of $\nabla_x \phi$ and $r^{-1} \phi$. 
\end{proof}

\begin{remark}
	The Morawetz multiplier $X = \partial_r + \tfrac{d - 2}{r}$ corresponds to the choice $\alpha(r) = \tfrac1{2r}$. 
\end{remark}

\begin{proof}[Modifying the proof of \eqref{boundpositive} to control $\partial_t$]
	Our proof of the positive commutator bound does not control the time-derivatives of $\phi$, since for anti-symmetric multipliers $\langle X \partial_t \phi, \partial_t \phi \rangle = 0$. On the other hand, if we were to choose a positive symmetric multiplier, then one could hope to recover the full estimate provided the lower-order error terms can be handled. Let $\beta(r)$ be a smooth radial function to be chosen later, set
		\[
			Y := X + \beta. 
		\]
	Then multiplying the equation by $Y\phi$, we obtain 
		\begin{align*}
			\langle Y \phi, \Box \phi \rangle 
				&= - \partial_t \langle X \phi, \partial_t \phi \rangle + \frac12 \langle [X, \Delta] \phi, \phi \rangle \\
				&\qquad - \partial_t \langle \beta \phi, \partial_t \phi \rangle + \langle \beta \partial_t \phi, \partial_t \phi \rangle + \langle \beta \phi, \Delta \phi \rangle. 
		\end{align*}
	Evidently the term $\langle \beta \partial_t \phi, \partial_t \phi \rangle = \int \beta |\partial_t \phi|^2$ gives the desired control over the time derivatives, provided we choose $\beta > 0$ and $\beta \approx r^{-1}$. However, we must contend with the final term having bad sign, indeed, integrating-by-parts and using symmetry, 
		\begin{equation}\tag{$\dagger \dagger$}\label{eq:bad}
		\begin{split}
			\langle \beta \phi, \Delta \phi \rangle 
				&= \langle - \beta \nabla \phi, \nabla \phi \rangle + \langle - \nabla \beta \phi, \nabla \phi \rangle \\
				&= -\langle  \beta \nabla \phi, \nabla \phi \rangle + \frac12 \langle \Delta \beta \phi,  \phi \rangle \\
				&=  \int_{\R^d} -\beta |\partial_r \phi|^2 - \frac{\beta}{r^2} |\slashed \nabla \phi|^2 + \frac12 \Delta \beta |\phi|^2 \, dx. 
		\end{split}
		\end{equation}
	We choose $\beta$ carefully such that the contributions of bad sign to \eqref{bad} are dominated by the good signs arising from the positive commutator identity \eqref{positive}. Choose for example
		\[
			\beta(r) := \frac12\alpha'(r)= \frac12\frac{N}{(r + N)^2},
		\]	
	then the coefficient in front of $|\partial_r \phi|^2$ is killed, the coefficient for $\tfrac1{r^2} |\slashed \nabla \phi|^2$ is dominated since $\alpha' < \tfrac\alpha r$, and the coefficients for $|\phi|^2$ are related by 
		\[
			-\frac12 (\Delta \partial_j) \left( \frac{x^j}{r} \alpha \right) = -\frac12 \Delta \beta - \frac{d - 1}2 \Delta \left( \frac{\alpha}{r} \right),
		\]
	so it suffices to show that $\Delta \tfrac{\alpha}{r} < 0$. Indeed, 
		\[
			\Delta \left( \frac{\alpha}{r} \right) = \frac{- (d - 3) r - (d - 1) N}{r (r + N)^3}. 
		\]	
	This shows that sign is not an issue. Furthermore, $\beta$ has the correct size on the dyadic annulus $A_N$, 
		\[
			\beta \sim N^{-1}, \qquad \text{when $r \sim N$}.
		\]	
	The non-negativity allows us to restrict our attention to the dyadic annulus, in which case we can bound from below
		\[
			\langle \beta \partial_t \phi, \partial_t \phi \rangle + \frac12 \langle [X, \Delta ] \phi, \phi \rangle + \langle \beta \phi, \Delta \phi \rangle \gtrsim ||r^{-\frac12} \partial_t \phi ||_{L^2_x (A_N)}^2. 
		\]	
	This estimate is uniform in $N$, taking the supremum over $N$ recovers the local energy norm of $\partial_t \phi$.
\end{proof}
	

\begin{proof}[Proof of $L^2$-bounds \eqref{boundl2} and $\LE_{t, x}$-bounds \eqref{boundle} and concluding \eqref{iled}]
	To control the terms $X\phi$ and $\beta \phi$, observe that the pointwise bounds on $\alpha$ and $\alpha'$ imply 
		\[
			|X\phi| + |\beta \phi| \lesssim |\nabla_x \phi| + \frac1r |\phi|. 
		\]
	The $\LE_{t,x}$-bounds follow immediately by definition, 
		\[
			||X\phi||_{\LE_{t, x}} + ||\beta\phi||_{\LE_{t, x}} \lesssim ||\nabla_x \phi||_{\LE_{t, x}} + ||r^{-1} \phi||_{\LE_{t, x}},
		\]	
	while one needs an application of Hardy's inequality to conclude the $L^2$-bounds,
		\[
			||X\phi||_{L^2_x} + ||\beta \phi||_{L^2_x}\lesssim ||\nabla_x \phi||_{L^2_x} + ||r^{-1} \phi||_{L^2_x} \lesssim ||\nabla_x \phi||_{L^2_x}.
		\]
	Collecting all our bounds, this completes the proof of \eqref{iled}.	
\end{proof}

\begin{remark}
	The integrated local energy decay estimate as stated in \eqref{iled} fails in dimensions $d = 1, 2$, though from our proof it is clear that one simply has to drop the Hardy term $||\langle r \rangle^{-1} \phi||_{\LE_{t, x}}$ to recover an admissible estimate. 
\end{remark}

\subsection{\eqref{iled} for small perturbations of $\Box$}\label{subsec:small}

To illustrate the robustness of the integrated local energy decay estimates, we prove the analogous results for lower-order perturbations of the wave operator. In the case where the decay condition \eqref{decay} for sufficiently small $\kappa \ll 1$, these perturbations can be easily absorbed by the left-hand side of \eqref{iled} for $\Box$. 

	
\begin{theorem}[\eqref{iled} for $P$]
	Let $L := - \Delta + b^j \partial_j + c$ be a lower-order perturbation of the Laplacian. Suppose the coefficients $b^j$ and $c$ satisfy the decay condition \eqref{decay} for sufficiently small $\kappa \ll 1$, i.e.
	\[
		\sum_{N \in 2^\N} \sup_{\R \times A_N} \langle r \rangle |b| + \langle r \rangle^2 |\partial_j b^j| + \langle r \rangle^2 |c| \leq \kappa \ll 1,
	\]
	then the wave operator $P := - \partial_t^2 - L$ satisfies \eqref{iled}.
\end{theorem}

\begin{proof}
	We want to treat the lower-order terms perturbatively, so to that end we write $\Box \phi =: P\phi + B\phi$ and apply \eqref{iled} for $\Box$ along with the triangle inequality to obtain 
		\begin{align*}
			||\nabla_{t, x} \phi||_{\LE_{t, x} \cap L^\infty_t L^2_x} + ||\langle r \rangle^{-1} \phi||_{\LE_{t, x} \cap L^\infty_t L^2_x} 
				&\lesssim ||\nabla_{t, x} \phi(0)||_{L^2_x} + ||P\phi||_{\LE^*_{t, x} + L^1_t L^2_x} + ||B \phi||_{\LE^*_{t, x}}.
		\end{align*}	
	It remains to show the $B\phi$ term can be absorbed into the left-hand side. Indeed, the triangle inequality, Holder's inequality and the decay \eqref{decay} respectively imply 
		\begin{align*}
			||B\phi||_{\LE^*_{t, x}} 
				&\leq \sum_{N \in 2^\N} || \langle r \rangle^{\frac12} b^j \partial_j \phi||_{L^2_{t, x} (\R \times A_N)} + || \langle r \rangle^{\frac12} c \phi||_{L^2_{t, x} (\R \times A_N)} \\
				&\leq \left( \sup_{N \in 2^\N} || \langle r \rangle^{-\frac12} \partial_j \phi ||_{L^2_{t, x} (\R \times A_N)} \right) \left( \sum_{N \in 2^\N} \sup_{\R \times A_N} \langle r \rangle |b| \right) \\
				& \qquad + \left( \sup_{N \in 2^\N} || \langle r \rangle^{-\frac12} \langle r \rangle^{-1} \phi ||_{L^2_{t, x} (\R \times A_N)} \right) \left( \sum_{N \in 2^\N} \sup_{\R \times A_N} \langle r \rangle^2 |c| \right) \\
				&\leq \kappa \left( ||\nabla_{t, x} \phi||_{\LE_{t, x}} + ||\langle r \rangle^{-1} \phi||_{\LE_{t, x}} \right).
		\end{align*}
	Choosing $\kappa$ smaller than the implicit constant in \eqref{iled} for $\Box$ completes the proof. 	
\end{proof}	

\section{Energy estimates}\label{sec:redshift}

With our discussion of linear wave equations at hand, we are ready to begin our study of non-linear wave equations. 
    \[
        \Box_{\bfg(\phi)} := \bfg^{\mu \nu} (\phi) \partial_\mu \partial_\nu
    \]

    \begin{equation}\tag{P}\label{eq:perturb}
        | \bfg^{\mu \nu} - \bfm^{\mu \nu} |
            \ll 1. 
    \end{equation}


\begin{equation}\label{eq:QLW}\tag{QLW}
	\begin{split}
		\Box_{\bfg(\phi)} \phi 
			&= \mathcal N(\phi, \partial\phi), \\
		(\phi, \partial_t \phi)_{|t = 0}
			&= (\phi_0, \phi_1),
	\end{split}
\end{equation}


The simplest model to consider is the semi-linear wave equation with power-type non-linearity,
    \[
        \Box \phi 
            = \phi^3
    \]




\section{Energy decay via \texorpdfstring{$r^p$}-method}

To obtain energy decay estimates, we employ the $r^p$-method of Dafermos-Rodnianski \cite{DafermosRodnianski2010} in neighborhoods of future null infinity $\cI^+$ \textit{and} the event horizon $\cH^+$. As the argument near $\cI^+$ is standard, applying to a larger class of spacetimes, we will only give a brief account as a primer for the argument near $\cH^+$, which is a novelty of the extremal Reissner-Nordstr\"om background. 


\begin{figure}[h]\label{fig:spacenull}
    \begin{center}
        \includegraphics[scale = 0.7]{graphics/klainerman.PNG}
        \caption{Spacelike-null folation of Minkowski space, taken from \cite{Klainerman2011a}; on the left is $\R^{1 + 3}$ and on the right is the compactified Penrose diagram. }
    \end{center}
\end{figure}

To capture the late-time asymptotics of linear waves, it is convenient to work with a spacelike-null foliation of spacetime, see Figures \ref{fig:spacenull} and \ref{fig:spacelikenull2}, rather than a spacelike foliation. As an extreme example, for compactly supported initial data in 3 spatial dimensions, the strong Huygen's principle implies that the solution remains supported within the \textit{wave zone}. A space-like foliation $\Sigma_\tau$ will intersect the wave zone for all $\tau$ and thus cannot decay, while the space-like null foliation $\widetilde{\Sigma}_\tau$ does not see the solution at all for large $\tau$. Working in double null coordinates $(u, v)$, the multiplier $V = r^p \partial_v$ yields the following hierarchy of integrated energy estimates for the radiation field $\psi := r \phi$, 

\begin{proposition}[$\cI^+$-localised $r^p$-hierarchy]
    For $p< 3$, the spherically-symmetric solutions $\phi$ to the wave equation \eqref{LW} on the extremal Reissner-Nordstr\"om background satisfy the following $r^p$-weighted estimates, 
        \begin{equation}
            \int_{\widetilde{\cN}_{\tau_2}} r^p \frac{|\partial_v \psi|^2}{r^2} + \int_{\widetilde{\cD}_{[\tau_1, \tau_2]}} r^{p - 1} (p + 2) \frac{|\partial_v \psi|^2}{r^2} 
                \lesssim \int_{\widetilde{\Sigma}_{\tau_1}} {^{(T)} J_\mu [\psi]} n^\mu_{\widetilde{\Sigma}_{\tau_1}} + \int_{\widetilde{\cN}_{\tau_1}} r^p \frac{|\partial_v \psi|^2}{r^2}. \label{eq:rp1}
        \end{equation}
\end{proposition}



\begin{figure}[h]\label{fig:spacelikenull2}
    \begin{center}
        \includegraphics[scale = 0.75]{graphics/foliation2.PNG}
    \end{center}
    \caption{The spacelike-null foliation $\widetilde{\Sigma}_\tau$ in \cite{Aretakis2011a} of extremal Reissner-Nordstr\"om, which is used to derive the $r^p$-hierarchy of integrated energy estimates.}
\end{figure}

Given an energy which is uniformly bounded, $\cE[\phi[\tau]] \lesssim \cE[\phi[0]]$, to prove decay it suffices to estimate the energy on a \textit{discrete} sequence of dyadic times $\tau_n \sim C^n$. And indeed, using the pigeonhole principle or a variant of the mean value theorem, one can leverage the hierarchy of integrated energy estimates \eqref{rp1} along with uniform boundedness of the $r^p$-weighted energies to obtain such a sequence such that 
    \[
    \int_{\widetilde{\cN}_{\tau_n}} r^{p - 2}\frac{|\partial_v \psi|^2}{r^2} \lesssim \frac1{\tau_{n - 1}} \int_{\widetilde{\cN}_{\tau_{n - 1}}} r^{p - 1} \frac{|\partial_v \psi|^2}{r^2} \lesssim \frac1{\tau_{n - 1}} \frac{1}{\tau_{n - 2}} \int_{\widetilde{\cN}_{0}} r^{p} \frac{|\partial_v \psi|^2}{r^2}.
    \]
As the sequence of times was chosen dyadically, this two-fold appliation of the hierarchy immediately implies a $\tau^{-2}$ decay rate. For more details, see \cite[Section 5]{Aretakis2011a}, while for the general scheme, see the original paper of Dafermos-Rodnianski \cite{DafermosRodnianski2010} or the lecture notes of Klainerman \cite[Section 9]{Klainerman2011a}.

\subsection{$T$-$P$-$N$-hierarchy in a neighborhood of $\cH^+$}

In view of the Couch-Torrence correspondence between $\cI^+$ and $\cH^+$ (see Figure \ref{fig:CT} and the discussion therein), one might expect that the $r^p$-hierarchy near future null infinity corresponds to an analogous $(r - M)^{-p}$-hierarchy near the event horizon. We will not explicitly adopt this perspective, as it is technically unnecessary for our purposes, however we point the interested reader to \cite{AngelopoulosEtAl2018} which studied precise late-time asymptotics using a Couch-Torrence invariant spacelike-null foliation as depicted in Figure \ref{fig:AAG}.


Recall that in Section \ref{sec:redshift}, the stationary vector field $T$, which is null on the horizon and thus does not see any of the redshift effect, yielded a \textit{degenerate} energy, while the redshift vector field $N$ yielded a \textit{non-degenerate} energy. The key idea of Aretakis \cite[Section 5.4]{Aretakis2011} was to introduce a vector field $P$ which captures less of the degenerate redshift effect near the horizon $N$ but more when compared to $T$, in the sense that we have the following hierarchy of energies,
    \begin{align*}
        ^{(T)} J_\mu [\phi] n^\mu_{\Sigma_\tau}
            &\sim |\partial_v \phi|^2 + D \cdot |\partial_r \phi|^2, \\
        ^{(P)} J_\mu [\phi] n^\mu_{\Sigma_\tau}
            &\sim |\partial_v \phi|^2 + \sqrt D \cdot |\partial_r \phi|^2 , \\
        ^{(N)} J_\mu [\phi] n^\mu_{\Sigma_\tau}
            &\sim |\partial_v \phi|^2 + |\partial_r \phi|^2.
    \end{align*}
To first approximation, one should think of the $P$-vector field as time-like in the domain of outer communication and degenerating to null in a linear fashion as one approaches the event horizon, 
    \[
        P \approx T - \sqrt{D} \cdot Y
    \]


\begin{proposition}[Microscopic $T$-$P$-$N$-hierarchy]
        There exists a $T$-invariant time-like vector field $P$ such that 
            \begin{align}\label{eq:TPmicro}
                {^{(T)}J_\mu [\phi]} n^\mu_\Sigma 
                    \lesssim {^{(P)} K[\phi]},
            \end{align}
        and 
            \begin{align}\label{eq:PNmicro}
                {^{(P)}J_\mu [\phi]} n^\mu_\Sigma 
                    \lesssim {^{(N, \delta, -\frac12)} K[\phi]},
            \end{align}
        in an appropriate neighborhood $\cA$ of the horizon. 
    \end{proposition}

\begin{proof}
Let us now turn to the derivation of the $P$-vector field, taking the ansatz $P = P^v \partial_v + P^r \partial_r$ for coefficients $P^v$ and $P^r$ to be determined later. Fix radii $r_0$ and $r_1$ between the horizon and photon sphere $M < r_0 < r_1 < 2M$, we divide our analysis between the region near the horizon $r \leq r_0$ and the region far from the horizon $r \geq r_1$, and smoothly interpolate between the two. Away from the horizon, we only care for causality, so we take $P^v = 1$ and $P^r = 0$. It remains to choose $P$ near the horizon to satisfy \eqref{TPmicro}-\eqref{PNmicro}. 

We would like the vector field $P$ to be time-like and to capture the redshift effect in a sense between $T$ and $N$. To that end, we take $P^r := -\sqrt D$, in which case the $0$-current takes the form 
\[
    ^{(P)} K [\phi] 
        = F_{vv} \cdot |\partial_v \phi|^2 + F_{rr} \cdot |\partial_r \phi|^2 + F_{vr} \cdot (\partial_v \phi \partial_r \phi)
\]
where the coefficients of the sign-definite terms are 
\begin{align*}
    F_{vv} 
        &= \partial_r P^v, \\
    F_{rr} 
        &= D \cdot \left( \frac{M}{2r^2} + \frac{\sqrt D}{r}\right),
\end{align*}
and the coefficient of the sign-indefinite term is
    \[
        F_{vr}
            = \sqrt D \left(  \sqrt D \cdot \partial_r P^v + \frac2r \right). 
    \]

We need the former to control the latter. Using Cauchy's inequality on $F_{vr}$ yields
\[
    F_{vr} \leq \epsilon D + \frac{1}{\epsilon} \left( \sqrt D \cdot \partial_r P^v + \frac2r \right)^2.
\]
From our computation we see that $F_{rr} \sim D$ near the horizon, so choosing $\epsilon \ll 1$ this term in the $0$-current is favourable for controlling the first term on the right. On the other hand, in view of the degeneracy $\sqrt{D(M)} = 0$ at the horizon, we can choose the coefficient $P^v$ such that $F_{vv}$ dominates the second term on the right, i.e. $\tfrac1\epsilon (\sqrt D \cdot \partial_r P^v + \frac2r )^2 < \partial_r P^v$.\footnote{We leave this as an exercise.} Collecting these judicious choices of coefficients and parameters, we see that the $0$-current of $P$ is comparable to the $T$-energy, 
\[
    ^{(P)} K [\phi] 
        \sim |\partial_v \phi|^2 + D \cdot |\partial_r \phi|^2 \sim {^{(T)}} J_\mu [\phi] n^\mu_{\Sigma_\tau},
\]
furnishing \eqref{TPmicro}. 

Near the horizon, we compute $- \bfg(P, P) \sim \sqrt D$, so the $P$-energy takes the form 
    \[
        {^{(P)}} J_\mu [\phi] n^\mu_{\Sigma_\tau} 
            \sim |\partial_v \phi|^2 + \sqrt D \cdot |\partial_r \phi|^2 \sim {^{(N, \delta, -\frac12)}} K [\phi],
    \]  
which proves \eqref{PNmicro}. 
\end{proof}

With these at hand, we are now ready to establish the integrated $T$-$P$-$N$-hierarchy. Let us recall boundedness of $T$-energy and $N$-energy; we also have boundedness of the $P$-energy, 

\begin{proposition}[Uniform boundedness of $P$-energy]
    Let $\phi$ be a solution to the wave equation \eqref{LW} on the extremal Reissner-Nordstr\"om background, then 
        \begin{equation}\label{eq:boundP}
            \int_{\widetilde{\Sigma}_\tau} {^{(P)} J_\mu [\phi]} n^\mu_{\widetilde{\Sigma}_\tau} 
                \lesssim \int_{\widetilde{\Sigma}_0} {^{(P)} J_\mu [\phi]} n^\mu_{\widetilde{\Sigma}_0}.
        \end{equation}
\end{proposition}

\begin{proof}
    By Stokes' theorem \eqref{stokes} and the divergence identity \eqref{currents}, we have
        \[
            \int_{\widetilde{\Sigma}_\tau} {^{(P)} J_\mu [\phi]} n^\mu +  \int_{\cH^+} {^{(P)} J_\mu [\phi]} n^\mu +  \int_{\cI^+} {^{(P)} J_\mu} [\phi] n^\mu + \int_{\cR} {^{(P)}} K  
                = \int_{\widetilde{\Sigma}_0} {^{(P)} J_\mu [\phi]} n^\mu.
        \]
    Since $P$ is time-like, the boundary integrals over $\cH^+$ and $\cI^+$ are non-negative. For the space-time integral, \eqref{TPmicro} implies ${^{(P)}K}$ is non-negative near the horizon $r \leq r_0$, while by construction of $P$ it vanishes far away from the horizon $r \geq r_1$. In the intermediate region $r_0 \leq r \leq r_1$, we can estimate it by the right-hand side using integrated local energy decay \eqref{iled}. 
\end{proof}

\begin{proposition}[$\cH^+$-localised $T$-$P$-$N$-hierarchy]
    Let $\phi$ be a spherically-symmetric solution to the wave equation \eqref{LW} on the extremal Reissner-Nordstr\"om background, then 
    \begin{equation}\label{eq:TP}
        \int_{\tau_1}^{\tau_2} \left(\int_{\widetilde{\Sigma}_\tau \cap \{ M \leq r \leq r_0 \}} {^{(T)} J_\mu [\phi]} n^\mu_{\Sigma_\tau} \right) d \tau
            \lesssim \int_{\widetilde{\Sigma}_{{\tau_1}}} {^{(P)} J_\mu [\phi]} n^\mu_{\widetilde{\Sigma}_{\tau_1}}
    \end{equation}
and 
    \begin{equation}\label{eq:PN}
        \int_{\tau_1}^{\tau_2} \left(\int_{\widetilde{\Sigma}_\tau \cap \{ M \leq r \leq r_0 \}} {^{(P)} J_\mu [\phi]} n^\mu_{\Sigma_\tau} \right) d \tau
            \lesssim \int_{\widetilde{\Sigma}_{{\tau_1}}} {^{(N)} J_\mu [\phi]} n^\mu_{\widetilde{\Sigma}_{\tau_1}}
    \end{equation}
in an appropriate neighborhood $\cA$ of the horizon. 
\end{proposition}

\begin{proof}
    Using Stokes' theorem \eqref{stokes} on the divergence identity \eqref{currents} and using boundedness \eqref{boundP} to handle the boundary terms, we have 
        \[
            \int_{\cA} {^{(P)} K} 
                \lesssim \int_{\Sigma_{\tau_1}} {^{(P)} J_\mu [\phi]} n^\mu_{\Sigma_{0}}.
        \]
    Then integral inequality \eqref{TP} follows from the microscopic inequality \eqref{TPmicro}. Similarly, the integral inequality \eqref{PN} follows from the divergence identity for the modified $N$-current ${^{(N, \delta, -\frac12)}} J_\mu [\phi]$, boundedness of the $N$-energy, and the microscopic inequality \eqref{PNmicro}. 
\end{proof}




\subsection{Energy decay via dyadic pigeonholing}

With the $T$-$P$-$N$-hierarchy of estimates localised near the horizon, we are ready to establish energy decay. Using the $P$-$N$ estimate \eqref{PN}, boundedness of the $P$-energy \eqref{boundP}, and the pigeonhole principle, we have a dyadic sequence $\tau_n$ such that 
    \begin{equation}
        \int_{\Sigma_{\tau_n} \cap \cA}  {^{(P)} J_\mu [\phi]} n^\mu_{\Sigma_{\tau_n}} 
            \lesssim \frac{1}{\tau_n} \int_{\Sigma_0 \cap \cA}  {^{(N)} J_\mu [\phi]} n^\mu_{\Sigma_{\tau_n}} 
    \end{equation}
By the mean value theorem, we can find $\tau_* \in [\tau_n, \tau_{n + 1}]$ such that 
    \begin{equation}
        \int_{\Sigma_{\tau_*} \cap \cA}  {^{(T)} J_\mu [\phi]} n^\mu_{\Sigma_{\tau_*}}
            = \frac{1}{\tau_{n + 1} - \tau_n} \int_{\tau_n}^{\tau_{n + 1}} \int_{\Sigma_{\tau} \cap \cA}  {^{(T)} J_\mu [\phi]} n^\mu_{\Sigma_{\tau}}.
    \end{equation}
By the $T$-$P$ estimate, 
    \begin{equation}
        \int_{\tau_n}^{\tau_{n + 1}} \int_{\Sigma_{\tau} \cap \cA}  {^{(T)} J_\mu [\phi]} n^\mu_{\Sigma_{\tau}} \lesssim \int_{\Sigma_{\tau_n} \cap \cA}  {^{(P)} J_\mu [\phi]} n^\mu_{\Sigma_{\tau_n}} . 
    \end{equation}
Collecting the inequalities and using the dyadic property $\tau_n \sim \tau_{n + 1} \sim \tau_{n + 1} - \tau_n$, we conclude decay of the $T$-energy, 
\begin{equation}
        \int_{\Sigma_{\tau_{n + 1}} \cap \cA}  {^{(T)} J_\mu [\phi]} n^\mu_{\Sigma_{\tau_{n + 1}}} 
            \lesssim \frac{1}{\tau_{n + 1}^2}. 
\end{equation}
This completes the proof of the energy decay statement \eqref{energydecay} of Theorem \ref{thm:stability}.

\section{Pointwise estimates}
Obtaining pointwise boundedness and decay estimates from the energy boundedness and decay estimates follow from standard applications of Sobolev embedding. That is, we can write $\phi$ using the fundamental theorem of calculus on the interval $[r, \infty)$ and applying Cauchy-Schwartz to obtain the estimate
    \begin{align} \label{eq:sobolev}
        |\phi(r)|^2
            &\leq \frac{1}{r}  \int_{r}^\infty |\partial_\rho \phi|^2 \, \rho^2 d \rho. 
    \end{align}
Far away from the horizon $r \geq R_0$, we have $|\partial_\rho \phi|^2 \lesssim {^{(T)} J_\mu [\phi]} n^\mu_{\widetilde{\Sigma}_\tau}$, so applying decay of the $T$-energy far away from the horizon furnishes 

\begin{proposition}[Decay away from the horizon]
    Let $\phi$ be a spherically-symmetric solution to the linear wave equation \eqref{LW} on an extremal Reissner-Nordstr\"om background, then 
        \begin{equation}
            |\phi(r)|^2 \lesssim \frac{1}{\tau^2}
        \end{equation}
    away from the horizon. 
\end{proposition}

\begin{proof}
    We have
    \begin{align*}
        |\phi(r)|^2
            &\lesssim \frac1r \int_{\widetilde{\Sigma}_\tau} {^{(T)} J_\mu [\phi]} n^\mu_{\widetilde{\Sigma}_\tau}.
    \end{align*}\
    Using decay of the $T$-energy \eqref{energydecay} completes the proof. 
\end{proof}

This completes the proof of the pointwise decay away from the horizon \eqref{pointwisedecay} from Theorem \ref{thm:stability}.

Close to the horizon $M \leq r \leq R_0$, the control over the transverse derivatives in the $T$-energy degenerates, so we are left with instead 

\begin{lemma}[Degenerate decay near horizon]
    Let $\phi$ be a spherically-symmetric solution to the linear wave equation \eqref{LW} on an extremal Reissner-Nordstr\"om background, then 
        \begin{equation}\label{eq:degenerate}
            |\phi (r)|^2
                \lesssim \frac{1}{(r - M)^2} \frac{1}{\tau^2}. 
        \end{equation}
\end{lemma}

\begin{proof}
    By Sobolev embedding \eqref{sobolev} and pointwise coercivity, we have
        \begin{align*}
            |\phi(r)|^2 
                &\lesssim \frac1r \int_{\Sigma_\tau \cap \{ r' \geq r\}} {^{(N)}J_\mu [\phi]} n^\mu \\
                &\lesssim \frac{1}{r D(r)} \int_{\Sigma_\tau} {^{(T)}J_\mu [\phi]} n^\mu.
        \end{align*}
    The result follows then from decay of the $T$-energy \eqref{energydecay}.
\end{proof}

Nonetheless, we can interpolate between the degenerate pointwise decay \textit{near} the horizon and integrated decay \textit{up to} the horizon to obtain decay \textit{at} the horizon. That is, using the fundamental theorem of calculus on the interval $[r_0, r_0 + \tau^{-\alpha}]$, see Figure \ref{fig:interpolate}, for $\alpha$ to be chosen later, we have

\begin{lemma}[Interpolation]
    Let $\phi$ be spherically-symmetric, then   
    \begin{align}\label{eq:interpolation}
        |\phi(r)|^2 
            \lesssim |\phi(r + \tau^{-\alpha})|^2 +2 \int_{\Sigma_\tau \cap \{ r_0 \leq r \leq r_0 + \tau^{-\alpha}\}} \phi \partial_\rho \phi .
    \end{align}
\end{lemma}

\begin{figure}[h]\label{fig:interpolate}
    \begin{center}
        \includegraphics[scale = 0.7]{graphics/foliation3.PNG}
    \end{center}
    \caption{Interpolation between the degenerate decay near the horizon and the integrated decay up to the horizon by using the fundamental theorem of calculus.}
\end{figure}

The first term on the right has favourable decay as it concerns a region away from the horizon, while the second term is favourable since $\partial_\rho \phi$ is uniformly bounded, $\phi$ decays in an integrated sense, and the region of integration decays. 

\begin{proposition}[Pointwise decay near the horizon]   
    Let $\phi$ be a spherically-symmetric solution to the linear wave equation \eqref{LW} on an extremal Reissner-Nordstr\"om background, then 
        \begin{equation}
            |\phi(r)|^2 \lesssim \frac{1}{\tau^{6/5}}
        \end{equation}
    near the horizon. 
\end{proposition}

\begin{proof}
    Continuing from \eqref{interpolation}, we estimate the first term on the right using the degenerate decay \eqref{degenerate}, and the second term on the right by placing $\partial_\rho \phi$ in $L^\infty$ and $\phi$ in $L^2$. Controlling the latter using Hardy's inequality and $T$-energy decay \eqref{energydecay}, we obtain
        \[
            |\phi(r)|^2 \lesssim \tau^{-2 + 2\alpha} + \tau^{-\frac\alpha2 - 1} .
        \]
    Optimising by taking $\alpha = \frac25$ furnishes the desired decay rate. 
\end{proof}

This completes the proof of the pointwise decay \eqref{pointwisedecay} near the horizon from Theorem \ref{thm:stability}.





\bibliographystyle{alpha}
\bibliography{external/biblio}

\end{document}
