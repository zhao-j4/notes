

\subsection{Degenerate energy monotonicity formula}

Using the Killing vector field $T$, we can obtain an exact conservation law for linear waves. Indeed, using Stokes' theorem and the divergence identity \eqref{div} on the region $\cR_{[0, \tau]}$, we obtain 
    \begin{equation}
        \int_{\Sigma_\tau} {^{(T)} J_\mu [\phi]} n^\mu_{\Sigma_\tau} + \int_{\cH^+} {^{(T)} J_\mu[\phi]} n^\mu_{\cH^+}
            = \int_{\Sigma_0} {^{(T)} J_\mu [\phi]} n^\mu_{\Sigma_0}.
    \end{equation}
Since $T$ and $n^\mu_{\cH^+}$ are null, the second boundary term on the left is non-negative. In fact, one may compute ${^{(T)} J_\mu [\phi]} n^\mu_{\cH^+} = |\partial_v \phi|^2$ on the event horizon. This yields the following monotonicity formula,

    \begin{proposition}[Boundedness of $T$-energy]
        Let $\phi$ be a solution to the wave equation \eqref{LW} on the extremal Reissner-Nordström background, then 
            \begin{equation}
                \int_{\Sigma_\tau} {^{(T)} J_\mu [\phi]} n^\mu_{\Sigma_\tau} 
                    \leq \int_{\Sigma_0} {^{(T)} J_\mu [\phi]} n^\mu_{\Sigma_0}.
            \end{equation}  
    \end{proposition}

However, since $T$ degenerates to null approaching the horizon, and in fact $\bfg (T, T) = - \tfrac12 D$ which vanishes to second-order at the horizon, the $T$-energy along the foliation takes the form 
    \[
        {^{(T)} J_\mu [\phi]} n^\mu_{\Sigma_\tau}  \sim |\partial_v \phi|^2 + D \cdot |\partial_r \phi|^2 .
    \]
In particular, the control over the ingoing null derivative $\partial_v \phi$ degenerates along the horizon. 

\subsection{Degeneracy of redshift}

To handle the degeneracy near the horizon, it is standard to use the redshift vector field $N$ introduced by Dafermos-Rodnianski \cite{DafermosRodnianski2009}. The basic strategy is to choose $N$ to be a time-like vector field, to first approximation taking in double null coordinates $(u, v)$ the form 
    \[
        N \approx \frac{\partial_u}{-(\partial_u r)}
    \]
satisfying 
    \[
        ^{(N) J_\mu [\phi]} n^\mu_{\Sigma_\tau}  \sim {^{(N)} K[\phi]}.
    \]
In particular, the bulk term ${^{(N)}K[\phi]}$ is non-negative. This construction relies crucially on the positivity of \textit{surface gravity} $\kappa$, which measures the strength of the redshift effect. Precisely, since $T$ is null, Killing, and tangential along the horizon, the surface gravity is defined by 
    \[
        \nabla_T T = \kappa T. 
    \]
Assuming $\kappa > 0$, one can show both non-degenerate energy boundedness and integrated local energy decay, 
\begin{equation}
    \int_{\Sigma_\tau} {^{(N)} J_\mu [\phi]} n^\mu_{\Sigma_\tau} + \int_\cA {^{(N)} J_\mu [\phi]} n^\mu 
        \lesssim \int_{\Sigma_0} {^{(N)} J_\mu [\phi]} n^\mu_{\Sigma_0}
\end{equation}
see \cite{DafermosRodnianski2009} and also the lecture notes \cite[Lectures on black holes and linear waves]{EllwoodEtAl2013}. 

In the extremal setting, the surface gravity vanishes, in which case one cannot choose $N$ such that the bulk term is non-negative on the horizon. Indeed, taking the ansatz $N = N^v \partial_v + N^r \partial_r$ we compute 
\[
    ^{(N)} K [\phi] 
        = F_{vv} \cdot |\partial_v \phi|^2 + F_{rr} \cdot |\partial_r \phi|^2 + F_{vr} \cdot (\partial_v \phi \partial_r \phi)
\]
where the coefficients are given by 
\begin{align*}
    F_{vv} 
        &= \partial_r N^v, \\
    F_{rr}
        &= D \cdot \left( \frac{\partial_r N^r}{2} - \frac{N^r}{r}  \right) - \frac{N^r D'}{2},\\
    F_{vr}
        &= D \cdot \partial_r N^v - \frac{2N^r}{r}.
\end{align*}
Since $N$ is necessarily time-like, we need $N^r (M) \neq 0$ since $\bfg(N, N) = - D \cdot |N^v|^2 + 2 N^v N^r$. It follows that $F_{rr} (M) = 0$, wherease the sign-indefinite coefficient $F_{vr}$ is non-zero. It follows that the bulk term is linear in $\partial_r \phi$ on the horizon and manifestly sign indefinite. 

\subsection{Generalised currents}

To rectify this issue, we introduce an extra Lagrangian term $w \cdot \phi \nabla_\mu \phi$ in the current; this corresponds to multiplying the wave equation by $w\phi$. Define the generalised currents by 
\begin{align*}
    ^{(N, w)} J_\mu [\phi] 
        &:= {^{(N)} J_\mu [\phi]} + w \cdot \psi \nabla_\mu \phi, \\ 
    ^{(N, w)} K[\phi]
        &:= {^{(N)} K}[\phi] + \nabla^\mu w \cdot \phi \nabla_\mu \phi + w \cdot \nabla^\alpha \phi \nabla_\alpha \phi.
\end{align*}
Then for solutions to the wave equation, we have the divergence identity
\begin{equation}\label{eq:div2}
    \nabla^\mu \left( ^{(N, w)} J_\mu [\phi]  \right) = ^{(N, w)} K[\phi].
\end{equation}
Taking 
    \[
        w = \frac{N^r(M)}{M} = - \frac12, \qquad N = 16 r \partial_v + \left( \frac32 r + M \right) \partial_r, 
    \]
we obtain 

\begin{proposition}[Coercivity of $N$-bulk term]
    Let $\phi$ be a solution to the wave equation \eqref{LW} on the extremal Reissner-Nordström background, then 
        \begin{equation}
            ^{(N, -\frac12)} K[\phi] 
                \gtrsim |\partial_v \phi|^2 + \sqrt D \cdot |\partial_r \phi|^2 
        \end{equation}
    in a neighborhood of the event horizon $\cA = \{ M \leq r \leq \tfrac98 M\}$
\end{proposition}

\begin{proof}
    We compute 
        \begin{align*}
            {^{N, w}} K[\phi]
                &= F_{vv} \cdot |\partial_v \phi|^2+ (F_{rr} + w D) \cdot |\partial_r \phi|^2 + (F_{vr} + 2w) \cdot (\partial_v \phi \partial_r\phi) .
        \end{align*}
    Then 
        \begin{align*}
            F_{vv}
                &= 16, \\
            F_{rr} + wD
                &= F_{rr} -\frac12 D,\\
            F_{vr} -1
                &= 16 D + 2 \sqrt D.
        \end{align*}
    Note the second coefficient is positive since the leading term $D'$ has a positive coefficient. Suitable application of Cauchy-Schwarz implies that the sign-indefinite term $F_{vr}$ can be controlled by the sign-definite terms. 
\end{proof}

\subsection{Cut-off generalised currents}

Since the bulk term ${^{(N, -\frac12)} K[\phi]}$ cannot be chosen to be non-negative far away from the horizon, we simply modify the current by choosing appropriate cut-offs. To this end, we extend the redshift vector field by 
    \begin{align*}
        N^v = 1, \qquad r \geq \tfrac87 M, \\
        N^r = 0, \qquad r \geq \tfrac87M. 
    \end{align*}
Then $N$ is future-directed time-like, and we define the cut-off currents by 
    \begin{align*}
        ^{(N, \delta, -\frac12)} J_\mu [\phi] 
            &:= {^{(N)} J_\mu [\phi]} - \frac12 \delta \phi \nabla_\mu \phi, \\
        ^{(N, \delta, -\frac12)} K[\phi]
            &:= \nabla^\mu \left( {^{(N, \delta, - \frac12)} J_\mu [\phi]}\right)
    \end{align*}
where $\delta$ is a cut-off satisfying $\delta \equiv 1$ for $M \leq r \leq \tfrac98 M$ and $\delta \equiv 0$ for $\tfrac87 M < r < \infty$. We arrive at the divergence identity, 
    \begin{equation}\label{eq:div3}
        ^{(N, \delta, -\frac12)} K[\phi] = \nabla^\mu \left( ^{(N, \delta, -\frac12)} J_\mu [\phi]  \right) 
    \end{equation}
By Hardy's inequality, one can verify 

\begin{lemma}[Comparable $N$-energies]
    For sufficiently regular $\phi$, we have 
        \begin{equation}
            \int_{\Sigma_\tau} {^{(N, \delta, -\frac12)} J_\mu [\phi]} n^\mu
                \sim \int_{\Sigma_\tau} {^{(N)} J_\mu [\phi]} n^\mu_{\Sigma_\tau} .
        \end{equation}
\end{lemma}


\begin{proposition}[Uniform boundedness of $N$-energy]
    Let $\phi$ be a solution to the wave equation \eqref{LW} on the extremal Reissner-Nordström background, then 
        \begin{equation}\label{eq:Nbound}
            \int_{\Sigma_\tau} {^{(N)} J_\mu [\phi]} n^\mu_{\Sigma_\tau} 
                \lesssim \int_{\Sigma_0} {^{(N)} J_\mu [\phi]} n^\mu_{\Sigma_0}.
        \end{equation}
\end{proposition}

\begin{proof}
    By Stokes' theorem \eqref{stokes} and the divergence identity \eqref{div3}, we have 
        \begin{align*}
            \int_{\Sigma_\tau} {^{(N, \delta, - \frac12)} J_\mu [\phi]} n^\mu + \int_{\cR} {^{(N, \delta, - \frac12)} K[\phi]} + \int_{\cH^+} {^{(N, \delta, - \frac12)} J_\mu[\phi]} n^\mu  
                &= \int_{\Sigma_0} {^{(N, \delta, - \frac12)} J_\mu [\phi]} n^\mu .
        \end{align*}
   The space-time integral has the correct sign close to the horizon, vanishes away from the horizon, while the intermediate region can be handled by a degenerate integrated local energy decay \eqref{iled}. The boundary term on the horizon has the correct sign. 
\end{proof}

\begin{proposition}[Integrated local energy decay]
    Let $\phi$ be a solution to the wave equation \eqref{LW} on the extremal Reissner-Nordström background, then for any $\tau_1 < \tau_2$, we have
        \begin{equation}
            \int_{\tau_1}^{\tau_2} \int_{\Sigma_\tau} {^{(N, - \frac12)} K[\phi]}
                \lesssim \int_{\Sigma_{\tau_1}} {^{(N)} J_\mu [\phi]} n^\mu_{\Sigma_{\tau_1}}
        \end{equation}
\end{proposition}


