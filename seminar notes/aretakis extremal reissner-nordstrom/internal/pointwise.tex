Obtaining pointwise boundedness and decay estimates from the energy boundedness and decay estimates follow from standard applications of Sobolev embedding. That is, we can write $\phi$ using the fundamental theorem of calculus on the interval $[r, \infty)$ and applying Cauchy-Schwartz to obtain the estimate
    \begin{align} \label{eq:sobolev}
        |\phi(r)|^2
            &\leq \frac{1}{r}  \int_{r}^\infty |\partial_\rho \phi|^2 \, \rho^2 d \rho. 
    \end{align}
Far away from the horizon $r \geq R_0$, we have $|\partial_\rho \phi|^2 \lesssim {^{(T)} J_\mu [\phi]} n^\mu_{\widetilde{\Sigma}_\tau}$, so applying decay of the $T$-energy far away from the horizon furnishes 

\begin{proposition}[Decay away from the horizon]
    Let $\phi$ be a spherically-symmetric solution to the linear wave equation \eqref{LW} on an extremal Reissner-Nordstr\"om background, then 
        \begin{equation}
            |\phi(r)|^2 \lesssim \frac{1}{\tau^2}
        \end{equation}
    away from the horizon. 
\end{proposition}

\begin{proof}
    We have
    \begin{align*}
        |\phi(r)|^2
            &\lesssim \frac1r \int_{\widetilde{\Sigma}_\tau} {^{(T)} J_\mu [\phi]} n^\mu_{\widetilde{\Sigma}_\tau}.
    \end{align*}\
    Using decay of the $T$-energy \eqref{energydecay} completes the proof. 
\end{proof}

This completes the proof of the pointwise decay away from the horizon \eqref{pointwisedecay} from Theorem \ref{thm:stability}.

Close to the horizon $M \leq r \leq R_0$, the control over the transverse derivatives in the $T$-energy degenerates, so we are left with instead 

\begin{lemma}[Degenerate decay near horizon]
    Let $\phi$ be a spherically-symmetric solution to the linear wave equation \eqref{LW} on an extremal Reissner-Nordstr\"om background, then 
        \begin{equation}\label{eq:degenerate}
            |\phi (r)|^2
                \lesssim \frac{1}{(r - M)^2} \frac{1}{\tau^2}. 
        \end{equation}
\end{lemma}

\begin{proof}
    By Sobolev embedding \eqref{sobolev} and pointwise coercivity, we have
        \begin{align*}
            |\phi(r)|^2 
                &\lesssim \frac1r \int_{\Sigma_\tau \cap \{ r' \geq r\}} {^{(N)}J_\mu [\phi]} n^\mu \\
                &\lesssim \frac{1}{r D(r)} \int_{\Sigma_\tau} {^{(T)}J_\mu [\phi]} n^\mu.
        \end{align*}
    The result follows then from decay of the $T$-energy \eqref{energydecay}.
\end{proof}

Nonetheless, we can interpolate between the degenerate pointwise decay \textit{near} the horizon and integrated decay \textit{up to} the horizon to obtain decay \textit{at} the horizon. That is, using the fundamental theorem of calculus on the interval $[r_0, r_0 + \tau^{-\alpha}]$, see Figure \ref{fig:interpolate}, for $\alpha$ to be chosen later, we have

\begin{lemma}[Interpolation]
    Let $\phi$ be spherically-symmetric, then   
    \begin{align}\label{eq:interpolation}
        |\phi(r)|^2 
            \lesssim |\phi(r + \tau^{-\alpha})|^2 +2 \int_{\Sigma_\tau \cap \{ r_0 \leq r \leq r_0 + \tau^{-\alpha}\}} \phi \partial_\rho \phi .
    \end{align}
\end{lemma}

\begin{figure}[h]\label{fig:interpolate}
    \begin{center}
        \includegraphics[scale = 0.7]{graphics/foliation3.PNG}
    \end{center}
    \caption{Interpolation between the degenerate decay near the horizon and the integrated decay up to the horizon by using the fundamental theorem of calculus.}
\end{figure}

The first term on the right has favourable decay as it concerns a region away from the horizon, while the second term is favourable since $\partial_\rho \phi$ is uniformly bounded, $\phi$ decays in an integrated sense, and the region of integration decays. 

\begin{proposition}[Pointwise decay near the horizon]   
    Let $\phi$ be a spherically-symmetric solution to the linear wave equation \eqref{LW} on an extremal Reissner-Nordstr\"om background, then 
        \begin{equation}
            |\phi(r)|^2 \lesssim \frac{1}{\tau^{6/5}}
        \end{equation}
    near the horizon. 
\end{proposition}

\begin{proof}
    Continuing from \eqref{interpolation}, we estimate the first term on the right using the degenerate decay \eqref{degenerate}, and the second term on the right by placing $\partial_\rho \phi$ in $L^\infty$ and $\phi$ in $L^2$. Controlling the latter using Hardy's inequality and $T$-energy decay \eqref{energydecay}, we obtain
        \[
            |\phi(r)|^2 \lesssim \tau^{-2 + 2\alpha} + \tau^{-\frac\alpha2 - 1} .
        \]
    Optimising by taking $\alpha = \frac25$ furnishes the desired decay rate. 
\end{proof}

This completes the proof of the pointwise decay \eqref{pointwisedecay} near the horizon from Theorem \ref{thm:stability}.



